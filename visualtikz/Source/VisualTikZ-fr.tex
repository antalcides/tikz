\input{versFR}



    \documentclass[a4paper,10pt]{article}

 \usepackage{fontspec}
\usepackage[frenchb,english]{babel}

%\TFRGB{\selectlanguage{french}}{\selectlanguage{english}}

\usepackage{amsmath,amsfonts,amssymb}

\usepackage{pdfpages}  


%\usepackage{pst-all}

\usepackage{graphicx} 
\usepackage{hyperref}

\usepackage{animate}
\usepackage{makeidx}
%\usepackage{wrapfig}

\usepackage{pgfplots} %<<<<<<<<<<<<<<<<<<<<<<<<<<<<< 
\usepackage{tikz}
\usepackage{tkz-tab} 
%\usepgflibrary{shapes.callouts}
\usepackage{tikz-qtree}
\usepackage{tkz-tab}
\usepackage{csquotes}

  
\usetikzlibrary{angles}
\usetikzlibrary{arrows}
\usetikzlibrary{patterns}
\usetikzlibrary{shadings}
\usetikzlibrary{calc}
\usetikzlibrary{backgrounds}
\usetikzlibrary{decorations.pathmorphing}

\usetikzlibrary{decorations.markings}
\usetikzlibrary{decorations.footprints}
\usetikzlibrary{decorations.shapes}
\usetikzlibrary{decorations.text}
\usetikzlibrary{decorations.fractals}
\usepgflibrary{shapes.geometric}
\usetikzlibrary{intersections}
\usetikzlibrary{scopes}
\usetikzlibrary{shapes.symbols}
\usetikzlibrary{shapes.arrows}
\usetikzlibrary{shapes.callouts}
\usetikzlibrary{shapes.misc}
\usepgflibrary{shapes.multipart}
\usetikzlibrary{plotmarks}
\usetikzlibrary{trees}
\usetikzlibrary{fadings}
\usetikzlibrary{arrows.meta}
\usetikzlibrary{bending}
\usetikzlibrary{fit}
%\usetikzlibrary{circuits}
\usetikzlibrary{circuits.ee.IEC}
%\usetikzlibrary{circuits.logic}

%\usetikzlibrary{decorations}





%\usetikzlibrary{babel}
\usetikzlibrary{math}


\usetikzlibrary{quotes}


\pgfplotsset{compat=1.8}
%\usetikzlibrary{positioning}

\usepackage{geometry}
\geometry{a4paper,top={3cm}}






%====================================================================

\makeindex


\newcommand{\AC}[1]{\{#1\}}

\newcommand{\BDD}[1]{{\color{blue}  #1}\index{\textbf{4 Variables Tikz}!#1}}

\newcommand{\BS}[1]{$\backslash$#1}

\newcommand{\BSB}[1]{\textbf{\color{blue} {$\backslash$#1}}}


\newcommand{\BSR}[1]{\textbf{\color{red}  $\backslash$#1}}




%\newcommand{\RDDX}[2]{{\color{red}#1} \index{\textbf{3 Paramètres et options}!#2=#1}}


\newcommand{\RRR}[1]{\tikz[baseline=-1mm]  \draw node[draw,fill=red!20] {{\footnotesize  PGFmanual section :  #1}} ; }

%\newcommand{\RRR}[1]{\tikz[baseline=-1mm]  \draw node[draw,fill=red!20] {{\footnotesize  PGFmanual section :  #1}} ;\index{\textbf{5 PGFmanual }!#1} }

\newcommand{\DFR}{ \tikzpicture[scale=.25]
\draw [fill=blue](0,0) rectangle (3,1.5);
\draw [fill=white](1,0) rectangle (2,1.5);
\draw [fill=red](2,0) rectangle (3,1.5);\endtikzpicture }

\newcommand{\DGB}{ \tikzpicture[scale=.25]
\draw [fill=blue](0,0) rectangle (3,1.5);
\draw [white,line width=.1cm](0,0) -- (3,1.5);
\draw [white,line width=.1cm](0,1.5) -- (3,0);
\draw [white,line width=.1cm](1.5,0) -- (1.5,1.5);
\draw [white,line width=.1cm](0,0.75) -- (3,0.75);
\draw [red,line width=.05cm](0,0) -- (3,1.5);
\draw [red,line width=.05cm](0,1.5) -- (3,0);
\draw [red,line width=.05cm](1.5,0) -- (1.5,1.5);
\draw [red,line width=.05cm](0,0.75) -- (3,0.75);
\endtikzpicture }



\TFRGB{
\newcommand{\ESS}[1]{\textbf{\textbackslash{#1}}\index{\textbf{1 Environnements}!#1 @\textbackslash{}#1}}
\newcommand{\BSS}[1]{\textbf{\textbackslash{#1}}\index{\textbf{2 Commandes}!#1 @\textbackslash{}#1}}
\newcommand{\BSI}[2]{ \index{\textbf{4) Par modules }!\textbf{#2}!#1@\textbackslash{}#1 (M)}}
\newcommand{\DDD}[1]{{\color{red}  #1}\index{\textbf{3 Paramètres et options!Elements}!#1}}
\newcommand{\FDD}[1]{{\color{red}  #1}\index{\textbf{5 Extrémit\'es}!#1}}
\newcommand{\RDD}[1]{{\color{red}  #1}\index{\textbf{3 Paramètres et options}!#1}}
\newcommand{\RDDD}[2]{{\color{red}#1} \index{\textbf{Paramètres et options}!#2!#1}}
\newcommand{\RDDX}[2]{{\color{red}#1} \index{\textbf{4 Options}!#1 (#2)}}
}
{
\newcommand{\ESS}[1]{\textbf{\textbackslash{#1}}\index{\textbf{1 Environments}!#1 @\textbackslash{}#1}}
\newcommand{\BSS}[1]{\textbf{\textbackslash{#1}}\index{\textbf{2 Commands}!#1 @\textbackslash{}#1}}
\newcommand{\BSI}[2]{ \index{\textbf{4) Par modules }!\textbf{#2}!#1@\textbackslash{}#1 (M)}}
\newcommand{\DDD}[1]{{\color{red}  #1}\index{\textbf{3 Parameters and options!Elements}!#1}}
\newcommand{\FDD}[1]{{\color{red}  #1}\index{\textbf{5 Extrémities}!#1}}
\newcommand{\RDD}[1]{{\color{red}  #1}\index{\textbf{3 Parameters and options}!#1}}
\newcommand{\RDDD}[2]{{\color{red}#1} \index{\textbf{Parameters and options}!#2!#1}}
\newcommand{\RDDX}[2]{{\color{red}#1} \index{\textbf{4 Options}!#1 (#2)}}
}

\newcommand{\rouge}[1] {{\color{red}  #1}}
\newcommand{\blll}[1] {{\color{blue}  #1}}



 \begin{document}
%  \newrgbcolor{Vert}{0 .8 0}
\selectlanguage{french}


%    \documentclass[a4paper,10pt]{article}

 \usepackage{fontspec}
\usepackage[frenchb,english]{babel}

%\TFRGB{\selectlanguage{french}}{\selectlanguage{english}}

\usepackage{amsmath,amsfonts,amssymb}

\usepackage{pdfpages}  


%\usepackage{pst-all}

\usepackage{graphicx} 
\usepackage{hyperref}

\usepackage{animate}
\usepackage{makeidx}
%\usepackage{wrapfig}

\usepackage{pgfplots} %<<<<<<<<<<<<<<<<<<<<<<<<<<<<< 
\usepackage{tikz}
\usepackage{tkz-tab} 
%\usepgflibrary{shapes.callouts}
\usepackage{tikz-qtree}
\usepackage{tkz-tab}
\usepackage{csquotes}

  
\usetikzlibrary{angles}
\usetikzlibrary{arrows}
\usetikzlibrary{patterns}
\usetikzlibrary{shadings}
\usetikzlibrary{calc}
\usetikzlibrary{backgrounds}
\usetikzlibrary{decorations.pathmorphing}

\usetikzlibrary{decorations.markings}
\usetikzlibrary{decorations.footprints}
\usetikzlibrary{decorations.shapes}
\usetikzlibrary{decorations.text}
\usetikzlibrary{decorations.fractals}
\usepgflibrary{shapes.geometric}
\usetikzlibrary{intersections}
\usetikzlibrary{scopes}
\usetikzlibrary{shapes.symbols}
\usetikzlibrary{shapes.arrows}
\usetikzlibrary{shapes.callouts}
\usetikzlibrary{shapes.misc}
\usepgflibrary{shapes.multipart}
\usetikzlibrary{plotmarks}
\usetikzlibrary{trees}
\usetikzlibrary{fadings}
\usetikzlibrary{arrows.meta}
\usetikzlibrary{bending}
\usetikzlibrary{fit}
%\usetikzlibrary{circuits}
\usetikzlibrary{circuits.ee.IEC}
%\usetikzlibrary{circuits.logic}

%\usetikzlibrary{decorations}





%\usetikzlibrary{babel}
\usetikzlibrary{math}


\usetikzlibrary{quotes}


\pgfplotsset{compat=1.8}
%\usetikzlibrary{positioning}

\usepackage{geometry}
\geometry{a4paper,top={3cm}}






%====================================================================

\makeindex


\newcommand{\AC}[1]{\{#1\}}

\newcommand{\BDD}[1]{{\color{blue}  #1}\index{\textbf{4 Variables Tikz}!#1}}

\newcommand{\BS}[1]{$\backslash$#1}

\newcommand{\BSB}[1]{\textbf{\color{blue} {$\backslash$#1}}}


\newcommand{\BSR}[1]{\textbf{\color{red}  $\backslash$#1}}




%\newcommand{\RDDX}[2]{{\color{red}#1} \index{\textbf{3 Paramètres et options}!#2=#1}}


\newcommand{\RRR}[1]{\tikz[baseline=-1mm]  \draw node[draw,fill=red!20] {{\footnotesize  PGFmanual section :  #1}} ; }

%\newcommand{\RRR}[1]{\tikz[baseline=-1mm]  \draw node[draw,fill=red!20] {{\footnotesize  PGFmanual section :  #1}} ;\index{\textbf{5 PGFmanual }!#1} }

\newcommand{\DFR}{ \tikzpicture[scale=.25]
\draw [fill=blue](0,0) rectangle (3,1.5);
\draw [fill=white](1,0) rectangle (2,1.5);
\draw [fill=red](2,0) rectangle (3,1.5);\endtikzpicture }

\newcommand{\DGB}{ \tikzpicture[scale=.25]
\draw [fill=blue](0,0) rectangle (3,1.5);
\draw [white,line width=.1cm](0,0) -- (3,1.5);
\draw [white,line width=.1cm](0,1.5) -- (3,0);
\draw [white,line width=.1cm](1.5,0) -- (1.5,1.5);
\draw [white,line width=.1cm](0,0.75) -- (3,0.75);
\draw [red,line width=.05cm](0,0) -- (3,1.5);
\draw [red,line width=.05cm](0,1.5) -- (3,0);
\draw [red,line width=.05cm](1.5,0) -- (1.5,1.5);
\draw [red,line width=.05cm](0,0.75) -- (3,0.75);
\endtikzpicture }



\TFRGB{
\newcommand{\ESS}[1]{\textbf{\textbackslash{#1}}\index{\textbf{1 Environnements}!#1 @\textbackslash{}#1}}
\newcommand{\BSS}[1]{\textbf{\textbackslash{#1}}\index{\textbf{2 Commandes}!#1 @\textbackslash{}#1}}
\newcommand{\BSI}[2]{ \index{\textbf{4) Par modules }!\textbf{#2}!#1@\textbackslash{}#1 (M)}}
\newcommand{\DDD}[1]{{\color{red}  #1}\index{\textbf{3 Paramètres et options!Elements}!#1}}
\newcommand{\FDD}[1]{{\color{red}  #1}\index{\textbf{5 Extrémit\'es}!#1}}
\newcommand{\RDD}[1]{{\color{red}  #1}\index{\textbf{3 Paramètres et options}!#1}}
\newcommand{\RDDD}[2]{{\color{red}#1} \index{\textbf{Paramètres et options}!#2!#1}}
\newcommand{\RDDX}[2]{{\color{red}#1} \index{\textbf{4 Options}!#1 (#2)}}
}
{
\newcommand{\ESS}[1]{\textbf{\textbackslash{#1}}\index{\textbf{1 Environments}!#1 @\textbackslash{}#1}}
\newcommand{\BSS}[1]{\textbf{\textbackslash{#1}}\index{\textbf{2 Commands}!#1 @\textbackslash{}#1}}
\newcommand{\BSI}[2]{ \index{\textbf{4) Par modules }!\textbf{#2}!#1@\textbackslash{}#1 (M)}}
\newcommand{\DDD}[1]{{\color{red}  #1}\index{\textbf{3 Parameters and options!Elements}!#1}}
\newcommand{\FDD}[1]{{\color{red}  #1}\index{\textbf{5 Extrémities}!#1}}
\newcommand{\RDD}[1]{{\color{red}  #1}\index{\textbf{3 Parameters and options}!#1}}
\newcommand{\RDDD}[2]{{\color{red}#1} \index{\textbf{Parameters and options}!#2!#1}}
\newcommand{\RDDX}[2]{{\color{red}#1} \index{\textbf{4 Options}!#1 (#2)}}
}

\newcommand{\rouge}[1] {{\color{red}  #1}}
\newcommand{\blll}[1] {{\color{blue}  #1}}

%
%
% \begin{document}
%%  \newrgbcolor{Vert}{0 .8 0}

\input{tkztitre}

%==========================================================


\setcounter{tocdepth}{4}
 \tableofcontents
 \setcounter{tocdepth}{5}
\addtolength{\hoffset}{-1.5cm} 
\setlength{\parindent}{0pt}
\setlength{\topmargin}{0pt}
\setlength{\headsep}{0pt}

 \newpage

%\section{Les figures de base}
\SSCT{Les figures de base}{Basic figures}
\input{tkz1}

\newpage

\input{tkz2}

\newpage
%
\input{tkz3}
%


\input{tkz3a}

\newpage

%\section{Insertion de petites images}
\SSCT{Insertion de petites images}{Small pictures}
\input{tkzpic}

\newpage

\input{tkzangles}

%%%%% % % %===================================

\newpage

%\section{Les coordonnées }
\SSCT{Les coordonnées }{Coordinates}
 
%\subsection{Quadrillage}
\SbSSCT{Quadrillage}{Grid}


\begin{tabular}{|c|}\hline 
\tikz \draw(0,0) grid (2,2); 
\\ \hline 
\BS{draw} (0,0) \RDD{grid} (2,2); \RRR{14-8}
\\ \hline 
\end{tabular} 


\bigskip
\begin{tabular}{|c|c|c|c|} \hline 
\multicolumn{4}{|c|}{ \BS{draw} (0,0) grid  [\RDD{step}=.75cm] (0,0) grid (3,3);   }\\ 
\hline  
\begin{tikzpicture}
\draw[dotted](0,0) grid (3,3); 
%\draw[thick] (0,0) grid [step=1] (3,2);
\draw[red] (0,0) grid [step=.75cm] (3,3);
\end{tikzpicture}
&  
\begin{tikzpicture}
\draw[dotted](0,0) grid (3,3); 
%\draw[thick] (0,0) grid [step=1] (3,2);
\draw[red] (0,0) grid [xstep=.75cm] (3,3);
\end{tikzpicture}
&  
\begin{tikzpicture}
\draw[dotted](0,0) grid (3,3); 
%\draw[thick] (0,0) grid [step=1] (3,2);
\draw[red] (0,0) grid [ystep=.75cm] (3,3);
\end{tikzpicture}
&
\begin{tikzpicture}
\draw[dotted](0,0) grid (3,3); 
%\draw[thick] (0,0) grid [step=1] (3,2);
\draw[red] (0,0) grid [step=(45:1)] (3,3);
\end{tikzpicture}
\\ \hline 
step=.75cm & x step=.75cm & ystep=.75cm  & step=(45:1)
\\ \hline 
\end{tabular} 

\bigskip

\begin{tabular}{|c|c|} \hline 
 
\BS{draw}[red] (0,0) grid [\RDD{rotate}=45] (3,3);
&  
\BS{draw}[\RDD{help lines}] (0,0) grid  (3,3);
\\ \hline  
\begin{tikzpicture}
\draw[dotted](0,0) grid (3,3); 
%\draw[thick] (0,0) grid [step=1] (3,2);
\draw[red] (0,0) grid [rotate=45] (3,3);
\end{tikzpicture}
& 
\tikz \draw[help lines] (0,0) grid (3,3); \\ 
\hline 
\end{tabular} 



%\begin{tabular}{|c|c|c|c|c|} \hline 
%\multicolumn{5}{|c|}{ \BS{tikz} \BS{draw} [\RDD{step}=1mm] (0,0) grid (2,2);   }\\ 
%\hline  
%\tikz \draw[step=1mm] (0,0) grid (2,2);
%&  
%\tikz[rotate=30] \draw (0,0) grid (2,2);
%&  
%\tikz \draw (0,0) grid [xstep=.5] (2,2);
%&  
%\tikz \draw (0,0) grid [ystep=.5] (2,2);
%&
%\tikz \draw[help lines] (0,0) grid (2,2);
%\\ \hline  
%[\RDD{step}=1mm] & [\RDD{rotate}=30] & [\RDD{xstep}=.5] & [\RDD{ystep}=.5] & [\RDD{help lines}] \\ 
%\hline 
%\end{tabular} 

\newpage 

\input{tkzcoord}
%
%%%%%==========================================================
 
\newpage
%\section[Les n\oe uds]{Les n\oe uds }
\SSCT{Les n\oe uds }{Nodes}


%\subsection{Définition des  n\oe uds}
\SbSSCT{Définition des  n\oe uds}{Creation of nodes}
\tikzset{blue}

\begin{tabular}{|c | c | c | c |} \hline
\multicolumn{4}{|c|}{  \BS{draw} (1,1) node[\RDD{fill}=red!20] \AC{};   }\\ 
\hline 
\tikz \draw (0,0) grid (2,2) (1,1) node[fill=red!20] {};
&
\tikz \draw (0,0) grid (2,2) (1,1) node[fill=red!20,draw] {}; 
&
\tikz \draw (0,0) grid (2,2) (1,1) node[circle,fill=red!20] {};
&
\tikz \draw (0,0) grid (2,2) (1,1) node[circle,fill=red!20,draw] {};
\\  \hline
\dft
&
node[\RDD{draw}] 
&
 node[\RDD{circle}]  
&
 node[\RDD{circle},\RDD{draw}]
 \\  \hline
\end{tabular}
\bigskip

\begin{tabular}{|c | c | c | c |} \hline
\multicolumn{4}{|c|}{ \BSS{node} \RDD{at} (1,1) [fill=red!20] \AC{};   }\\ 
\hline 
 \begin{tikzpicture}
\draw (0,0) grid (2,2) ; 
\node at (1,1) [fill=red!20] {};
 \end{tikzpicture}
&
 \begin{tikzpicture}
\draw (0,0) grid (2,2) ; 
\node at (1,1) [draw] {};
 \end{tikzpicture}
&
 \begin{tikzpicture}
\draw (0,0) grid (2,2) ; 
\node at (1,1) [fill=red!20,circle] {};
 \end{tikzpicture}
&
 \begin{tikzpicture}
\draw (0,0) grid (2,2) ; 
\node at (1,1) [circle,draw] {};
 \end{tikzpicture}
\\  \hline
[fill=red!20]
&
[\RDD{draw}] 
&
[\RDD{circle},fill=red!20]
 &
[\RDD{circle},draw] 
 \\  \hline
\end{tabular}
\bigskip

\TFRGB{Autres types de n\oe uds voir page}{Other type of nodes see page} \pageref{noeudboite}



%-------------------------------------------------------------------------------
%\subsection{Liaisons}
\SbSSCT{Liaisons}{Links}
\label{liaisons}

\begin{tabular}{|c|c|c|} \hline  
\begin{tikzpicture}[blue]
\node[draw] (A) at (0,0) {A};
\node[draw] (B) at (1.5,1.5) {B};
\draw (A) -- (B);
\end{tikzpicture}
&  
\begin{tikzpicture}[blue]
\node[draw] (A) at (0,0) {A};
\node[draw] (B) at (1.5,1.5) {B};
\draw (A) |- (B);
\end{tikzpicture}
&  
\begin{tikzpicture}[blue]
\node[draw] (A) at (0,0) {A};
\node[draw] (B) at (1.5,1.5) {B};
\draw (A) -| (B);
\end{tikzpicture}
\\ \hline  
(A){\color{red} - -} (B) & (A) {\color{red}|-} (B) &  (A) {\color{red}-|} (B)
\\ \hline 
\begin{tikzpicture}[blue]
\node[draw] (A) at (0,0) {A};
\node[draw] (B) at (1.5,1.5) {B};
\draw (A) to [bend right] (B);
\end{tikzpicture}
&  
\begin{tikzpicture}[blue]
\node[draw] (A) at (0,0) {A};
\node[draw] (B) at (1.5,1.5) {B};
\draw (A) to [bend left] (B);
\end{tikzpicture}
&  
\begin{tikzpicture}[blue]
\node[draw] (A) at (0,0) {A};
\node[draw] (B) at (1.5,1.5) {B};
\draw (A) to[bend left=0] (B);
\end{tikzpicture}
\\ \hline  
(A) to [\RDD{bend right}] (B) & (A) to [\RDD{bend left}] (B) &  (A) to[\RDD{bend left}=0] (B)
\\ \hline 
\begin{tikzpicture}[blue]
\node[draw] (A) at (0,0) {A};
\node[draw] (B) at (1.5,1.5) {B};
\draw (A) to[bend left=120]  (B);
\end{tikzpicture}
&  
\begin{tikzpicture}[blue]
\node[draw] (A) at (0,0) {A};
\node[draw] (B) at (1.5,1.5) {B};
\draw (A) to[bend left=45] (B);
\end{tikzpicture}
&  
\begin{tikzpicture}[blue]
\node[draw] (A) at (0,0) {A};
\node[draw] (B) at (1.5,1.5) {B};
\draw (A) to[bend left=90] (B);
\end{tikzpicture}
\\ \hline  
(A)  to[\RDD{bend left}=120]  (B) & (A) to[\RDD{bend left}=45] (B) &  (A) to[\RDD{bend left}=90] (B)
\\ \hline 
\begin{tikzpicture}[blue]
\node[draw] (A) at (0,0) {A};
\node[draw] (B) at (1.5,1.5) {B};
\draw (A)  to[out=90]  (B);
\end{tikzpicture}
&  
\begin{tikzpicture}[blue]
\node[draw] (A) at (0,0) {A};
\node[draw] (B) at (1.5,1.5) {B};
\draw (A) to[out=30] (B);
\end{tikzpicture}
&  
\begin{tikzpicture}[blue]
\node[draw] (A) at (0,0) {A};
\node[draw] (B) at (1.5,1.5) {B};
\draw (A)  to[in=-90]  (B);
\end{tikzpicture}
\\ \hline  
(A)  to[\RDD{out}=90] (B) & (A) to[\RDD{out}=30]  (B) &  (A)  to[\RDD{in}=-90]  (B)
\\ \hline 
%\begin{tikzpicture}[blue]
%\node[draw] (A) at (0,0) {A};
%\node[draw] (B) at (2,2) {B};
%\draw (A)  to[in=90]  (B);
%\end{tikzpicture}
%&  
%\begin{tikzpicture}[blue]
%\node[draw] (A) at (0,0) {A};
%\node[draw] (B) at (2,2) {B};
%\draw (B) to[in=0,out=90]  (B);
%\end{tikzpicture}
%&  
%\begin{tikzpicture}[blue]
%\node[draw] (A) at (0,0) {A};
%\node[draw] (B) at (2,2) {B};
%%\draw (A)  to[out=45,in=-90]  (A);
%\draw (B) to[out=45,in=135] (B);
%\end{tikzpicture}
%\\ \hline  
%(A)  to[\RDD{in}=90] (B) & (B) to[bend left]  (B) & (B) to[out=45,in=135] (B)
%\\ \hline 
\end{tabular} 

\bigskip
\begin{tabular}{|c|c|c|} \hline  
\multicolumn{2}{|c|}{ \BS{draw} (A) .. controls +(right:2cm) and +(down:2cm) .. (B);  }\\ 
\hline  
\begin{tikzpicture}[blue]
\node[draw] (A) at (0,0) {A};
\node[draw] (B) at (2,2) {B};
\draw  (A) .. controls +(right:2cm) and +(down:2cm) .. (B);
\end{tikzpicture}
&
\begin{tikzpicture}[blue]
\node[draw] (A) at (0,0) {A};
\node[draw] (B) at (2,2) {B};
\draw  (A) .. controls +(up:1cm) and +(left:1cm) .. (B);
\end{tikzpicture}
\\ \hline 
controls +(right:2cm) and +(down:2cm)  &
controls +(up:1cm) and +(left:1cm)
\\ \hline 
\begin{tikzpicture}[blue]
\node[draw] (A) at (0,0) {A};
\node[draw] (B) at (2,2) {B};
\draw  (A) .. controls +(right:1cm) and +(right:2cm) .. (B);
\end{tikzpicture}
&
\begin{tikzpicture}[blue]
\node[draw] (A) at (0,0) {A};
\node[draw] (B) at (2,2) {B};
\draw  (A) .. controls +(up:1cm) and +(right:2cm) .. (B);
\end{tikzpicture}
\\ \hline 
controls +(right:1cm) and +(right:2cm)  &
controls +(up:1cm) and +(right:2cm) 
\\ \hline 
\begin{tikzpicture}[blue]
\node[draw] (A) at (0,0) {A};
\node[draw] (B) at (2,2) {B};
\draw  (A) .. controls +(120:2cm) and +(200:1cm) .. (B);
\end{tikzpicture}
 &
 \begin{tikzpicture}[blue]
 \node[draw] (A) at (0,0) {A};
 \node[draw] (B) at (2,2) {B};T
 \draw  (A) .. controls +(120:2cm) and +(200:1cm) .. (A);
 \end{tikzpicture}
\\  \hline  
controls +(120:2cm) and +(200:1cm) & controls +(120:2cm) and +(200:1cm) 
\\ \hline 
\begin{tikzpicture}[blue]
\node[draw] (A) at (0,0) {A};
\node[draw] (B) at (2,2) {B};
\node[draw] (C) at (0,1) {C};
\node[draw] (D) at (3,0) {D};
\draw  (A) .. controls +(C) and +(D) .. (B);
\end{tikzpicture}
&
\begin{tikzpicture}[blue]
\node[draw] (A) at (0,0) {A};
\node[draw] (B) at (2,2) {B};
\node[draw] (C) at (0,1) {C};
\node[draw] (D) at (3,0) {D};
\draw (A) .. controls +(D)  .. (B);
\end{tikzpicture}
\\ \hline 
controls +(C) and +(D) &
controls +(D) 
\\ \hline 
\end{tabular} 
 \bigskip
 
\begin{tabular}{|c|c|c|} \hline 
\multicolumn{3}{|l|}{ \BS{node}[draw] (A) at (0,0) \AC{A}  }\\

\multicolumn{3}{|l|}{ \BS{node}[draw] (B) at (2,2) \AC{B} \RDD{edge}  [->] (A);  }\\
\multicolumn{3}{|c|}{\RRR{17-12-1}}  \\
\hline 
 \begin{tikzpicture}
 \node[draw] (A) at (0,0) {A};
 \node[draw] (B) at (2,2) {B} edge [->] (A);
 \end{tikzpicture}
 &
 \begin{tikzpicture}
 \node[draw] (A) at (0,0) {A};
 \node[draw] (B) at (2,2) {B} edge [red]  (A);
 \end{tikzpicture}
 &
 \begin{tikzpicture}
 \node[draw] (A) at (0,0) {A};
 \node[draw] (B) at (2,2) {B} edge [dashed] (A);
 \end{tikzpicture}
\\ \hline 
[->] & [red]  & [dashed]
\\ \hline 
\end{tabular}

%---------------------------------------------------------------------------------
%\subsection{\'Etiquettes sur les n\oe uds}
\SbSSCT{\'Etiquettes sur les n\oe uds}{Node labels}

\begin{tabular}{|c|c|c|c|} \hline
\multicolumn{4}{|c|}{  \BS{fill}(0,0) circle (2pt) node[\RDD{above}] \AC{texte} ;   }\\ 
\hline 
  
\begin{tikzpicture} \draw[help lines] (-1,-1) grid (1,1) ;\fill (0,0) circle (2pt) node[above] {texte};\end{tikzpicture}
& 
\begin{tikzpicture} \draw[help lines] (-1,-1) grid (1,1) ;\fill (0,0) circle (2pt) node[below] {texte};\end{tikzpicture}
 &  
\begin{tikzpicture} \draw[help lines] (-1,-1) grid (1,1);\fill (0,0) circle (2pt) node[left] {texte};\end{tikzpicture}
 &  
\begin{tikzpicture} \draw[help lines] (-1,-1) grid (1,1); \fill (0,0) circle (2pt) node[right] {texte};\end{tikzpicture}
 \\  \hline 
 [\RDD{above}] & [\RDD{below}] & [\RDD{left}] &  [\RDD{right}]
 \\ \hline 
 \begin{tikzpicture} \draw[help lines] (-1,-1) grid (1,1) ;\fill (0,0) circle (2pt) node[above left] {texte};\end{tikzpicture}
 & 
 \begin{tikzpicture} \draw[help lines] (-1,-1) grid (1,1) ;\fill (0,0) circle (2pt) node[below left] {texte};\end{tikzpicture}
  &  
 \begin{tikzpicture} \draw[help lines] (-1,-1) grid (1,1);\fill (0,0) circle (2pt) node[above right] {texte};\end{tikzpicture}
  &  
 \begin{tikzpicture} \draw[help lines] (-1,-1) grid (1,1); \fill (0,0) circle (2pt) node[below right] {texte};\end{tikzpicture}
  \\  \hline 
  [\RDD{above left}] & [\RDD{below left}] & [\RDD{above right}] &  [\RDD{below right}]
  \\ \hline 
 \begin{tikzpicture} \draw[help lines] (-1,-1) grid (1,1) ;\fill (0,0) circle (2pt) node[anchor=south] {texte};\end{tikzpicture}
 & 
 \begin{tikzpicture} \draw[help lines] (-1,-1) grid (1,1) ;\fill (0,0) circle (2pt) node[anchor=west] {texte};\end{tikzpicture}
  &  
 \begin{tikzpicture} \draw[help lines] (-1,-1) grid (1,1);\fill (0,0) circle (2pt) node[anchor=north] {texte};\end{tikzpicture}
  &  
 \begin{tikzpicture} \draw[help lines] (-1,-1) grid (1,1); \fill (0,0) circle (2pt) node[anchor=east] {texte};\end{tikzpicture}
  \\  \hline 
  [\RDD{anchor=south}] & [\RDD{anchor=west}] & [\RDD{anchor=north}] & [\RDD{anchor=east                                                                                                                                                               }]
  \\ \hline 
 \begin{tikzpicture} \draw[help lines] (-1,-1) grid (1,1) ;\fill (0,0) circle (2pt) node[anchor=south east] {texte};\end{tikzpicture}
 & 
\begin{tikzpicture} \draw[help lines] (-1,-1) grid (1,1) ;\fill (0,0) circle (2pt) node[anchor=south west] {texte};\end{tikzpicture}
&  
\begin{tikzpicture} \draw[help lines] (-1,-1) grid (1,1);\fill (0,0) circle (2pt) node[anchor=north west] {texte};\end{tikzpicture}
&  
\begin{tikzpicture} \draw[help lines] (-1,-1) grid (1,1); \fill (0,0) circle (2pt) node[anchor=east] {texte};\end{tikzpicture}
\\  \hline 
[\RDD{anchor=south east}] & [\RDD{anchor=south west}] & [\RDD{anchor=north west}] & [\RDD{anchor=north east                                                                                                                                                              }]
  \\ \hline 
\end{tabular} 


\bigskip
\begin{tabular}{|c|c|c|c|} \hline
\multicolumn{4}{|c|}{  \BS{fill}(0,0) circle (2pt) node[\RDD{above}=.3cm] \AC{texte} ;   }\\ 
\hline 
  
\begin{tikzpicture} \draw[help lines] (-1,-1) grid (1,1) ;\fill (0,0) circle (2pt) node[above=.3cm] {texte};\end{tikzpicture}
& 
\begin{tikzpicture} \draw[help lines] (-1,-1) grid (1,1) ;\fill (0,0) circle (2pt) node[below=.3cm] {texte};\end{tikzpicture}
 &  
\begin{tikzpicture} \draw[help lines] (-1,-1) grid (1,1);\fill (0,0) circle (2pt) node[left=.3cm] {texte};\end{tikzpicture}
 &  
\begin{tikzpicture} \draw[help lines] (-1,-1) grid (1,1); \fill (0,0) circle (2pt) node[right=.3cm] {texte};\end{tikzpicture}
 \\  \hline 
 [\RDD{above}=.3cm] & [\RDD{below}=.3cm] & [\RDD{left}=.3cm] &  [\RDD{right}=.3cm]]
 \\ \hline 
\begin{tikzpicture} \draw[help lines] (-1,-1) grid (1,1) ;\fill (0,0) circle (2pt) node[above left=.3cm] {texte};\end{tikzpicture}
& 
\begin{tikzpicture} \draw[help lines] (-1,-1) grid (1,1) ;\fill (0,0) circle (2pt) node[below left=.3cm] {texte};\end{tikzpicture}
 &  
\begin{tikzpicture} \draw[help lines] (-1,-1) grid (1,1);\fill (0,0) circle (2pt) node[above right=.3cm] {texte};\end{tikzpicture}
 &  
\begin{tikzpicture} \draw[help lines] (-1,-1) grid (1,1); \fill (0,0) circle (2pt) node[below right=.3cm] {texte};\end{tikzpicture}
 \\  \hline 
 [\RDD{above left}=.3cm] & [\RDD{below left}=.3cm] & [\RDD{above right}=.3cm] &  [\RDD{below right}=.3cm]]
 \\ \hline 
 
 \end{tabular} 
 
%\begin{tikzpicture} \draw[help lines] (-1,-1) grid (1,1);\fill (0,0) circle (2pt) node[distance=.3cm] {texte};\end{tikzpicture} 
 
 \newpage
\selectlanguage{french}
 
 \begin{tabular}{|c|c|c|c|c|} \hline
 \multicolumn{5}{|l|}{ \BSS{shorthandoff}\AC{:} \footnotemark[1]  } \\
 \multicolumn{5}{|l|}{  \BS{node} [draw,\RDD{label}=right:texte] \AC{}   }\\
 \multicolumn{5}{|l|}{ \BSS{shorthandon}\AC{:} } \\ 
 \hline 
     \shorthandoff{:} 
 \tikz \node [draw,label=right:texte] {};
 \shorthandon{:}
 &
  \shorthandoff{:}
 \tikz \node [draw,label=left:texte] {};
 \shorthandon{:}
 &
  \shorthandoff{:}
 \tikz \node [draw,label=above:texte] {};
 \shorthandon{:}
 &
  \shorthandoff{:}
 \tikz \node [draw,label=below:texte] {};
 \shorthandon{:}
 &
  \shorthandoff{:}
 \tikz \node [draw,label=45:texte] {};
    \shorthandon{:}
   \\ \hline
  label=right & label=left &  label=above & label=below & label=45
    \\ \hline 
 \end{tabular}
 \footnotetext[1]{\TFRGB{désactivation et ré-activation de \og : \fg  conflit entre les modules Tikz et Babel en français}{Only useful when the package babel is loaded with the frenchb option    }}
 
 \bigskip
  \begin{tabular}{|c|c|c|c|c|} \hline
  \BS{fill}(0,0) circle (2pt) node[below right=.3cm,draw,label=45:étiquette] \AC{texte} ;
      \\ \hline 
  
  \shorthandoff{:}
\begin{tikzpicture} \draw[help lines] (-1,-1) grid (2,1); \fill (0,0) circle (2pt) node[below right=.3cm,draw,label=45:étiquette] {texte};\end{tikzpicture}
 \shorthandon{:}
 
    \\ \hline 
 \end{tabular}
\bigskip

 \shorthandoff{:}
 

 
\begin{tabular}{|c|c|c|} \hline
\multicolumn{3}{|c|}{  \BSS{shorthandoff}\AC{:} \BS{node}[circle,draw,blue,\RDD{pin}=texte] \AC{} ;   \BSS{shorthandon}\AC{:}  \footnotemark[1] }\\ 
\hline
\begin{tikzpicture} 
\node [circle,draw,blue,pin=texte] {};
\end{tikzpicture}
&
\begin{tikzpicture} 
\node [circle,draw,blue,pin=60:texte] {};
\end{tikzpicture}
&
\begin{tikzpicture} 
\node [circle,draw,blue,pin=right:texte] {};
\end{tikzpicture}
 \\ \hline
[circle,pin=texte] &   [circle,pin=60:texte] & [circle,pin=right:texte]
 \\ \hline 
\end{tabular}  

\bigskip
\begin{tabular}{|c|c|c|} \hline
\multicolumn{3}{|c|}{  \BS{tikz}[\RDD{pin position}=60] \BS{node} [circle,pin=texte] \AC{} ;   }\\ 
\hline 
\tikz[pin position=60] \node [circle,draw,blue,pin=texte] {};
&
\tikz[pin distance=0 cm] \node [circle,draw,blue,pin=60:texte] {};
&
\tikz[pin distance=2 cm] \node [circle,draw,blue,pin=60:texte,pin distance=0cm] {};
  \\ \hline
  [\RDD{pin position}=60] & [\RDD{pin distance}=0 cm] & [\RDD{pin distance}=2 cm]
    \\ \hline
  \dft{ : above} & \multicolumn{2}{|c|}{ \dft{ : 3 ex}}
      \\ \hline
\end{tabular}  

% % % % % % % % % % >>>>>>>>>> a voir : option edge <<<<<<<<<<<<<<<<<<<<<<<<<<<<<<<<<<<<<<

   \shorthandon{:} 
   
\selectlanguage{english}   
% >>>>>>>>>>>>>>>>>>>>>> A Voir : positioning librairy <<<<<<<<<<<<<<<<<<<<<<<<<<<<<<<<<<<<<<<

%\subsection{ N\oe uds  sur un chemin}
\SbSSCT{ N\oe uds  sur un chemin}{Nodes on a path}

\begin{tabular}{|c|c|c|} \hline
\multicolumn{3}{|c|}{  \BS{draw}(0,0) .. controls (1,2) and (2,-1) .. (4,0) node[\RDD{at end}] \AC{texte} ;   }\\ 
\hline 
\tikz \draw (0,0) .. controls (1,2) and (2,-1) .. (4,0) node[pos=0] {texte}; 
&
\tikz \draw (0,0) .. controls (1,2) and (2,-1) .. (4,0) node[pos=.33] {texte}; 
&
\tikz \draw (0,0) .. controls (1,2) and (2,-1) .. (4,0) node[at end] {texte}; 
  \\ \hline 
\RDD{pos}{\color{red}  =0} & \RDD{pos}{\color{red}  =.33} & \RDD{at end} (pos=1)
  \\ \hline 

\tikz \draw (0,0) .. controls (1,2) and (2,-1) .. (4,0) node[very near end] {texte}; 
&
\tikz \draw (0,0) .. controls (1,2) and (2,-1) .. (4,0) node[near end] {texte}; 
&
\tikz \draw (0,0) .. controls (1,2) and (2,-1) .. (4,0) node[midway] {texte}; 
  \\ \hline 
\RDD{very near end} (pos=0.875.) & \RDD{ near end} (pos=0.75) & \RDD{midway} (pos=0.5)
  \\ \hline 
  
\tikz \draw (0,0) .. controls (1,2) and (2,-1) .. (4,0) node[near start] {texte}; 
&
\tikz \draw (0,0) .. controls (1,2) and (2,-1) .. (4,0) node[very near start] {texte}; 
&
\tikz \draw (0,0) .. controls (1,2) and (2,-1) .. (4,0) node[at start] {texte};
\\ \hline 
\RDD{near start} (pos=0.25) & \RDD{very near start} (pos=0.125) & \RDD{at start} (pos=0)
  \\ \hline 
  
\end{tabular} 

\bigskip
\begin{tabular}{|c|c|c|} \hline
\multicolumn{3}{|c|}{  \BS{draw}(0,0) .. controls (1,2) and (2,1) .. (4,0) node[\RDD{sloped},midway] \AC{texte} ;   }\\ 
\hline 
\tikz \draw (0,0) .. controls (1,2) and (2,-1) .. (4,0) node[sloped,midway] {texte};
&
\tikz \draw (0,0) .. controls (1,2) and (2,-1) .. (4,0) node[above,midway] {texte};
&
\tikz \draw (0,0) .. controls (1,2) and (2,-1) .. (4,0) node[below,midway] {texte};
  \\ \hline
\RDD{sloped} & \RDD{above} &\RDD{below}
  \\ \hline
\end{tabular}
\bigskip

\begin{tabular}{|c|c|c|} \hline
\multicolumn{3}{|c|}{  \BS{draw}(0,0) .. controls (1,2) and (2,1) .. (5,0) node[\RDD{sloped},midway,allow upside down] \AC{texte} ;   }\\ 
\hline 
\tikz \draw (0,0) .. controls (1,2) and (2,-1) .. (4,0) node[sloped,midway,allow upside down] {texte};
&
\tikz \draw (0,0) .. controls (1,2) and (2,-1) .. (4,0) node[above,midway,allow upside down] {texte};
&
\tikz \draw (0,0) .. controls (1,2) and (2,-1) .. (4,0) node[below,midway,allow upside down] {texte};
  \\ \hline
\RDD{sloped} & \RDD{above} &\RDD{below}
  \\ \hline
\end{tabular}  


\begin{tabular}{|c|c|c|} \hline
\multicolumn{3}{|c|}{  \BS{draw}(A)  to [bend right]  node [\RDD{bend right}] \AC{texte} (B);   }\\ 
\hline 
\begin{tikzpicture} %[auto,bend right]
\node[draw] (A) at (0,0) {A};
\node[draw] (B) at (2,2) {B};
\draw (A) to [bend right] node [bend right] {texte} (B);
\end{tikzpicture}
&
\begin{tikzpicture} 
\node[draw] (A) at (0,0) {A};
\node[draw] (B) at (2,2) {B};
\draw (A) to [bend right] node [auto,bend right] {texte} (B);
\end{tikzpicture}
&
\begin{tikzpicture} 
\node[draw] (A) at (0,0) {A};
\node[draw] (B) at (2,2) {B};
\draw (A) to[bend right] node [auto,swap,bend right] {texte} (B);
\end{tikzpicture}
  \\ \hline
[bend right]  & [\RDD{auto},bend right] & [auto,\RDD{swap},bend right] 
  \\ \hline
\end{tabular}  

\SbSSCT{ N\oe uds  sur un \og edge\fg}{Nodes on an edge}

\begin{tabular}{|c|c|c|}\hline  
\multicolumn{3}{|c|}{  \BS{draw}(0,0) edge \BDD{["abc", ->]} (4,0);  }\\ 
\multicolumn{3}{|c|}{  \RRR{17-12-2} }\\ 
\hline 
\begin{tikzpicture}[blue] 
\useasboundingbox  (0,-.5) rectangle (4,.5); 
\draw (0,0) edge ["abc", ->] (4,0);
\end{tikzpicture}
&
\begin{tikzpicture}[blue] 
\useasboundingbox  (0,-.5) rectangle (4,.5); 
\draw (0,0) edge ["abc", near start] (4,0);
\end{tikzpicture}
&
\begin{tikzpicture}[blue] 
\useasboundingbox  (0,-.5) rectangle (4,.5); 
\draw (0,0) edge ["abc", style={auto=right}] (4,0);
\end{tikzpicture}
\\ \hline 
["abc", ->]
& 
["abc", near start] &  ["abc", style=\AC{auto=right}] 
\\ \hline  
\begin{tikzpicture}[blue] 
\useasboundingbox  (0,-.5) rectangle (4,.5); 
\draw (0,0) edge [font=\Large,"abc" ] (4,0);
\end{tikzpicture}
&
\begin{tikzpicture}[blue] 
\useasboundingbox  (0,-.5) rectangle (4,.5); 
\draw (0,0) edge ["abc" color=red ] (4,0);
\end{tikzpicture}
&
\begin{tikzpicture}[blue] 
\useasboundingbox  (0,-.5) rectangle (4,.5); 
 \draw (0,0) edge ["abc" '] (4,0);
\end{tikzpicture}
\\ \hline 
[font=\BS{Large},"abc" ] & ["abc" color=red ]
&["abc" ' ]
\\ \hline 

\begin{tikzpicture}[blue] 
\useasboundingbox  (0,-.5) rectangle (4,.75); 
\draw (0,0) edge ["abc" draw ] (4,0);
\end{tikzpicture}
&
\begin{tikzpicture}[blue] 
\useasboundingbox  (0,-.5) rectangle (4,.5); 
\draw (0,0) edge ["abc" inner sep=0pt ] (4,0);
\end{tikzpicture}
&
\begin{tikzpicture}[blue] 
\useasboundingbox  (0,-.5) rectangle (4,.5); 
\draw (0,0) edge ["abc" fill ,fill=yellow ] (4,0);
\end{tikzpicture}
\\ \hline
["abc" draw ]
&
["abc" inner sep=0pt ]
&
["abc" fill ,fill=yellow ]
\\ \hline
\end{tabular} 



\bigskip

\begin{tabular}{|c|} \hline  
\BS{draw}[every edge quotes/.style=\AC{fill=yellow}] (0,0) edge ["abc"] (4,0);
\\ \hline  
\begin{tikzpicture}[blue] 
\useasboundingbox  (0,-.5) rectangle (4,.5); 
 \draw[every edge quotes/.style={fill=yellow}] (0,0) edge ["abc"] (4,0);
\end{tikzpicture}
\\ \hline 
\end{tabular} 






\input{tkzfit}

%
\newpage

%%%======================================================
%\section[Constructions particulières]{Constructions particulières  }
\SSCT{Constructions particulières  }{Transformations}
%

%\subsubsection{Transformations}

\begin{center}
\RRR{25-3}
\end{center}


\begin{tabular}{|c|c|c|c|} \hline 
\multicolumn{4}{|c|}{  \BS{draw}[\RDD{rotate},blue] (0,0)  rectangle  (2,2) ;   }\\ 
\hline  
\begin{tikzpicture}
\draw[dashed,red] (0,0) rectangle  (2,2) ; 
\draw[rotate=40,blue] (0,0) rectangle  (2,2) ;
\end{tikzpicture}
&  
\begin{tikzpicture}
\draw[dashed,red] (0,0) rectangle  (2,2) ; 
\draw[x=1cm,y=.5cm,blue] (0,0) rectangle  (2,2); 
\end{tikzpicture}
&  
\begin{tikzpicture}
\draw[dashed,red] (0,0) rectangle  (2,2) ; 
\draw[xslant=.75,blue] (0,0) rectangle  (2,2);  
\end{tikzpicture}
&
\begin{tikzpicture}
\draw[dashed,red] (0,0) rectangle  (2,2) ; 
\draw[yslant=.75,blue] (0,0) rectangle  (2,2);  
\end{tikzpicture}
\\ \hline  
\RDD{rotate}=40 & \RDD{x}=1cm,\RDD{y}=0.5cm & \RDD{xslant}=0.75 & \RDD{yslant}=0.75\\ 
\hline 
  
\begin{tikzpicture}
\draw[dashed,red] (0,0) rectangle  (2,2) ; 
\draw[scale=1.5,blue] (0,0) rectangle  (2,2) ; 
\end{tikzpicture}
&  
\begin{tikzpicture}
\draw[dashed,red] (0,0) rectangle  (2,2) ; 
\draw[scale=-1,y=.5cm,blue] (0,0) rectangle  (2,2); 
\end{tikzpicture}
&  
\begin{tikzpicture}
\draw[dashed,red] (0,0) rectangle  (2,2) ; 
\draw[xshift=.5cm,blue] (0,0) rectangle  (2,2); 
\end{tikzpicture}
&
\begin{tikzpicture}
\draw[dashed,red] (0,0) rectangle  (2,2) ; 
\draw[yshift=.5cm,blue] (0,0) rectangle  (2,2);  
\end{tikzpicture}
\\ \hline  
\RDD{scale}=1.5 & \RDD{scale}=-1 & \RDD{xshift}=0.5cm & \RDD{yshift}=0.5cm
\\ \hline 
\end{tabular} 

\bigskip

%==============================================
%\begin{tikzpicture}
%\draw[help lines] (0,0) grid (3,2);
%\draw (0,0) - - (1,1) - - (1,0);
%\draw[rotate=40,blue] (0,0) - - (1,1) - - (1,0); % rotation 40°
%\draw[rotate=-20,red] (0,0) - - (1,1) - - (1,0); % rotation -20°
%\end{tikzpicture}

%\tikz \draw[x=1cm,y=.5cm] (0,0) rectangle(2,2);

%\tikz \draw (0,0) rectangle (1,0.5) [xshift=2cm] (0,0) rectangle (1,0.5);

%\begin{tikzpicture}
%\draw[help lines] (0,0) grid (3,2);
%\draw (0,0) - - (1,1) - - (1,0);
%\draw[scale=2,blue] (0,0) - - (1,1) - - (1,0); % échelle 2
%\draw[scale=-1,red] (0,0) - - (1,1) - - (1,0); % échelle -1
%\end{tikzpicture}

%\begin{tikzpicture}
%\draw[help lines] (0,0) grid (3,2);
%\draw (0,0) - - (1,1) - - (1,0);
%\draw[xslant=2,blue] (0,0) - - (1,1) - - (1,0);
%\draw[xslant=-1,red] (0,0) - - (1,1) - - (1,0);
%\end{tikzpicture}

%\begin{tikzpicture}
%\draw[help lines] (0,0) grid (3,2);
%\draw (0,0) rectangle (1,0.5);
%\beginscope[xshift=2cm] % Décalage en X de 2cm
%\draw [red] (0,0) rectangle (1,0.5);
%\draw[yshift=1cm,blue] (0,0) rectangle (1,0.5);
%\draw[rotate=30,orange] (0,0) rectangle (1,0.5);
%\endscope
%\end{tikzpicture}
 

\newpage

%\section{Placer son dessin}
\SSCT{Placer son dessin}{Placing the picture}
%
\input{tkzfig}

\newpage

\section{Scope}
%
\input{tkzscope} 

\newpage

 
%\section{Position absolue sur une page}
\SSCT{Position absolue sur une page}{Absolute position on a page}

 \input{tkzpage}
 
\newpage 

%\section{Arrière plan du dessin}
\SSCT{Arrière plan du dessin}{Background}

 \input{tkzbackground}

\newpage 

%%\section{Placer des objets}
%%
%%%\input{plac}
%%
%
%%\newpage
%%
%%%===================================================
%\section{Créer ses couleurs}
\SSCT{Créer ses couleurs}{Defining your own colors}

 \input{tkzcoul}
 

\newpage

%\sectionCreate {Créer ses commandes}
\SSCT{Créer ses commandes}{Create command}

\input{tkzcde}


\newpage

%\section[Créer ses styles]{Créer ses styles}
\SSCT{Créer ses styles}{Creating styles}

\input{tkzstyl}

%%%%%%%======================================================================

\newpage

%\section{Mettre du texte  en valeur}
\SSCT{Mettre du texte  en valeur}{Text highlighting}

\label{ndbt}

\tikzset{blue}

%\subsection{Dans un n\oe ud de Tikz}
\SbSSCT{Dans un n\oe ud de Tikz}{In a TikZ node}
\label{noeudboite}

\begin{tabular}{|c | c | c | c |} \hline
\multicolumn{4}{|c|}{ \BS{tikz} \BS{draw} (0,0) grid (2,2) (1,1) node[fill=red!20,] \AC{texte};   }\\ 
\hline 
\tikz \draw (0,0) grid (2,2) (1,1) node[fill=red!20] {texte};
&
\tikz \draw (0,0) grid (2,2) (1,1) node[fill=red!20,draw] {texte}; 
&
\tikz \draw (0,0) grid (2,2) (1,1) node[circle,fill=red!20] {texte};
&
\tikz \draw (0,0) grid (2,2) (1,1) node[circle,fill=red!20,draw] {texte};
\\  \hline
node[fill=red!20] 
&
node[fill=red!20,\RDD{draw}] 
&
 node[fill=red!20,\RDD{circle}]  
&
 node[fill=red!20,\RDD{circle},\RDD{draw}]
 \\  \hline
\end{tabular}
\bigskip


\subsubsection{Options}
\begin{tabular}{|c | c | c | c |c |c |c |c |} \hline
\multicolumn{8}{|c|}{ \BS{tikz} \BS{draw} node[draw,\RDD{double},blue] \AC{texte};   }\\ 
\hline 

\tikz \draw  node[draw,double,blue] {texte};
&
\tikz \draw  node[draw,rounded corners,blue] {texte};
&
\tikz \draw  node[draw,ultra thick,blue] {texte};
&
\tikz \draw  node[draw,dashed,blue] {texte};
&
\tikz \draw  node[draw,red] {texte};
&
\tikz \draw  node[draw,rotate=45,blue] {texte};
&
\tikz \draw  node[draw,shading=radial,blue] {texte};
&
\tikz \draw  node[draw,blue,text=red] {texte};
\\ \hline
\RDD{double} & \RDD{rounded corners} &  ultra thick & dashed & red & rotate=45 & shading=radial & text=red 
\\ \hline
\end{tabular}
\bigskip


\begin{tabular}{|c | c | c | c |c |} \hline
\multicolumn{4}{|c|}{ \BS{tikz} \BS{draw}  node[draw,\RDD{inner sep}=0pt] \AC{texte};   }\\ 
\hline 
\tikz \draw  node[draw,inner sep=0pt,blue] {texte};
&
\tikz \draw node[draw,inner sep=1cm,blue] {texte};
&
\tikz \draw  node[draw,inner xsep=1cm,blue] {texte};
&
\tikz \draw  node[draw,inner ysep=1cm,blue] {texte};
\\ \hline
 \RDD{inner sep}=0pt & \RDD{inner sep}=1cm & \RDD{inner xsep}=1cm & \RDD{inner ysep}=1cm
\\ \hline
\multicolumn{4}{|c|}{ \dft{} : 0.3333em }\\ 
\hline 

\end{tabular}

\bigskip

\begin{tabular}{|c | c | c | c |} \hline
\multicolumn{4}{|l|}{ \BS{node} [fill=red!20,\RDD{outer sep}=1cm] (A) at (1,1) \AC{texte};   }\\ 
\multicolumn{4}{|l|}{ \BS{fill} (node cs:name=A,anchor=east) circle (3pt);  }\\ 
\multicolumn{4}{|l|}{ \BS{fill} (node cs:name=A,anchor=south) circle (3pt);  }\\ 
\hline 
\begin{tikzpicture}
\draw[help lines] (0,0) grid (3,2);
\node[fill=red!20,outer sep=1cm] (A) at (1,1) {texte};
\fill[red] (node cs:name=A,anchor=east) circle (3pt);
\fill[red] (node cs:name=A,anchor=south) circle (3pt);
\end{tikzpicture}
&
\begin{tikzpicture}
\draw[help lines] (0,0) grid (3,2);
\node[fill=red!20,outer sep=0pt] (A) at (1,1) {texte};
\fill[red] (node cs:name=A,anchor=east) circle (3pt);
\fill[red] (node cs:name=A,anchor=south) circle (3pt);
\end{tikzpicture}
&
\begin{tikzpicture}
\draw[help lines] (0,0) grid (3,2);
\node[fill=red!20,outer xsep=1cm] (A) at (1,1){texte};
\fill[red] (node cs:name=A,anchor=east) circle (3pt);
\fill[red] (node cs:name=A,anchor=south) circle (3pt);
\end{tikzpicture}
&
\begin{tikzpicture}
\draw[help lines] (0,0) grid (3,2);
\node[fill=red!20,outer ysep=1cm] (A) at (1,1) {texte};
\fill[red] (node cs:name=A,anchor=east) circle (3pt);
\fill[red] (node cs:name=A,anchor=south) circle (3pt);
\end{tikzpicture}
\\ \hline
 \RDD{outer sep}=1cm & \RDD{outer sep}=0pt & \RDD{outer xsep}=1cm & \RDD{outer ysep}=1cm
\\ \hline
\multicolumn{4}{|c|}{ \dft{} : 0.5\BS{pgflinewidth} }\\ 
\hline 
\end{tabular}
%----------------------------------------------------------------------------------
%\subsubsection{Taille minimale des noeuds}
\SbSbSSCT{Taille minimale des noeuds}{Minimum size}

\begin{tabular}{|c|c|} \hline  
\multicolumn{2}{|c|}{  \BS{draw}((0,0) node[fill=blue!20,\RDD{minimum height}=1.5cm,draw]  \AC{texte} ;   }\\ 
\hline 
\tikz \draw (0,0) node[fill=red!20,minimum height=1.5cm,draw] {texte};
&  
\tikz \draw (0,0) node[fill=red!20,minimum width=3cm,draw] {texte};

\\ \hline  

\RDD{minimum height}=1.5cm
&  
\RDD{minimum width}=3cm
\\ \hline  
\tikz \draw (0,0) node[fill=red!20,minimum size=1.5cm,draw] {texte};
&  
\tikz \draw (0,0) node[fill=red!20,minimum size=1.5cm,draw,circle] {texte};

\\ \hline 
\RDD{minimum size}=1.5cm,draw
&  
\RDD{minimum size}=1.5cm,circle

\\ \hline 
\end{tabular} 

\newpage
%-----------------------------------------------
%\subsection{Dans un n\oe ud à formes géométriques}
\SbSSCT{Dans un n\oe ud à formes géométriques}{Geometric Shapes nodes}

\label{lib-geom}
\label{formes}
%Insérer dans le préambule :

 \maboite{\BS{usetikzlibrary}\AC{shapes.geometric}}
 
 
\begin{center}
\RRR{67-3}
\end{center}
%\subsubsection{Formes disponibles}
\SbSbSSCT{Formes disponibles}{Available shapes}

\label{nd1}

\begin{tabular}{|c|c|c|c|} \hline  
\multicolumn{4}{|l|}{ 2 syntaxes :   }\\ 
\multicolumn{4}{|l|}{ \BS{tikz} \BS{node}[fill=green!20,\RDD{shape}=diamond,draw,blue] \AC{texte};   }\\ 
\multicolumn{4}{|l|}{ \BS{tikz} \BS{node}[fill=green!20,\RDD{diamond},draw] \AC{texte};   }\\ 
\hline 
\tikz  \node[fill=green!20,diamond,draw] {texte}; 
&  
\tikz  \node[fill=green!20,ellipse,draw] {texte};
&  
\tikz  \node[fill=green!20,trapezium, regular polygon sides=6,draw] {texte};
&
\tikz  \node[fill=green!20,semicircle,draw] {texte}; 
\\ \hline 
diamond & ellipse  & trapezium & semicircle
\\ \hline 
\tikz  \node[fill=green!20,star,draw] {texte};
&  
\tikz  \node[fill=green!20,regular polygon,draw] {texte};
&  
\tikz  \node[fill=green!20,isosceles triangle,draw] {texte};
&
\tikz  \node[fill=green!20,kite,draw] {texte};
\\ \hline 
star & regular polygon  & isosceles triangle & kite 
\\ \hline 
\tikz  \node[fill=green!20,dart,draw] {texte};
&
\tikz  \node[fill=green!20,circular sector,draw] {texte};
&
\tikz  \node[fill=green!20,cylinder,draw] {texte};
&

\\ \hline 
dart & circular sector & cylinder &
\\ \hline 
\end{tabular} 

%---------------------------------------------------------------------------------------
\subsubsection{Options}

\begin{tabular}{|c|c|c|} \hline
\multicolumn{3}{|c|}{  \BS{node} [trapezium,draw,\RDD{trapezium left angle}=90,draw,blue] \AC{texte};   }\\ 
\hline
\begin{tikzpicture}
\node[trapezium,draw,red,dashed] {texte};
\node[trapezium,draw,trapezium left angle=90,draw,blue] {texte};
\end{tikzpicture}
& 
\begin{tikzpicture}
\node[trapezium,draw,red,dashed] {texte};
\node[trapezium,draw,trapezium right angle=90,draw,blue] {texte};
\end{tikzpicture} 
& 
\begin{tikzpicture}
\node[trapezium,draw,red,dashed] {texte};
\node[trapezium,draw,trapezium angle=120,draw,blue] {texte};
\end{tikzpicture} 
\\ \hline
\RDD{trapezium left angle}=90  & \RDD{trapezium right angle}=90  & \RDD{trapezium  angle}=120 \\ 
\hline 
\begin{tikzpicture}
\node[trapezium,draw,red,dashed] {texte};
\node[trapezium,draw,minimum height=1.5cm,trapezium stretches=true,draw,blue] {texte};
\end{tikzpicture}
& 
\begin{tikzpicture}
\node[trapezium,draw,red,dashed] {texte};
\node[trapezium,draw,minimum height=1.5cm,trapezium stretches=false,draw,blue] {texte};
\end{tikzpicture} 
& 
\begin{tikzpicture}
\node[trapezium,draw,red,dashed] {texte};
\node[trapezium,draw,minimum width=3cm,trapezium stretches =false,draw,blue] {texte};
\end{tikzpicture} 

\\ \hline
minimum height=1.5cm & minimum height=1.5cm & minimum width=1.5cm \\
\RDD{trapezium stretches}=true & \RDD{trapezium stretches}=false & \RDD{trapezium stretches}  \\ 
\hline
%
%& 
%\begin{tikzpicture}
%\node[trapezium,draw,red,dashed] {texte};
%\node[trapezium,draw,minimum width=1.5cm,trapezium stretches body=false,draw,blue] {texte};
%\end{tikzpicture} 
%&
%\\
\end{tabular} 

%\tikz  \draw (-1,-1) grid (1,1) (0,0) node[fill=red!20,shape=trapezium,draw,minimum height=1.5cm,trapezium stretches=true] {texte};
%
%\tikz  \draw (-1,-1) grid (1,1) (0,0) node[fill=red!20,shape=trapezium,draw,minimum height=1.5cm,trapezium stretches=false] {texte};
%
%\tikz  \draw (-1,-1) grid (1,1) (0,0) node[fill=red!20,shape=trapezium,draw,minimum width=1.5cm,trapezium stretches] {texte};
%
%\tikz  \draw (-1,-1) grid (1,1) (0,0) node[fill=red!20,shape=trapezium,draw,minimum width=1.5cm,trapezium stretches body] {texte};


\bigskip
\begin{tabular}{|c|c|c|} \hline
\multicolumn{3}{|c|}{ \BS{tikz} \BS{node} [fill=green!20,star,\RDD{star points}=6,draw] \AC{texte};   }\\ 
\hline
\begin{tikzpicture}
\node[star,draw,red,dashed] {texte};
\node[star,star points=7,draw,blue] {texte};
\end{tikzpicture}
&  
\begin{tikzpicture}
\node[star,draw,red,dashed] {texte};
\node[star,star point height = 2cm,draw,blue] {texte};
\end{tikzpicture} 
&  
\begin{tikzpicture}
\node[star,draw,red,dashed] {texte};
\node[star,star point ratio = 3,draw,blue] {texte};
\end{tikzpicture} 
\\ \hline  
\RDD{star points}=7 & \RDD{star point height} = 2cm & \RDD{star point ratio} = 3 \\ \hline
\dft{5} & \dft.5cm &  \dft{1.5}\\ 
\hline 
\end{tabular} 
\bigskip

\begin{tabular}{|c|c|c|} \hline
\multicolumn{3}{|c|}{  \BS{node} [isosceles triangle,\RDD{isosceles triangle apex angle}=90,draw,blue] \AC{texte};   }\\ 
\multicolumn{3}{|c|}{  \BS{node} [regular polygon, \RDD{regular polygon sides}=6,draw,blue] \AC{texte};   }\\ 
\hline
\begin{tikzpicture}
\node[isosceles triangle,draw,red,dashed] {texte};
 \node[isosceles triangle,isosceles triangle apex angle=90,draw,blue] {texte};
\end{tikzpicture} 
& 
\begin{tikzpicture}
\node[isosceles triangle,draw,red,dashed] {texte};
 \node[isosceles triangle,isosceles triangle stretches=true,draw,blue] {texte};
\end{tikzpicture}
&  
\begin{tikzpicture}
\node[regular polygon,draw,red,dashed] {texte};
\node[regular polygon, regular polygon sides=6,draw,blue] {texte};
\end{tikzpicture} 
\\ \hline  
\RDD{isosceles triangle apex angle}=90 & \RDD{isosceles triangle stretches} & \RDD{regular polygon sides}=6 \\ 
\hline 
\end{tabular} 
\bigskip

\begin{tabular}{|c|c|c|} \hline 
\multicolumn{3}{|c|}{  \BS{node} [kite,\RDD{kite upper vertex angle}=90,draw,blue] \AC{texte};   }\\ 
\hline 
\begin{tikzpicture}
\node[red,kite,draw,dashed] {texte} ;
 \node[kite,kite upper vertex angle=90,draw,blue] {texte};
\end{tikzpicture} 
&  
\begin{tikzpicture}
\node[red,kite,draw,dashed] {texte} ;
 \node[kite,kite lower vertex angle=90,draw,blue] {texte};
\end{tikzpicture} 
&  
\begin{tikzpicture}
\node[red,kite,draw,dashed] {texte} ;
\node[kite,kite vertex angles=90,draw,blue] {texte};
\end{tikzpicture} 
\\ \hline  
\RDD{kite upper vertex angle}=90 & \RDD{kite lower vertex angle}=90 &\RDD{kite vertex angles}=90
\\ \hline 
initially 120 & initially 60 &  \\ 
\hline 
\end{tabular} 

\bigskip

\begin{tabular}{|c|c|c|} \hline
\multicolumn{3}{|c|}{  \BS{node} [dart,\RDD{dart tip angle}=90,draw,blue] \AC{texte};   }\\ 
\hline 
\begin{tikzpicture}
\node[dart,draw,red,dashed] {texte};
\node[dart,dart tip angle=90,draw,blue] {texte};
\end{tikzpicture} 
&  
\begin{tikzpicture}
\node[dart,draw,red,dashed] {texte};
\node[dart,dart tail angle=90,draw,blue] {texte};
\end{tikzpicture} 
&  
\begin{tikzpicture}
\node[,circular sector,draw,red,dashed] {texte};
\node[circular sector,circular sector angle=90,draw,blue] {texte};
\end{tikzpicture} 
\\ \hline  
\RDD{dart tip angle}=90 & \RDD{dart tail angle}=90  & \RDD{circular sector angle}=90
\\ \hline  
initially 45 & initially 135 & initially 60  \\ 
\hline 
\end{tabular} 

\bigskip

\begin{tabular}{|c|c|} \hline  
\multicolumn{2}{|c|}{  \BS{node} [cylinder,\RDD{aspect=2},draw,blue] \AC{texte};   }\\ 
\hline
\tikz  \node[cylinder,aspect=2,draw,blue] {texte};
& 
 \tikz  \node[cylinder,aspect=4,draw,blue] {texte};
\\ \hline 
\RDD{aspect}=2 & \RDD{aspect}=4 
\\ \hline
\tikz  \node[cylinder,cylinder uses custom fill, cylinder end fill=yellow,draw,blue] {texte};
&  
\tikz  \node[cylinder,cylinder uses custom fill, cylinder body fill=yellow,draw,blue] {texte};
\\ \hline
\RDD{cylinder uses custom fill}, & \RDD{cylinder uses custom fill}, \\ 
\RDD{cylinder end fill}=yellow & \RDD{cylinder body fill}=yellow  \\ 
\hline 
\end{tabular} 

%\subsection{Ratio hauteur/largeur}
\bigskip

\begin{tabular}{|c|c|c|c|} \hline 
\multicolumn{4}{|c|}{  \BS{draw}(0,0) node[\RDD{shape aspect}=1,diamond,draw]  \AC{texte} ;   }
\\ \hline
 
\tikz \draw (0,0) node[shape aspect=1,diamond,draw,blue] {texte};
&  
\tikz \draw (0,-2) node[shape aspect=2,diamond,draw,blue] {texte};
&
\tikz \draw (0,0) node[shape aspect=3,diamond,draw,blue] {texte};
&
\tikz \draw (0,0) node[shape aspect=4,diamond,draw,blue] {texte};
\\ \hline  
\RDD{shape aspect}=1
&  
\RDD{shape aspect}=2
&
\RDD{shape aspect}=3
&
\RDD{shape aspect}=4
\\ \hline 
\end{tabular} 


%==============================================================
\newpage
%\subsection{Dans un n\oe ud en forme de symboles}
\SbSSCT{Dans un n\oe ud en forme de symboles}{Symbol Shapes nodes}
\label{lib-symb}

\maboite{\BS{usetikzlibrary}\AC{shapes.symbols}}

\begin{center}
\RRR{67-4}
\end{center}

%\subsubsection{Formes disponibles}
\SbSbSSCT{Formes disponibles}{Available shapes}

\label{nd2}

\begin{tabular}{|c|c|c|} \hline  
\tikz  \node[fill=green!20,forbidden sign,draw] {texte};
&  
\tikz  \node[fill=green!20,magnifying glass,draw] {texte};
&  
\tikz  \node[fill=green!20,cloud,draw] {texte};
\\ \hline 
forbidden sign & magnifying glass & cloud
\\ \hline  
\tikz  \node[fill=green!20,starburst,draw] {texte};
&  
\tikz  \node[fill=green!20,signal,draw] {texte};

&  
\tikz  \node[fill=green!20,tape,draw] {texte};
\\ \hline 
starburst & signal & tape
\\ \hline 
\end{tabular} 
\bigskip

\subsubsection{Options}

\begin{tabular}{|c|c|c|} \hline  
\multicolumn{3}{|c|}{  \BS{node}[magnifying glass,\RDD{magnifying glass handle angle}=45,draw,blue]  \AC{texte} ;   }
\\ \hline
\tikz  \node[magnifying glass,magnifying glass handle angle=45,draw,blue] {texte};
&  
\tikz  \node[,magnifying glass,magnifying glass handle aspect=3,draw,blue] {texte};
& 
\tikz  \node[magnifying glass,line width=1ex,draw,blue] {texte};

\\ \hline  
\RDD{magnifying glass handle angle}=45 & \RDD{magnifying glass handle aspect}=3  & line width=1ex  
\\ \hline 
\dft{ : -45} & \dft{ : 1.5}& 
\\ \hline 
\end{tabular} 

\bigskip

\begin{tabular}{|c|c|c|c|} \hline 
\multicolumn{4}{|c|}{  \BS{node} [cloud,\RDD{cloud puffs}=5,draw,blue] \AC{texte};   }\\ 
\hline 
\begin{tikzpicture}
\node[cloud,draw,red,dashed] {texte};
\node[cloud,cloud puffs=5,draw,blue] {texte};
\end{tikzpicture} 
&  
\begin{tikzpicture}
\node[cloud,draw,red,dashed] {texte};
\node[cloud,cloud puff arc=270,draw,blue] {texte};
\end{tikzpicture} 
&  
\begin{tikzpicture}
\node[cloud,draw,red,dashed] {texte};
\node[cloud,cloud ignores aspect=true,draw,blue] {texte};
\end{tikzpicture} 
&
\begin{tikzpicture}
\node[cloud,draw,red,dashed] {texte};
\node[cloud,cloud ignores aspect=false,draw,blue] {texte};
\end{tikzpicture} 
\\ \hline  
\RDD{cloud puffs}=5 & \RDD{cloud puff arc}=270 & \RDD{cloud ignores aspect}=false & \RDD{cloud ignores aspect}=true  \\ 
\hline 
\dft :  10 & \dft :  135 &\multicolumn{2}{|c|}{ \dft :  true } \\ \hline
\end{tabular} 

\bigskip

\begin{tabular}{|c|c|c|c|} \hline 
\multicolumn{4}{|c|}{  \BS{node} [starburst,\RDD{starburst points}=5,draw,blue] \AC{texte};   }\\ 
\hline  
\tikz  \node[starburst,starburst points=5,draw,blue] {texte};
&  
\tikz  \node[starburst,starburst point height=1cm,draw,blue] {texte};
&  
\tikz  \node[starburst,random starburst=50,draw,blue] {texte};
&
\tikz  \node[,starburst,random starburst=0,draw,blue] {texte};
\\ \hline  
\RDD{starburst points}=5 & \RDD{starburst point height}=1cm & \RDD{random starburst}=50 & \RDD{random starburst}=0  \\ 
\hline 
\end{tabular} 

\bigskip


\begin{tabular}{|c|c|c|} \hline 
\multicolumn{3}{|c|}{  \BS{node} [signal,\RDD{signal pointer angle}=45,draw,blue] \AC{texte};   }\\ 
\hline 
\tikz  \node[signal,signal pointer angle=45,draw,blue] {texte};
&
\tikz  \node[signal,signal pointer angle=10,draw,blue] {texte};
&
\tikz  \node[signal,signal pointer angle=300,draw,blue] {texte};
\\ \hline 
\RDD{signal pointer angle}=45
&
signal pointer angle=10
&
signal pointer angle=300
\\ \hline 
\multicolumn{3}{|c|}{  \dft{ : signal pointer angle= 90}  }
\\  \hline 

\end{tabular} 
\bigskip

\begin{tabular}{|c|c|c|c|c|} \hline 
\multicolumn{4}{|c|}{  \BS{node} [signal,\RDD{signal to}=above,draw,blue] \AC{texte};   }
\\ \hline 
\tikz  \node[signal,signal to=above,draw,blue] {texte};
&  
\tikz  \node[signal,signal to=below,draw,blue] {texte};
&
\tikz  \node[signal,signal to=right,draw,blue] {texte};
&
\tikz  \node[signal,signal to=above,draw,blue] {texte};
\\ \hline  
  \RDD{signal to}=above  & \RDD{signal to}=below & \RDD{signal to}=right  & \RDD{signal to}=above \\ 
\hline 
\end{tabular} 
\bigskip

\begin{tabular}{|c|c|c|c|c|} \hline 
\multicolumn{4}{|c|}{ \BS{tikz} [signal to=nowhere] \BS{node} [signal,\RDD{signal from=above}=45,draw,blue] \AC{texte};   }\\ 
\hline 
\tikz [signal to=nowhere] \node[signal,signal from=above,draw,blue] {texte};
&  
\tikz [signal to=nowhere] \node[signal,signal from=below,draw,blue] {texte};
&
\tikz [signal to=nowhere] \node[signal,signal from=right,draw,blue] {texte};
&
\tikz [signal to=nowhere] \node[signal,signal from=above,draw,blue] {texte};
\\ \hline  
  \RDD{signal from}=above  & \RDD{signal from}=below & \RDD{signal from}=right  & \RDD{signal from}=above \\ 
\hline 
\end{tabular} 

\bigskip
\begin{tabular}{|c|c|c|c|} \hline
\multicolumn{2}{|c|}{ \tikz  \node[draw,signal, signal from=east , signal to=west,blue] at (0,0) {texte};}
&
\multicolumn{2}{|c|}{ \tikz  \node[draw,signal,signal from=south, signal to=north,blue] at (0,0) {texte};}
\\ \hline 
\multicolumn{2}{|c|}{ \RDD{signal from}=east , \RDD{signal to}=west}
&
\multicolumn{2}{|c|}{\RDD{signal from}=south, \RDD{signal to}=north}

\\ \hline 
\end{tabular}
\bigskip

\begin{tabular}{|c | c | c | c |} \hline
\multicolumn{3}{|c|}{ \BS{tikz} \BS{node}  [tape, draw,\RDD{tape bend top}=out and in] \AC{texte};   }\\ 
\hline  
\tikz \node [tape, draw,tape bend top=out and in,blue] {texte};
&
\tikz \node [tape, draw, tape bend bottom=out and in,blue] {texte};
&
\tikz \node [tape, draw, tape bend bottom=in and in,blue] {texte};
 \\  \hline
 \RDD{tape bend top}=out and in & \RDD{tape bend bottom}=out and in &  \RDD{tape bend bottom}=in and in 
  \\  \hline
 \tikz \node [tape, draw, tape bend top=none,blue] {texte};
 &
 \tikz \node [tape, draw,tape bend top=out and in,tape bend bottom=out and in,blue] {texte};
 &
  \tikz \node [tape, draw,tape bend top=in and out,tape bend bottom=in and out,blue] {texte};
  \\  \hline
 \RDD{tape bend top}=none & \RDD{tape bend bottom}=out and in 	&  \RDD{tape bend bottom}=in and out  \\
 					& \RDD{tape bend top}=out and in 		& \RDD{tape bend top}=in and out  \\
 					& & (\dft{} ) 
  \\  \hline 
\end{tabular}
\bigskip

\begin{tabular}{|c | c | c | c |} \hline
\BS{tikz} \BS{node} [tape, draw, \RDD{tape bend height}=1cm,blue] \AC{texte}; 
  \\  \hline 
\tikz \node [tape, draw, tape bend height=1cm,blue] {texte};

  \\  \hline 
\dft{ : tape bend height = 5pt}
  \\  \hline 
\end{tabular}
%=============================================================
\newpage
%\subsection{Dans un n\oe ud en forme de flèche}
\SbSSCT{Dans un n\oe ud en forme de flèche}{Arrow Shapes nodes}

\label{lib-arr}

\maboite{\BS{usetikzlibrary}\AC{shapes.arrows}}

\begin{center}
\RRR{67-5}
\end{center}
%\subsubsection{Formes disponibles}
\SbSbSSCT{Formes disponibles}{Available shapes}
\label{nd3}

\begin{tabular}{|c|c|c|} \hline  
\tikz \node[fill=green!20,single arrow,draw] {texte};
&  
\tikz  \node[fill=green!20,double arrow,draw] {texte};
&  
\tikz  \node[fill=green!20,arrow box,draw] {texte};
\\ \hline 
single arrow & double arrow & arrow box \\ 
\hline 
\end{tabular} 

\subsubsection{Options}

\begin{tabular}{|c|c|c|c|c|} \hline  
 \multicolumn{5}{|c|}{  \BS{node}[single arrow,draw,\RDD{single arrow tip angle}=45] \AC{texte};   }\\ 
  \multicolumn{5}{|c|}{  \BS{node}[single arrow,draw,\RDD{single arrow head extend}=.75cm] \AC{texte};   }\\
 \hline
\begin{tikzpicture}
 \node[single arrow,draw,red,dashed,text=black] {texte};
 \node[single arrow,draw,single arrow tip angle=45,blue] {texte};
\end{tikzpicture}
&
\begin{tikzpicture}
 \node[single arrow,draw,red,dashed,text=black] {texte};
\node[single arrow,draw,single arrow tip angle=120,blue] {texte};
\end{tikzpicture}
&
\begin{tikzpicture}
 \node[single arrow,draw,red,dashed,text=black] {texte};
 \node[single arrow,draw,single arrow head extend=.75cm,blue] {texte};
\end{tikzpicture}
&
\begin{tikzpicture}
 \node[single arrow,draw,red,dashed,text=black] {texte};
 \node[single arrow,draw,single arrow head extend=0cm,blue] {texte};
 \end{tikzpicture}
 &
 \begin{tikzpicture}
  \node[single arrow,draw,red,dashed,text=black] {texte};
  \node[single arrow,draw,single arrow head extend=-1mm,blue] {texte};
 \end{tikzpicture}

\\ \hline
angle=45 & angle=120 & extend=.75cm] & extend=0cm & extend=-1mm
\\ \hline 
\multicolumn{2}{|c|}{  \dft : single arrow tip angle= 90   }
&
\multicolumn{3}{|c|}{  \dft : single arrow head extend=0.5cm   }
\\ \hline 
\end{tabular} 
\bigskip


\begin{tabular}{|c|c|c|c|} \hline
 \multicolumn{4}{|c|}{  \BS{node}[minimum size=2cm,single arrow,draw,\RDD{single arrow head indent}=1cm,blue] \AC{texte};   }\\ 
 \hline   
\begin{tikzpicture}
 \node[minimum size=2cm,single arrow,draw,red,dashed,text=black] {texte};
\node[minimum size=2cm,single arrow,draw,single arrow head indent=1cm,blue] {texte};
\end{tikzpicture}
&
\begin{tikzpicture}
 \node[minimum size=2cm,single arrow,draw,red,dashed,text=black] {texte};
  \node[minimum size=2cm,single arrow,draw,single arrow head indent=10pt,blue] {texte};
  \end{tikzpicture}
&
\begin{tikzpicture}
 \node[minimum size=2cm,single arrow,draw,red,dashed,text=black] {texte};
  \node[minimum size=2cm,single arrow,draw,single arrow head indent=1ex,blue] {texte};
  \end{tikzpicture}
  &
  \begin{tikzpicture}
   \node[minimum size=2cm,single arrow,draw,red,dashed,text=black] {texte};
    \node[minimum size=2cm,single arrow,draw,single arrow head indent=-1ex,blue] {texte};
    \end{tikzpicture}
\\ \hline
indent=1cm & indent=10pt & indent=1ex & indent=-1ex
\\ \hline 
\end{tabular}
\bigskip

 



\begin{tabular}{|c|c|c|c|c|} \hline
 \multicolumn{5}{|c|}{  \BS{node}[minimum size=2cm,double arrow,draw,\RDD{double arrow tip angle}=45] \AC{texte};   }\\ 
  \multicolumn{5}{|c|}{  \BS{node}[minimum size=2cm,double arrow,draw,\RDD{double arrow head extend}=1ex] \AC{texte};   }\\
   \multicolumn{5}{|c|}{  \BS{node}[minimum size=2cm,double arrow,draw,\RDD{double arrow head indent}=1ex] \AC{texte};   }\\ 
 \hline  
\begin{tikzpicture}
\node[minimum size=2cm,double arrow,draw,red,dashed,text=black] {texte};
\node[minimum size=2cm,double arrow,draw,double arrow tip angle=45,blue] {texte};
\end{tikzpicture}
&
\begin{tikzpicture}
\node[minimum size=2cm,double arrow,draw,red,dashed,text=black] {texte};
\node[minimum size=2cm,double arrow,draw,double arrow tip angle=120,blue] {texte};
\end{tikzpicture}
&
\begin{tikzpicture}
 \node[minimum size=2cm,double arrow,draw,red,dashed,text=black] {texte};
 \node[minimum size=2cm,double arrow,draw,double arrow head extend=1ex,blue] {texte};
   \end{tikzpicture}
&
\begin{tikzpicture}
 \node[minimum size=2cm,double arrow,draw,red,dashed,text=black] {texte};
  \node[minimum size=2cm,double arrow,draw,double arrow head extend=0,blue] {texte};
    \end{tikzpicture}
&
\begin{tikzpicture}
 \node[minimum size=2cm,double arrow,draw,red,dashed,text=black] {texte};
  \node[,minimum size=2cm,double arrow,draw,double arrow head indent=1ex,blue] {texte};
    \end{tikzpicture}
\\ \hline 
angle=45 & angle=120 & extend=1ex & extend=0 & indent=1ex
\\ \hline
\end{tabular}

\bigskip

\begin{tabular}{|c|c|c|c|c|} \hline
\multicolumn{4}{|c|}{ \BS{node} [arrow box, draw, \RDD{arrow box arrows}=\AC{north:.25cm}] \AC{texte}; }\\ 
\hline 
\begin{tikzpicture}
\node[arrow box, draw,red,text=white,dashed] {texte};
\node[arrow box, draw, arrow box arrows={north:.25cm},blue] {texte};
\end{tikzpicture}
& 
\begin{tikzpicture}
\node[arrow box, draw,red,text=white,dashed] {texte};
\node[arrow box, draw, arrow box arrows={west:.25cm},blue] {texte};
\end{tikzpicture}
 &
 \begin{tikzpicture}
 \node[arrow box, draw,red,text=white,dashed] {texte};
 \node[arrow box, draw, arrow box arrows={south:.25cm},blue] {texte};
 \end{tikzpicture}
&
 \begin{tikzpicture}
 \node[arrow box, draw,red,text=white,dashed] {texte};
 \node[arrow box, draw, arrow box arrows={east:.25cm},blue] {texte};
 \end{tikzpicture}   
 \\ \hline
\AC{north:.25cm} & \AC{west:.25cm} & \AC{south:.25cm}& \AC{east:.25cm} 
\\ \hline
\multicolumn{4}{|c|}{  \dft{} : 0.5 cm}
 \\ \hline 
 \end{tabular}
 
 
 \bigskip
 
 \begin{tabular}{|c|c|} \hline
 \multicolumn{2}{|c|}{ \BS{node} [arrow box, draw, \RDD{arrow box tip angle}=45] \AC{texte}; }\\ 
 \hline 
  \begin{tikzpicture}
  \node[arrow box, draw,red,text=white,dashed] {texte};
  \node[arrow box, draw, arrow box tip angle=45,blue] {texte};
  \end{tikzpicture} 
  &
    \begin{tikzpicture}
   \node[arrow box, draw,red,text=white,dashed] {texte};
   \node[arrow box, draw, arrow box head extend=.25cm,blue] {texte};
   \end{tikzpicture}
\\ \hline  
\RDD{arrow box tip angle}=45 & \RDD{arrow box head extend}=.25cm
\\ \hline 
\dft : 90  & \dft : 0.125cm 
\\ \hline 
   \begin{tikzpicture}
   \node[arrow box, draw,red,text=white,dashed] {texte};
   \node[arrow box, draw, arrow box head indent=.25cm,blue] {texte};
   \end{tikzpicture} 
 &
    \begin{tikzpicture}
    \node[arrow box, draw,red,text=white,dashed] {texte};
    \node[arrow box, draw,arrow box shaft width=.25cm,blue] {texte};
    \end{tikzpicture} 
 \\ \hline 
\RDD{arrow box head indent}=.25cm  &  \RDD{arrow box shaft width}=.25cm
 \\ \hline  
 \dft{ : 0cm } &  \dft{ : 0.125cm }
 \\ \hline  
 \end{tabular}



\newpage
%-----------------------------------------------------------------------
%\subsection{Dans un n\oe ud en forme de bulle}
\SbSSCT{Dans un n\oe ud en forme de bulle}{Callout Shapes nodes}
\label{lib-call}

%insérer dans le préambule : 

 \maboite{\BS{usetikzlibrary}\AC{shapes.callouts}}
 
\begin{center}
\RRR{67-7}
\end{center}
%\subsubsection{Formes disponibles}
\SbSbSSCT{Formes disponibles}{Available shapes}

\begin{tabular}{|c|c|c|} \hline 
\tikz  \node[fill=green!20,ellipse callout,draw] {texte};
 &  
 \tikz  \node[fill=green!20,rectangle callout,draw] {texte};
  &  
  \tikz  \node[fill=green!20,cloud callout,draw] {texte};
 \\ \hline
 ellipse callout  &  rectangle callout  & cloud callout \\ 
\hline 
\end{tabular} 
%------------------------------------------------

\subsubsection{Options}


\begin{tabular}{|c | c | c | c |} \hline
\multicolumn{4}{|c|}{  \BS{node} [rectangle callout,draw,\RDD{callout absolute pointer}={(0,1)}] at (2,1) \AC{texte};   }\\ 
\hline 
\begin{tikzpicture} 
\draw [help lines] grid(3,3);
\node [rectangle callout,draw,blue, callout relative pointer={(0,1)}] at (2,1) {texte};
\end{tikzpicture}
&
\begin{tikzpicture} 
\draw [help lines] grid(3,3);
\node [ellipse callout,draw, callout relative pointer={(0,1)},blue] at (2,1) {texte};
\end{tikzpicture}
&
\begin{tikzpicture} 
\draw [help lines] grid(3,3);
\node [rectangle callout,draw,blue,callout absolute pointer={(0,1)}] at (2,1) {texte};
\end{tikzpicture}
&
\begin{tikzpicture} 
\draw [help lines] grid(3,3);
\node [ellipse callout,draw, callout absolute pointer={(0,1)},blue] at (2,1) {texte};
\end{tikzpicture}
 \\  \hline
\multicolumn{2}{|c|}{ \RDD{callout relative pointer}=\AC{(0,1)} } & 
\multicolumn{2}{|c|}{  \RDD{callout absolute pointer}=\AC{(0,1)} }
 \\  \hline 
 \begin{tikzpicture} 
 \draw [help lines] grid(3,3);
 \node [rectangle callout,draw, callout relative pointer={(0,1)},callout pointer shorten=.5cm,blue] at (2,1) {texte};
 \end{tikzpicture}
 &
  \begin{tikzpicture} 
  \draw [help lines] grid(3,3);
  \node [ellipse callout,draw, callout relative pointer={(0,1)},callout pointer shorten=.5cm,blue] at (2,1) {texte};
  \end{tikzpicture}
  &
 \begin{tikzpicture} 
 \draw [help lines] grid(3,3);
 \node [rectangle callout,draw, callout absolute pointer={(0,1)},callout pointer shorten=.5cm,blue] at (2,1) {texte};
 \end{tikzpicture}
  &
  \begin{tikzpicture} 
  \draw [help lines] grid(3,3);
  \node [ellipse callout,draw, callout absolute pointer={(0,1)},callout pointer shorten=.5cm,blue] at (2,1) {texte};
  \end{tikzpicture}
  \\  \hline
\multicolumn{4}{|c|}{ \RDD{callout pointer shorten}=.5cm} 
  \\  \hline 
\end{tabular}

%-------------------------------------------------------------

\bigskip
 


\bigskip
\begin{tabular}{|c | c | c | c |} \hline
\multicolumn{3}{|c|}{  \BS{node} [ellipse callout,draw,\RDD{callout pointer arc}=1] at (0,1.5) \AC{texte};   }\\ 
\hline
\begin{tikzpicture}
\node[ellipse callout,draw, callout pointer arc=1,blue] at (0,1.5) {texte};
\end{tikzpicture}
&
\begin{tikzpicture}
\node[ellipse callout,draw, callout pointer arc=30,blue] at (0,1.5) {texte};
\end{tikzpicture}
 &
\begin{tikzpicture}
\node[ellipse callout,draw, callout pointer arc=90,blue] at (0,1.5) {texte};
\end{tikzpicture}
  \\  \hline 
   callout pointer arc=1 & callout pointer arc=30 & callout pointer arc=90
  \\  \hline  
  \multicolumn{3}{|c|}{  \dft{ : callout pointer arc=15}}
 \\  \hline  
 \end{tabular}

\bigskip

\begin{tabular}{|c | c | c | c |} \hline
\multicolumn{3}{|c|}{  \BS{node}[draw,cloud callout, aspect=2.5] \AC{texte};   }\\ 
\hline 
 \begin{tikzpicture}
  \node[draw,cloud callout, dashed,red,text=black] {texte};
 \node[draw,cloud callout, cloud puffs=5,blue] {texte};
 \end{tikzpicture}
&
 \begin{tikzpicture}
 \node[draw,cloud callout, dashed,red,text=black] {texte};
 \node[draw,cloud callout, aspect=2.5,blue] {texte};
 \end{tikzpicture}
&
  \begin{tikzpicture}
  \node[draw,cloud callout, dashed,red,text=black] {texte};
  \node[draw,cloud callout,cloud puff arc=120,blue] {texte};
  \end{tikzpicture}
   \\  \hline 
cloud puffs=5 & aspect=2.5 &  cloud puff arc=120
\\  \hline 
 \end{tabular}

\bigskip

\begin{tabular}{|c | c | c | c |c |} \hline
\multicolumn{3}{|c|}{  \BS{node} [draw,cloud callout,\RDD{callout pointer start size}=.1] \AC{texte};   }\\ 
\hline 
  \begin{tikzpicture}
  \node[draw,cloud callout, dashed,red,text=black] {texte};
  \node[draw,cloud callout,callout pointer start size=.1,blue] {texte};
  \end{tikzpicture}
&
  \begin{tikzpicture}
  \node[draw,cloud callout, dashed,red,text=black] {texte};
  \node[draw,cloud callout,callout pointer start size=.8cm,blue] {texte};
  \end{tikzpicture}
&
  \begin{tikzpicture}
  \node[draw,cloud callout, dashed,red,text=black] {texte};
 \node[draw,cloud callout,callout pointer start size=1cm and 0.1cm,blue] {texte};
  \end{tikzpicture}
\\  \hline 
\RDD{callout pointer start size}=.1 &start size=.8cm & start size=20pt and 1pt
\\  \hline 
\multicolumn{3}{|c|}{  \dft{} : callout pointer start size =.2 of callout  }
\\ 
\hline 
  \begin{tikzpicture}
  \node[draw,cloud callout, dashed,red,text=black] {texte};
  \node[draw,cloud callout,callout pointer end size=5,blue] {texte};
  \end{tikzpicture}
&
  \begin{tikzpicture}
  \node[draw,cloud callout, dashed,red,text=black] {texte};
  \node[draw,cloud callout,callout pointer end size=.8cm,blue] {texte};
  \end{tikzpicture}
&
    \begin{tikzpicture}
    \node[draw,cloud callout, dashed,red,text=black] {texte};
    \node[draw,cloud callout,callout pointer segments=3,blue] {texte};
    \end{tikzpicture}
\\  \hline 
\RDD{callout pointer end size}=.5 & \RDD{callout pointer end size}=.8cm & \RDD{callout pointer segments}=3
\\  \hline 
\multicolumn{2}{|c|}{  \dft{} : callout pointer start size = .1 of callout  }
& \dft{} : segments=2
\\  \hline  

 \end{tabular}
 


%----------------------------------------------------------------------
\newpage

%\subsection{Dans un n\oe ud en diverses formes  diverses}

\SbSSCT{Dans un n\oe ud en diverses formes  diverses}{Miscellaneous Shapes nodes}

\label{lib-misc}

%insérer dans le préambule:

 \maboite{\BS{usetikzlibrary}\AC{shapes.misc}}
 
\begin{center}
\RRR{67-8}
\end{center}

%\subsubsection{formes disponibles}
\SbSbSSCT{Formes disponibles}{Available shapes}

\begin{tabular}{|c|c|c|c|} \hline  
\tikz  \node[fill=green!20,cross out,draw] {texte};
&  
\tikz  \node[fill=green!20,strike out,draw] {texte};
&  
\tikz  \node[fill=green!20,rounded rectangle,draw] {texte};
&  
\tikz  \node[fill=green!20,chamfered rectangle,draw] {texte};
\\ \hline  
cross out & strike out & rounded rectangle & chamfered rectangle \\ 
\hline 
\end{tabular} 


\subsubsection{Options}

\paragraph{Options \TFRGB{pour}{for} \og rounded rectangle \fg} :


%
\begin{tabular}{|c|c|c|c|c|} \hline
\multicolumn{5}{|c|}{  \BS{node} [draw, rounded rectangle,\RDD{rounded rectangle arc length}=270] \AC{texte};   }\\ 

\hline 

%\begin{tikzpicture}
\tikz \node[draw, rounded rectangle,rounded rectangle arc length=270,blue] {texte}; 
&
\tikz \node[draw, rounded rectangle,rounded rectangle arc length=180,blue]  {texte}; 
&
\tikz \node[draw, rounded rectangle,rounded rectangle arc length=120,blue] {texte}; 
&
\tikz \node[draw, rounded rectangle,rounded rectangle arc length=90,blue]  {texte}; 
&
\tikz \node[draw, rounded rectangle,rounded rectangle arc length=45,blue] {texte}; 
 \\ \hline 
270 & 180 & 120 & 90& 45 
\\ \hline 
%\end{tikzpicture}

\end{tabular} 

\bigskip


\begin{tabular}{|c|c|c|c|} \hline 
\multicolumn{4}{|c|}{  \BS{node} [draw, rounded rectangle,\RDD{rounded rectangle west arc}=concave] \AC{texte};   }\\ 
\multicolumn{4}{|c|}{  \BS{node} [draw, rounded rectangle,\RDD{rounded rectangle left arc}=concave] \AC{texte};   }\\ 
\hline 
\tikz \node[draw, rounded rectangle,rounded rectangle west arc=concave,blue] {texte}; 
&
\tikz \node[draw, rounded rectangle,rounded rectangle left arc=concave,blue] {texte}; 
&
\tikz \node[draw, rounded rectangle,rounded rectangle west arc=convex,blue] {texte}; 
&
\tikz \node[draw, rounded rectangle,rounded rectangle left arc=none,blue] {texte};
 \\\hline 
concave & convex & none 
 \\\hline 
\end{tabular} 

\bigskip

\begin{tabular}{|c|c|c|c|} \hline 
\multicolumn{3}{|c|}{  \BS{node} [draw, rounded rectangle,\RDD{rounded rectangle east arc}=concave] \AC{texte};   }\\ 
\multicolumn{3}{|c|}{  \BS{node} [draw, rounded rectangle,\RDD{rounded rectangle right arc}=concave] \AC{texte};   }\\ 

\hline 
\tikz \node[draw, rounded rectangle,rounded rectangle east arc=concave,blue] {texte}; 
&
\tikz \node[draw, rounded rectangle,rounded rectangle  east arc=convex,blue] {texte}; 
&
\tikz \node[draw, rounded rectangle,rounded rectangle right arc=none,blue] {texte};
 \\\hline 
concave & convex & none 
 \\\hline 
\end{tabular} 

\paragraph{Options  \TFRGB{pour}{for} \og chamfered rectangle \fg} :


\begin{tabular}{|c|c|c|c|} \hline 
\multicolumn{4}{|c|}{  \BS{node} [draw, chamfered rectangle,\RDD{chamfered rectangle angle}=30] \AC{texte};   }\\ 
\hline 
\tikz \node[draw, chamfered rectangle,chamfered rectangle angle=10,blue] {texte}; 
&
\tikz \node[draw, chamfered rectangle,chamfered rectangle angle=30,blue] {texte}; 
&
\tikz \node[draw,chamfered rectangle,chamfered rectangle angle=60,blue] {texte};
&
\tikz \node[draw,chamfered rectangle,chamfered rectangle angle=80,blue] {texte};
 \\ \hline 
10 & 30 & 60 & 80
\\ \hline 
\multicolumn{4}{|c|}{  \dft :  45 }
  \\\hline  

\end{tabular}

\bigskip

\begin{tabular}{|c|c|c|c|c|} \hline 
\multicolumn{5}{|c|}{  \BS{node} [draw, chamfered rectangle,\RDD{chamfered rectangle xsep}=10pt] \AC{texte};   }\\ 
\hline 
\tikz \node[draw, chamfered rectangle,chamfered rectangle xsep=0pt,blue] {texte}; 
&
\tikz \node[draw, chamfered rectangle,chamfered rectangle xsep=5pt,blue] {texte}; 
&
\tikz \node[draw, chamfered rectangle,chamfered rectangle xsep=10pt,blue] {texte}; 
&
\tikz \node[draw,chamfered rectangle,chamfered rectangle xsep=-10pt,blue] {texte};
&
\tikz \node[draw,chamfered rectangle,chamfered rectangle xsep=2cm,blue] {texte};
 \\\hline 
  xsep=0pt & xsep=5pt & xsep=10pt & xsep=-10pt  & xsep=2cm
  \\\hline  
\multicolumn{5}{|c|}{  \dft :  0.666ex }
  \\\hline   
\end{tabular}

\bigskip

\begin{tabular}{|c|c|c|c|c|} \hline 
\multicolumn{5}{|c|}{  \BS{node} [draw, chamfered rectangle,\RDD{chamfered rectangle ysep}=10pt] \AC{texte};   }\\ 
\hline 
\tikz \node[draw, chamfered rectangle,chamfered rectangle ysep=0pt,blue] {texte}; 
&
\tikz \node[draw, chamfered rectangle,chamfered rectangle ysep=5pt,blue] {texte}; 
&
\tikz \node[draw,chamfered rectangle,chamfered rectangle ysep=10pt,blue] {texte};
&
\tikz \node[draw,chamfered rectangle,chamfered rectangle ysep=-10pt,blue] {texte};
&
\tikz \node[draw,chamfered rectangle,chamfered rectangle ysep=1cm,blue] {texte};
 \\ \hline 
 ysep=0pt & ysep=5pt & ysep=10pt & ysep=-10pt & ysep=1cm
 \\\hline  
\end{tabular}

\bigskip

\begin{tabular}{|c|c|c|c|c|} \hline 
\multicolumn{5}{|c|}{  \BS{node} [draw, chamfered rectangle,\RDD{chamfered rectangle ysep}=10pt] \AC{texte};   }\\ 
\hline 
\tikz \node[draw, chamfered rectangle,chamfered rectangle sep=0pt,blue] {texte}; 
&
\tikz \node[draw, chamfered rectangle,chamfered rectangle sep=5pt,blue] {texte}; 
&
\tikz \node[draw, chamfered rectangle,chamfered rectangle sep=10pt,blue] {texte}; 

&
\tikz \node[draw, chamfered rectangle,chamfered rectangle sep=-10pt,blue] {texte}; 
&
\tikz \node[draw,chamfered rectangle,chamfered rectangle sep=1cm,blue] {texte};
 \\\hline 
 sep=0pt & sep=5pt & sep=10pt& sep=-10pt & sep=1cm
 \\\hline  
\end{tabular}

\bigskip

\begin{tabular}{|c|c|c|c|} \hline 
\multicolumn{3}{|c|}{  \BS{node} [draw, chamfered rectangle,\RDD{chamfered rectangle corners}=north west] \AC{texte};   }\\ 
\hline
\tikz \node[draw, chamfered rectangle,chamfered rectangle corners=north west,blue] {texte}; 
&
\tikz \node[draw, chamfered rectangle,chamfered rectangle corners={north east, south east},blue] {texte}; 
&
\tikz \node[draw,chamfered rectangle,chamfered rectangle corners={north east, south west},blue] {texte};
 \\ \hline 
 north west & \AC{north east, south east}  & \AC{north east, south west}
 \\ \hline 
\end{tabular}





%\begin{tikzpicture}
%\tikzset{every node/.style={chamfered rectangle, draw}}
%\node[chamfered rectangle corners=north west] {ghi};
%\node[chamfered rectangle corners={north east, south east}] at (1.5,0) {789};
%\end{tikzpicture}


%\begin{tikzpicture}
%\tikzset{every node/.style={chamfered rectangle, draw}}
%\node[chamfered rectangle xsep=2pt] {def};
%\node[chamfered rectangle xsep=2cm] at (1.5,0) {456};
%\end{tikzpicture}

%\begin{tikzpicture}
%\tikzset{every node/.style={chamfered rectangle, draw}}
%\node[chamfered rectangle angle=30] {abc};
%\node[chamfered rectangle angle=60] at (1.5,0) {123};
%\end{tikzpicture}

%\begin{tikzpicture}
%\matrix[row sep=5pt, every node/.style={draw, rounded rectangle}]{
%\node[rounded rectangle west arc=concave] {Concave}; \\
%\node[rounded rectangle west arc=convex] {Convex}; \\
%\node[rounded rectangle left arc=none] {None}; \\};
%\end{tikzpicture}
%\tikz  \draw (-1,-1) grid (1,1) (0,0) node[fill=red!20,diamond,draw,rounded corners] {texte};&
 
%------------------------------------------------------------------------------------------

\newpage
%\subsection{N\oe uds à plusieurs parties}
\SbSSCT{N\oe uds à plusieurs parties}{Shapes with Multiple Text Parts}

\label{lib-mult}

%insérer dans le préambule :

 \maboite{\BS{usetikzlibrary}\AC{shapes.multipart}}

\begin{center}
\RRR{67-6}
\end{center}



\begin{tabular}{|c|c|c|c|} \hline 
\multicolumn{4}{|c|}{  \BS{node} [\RDD{circle split},draw,fill=green!20]\AC{haut  \BSS{nodepart}\AC{lower} bas };   }\\ 
\hline 
 
\tikz  \node [circle split,draw,blue,fill=green!20] {haut  \nodepart{lower} bas }; % \filldraw[fill=red] (0,0) circle (3pt);

&  
\tikz  \node [circle solidus,draw,blue,fill=green!20]{haut  \nodepart{lower} bas };
&  
\tikz  \node [ellipse split,draw,blue,fill=green!20]{texte haut  \nodepart{lower} texte bas };
& 
\tikz  \node [rectangle split,draw,blue,fill=green!20]{haut  \nodepart{lower} bas}; 
%\tikz  \node [rectangle split ,draw,fill=green!20]{a\nodepart{two}b\nodepart{three}c\nodepart{four}d\nodepart{five}e};
\\ \hline 
\RDD{circle split} & \RDD{circle solidus} & \RDD{ellipse split} & \RDD{rectangle split} \\ 
\hline 
\end{tabular} 

 \bigskip
 
 \begin{tabular}{|c|c|}  \hline  
 \begin{tikzpicture} [baseline=0pt]%[every text node part/.style={text centered}]
 \node[rectangle split,rectangle split parts=5,draw,blue,fill=green!20] at(0,0)
 {texte 1
 \nodepart{second}
 texte 2
 \nodepart{four}
 texte 3};
 \end{tikzpicture}
&
\parbox[c]{10cm}{
 \BS{node}[rectangle split,\RDD{rectangle split parts}=5,\\
 draw] \\
 \AC{texte 1 \\
 \BSS{nodepart}\AC{second} texte 2 \\
 \BSS{nodepart}\AC{four} texte 3}; \\
 \\
\dft : rectangle split parts=4 }
 \\  \hline 
 \end{tabular} 
 
\bigskip

\begin{tabular}{|c|}\hline  
\BS{node} [rectangle split,rectangle split parts=3,\RDD{rectangle split horizontal},draw,blue] \\
\AC{texte1\BSS{nodepart}\AC{two}texte2\BSS{nodepart}\AC{three}texte3};
\\ \hline  
\tikz \node [rectangle split,rectangle split parts=3, rectangle split horizontal,draw,blue]
{texte 1\nodepart{two}texte 2\nodepart{three}texte 3}; 
\\ \hline 
\end{tabular} 
 
 \bigskip
 
% % % <<<<<<<<<<<<<<<<< A Voir rectangle split allocate boxes= >>>>>>>>>>>>>>>>>>>>>>>>>>>>>>>>

% \begin{tikzpicture} [baseline=0pt]%[every text node part/.style={text centered}]
% \node[rectangle split,draw,rectangle split parts=5,fill=green!20,rectangle split allocate boxes=3] at(0,0)
% {texte 1  \nodepart{second}  texte 2  \nodepart{four}  texte 3};
% \end{tikzpicture}
% 
 
\bigskip
 \begin{tabular}{|c|c|}  \hline  
\begin{tikzpicture}[baseline=0pt] %[every text node part/.style={align=center}]
\node[rectangle split, rectangle split parts=3, draw,blue, text width=2.75cm]
{texte 1
\nodepart{two}
texte 2a \\
texte 2b \\
texte 2c
\nodepart{three}
texte 3a \\
texte 3b};
\end{tikzpicture}
&
\parbox{8cm}{
 \BS{node}[rectangle split,\RDD{rectangle split parts}=5, draw] \\
 \AC{texte 1 \\
 \BSS{nodepart}\AC{second} texte 2a  \BS{}\BS{}texte 2b  \BS{}\BS{}  texte 2c \\
 \BSS{nodepart}\AC{three} texte 3a \BS{}\BS{} texte 3b }; \\
}
 \\  \hline 
 \end{tabular} 
\bigskip
%---------------------------------------------------------------------------------

 \begin{tabular}{|c|c|}  \hline  
 \multicolumn{2}{|c|}{  \BS{node}[rectangle split, draw,blue,minimum size = 2cm,\RDD{rectangle split draw splits}= true] } \\
  \multicolumn{2}{|c|}{ 
  \AC{texte 1 \BS{nodepart}\AC{two} texte 2 \BS{nodepart}\AC{three} texte 3 \BS{nodepart}\AC{four} texte 4};   }\\ 
 \hline 
\tikz \node[rectangle split, draw,blue,minimum size = 2cm,rectangle split draw splits= true] {texte 1 \nodepart{two} texte 2 \nodepart{three} texte 3 \nodepart{four} texte 4};
&
\tikz \node[rectangle split, draw,blue,minimum size = 2cm,rectangle split draw splits= false] {texte 1 \nodepart{two} texte 2 \nodepart{three} texte 3 \nodepart{four} texte 4};
 \\ \hline
 \RDD{rectangle split draw splits}= true & \RDD{rectangle split draw splits}= false \\
 \dft &
 \\ \hline 
 \end{tabular}
 
\bigskip

 \begin{tabular}{|c|c|}  \hline  
\multicolumn{2}{|c|}{  
\BS{node} [rectangle split,rectangle split parts=3,draw,\RDD{rectangle split ignore empty parts}=false] }\\
 \multicolumn{2}{|c|}{ \AC{texte 1 \BS{nodepart}\AC{second} \BS{nodepart}\AC{third}texte 3};} 
\\ \hline  
\begin{tikzpicture} 
\node[rectangle split,rectangle split parts=3,draw,blue,rectangle split ignore empty parts=false] {texte 1 \nodepart{second} \nodepart{third}texte 3};
\end{tikzpicture}
&
\begin{tikzpicture}
\node[rectangle split,rectangle split parts=3,draw,blue,rectangle split ignore empty parts] 
{texte 1 \nodepart{second} \nodepart{third}texte 3};
\end{tikzpicture}
 \\  \hline 
\RDD{rectangle split ignore empty parts}=false & \RDD{rectangle split ignore empty parts}=true 
\\ \hline
 \end{tabular}
 
\bigskip

 \begin{tabular}{|c|c|}  \hline  
\multicolumn{2}{|c|}{  
\BS{node} [rectangle split,rectangle split parts=3,draw,\RDD{rectangle split empty part depth}=1cm] }\\
 \multicolumn{2}{|c|}{ \AC{texte 1 \BS{nodepart}\AC{second} \BS{nodepart}\AC{third}texte 3};} 
\\ \hline 
\begin{tikzpicture} 
\node[rectangle split,rectangle split parts=3,draw,blue,rectangle split empty part depth=1cm] {texte 1 \nodepart{second} \nodepart{third}texte 3};
\end{tikzpicture}
&
\begin{tikzpicture} 
\node[rectangle split,rectangle split parts=3,draw,blue,text depth=1cm] {texte 1 \nodepart{second} \nodepart{third}texte 3};
\end{tikzpicture}
\\ \hline 
\RDD{rectangle split empty part depth}=1cm & \RDD{text depth}=1cm
\\ \hline
\dft : 0ex & \dft : 0ex
\\ \hline 
\begin{tikzpicture}
\node[rectangle split,rectangle split parts=3,draw,blue,rectangle split empty part  height=1cm] 
{texte 1 \nodepart{second} \nodepart{third}texte 3};
\end{tikzpicture}
&
\begin{tikzpicture}
\node[rectangle split,rectangle split parts=3,draw,blue,text height=1cm] 
{texte 1 \nodepart{second} \nodepart{third}texte 3};
\end{tikzpicture}
\\  \hline 
\RDD{rectangle split empty part height}=1cm & \RDD{text height}=1cm
\\ \hline
\dft : 1ex & \dft : 1ex
\\ \hline 
 \end{tabular}
 
\bigskip



 \begin{tabular}{|c|c|}  \hline 
 \multicolumn{2}{|c|}{ 
 \BS{node} [rectangle split,rectangle split parts=3,draw,\RDD{rectangle split empty part width}=1cm]   \AC{};  } 
 \\ \hline 
\begin{tikzpicture} 
\node[rectangle split,rectangle split parts=3,draw,blue,rectangle split empty part width=2cm]{}; % {texte 1 \nodepart{second} \nodepart{third}texte 3};
\end{tikzpicture}
%\rule{6cm}{0pt}
&
\begin{tikzpicture} 
\node[rectangle split,rectangle split parts=3,draw,blue]{}; % {texte 1 \nodepart{second} \nodepart{third}texte 3};
\end{tikzpicture}
\\  \hline 
 \RDD{rectangle split empty part width}=2cm  &  \dft : 1ex
\\ \hline
 \end{tabular} 
 
 \bigskip



% % % % <<<<<<<<<< A voir   /pgf/rectangle split use custom fill= (default true) <<<<<<<<<<<<<<<<<<<<<<<<<<<<
 
 

%--------------------------------------------------------------------------------------

 \begin{tabular}{|c|c|}  \hline 
 \tikz[baseline=0pt] \node[rectangle split, draw,blue,minimum size = 2cm,rectangle split part align={center, left,right}] {texte 1 \nodepart{two} texte 2 \nodepart{three} texte 3 \nodepart{four} texte 4};
&
\parbox{8cm}{
\BS{node}[rectangle split, draw,blue,minimum size = 2cm,\\
\RDD{rectangle split part align}=\AC{center, left,right}]\\
 \AC{texte 1 \BS{nodepart}\AC{two} texte 2  \\
 \BS{nodepart}\AC{three} texte 3  \BS{nodepart}\AC{four} texte 4};
}
\\ \hline
 \tikz[baseline=0pt] \node[rectangle split, draw,blue,minimum size = 2cm, rectangle split horizontal,rectangle split part align={center,base, top,bottom}] {texte 1 \nodepart{two} texte 2 \nodepart{three} texte 3 \nodepart{four} texte 4};
 &
 \parbox{8cm}{
 \BS{node}[rectangle split, draw,blue,minimum size = 2cm,\\
 rectangle split horizontal,\\
 \RDD{rectangle split part align}=\AC{center,base, top,bottom}]\\
  \AC{texte 1 \BS{nodepart}\AC{two} texte 2  \\
  \BS{nodepart}\AC{three} texte 3  \BS{nodepart}\AC{four} texte 4};
 }
 \\ \hline
 \end{tabular}
 
\bigskip
%--------------------------------------------------------------------

 \begin{tabular}{|c|c|}  \hline  
\tikz[baseline=0pt] \node[rectangle split, draw,blue, minimum width=1cm,rectangle split part fill={red, green,cyan}]{};
&
\parbox{12cm}{
\BS{node}[rectangle split, draw,blue, minimum width=1cm,\\
 \RDD{rectangle split part fill}=\AC{red, green,cyan}]\AC{};}
\\ \hline
\end{tabular} 

%--------------------------------------------
\newpage
%\subsection{Mise en forme du texte}
\SbSSCT{Mise en forme du texte}{Text attributes}

\subsubsection{Position}

\begin{center}
\RRR{17-4-3}
\end{center}

\begin{tabular}{|c|c|c|c|} \hline  
\multicolumn{4}{|l|}{ \BS{tikz} \BS{draw} (0,0) node[fill=blue!10,text width=2cm,\RDD{text justified}]   }\\ 

\multicolumn{4}{|l|}{ \AC{Ceci est une démonstration d'un texte  sur une largeur de 2cm};  }\\ 
\hline 
\tikz \draw (0,0) node[fill=blue!10,text width=2cm]
{Ceci est une démonstration d'un texte  sur une largeur de 2cm.};
&  
\tikz \draw (0,0) node[fill=blue!10,text width=2cm,text justified]
{Ceci est une démonstration d'un texte  sur une largeur de 2cm};
&  
\tikz \draw (0,0) node[fill=blue!10,text width=2cm,text centered]
{Ceci est une démonstration d'un texte  sur une largeur de 2cm .};
&  
\tikz \draw (0,0) node[fill=blue!10,text width=2cm,text ragged]
{Ceci est une démonstration d'un texte  sur une largeur de 2cm .};
\\  \hline  
\TFRGB{sans}{without} option & text justified & text centered & text ragged   
\\ \hline  
\tikz \draw (0,0) node[fill=blue!10,text width=2cm,text badly ragged]
{Ceci est une démonstration d'un texte  sur une largeur de 2cm.};
&  
\tikz \draw (0,0) node[fill=blue!10,text width=2cm,text badly centered]
{Ceci est une démonstration d'un texte  sur une largeur de 2cm .};
&
\tikz \draw (0,0) node[fill=blue!10,text width=2cm,align=center]
{Ceci est une démonstration d'un texte  sur une largeur de 2cm .};
&
\tikz \draw (0,0) node[fill=blue!10,text width=2cm,align=flush center]
{Ceci est une démonstration d'un texte  sur une largeur de 2cm .};
\\  \hline 
text badly ragged &  text badly centered &  align=center & align=flush center 
\\  \hline 
\tikz \draw (0,0) node[fill=blue!10,text width=2cm,align=justify]
{Ceci est une démonstration d'un texte  sur une largeur de 2cm .};
&
\tikz \draw (0,0) node[fill=blue!10,text width=2cm,align=flush right]
{Ceci est une démonstration d'un texte  sur une largeur de 2cm .};
&
\tikz \draw (0,0) node[fill=blue!10,text width=2cm,align=right]
{Ceci est une démonstration d'un texte  sur une largeur de 2cm .};
&
\tikz \draw (0,0) node[fill=blue!10,text width=2cm,align=flush left]
{Ceci est une démonstration d'un texte  sur une largeur de 2cm .};
\\ \hline 
 align=justify & align=flush right &  align=right & align=flush left
\\ \hline 

\end{tabular} 
\bigskip

%--------------------------------------------------------------
%\subsubsection{Couleur et fontes } 
\SbSbSSCT{Couleur et fontes }{Colors and Fonts}

\begin{tabular}{|c|c|c|c|c|c|} \hline  
\tikz \draw (0,0) node[text= red]{Texte.};
&
\tikz \draw (0,0) node[font=\itshape]{Texte.};
&
\tikz \draw (0,0) node[font=\slshape]{Texte.};
&
\tikz \draw (0,0) node[font=\scshape]{Texte.};
&
\tikz \draw (0,0) node[font=\upshape]{Texte.};
&
\tikz \draw (0,0) node[font=\bfseries]{Texte.};
\\ \hline 



[text= red] & [font=\BS{itshape}]  & [font=\BS{slshape}] & [font=\BS{scshape}] & [font=\BS{upshape}] & [font=\BS{bfseries}]
\\ \hline 
\end{tabular} 



\bigskip

%\subsubsection{Taille des fontes} 
\SbSbSSCT{Taille des fontes}{Font Sizes}

\begin{tabular}{|c|c|c|c|c|c|c|}\hline
\multicolumn{7}{|c|}{ \BS{tikz} \BS{draw} (0,0) node[\RDD{font}=\BS{tiny}]\AC{Texte.}   }
\\  \hline
\tikz \draw (0,0) node[font=\tiny]{Texte.};
&
\tikz \draw (0,0) node[font=\footnotesize]{Texte.};
&
\tikz \draw (0,0) node[font=\small]{Texte.};
&
\tikz \draw (0,0) node[font=\large]{Texte.};
&
\tikz \draw (0,0) node[font=\Large]{Texte.};
&
\tikz \draw (0,0) node[font=\huge]{Texte.};
&
\tikz \draw (0,0) node[font=\Huge]{Texte.};
\\ \hline \BS{tiny} & \BS{footnotesize}  & \BS{small} & \BS{large} & \BS{Large} & \BS{huge} & \BS{Huge} \\ 
\hline 
\end{tabular} 

\bigskip
\begin{center}
\RRR{17-4-4}
\end{center}

\begin{tabular}{|c|c|} \hline  
\tikz \draw (0,0) node[fill=blue!10,text height=1cm,draw]{Texte.};
&  
\tikz \draw (0,0) node[fill=blue!10,text depth=1cm,draw]{Texte.};
\\ \hline  
\RDD{text height}=1cm
&  
\RDD{text depth}=1cm
\\ \hline 
\end{tabular} 

%\subsection{Positions prédéfinies  sur un n\oe ud}
\SbSSCT{Positions prédéfinies  sur un n\oe ud}{Positions on a node}
\label{nomnoeud}

%\subsubsection{pour l'ensemble des n\oe uds}
\SbSbSSCT{pour l'ensemble des n\oe uds}{For all types of node}
\begin{center}
\RRR{17-5-1}
\end{center}

\begin{tabular}{|c|c|c|c|} \hline  
\begin{tikzpicture}
\node[rectangle,draw,minimum size=3cm] (A) at (1,1) {\Huge texte};
\fill[red] (node cs:name=A,anchor=north west) circle (3pt);
\end{tikzpicture}
&
\begin{tikzpicture}
\node[rectangle,draw,minimum size=3cm] (A) at (1,1) {\Huge texte};
\fill[red] (node cs:name=A,anchor=north) circle (3pt);
\end{tikzpicture}
&
\begin{tikzpicture}
\node[rectangle,draw,minimum size=3cm] (A) at (1,1) {\Huge texte};
\fill[red] (node cs:name=A,anchor=north east) circle (3pt);
\end{tikzpicture}
&
\begin{tikzpicture}
\node[rectangle,draw,minimum size=3cm] (A) at (1,1) {\Huge texte};
\fill[red] (node cs:name=A,anchor=text) circle (3pt);
\end{tikzpicture}
\\ \hline 
north west & north & north east & text
\\ \hline 
%---------------------------------------------------------------
\begin{tikzpicture}
\node[rectangle,draw,minimum size=3cm] (A) at (1,1) {\Huge texte};
\fill[red] (node cs:name=A,anchor= west) circle (3pt);
\end{tikzpicture}
&
\begin{tikzpicture}
\node[rectangle,draw,minimum size=3cm] (A) at (1,1) {\Huge texte};
\fill[red] (node cs:name=A,anchor=mid  west) circle (3pt);
\end{tikzpicture}
&
\begin{tikzpicture}
\node[rectangle,draw,minimum size=3cm] (A) at (1,1) {\Huge texte};
\fill[red] (node cs:name=A,anchor= base west) circle (3pt);
\end{tikzpicture}
&
\begin{tikzpicture}
\node[rectangle,draw,minimum size=3cm] (A) at (1,1) {\Huge texte};
\fill[red] (node cs:name=A,anchor= base) circle (3pt);
\end{tikzpicture}
\\ \hline 
west & mid west & base west &  base
\\ \hline
%------------------------------------------------------------ 
\begin{tikzpicture}
\node[rectangle,draw,minimum size=3cm] (A) at (1,1) {\Huge texte};
\fill[red] (node cs:name=A,anchor=east) circle (3pt);
\end{tikzpicture}
&
\begin{tikzpicture}
\node[rectangle,draw,minimum size=3cm] (A) at (1,1) {\Huge texte};
\fill[red] (node cs:name=A,anchor=mid east) circle (3pt);
\end{tikzpicture}
&
\begin{tikzpicture}
\node[rectangle,draw,minimum size=3cm] (A) at (1,1) {\Huge texte};
\fill[red] (node cs:name=A,anchor=base east) circle (3pt);
\end{tikzpicture}
&
\begin{tikzpicture}
\node[rectangle,draw,minimum size=3cm] (A) at (1,1) {\Huge texte};
\fill[red] (node cs:name=A,anchor= mid) circle (3pt);
\end{tikzpicture}
\\ \hline 
east & mid esat & base east & mid
\\ \hline 
%--------------------------------------
\begin{tikzpicture}
\node[rectangle,draw,minimum size=3cm] (A) at (1,1) {\Huge texte};
\fill[red] (node cs:name=A,anchor= south east) circle (3pt);
\end{tikzpicture}
&
\begin{tikzpicture}
\node[rectangle,draw,minimum size=3cm] (A) at (1,1) {\Huge texte};
\fill[red] (node cs:name=A,anchor= south) circle (3pt);
\end{tikzpicture}
&
\begin{tikzpicture}                                       
\node[rectangle,draw,minimum size=3cm] (A) at (1,1) {\Huge texte};
\fill[red] (node cs:name=A,anchor= south west) circle (3pt);
\end{tikzpicture}
&
\begin{tikzpicture}
\node[rectangle,draw,minimum size=3cm] (A) at (1,1) {\Huge texte};
\fill[red] (node cs:name=A,anchor=center ) circle (3pt);
\end{tikzpicture}
\\ \hline 
south east & south & south west & center
\\ \hline
%------------------------------------------------------------------------- 
\begin{tikzpicture}
\node[rectangle,draw,minimum size=3cm] (A) at (1,1) {\Huge texte};
\fill[red] (node cs:name=A,anchor=0) circle (3pt);
\end{tikzpicture}
&
\begin{tikzpicture}
\node[rectangle,draw,minimum size=3cm] (A) at (1,1) {\Huge texte};
\fill[red] (node cs:name=A,anchor=120) circle (3pt);
\end{tikzpicture}
&
\begin{tikzpicture}
\node[rectangle,draw,minimum size=3cm] (A) at (1,1) {\Huge texte};
\fill[red] (node cs:name=A,anchor=-60) circle (3pt);
\end{tikzpicture}
&
%\begin{tikzpicture}
%\node[rectangle,draw,minimum size=3cm] (A) at (1,1) {\Huge texte};
%\fill[red] (node cs:name=A,anchor=text) circle (3pt);
%\end{tikzpicture}

\\ \hline 
0 & 120 & -60 & %text  
\\ \hline 
\end{tabular}
 
%\subsubsection{spécifique à un n\oe ud}
\SbSbSSCT{spécifique à un n\oe ud}{Specific to a node}

\TFRGB{Dans une prochaine version !}{In a future version}







 


%============\newpage

\section{Decorations}

 \input{tkzdeco}
% 
% ======================================================================
\newpage

%\section{Insertion images dans un environnement TikZ}
\SSCT{Insertion images dans un environnement TikZ}{Pictures in a TikZ picture}

\input{tkzimage}


%%
%%>>>> \section[Mettre des objets en cadre]{Mettre des objets en cadre }
%%
%
%%
%%\newpage
%%>>>>> \section[Mettre des objets en bouton]{Mettre des objets en bouton }
%
%
%%%%%%=============================================================
%

%\section{Des lignes et liaisons spéciales}
%\subsection[Trait à main levé]{Trait à main levée }
\SSCT{Trait à main levée }{Freehand drawing}

\input{tkzalea}

%% >>>> \subsection{Tracer avec des symboles}
%
%%
%%\newpage

%%>>>>> \subsection[Les bobines]{Les bobines \cite{pst-user} \cite{pst-coil}}
%%

%%\newpage
%%
%%>>>> \subsection[Les accolades]{Les accolades }
%%
%
%%%%%======================================================================
%%\section{Des remplissages spéciaux}
%%\subsection{Des gradients de couleurs}
%%
%%\subsubsection[Module pst-grad]{Module pst-grad \cite{pst-user} \cite{pst-grad}}
%
%%%
%%\newpage
%%\subsubsection[Module pst-slpe]{Module pst-slpe  \cite{pst-slpe}}
%%
%
%%
%%\newpage
%%\subsection[Remplissage par des motifs]{Remplissage par des motifs \cite{pst-fill}}
%%
%
%%
%%\subsection[Remplissage par des points aléatoires]{Remplissage par des points aléatoires \cite{pst-add}}
%
%%\newpage
%%
%%% ========================================================================
%%\section[Effets spéciaux avec du texte ]{Effets spéciaux avec du texte  \cite{pst-user}  \cite{pst-text}}
%
%
%\newpage
%%% % % % %======================================================================
%\section[Créer un graphe]{Créer un graphe }
\SSCT{Créer un graphe }{Creating Graphs}


%\subsection{Graphe avec Tikz}
\SbSSCT{Graphe avec TikZ}{Graph with TikZ}
%\subsubsection{Graphe à partir d'une liste de points}
\SbSbSSCT{Graphe à partir d'une liste de points}{From a list of points}
\label{plot}

\begin{tabular}{|c | } \hline
\BS{tikz} \BS{draw} plot \RDD{coordinates} \AC{(0,0) (1,1) (2,0) (3,1) (4,1) (5,2)}; \\ 
\hline
\tikz \draw plot coordinates {(0,0) (1,1) (2,0) (3,1) (4,1) (5,2)};
\\ \hline
\end{tabular}

%\subsubsection{Graphe à partir partir d'un fichier de données}
\SbSbSSCT{Graphe à partir partir d'un fichier de données}{From a data file}

\begin{tabular}{|c | c | c | c |} \hline
\multicolumn{4}{|c|}{ \BS{tikz} \BS{draw}  plot[mark=x] \RDD{file} \AC{table.dat} ;   }\\ 
\hline
%\draw plot[mark=x] file {table.dat};
& 
\tikz \draw plot[mark=x,smooth] file {table.dat};
&
\tikz \draw plot[mark=x,smooth,tension=.2] file {table.dat};
&
\tikz \draw plot[mark=x,smooth,tension=1] file {table.dat};
\\ \hline
[mark=x] & [mark=x,\RDD{smooth}] & [mark=x,smooth,\RDD{tension}=.2] & [mark=x,smooth,\RDD{tension}=1]
\\ \hline
\multicolumn{4}{|c|}{ \dft : tension= 0:55}
\\ \hline
\end{tabular}

\bigskip


\begin{tabular}{|c  c |} \hline
\multicolumn{2}{|c|}{\TFRGB{Contenu du fichier}{content of the file} table.dat}
\\ \hline
0.0 & 0.3 \\
0.3 & 0.6 \\
0.6 & 0.9 \\
0.9 & 1.5  \\
1.2 & 0.6  \\
1.5 & 1.2  \\
1.8 & 1.5  \\
2.1 & 2.0 \\
2.4 & 3.0 \\
\hline
\end{tabular}

\bigskip

%\subsubsection{Les types de graphes}
\SbSbSSCT{Les types de graphes}{Graph types}

\begin{tabular}{|c | c | c | c |} \hline
\multicolumn{4}{|c|}{ \BS{tikz} \BS{draw}  plot[mark=*,\RDD{const plot}] file \AC{table.dat} ;   }\\ 
\hline
\tikz \draw plot[mark=*,const plot] file {table.dat};
&

\tikz \draw plot[const plot mark left,mark=*] file {table.dat};
&
\tikz \draw plot[const plot mark right,mark=*] file {table.dat};
&
\tikz \draw plot[jump mark left, mark=*] file {table.dat};
\\ \hline
\RDD{const plot} & \RDD{const plot mark left} & \RDD{const plot mark right} & \RDD{jump mark left}
\\ \hline
\tikz \draw plot[jump mark right, mark=*] file {table.dat};
&
\tikz \draw plot[ycomb,thin,mark=*] file {table.dat};
&
\tikz \draw plot[xcomb,mark=*] file {table.dat};
&
\tikz \draw plot[only marks,mark=*] file {table.dat};
\\ \hline
\RDD{jump mark right} & \RDD{ycomb} & \RDD{xcomb} & \RDD{only marks}
\\ \hline
\end{tabular}

\bigskip
\begin{tabular}{|c | c | c |c |} \hline
%\begin{tikzpicture}
%\draw[help lines] (-2,-3) grid (2,2);
\tikz  \draw plot[polar comb,mark=*]coordinates {(0:1) (60:0.5) (120:1.5) (180:3) (240:.5) (300:1) (0:1)};
%\draw[line width=1pt,color=red] plot coordinates (0:1cm)(60:0.5)(120:1.5)(180:1)(240:3)(300:1)(0:1cm);
%\end{tikzpicture}
\\ \hline
\BS{tikz}  \BS{draw} plot[\RDD{polar comb},mark=*]coordinates \\
\AC{(0:1) (60:0.5) (120:1.5) (180:3) (240:.5) (300:1) (0:1)};
\\ \hline
\end{tabular}

\bigskip

\begin{tabular}{|c | c | c |c |} \hline
\multicolumn{4}{|c|}{ \BS{tikz} \BS{draw}  plot[\RDD{ybar}] file \AC{table.dat} ;   }\\ 
\hline
\tikz \draw plot[ybar] file {table.dat};
&
\tikz \draw plot[ybar interval] file {table.dat};
&
\tikz \draw plot[ybar interval,x=2cm] file {table.dat};
&
\tikz \draw plot[ybar interval,y=.5cm] file {table.dat};
\\ \hline
[\RDD{ybar}] & [\RDD{ybar interval}] & [ybar interval,\RDD{x}=2cm] & [ybar interval,\RDD{y}=.5cm]
\\ \hline
\end{tabular}

\bigskip
 \begin{tabular}{|c|c|}  \hline 
\begin{tikzpicture}[baseline=0pt]
\draw[red,fill=cyan,ybar,bar width=.5cm]plot coordinates{(0,1) (1,1.2) (2,.6) (3,.7) (4,.9)};
\draw[blue,fill=green,ybar,bar width=.5cm,bar shift=.3cm]plot coordinates{(0,1.2) (1,1.3) (2,.5) (3,.2) (4,.5)};
\end{tikzpicture}
&
\parbox[c]{10cm}{
\BS{begin}\AC{tikzpicture} \\
\BS{draw}[red,fill=cyan,ybar,bar width=.5cm] \\
\rule{1cm}{.0pt} plot coordinates \AC{(0,1) (1,1.2) (2,.6) (3,.7) (4,.9)}; \\
\BS{draw}[blue,fill=green,ybar,bar width=.5cm,\RDD{bar shift}=.3cm] \\
\rule{1cm}{.0pt} plot coordinates \AC{(0,1.2) (1,1.3) (2,.5) (3,.2) (4,.5)}; \\
\BS{end}\AC{tikzpicture} }
 \\  \hline 
 \end{tabular} 

\bigskip

\begin{tabular}{|c | c | c | c |c |} \hline
\multicolumn{4}{|c|}{ \BS{tikz} \BS{draw}  plot[xbar interval] file \AC{table.dat} ;   }\\ 
\hline
\tikz \draw[blue] plot[xbar] coordinates{(2,0) (3,1) (1,2) (2,3)};
&
\tikz \draw[blue] plot[xbar interval]  coordinates {(2,0) (3,1) (1,2) (2,3)};
&
\tikz \draw[blue] plot[xbar interval,x=.5cm]  coordinates {(2,0) (3,1) (1,2) (2,3)};
&
\tikz \draw[blue] plot[xbar interval,y=.5cm]  coordinates {(2,0) (3,1) (1,2) (2,3)};
%&
%\tikz \draw[blue!20] plot[xbar interval,x=.5cm,y=.5cm]  coordinates {(2,0) (3,1) (1,2) (2,3)};
\\ \hline
[\RDD{xbar}] & [\RDD{xbar interval}] & [xbar interval,\RDD{x}=.5cm] & [xbar interval,\RDD{y}=.5cm] 
\\ \hline
\end{tabular}

\newpage
%--------------------------------------------------------------
%\subsubsection{Graphe à partir d'une fonction}
\SbSbSSCT{Graphe à partir d'une fonction}{Graph of a function}


\begin{tabular}{|c | c | c | } \hline
\multicolumn{3}{|c|}{  \BS{draw}  [color=red] plot (\BS{x},\BS{x});   }\\ 
\hline
\begin{tikzpicture}[domain=0:4,ultra thick]
%\draw[very thin,color=gray] (-0.1,-1.1) grid (4.1,4.1);
\draw[->,blue,ultra thick] (-.1,0) -- (4.5,0);
\draw[->,blue,ultra thick] (0,-1.1) -- (0,04);
\draw[color=red] plot (\x,\x);
\end{tikzpicture} 
&
\begin{tikzpicture}[domain=0:6.28,ultra thick,x=0.7cm]
%\draw[very thin,color=gray] (-0.1,-2.1) grid (4.1,2.1);
\draw[->,blue,ultra thick] (-.1,0) -- (7,0);
\draw[->,blue,ultra thick] (0,-2.5) -- (0,2.5);
\draw[color=red] plot  (\x,{sin(\x r)});
\end{tikzpicture} 
&
\begin{tikzpicture}[domain=0:360,x=0.3,ultra thick]
%\draw[very thin,color=gray] (-0.1,-2.1) grid (4.1,2.1);
\draw[->,blue,ultra thick] (-.1,0) -- (370,0);
\draw[->,blue,ultra thick] (0,-2.5) -- (0,2.5);
\draw[color=red] plot (\x,{sin(\x)});
\end{tikzpicture} 
\\ \hline
(\BS{x},\BS{x}) &  (\BS{x},\AC{sin(\BS{x} r)}) & (\BS{x},\AC{sin(\BS{x})}) \\
& x en radian & x en degré
\\ \hline
\end{tabular}

Options 

\begin{tabular}{|c | c |} \hline
\multicolumn{2}{|l|}{ \BS{draw}[color=red,dashed] plot(\BS{x},\AC{sin(\BS{x} r)});}  \\
\multicolumn{2}{|l|}{ \BS{draw}[color=blue,\RDD{samples}=5,mark=*,ultra thick] plot(\BS{x},\AC{sin(\BS{x} r)});} 
\\ \hline
\begin{tikzpicture}[domain=0:6.28]
\draw[very thin,color=gray] (-0.1,-1.1) grid (6.28,1.1);
\draw[color=red,dashed] plot  (\x,{sin(\x r)});
\draw[color=blue,samples=5,mark=*,ultra thick] plot  (\x,{sin(\x r)});
\end{tikzpicture} 
&
\begin{tikzpicture}
\draw[very thin,color=gray] (-0.1,-1.1) grid (6.28,1.1);
\draw[color=red,dashed,domain=0:6.28] plot  (\x,{sin(\x r)});
\draw[color=blue,domain=0:4,ultra thick] plot  (\x,{sin(\x r)});
\end{tikzpicture} 
  \\ \hline
[color=blue,\RDD{samples}=5,mark=*] & [color=blue,\RDD{domain}=0:4]
\\ \hline
\begin{tikzpicture}
\draw[very thin,color=gray] (-0.1,-1.1) grid (6.28,1.1);
\draw[color=red,dashed,domain=0:6.28] plot  (\x,{sin(\x r)});
\draw[color=blue,domain=1:5,ultra thick] plot  (\x,{sin(\x r)});
\end{tikzpicture} 
&
\begin{tikzpicture}[domain=0:6.28]
\draw[very thin,color=gray] (-0.1,-1.1) grid (6.28,1.1);
\draw[color=red,dashed] plot  (\x,{sin(\x r)});
\draw[color=blue,samples at={1,2,4,5},mark=*,ultra thick] plot  (\x,{sin(\x r)});
\end{tikzpicture} 
\\ \hline
[color=blue,\RDD{domain}=1:5] & [color=blue,\RDD{samples at}=\AC{1,2,4,5},mark=*]
\\ \hline
\end{tabular}


%-------------------------------------------------------------------------
%\subsubsection{Fonctions paramétriques}
\SbSbSSCT{Fonctions paramétriques}{Parametric function}


\begin{tabular}{|c | c |} \hline
\multicolumn{2}{|l|}{  \BS{draw}[domain=-3.141:3.141,smooth,variable=\BS{t}]plot (\AC{sin(\BS{t} r)},\AC{sin(2 *\BS{t} r)});} \\
\multicolumn{2}{|l|}{  \BS{draw}[domain=0:720,smooth,variable=\BS{t}]plot (\AC{sin(\BS{t})},\BS{t}/360,\AC{cos(\BS{t})});} 
\\ \hline

\tikz \draw[domain=-3.141:3.141,smooth,variable=\t,ultra thick]plot ({sin(\t r)},{sin(2*\t r)});
&
\tikz \draw[domain=0:720,smooth,variable=\t,ultra thick] plot ({sin(\t)},\t/360,{cos(\t)});
\\ \hline
(\AC{sin(\BS{t} r)},\AC{sin(2 *\BS{t} r)}) & (\AC{sin(\BS{t})},\BS{t}/360,\AC{cos(\BS{t})})
\\ \hline
\end{tabular} 
%\tikz \draw plot[mark=x,mark repeat=3,smooth] file {plots/pgfmanual-sine.table};
 

%\subsection{Marques}
\SbSSCT{Marques}{Marks}

%\subsubsection{Marques avec Tikz}
\SbSbSSCT{Marques avec TikZ}{Marks with TikZ}

\begin{tabular}{|c | c | c | c |} \hline
\tikz \draw plot[mark=+,mark size=5pt] coordinates {(0,0) (1,1) (2,0)};
&
\tikz \draw plot[mark=x,mark size=5pt] coordinates {(0,0) (1,1) (2,0) };
&
\tikz \draw plot[mark=*,mark size=5pt] coordinates {(0,0) (1,1) (2,0)};
&
\tikz \draw plot[mark=ball,mark size=5pt] coordinates {(0,0) (1,1) (2,0)};
\\ \hline
mark=+ & mark=x & mark=* & mark=ball
\\ \hline
\end{tabular}

\bigskip

\begin{tabular}{|c | c |} \hline
\begin{tikzpicture}[domain=0:6.28]
\draw[very thin,color=gray] (-0.1,-1.1) grid (6.28,1.1);
\draw[color=red,dashed,mark=+] plot  (\x,{sin(\x r)});
\draw[color=blue,mark repeat=3,mark=*] plot  (\x,{sin(\x r)});
\end{tikzpicture} 
&
\begin{tikzpicture}[domain=0:6.28]
\draw[very thin,color=gray] (-0.1,-1.1) grid (6.28,1.1);
\draw[color=red,dashed,mark=+] plot  (\x,{sin(\x r)});
\draw[color=blue,mark repeat=3,mark phase=5,mark=*] plot  (\x,{sin(\x r)});
\end{tikzpicture} 
\\ \hline
[color=blue,\RDD{mark repeat}=3,mark=*] & [color=blue,mark repeat=3,\RDD{mark phase}=5,mark=*]
\\ \hline
\begin{tikzpicture}[domain=0:6.28]
\draw[very thin,color=gray] (-0.1,-1.1) grid (6.28,1.1);
\draw[color=red,dashed,mark=+] plot  (\x,{sin(\x r)});
\draw[color=blue,mark indices={1,4,...,15,17,20},mark=*] plot  (\x,{sin(\x r)});
\end{tikzpicture} 
&
\begin{tikzpicture}[domain=0:6.28]
\draw[very thin,color=gray] (-0.1,-1.1) grid (6.28,1.1);
\draw[color=red,dashed,mark=+] plot  (\x,{sin(\x r)});
\draw[color=blue,mark size=5pt,mark=*] plot  (\x,{sin(\x r)});
\end{tikzpicture} 
\\ \hline
[color=blue,\RDD{mark indices}={1,4,...,15,17,20},mark=*] & [color=blue,\RDD{mark size}=5pt,mark=*]
\\ \hline
\begin{tikzpicture}[domain=0:6.28]
\draw[very thin,color=gray] (-0.1,-1.1) grid (6.28,1.1);
%\draw[color=red,dashed,mark=*] plot  (\x,{sin(\x r)});
\draw[color=blue,mark size=5pt,mark options={color=magenta},mark=+] plot  (\x,{sin(\x r)});
\end{tikzpicture}
&
\begin{tikzpicture}[domain=0:6.28]
\draw[very thin,color=gray] (-0.1,-1.1) grid (6.28,1.1);
%\draw[color=red,dashed,mark=*] plot  (\x,{sin(\x r)});
\draw[color=blue,mark size=5pt,mark options={rotate=10},mark=+] plot  (\x,{sin(\x r)});
\end{tikzpicture}
\\ \hline
\RDD{mark options}=\AC{color=magenta},mark=+ & \RDD{mark options}=\AC{rotate=10},mark=+
\\ \hline
\end{tabular}
 

%\subsubsection{Marques personnalisées avec text mark}
\SbSbSSCT{Marques personnalisées avec text mark}{Marks with text mark}

\begin{tabular}{|c | c | c |} \hline
\multicolumn{3}{|l|}{ \BS{draw}[\RDD{mark=text},\RDD{text mark}=A,mark size=5pt] coordinates \AC{(0,0) (1,1) (2,0)};} 
\\ \hline
\tikz \draw plot[mark=text,text mark=A,mark size=5pt] coordinates {(0,0) (1,1) (2,0)};
&
\tikz \draw plot[mark=text,text mark=Texte,mark size=5pt] coordinates {(0,0) (1,1) (2,0)};
&
\begin{tikzpicture}
\draw[white]  (-1,0)-- (-1,1.5);
 \draw plot[mark=text,text mark=\DFR,mark size=5pt] coordinates {(0,0) (1,1) (2,0)};
\end{tikzpicture} 
\\ \hline
\RDD{text mark}=A &  \RDD{text mark}=Texte & \RDD{text mark}=\BS{DFR} \pageref{DFR} 
\\ \hline 
\multicolumn{3}{|c|}{ 
\begin{tikzpicture}
\draw[white]  (-1,0)-- (-1,1.5);
\draw plot[mark=text,text mark={\includegraphics[width=.5cm]{tiger}} ,mark size=5pt] coordinates {(0,0) (1,1) (2,0)};  
\end{tikzpicture} }
\\ \hline  
\multicolumn{3}{|c|}{ \RDD{text mark}=\AC{\BS{includegraphics}[width=.5cm]\AC{tiger}} }
\\ \hline   
\end{tabular}


\newpage
%\subsubsection{Marques avec l'extension plotmarks }
\SbSbSSCT{Marques avec l'extension plotmarks }{Marks with plotmarks library}

\label{plotmarks}

%Insérer dans le préambule :

 \maboite{\BS{usetikzlibrary}\AC{plotmarks}}
 
\begin{center}
\RRR{63}
\end{center}

\begin{tabular}{|c | c | c | c |} \hline
\tikz \draw plot[mark=-,mark size=5pt] coordinates {(0,0) (1,1) (2,0)};
& 
\tikz \draw plot[mark=|,mark size=5pt] coordinates {(0,0) (1,1) (2,0)};
 &
\tikz \draw plot[mark=o,mark size=5pt] coordinates {(0,0) (1,1) (2,0)};
 &
\tikz \draw plot[mark=asterisk,mark size=5pt] coordinates {(0,0) (1,1) (2,0)};
\\ \hline 
mark=- & mark=| & mark=o &mark=asterisk
\\ \hline
\tikz \draw plot[mark=star,mark size=5pt] coordinates {(0,0) (1,1) (2,0)};
&
\tikz \draw plot[mark=10-pointed star,mark size=5pt] coordinates {(0,0) (1,1) (2,0)};
&
\tikz \draw plot[mark=oplus,mark size=5pt] coordinates {(0,0) (1,1) (2,0)};
&
\tikz \draw plot[mark=oplus*,mark size=5pt] coordinates {(0,0) (1,1) (2,0)};
\\ \hline
mark=star & mark=10-pointed star & mark=oplus & mark=oplus*
\\ \hline
 
\tikz \draw plot[mark=otimes,mark size=5pt] coordinates {(0,0) (1,1) (2,0)};
&
\tikz \draw plot[mark=otimes*,mark size=5pt] coordinates {(0,0) (1,1) (2,0)};
&
\tikz \draw plot[mark=square,mark size=5pt] coordinates {(0,0) (1,1) (2,0)};
&
\tikz \draw plot[mark=square*,mark size=5pt] coordinates {(0,0) (1,1) (2,0)};
\\ \hline
 mark=otimes & mark=otimes* & mark=square & mark=square*
  \\ \hline
  
\tikz \draw plot[mark=triangle,mark size=5pt] coordinates {(0,0) (1,1) (2,0)};
& 
\tikz \draw plot[mark=triangle*,mark size=5pt] coordinates {(0,0) (1,1) (2,0)};
& 
\tikz \draw plot[mark=diamond,mark size=5pt]  coordinates {(0,0) (1,1) (2,0)};
 &
\tikz \draw plot[mark=diamond*,mark size=5pt] coordinates {(0,0) (1,1) (2,0)};
\\ \hline 
mark=triangle & mark=triangle* & mark=diamond & mark=diamond*
\\ \hline 

\tikz \draw plot[mark=halfdiamond*,mark size=5pt] coordinates {(0,0) (1,1) (2,0)};
 &
\tikz \draw plot[mark=halfsquare*,mark size=5pt] coordinates {(0,0) (1,1) (2,0)};
 &
\tikz \draw plot[mark=halfsquare right*,mark size=5pt] coordinates {(0,0) (1,1) (2,0)};
 &
\tikz \draw plot[mark=halfsquare left*,mark size=5pt] coordinates {(0,0) (1,1) (2,0)};
\\ \hline 
mark=halfdiamond* & mark=halfsquare* & mark=halfsquare right* & mark=halfsquare left*
\\ \hline 

\tikz \draw plot[mark=pentagon,mark size=5pt] coordinates {(0,0) (1,1) (2,0)};
 &
\tikz \draw plot[mark=pentagon*,mark size=5pt] coordinates {(0,0) (1,1) (2,0)};
 &
\tikz \draw plot[mark=Mercedes star,mark size=5pt] coordinates {(0,0) (1,1) (2,0)};
 &
\tikz \draw plot[mark=Mercedes star flipped,mark size=5pt] coordinates {(0,0) (1,1) (2,0)};
 \\ \hline
 mark=pentagon & mark=pentagon* & mark=Mercedes star & mark=Mercedes star flipped
 \\ \hline 
 
\tikz \draw plot[mark=halfcircle,mark size=5pt] coordinates {(0,0) (1,1) (2,0)};
 &
\tikz \draw plot[mark=halfcircle*,mark size=5pt] coordinates {(0,0) (1,1) (2,0)};
& 
\tikz \draw plot[mark=heart,mark size=5pt] coordinates {(0,0) (1,1) (2,0)};
 &
\tikz \draw plot[mark=text,mark size=5pt] coordinates {(0,0) (1,1) (2,0)};
 \\ \hline
 mark=halfcircle & mark=halfcircle* & mark=heart & mark=text
  \\ \hline
\end{tabular}

\bigskip

\begin{tabular}{|c | c | c | c |} \hline
\multicolumn{4}{|l|}{ \BS{draw}[mark=halfcircle,\RDD{mark color}=red,mark size=5pt] coordinates \AC{(0,0) (1,1) (2,0)};} 
\\ \hline
\tikz \draw plot[mark=halfcircle,mark color=red,mark size=5pt] coordinates {(0,0) (1,1) (2,0)};
&
\tikz \draw plot[mark=halfcircle*,mark color=red,mark size=5pt] coordinates {(0,0) (1,1) (2,0)};
&
\tikz \draw plot[mark=halfdiamond*,mark color=red,mark size=5pt] coordinates {(0,0) (1,1) (2,0)};
&
\tikz \draw plot[mark=halfsquare*,mark color=red,mark size=5pt] coordinates {(0,0) (1,1) (2,0)};
  \\ \hline
  mark=halfcircle & mark=halfcircle* & mark=halfdiamond* & mark=halfsquare*
   \\ \hline 
\end{tabular}



% \subsection{Graphes avec Gnuplot}
\SbSSCT{Graphes avec Gnuplot}{Graph with Gnuplot}
 
 \begin{tabular}{|l| } \hline
%\begin{tikzpicture}[domain=0:6.28]
%%\draw[very thin,color=gray] (-0.1,-1.1) grid (7.1,1.1);
%%\draw[->,ultra thick,blue] (-0.2,0) -- (7,0) node[right] {$x$};
%%\draw[->,ultra thick,blue] (0,-1.2) -- (0,1.2) node[above] {$f(x)$};
%%\draw[color=red] plot[id=x] function{x} node[right] {$f(x) =x$};
%\draw[color=red] plot[id=sin] function{sin(x)} ;
%%\draw[color=orange] plot[id=exp] function{0.05*exp(x)} node[right] {$f(x) = \frac{1}{20} \mathrm e^x$};
%\end{tikzpicture}
\BS{draw}[color=red] plot[\RDD{id}=sin] function\AC{sin(x)} ;
   \\ \hline
\\
==> plot[id=sin] \TFRGB{crée le fichier}{create the file} \og sin.gnuplot \fg \\
==>  \TFRGB{Ouvrir le fichier}{Open the file} \og sin.gnuplot \fg \TFRGB{avec le programme gnuplot pour créer le fichier}{with the program gnuplot : creation of the file }   \og sin.table \fg\\
==> \TFRGB{Utiliser le fichier de données} {Use the datafile }
 \og sin.table  \fg   \\ \hline 
\end{tabular}

\newpage

\SSCT{Créer un graphe avec pgfplot}{Creation of a graph with pgfplots}

\input{tkzgraph2} % <<<<<<<<<<<<<<<<<<<<<<<<<<<<<
%
%\subsection{Ccùourbes 3D}
\SSCT{Courbes 3D}{3D graph}

\input{tkzgraph3D} % très lourd à compiler

%
%%%
%%%\essais{pstgraph2ess.tex}
%%\newpage
%%\section[Créer un graphe d'après une équation]{Créer un graphe d'après une équation  \cite{pst-user} \cite{pst-plot}}
%%
%
%%%
%%%\essais{pstgraph3ess.tex} 
%%\newpage
%% \section[Des outils pour les graphes]{Des outils pour les graphes \cite{pst-add} }
%% 
%
%%
%%\newpage
%% \section[Créer un graphe en camembert]{Créer un graphe en camembert \cite{pst-add} }
%% 
%%\input{chart} % camembert

\newpage

%\section{Les Tableaux de variation}
\SSCT{Les Tableaux de variation}{Table of a function variation }

\input{tkztab}

\newpage
%%%=============================================
%\section{Les répétitions}
\SSCT{Les répétitions}{Repetitions}


\input{tkzrep1}  % OK

%%\subsection[Commande multido]{Commande multido \cite{pst-user} \cite{multido} }
%%
%
%%
%%\essais{pstrep2ess.tex}
%%
%%\subsection[Commande psforeach]{Commande psforeach \cite{pst-news10} }
%
%
%%
%%\newpage
%%% % % %======================================================================
%%\section[La géométrie]{La géométrie  \cite{pst-eucl} }
%%
%%Utilisation du module \textbf{pst-eucl} \label{pst-eucl}(consultez le fichier\textbf{ pst-eucl-doc.pdf} )
%%
%%
%%\psset{fillcolor=yellow,linecolor=blue,dotscale=2}
%%\subsection{\'Elements de base}
%%
%
%%
%%\subsection[Transformations géométriques]{Transformations géométriques \cite{pst-eucl} }
%%
%%
%
%%
%%
%%\subsection[Constructions particulières en géométrie ]{Constructions particulières en géométrie }
%%
%
%%
%%\subsection[Intersections]{Intersections  }
%%
%
%%
%%%--------------------------------------------------------------
%%\section[Les vecteurs]{Les vecteurs  }
%%
%
%%%==============================================================
\newpage
 
%\section[Les diagrammes arborescents ]{Les diagrammes arborescents }
\SSCT{Les diagrammes arborescents }{Tree diagram}


\input{tkztree}

%%%==============================================================
\newpage

\SSCT{Les schemas électriques }{Electrical Engineering Circuits}

\input{tkzelect}


\newpage

%\section[Les animations ]{Les animations }
\SSCT{Les animations }{Animate a TikZ picture}


\input{tkzanim}
%%
%%\newpage
%%
%%\section[Créer un dessin en 3D]{Créer un dessin en 3D  }
%

%%\subsection{Les objets en 3D}
%%

%%\newpage
%%\subsection[Créer un graphe en 3D]{Créer un graphe en 3D } 
%%

\newpage
%%=======================================================================
%\section{Les modules étudiés dans ce document}
\SSCT{Les modules étudiés dans ce document}{Packages studied in this document}


\TFRGB{module de base TikZ}{Basic TikZ package} : 

\maboite{\BS{usepackage}\AC{tikz} }

%\bigskip
\bigskip
\textbf{\TFRGB{Autres modules}{Other packages}}

%
\begin{tabular}{|c|c|l c|}\hline 
\TFRGB{nom}{name} 			& \TFRGB{voir page} 				& documentation\footnotemark[1] & \\  \hline 

animate 		& \pageref{anim} 			& animate.pdf 			& \DGB\\
tkz-tab  		& \pageref{tab} 			& tkz-tab-screen.pdf 	& \DFR \\
\hline 
\end{tabular} 
\bigskip

\textbf{\TFRGB{Compléments optionnels}{Optional library} :}

\begin{tabular}{|l|c|l|}\hline 
\TFRGB{nom}{name} 				& \TFRGB{voir page}{see page}						& \TFRGB{A insérer dans le préambule}{Load package}\\ \hline 
angles				& \pageref{lib-angles}			&  \BS{usetikzlibrary}\AC{angles}
\\
arrows.meta				& \pageref{lib-arrows.meta}			&  \BS{usetikzlibrary}\AC{arrows.meta}
\\
bending				& \pageref{lib-bending}			&  \BS{usetikzlibrary}\AC{bending}
\\
backgrounds			& \pageref{lib-bkgd} 			&  \BS{usetikzlibrary}\AC{backgrounds}
\\
calc				& \pageref{lib-calc}			&  \BS{usetikzlibrary}\AC{calc}
\\
circuits.ee.IEC				& \pageref{lib-ee}			&  \BS{usetikzlibrary}\AC{circuits.ee.IEC}
\\
 
fit & \pageref{lib-fit} 	& \BS{usetikzlibrary}\AC{fit} 
\\
decorations.footprints & \pageref{lib-footprints} 	& \BS{usetikzlibrary}\AC{decorations.footprints} 
\\
decorations.fractals & \pageref{lib-fractals} 		& \BS{usetikzlibrary}\AC{decorations.fractals} 
\\
decorations.markings & \pageref{lib-mark} 			& \BS{usetikzlibrary}\AC{decorations.markings} 
\\
decorations.pathmorphing  & \pageref{lib-morph}		& \BS{usetikzlibrary}\AC{decorations.pathmorphing}
\\
decorations.pathreplacing & \pageref{lib-replac}	& \BS{usetikzlibrary}\AC{decorations.pathreplacing} 
\\
decorations.shapes & \pageref{lib-shapes} 			& \BS{usetikzlibrary}\AC{decorations.shapes} 
\\
decorations.text & \pageref{lib-text} 				& \BS{usetikzlibrary}\AC{decorations.text} 
\\

fadings 			& \pageref{lib-fadings}			&  \BS{usetikzlibrary}\AC{fadings }
\\
intersections		& \pageref{lib-intersections}	&  \BS{usetikzlibrary}\AC{intersections}
\\
patterns			& \pageref{lib-patterns}		&  \BS{usetikzlibrary}\AC{patterns}
\\
plotmarks			& \pageref{plotmarks} 			&  \BS{usetikzlibrary}\AC{plotmarks}
\\ 
scopes				& \pageref{lib-scopes}			&  \BS{usetikzlibrary}\AC{scopes}
\\
shadings			& \pageref{lib-shadings}		&  \BS{usetikzlibrary}\AC{shadings}
\\
shapes.arrows		& \pageref{lib-arr}				&\BS{usetikzlibrary}\AC{shapes.arrows} 
\\shapes.callouts		& \pageref{lib-call}			& \BS{usetikzlibrary}\AC{shapes.callouts} 
\\
shapes.geometric	& \pageref{lib-geom} 			& \BS{usetikzlibrary}\AC{shapes.geometric}
\\  
shapes.misc			& \pageref{lib-misc} 			& \BS{usetikzlibrary}\AC{shapes.misc} 
\\
shapes.multipart	& \pageref{lib-mult} 			& \BS{usetikzlibrary}\AC{shapes.multipart} 
\\
shapes.symbols		& \pageref{lib-symb}			& \BS{usetikzlibrary}\AC{shapes.symbols} 
\\
trees				& \pageref{lib-trees}			&  \BS{usetikzlibrary}\AC{trees}
\\ 

\hline
 \end{tabular} 


\bigskip



\begin{tabular}{|l|c|}\hline
\multicolumn{2}{|c|}{ \TFRGB{dans une prochaine mise à jour}{In a a future update } }
\\ \hline
automata									& \RRR{41} \\
babel										& \RRR{42} \\
calendar									& \RRR{45} \\
chains										& \RRR{46} \\ 
%circuits.ee									& \RRR{47-4} \\ 
circuits.logic								& \RRR{47-3} \\ 
circular graph drawing library 				& \RRR{32} \\
curvilinear library 						& \RRR{103-4-7} \\
datavisualization library					& \RRR{75} \\
datavisualization.formats.functions library & \RRR{76-4} \\
datavisualization.polar library 			& \RRR{80}  \\
 er 										& \RRR{49}  \\
examples graph drawing library 				& \RRR{35-8} \\ 
external 									& \RRR{50}  \\  
%fit 										& \RRR{52} \\ 
fixedpointarithmetic 						& \RRR{53} \\ 
folding 									& \RRR{59} \\
force graph drawing library 				& \RRR{31}  \\
fpu											& \RRR{54}  \\
graph.standard library 						& \RRR{19-10}\\
graphdrawing library 						& \RRR{27} \\
graphs library 								& \RRR{19} \\ 
layered graph drawing library 				& \RRR{30}  \\
lindenmayersystems							& \RRR{55}  \\ 
matrix										& \RRR{57}  \\ 
mindmap										& \RRR{58} \\ 
petri										& \RRR{61}  \\ 
phylogenetics graph drawing library 		& \RRR{33} \\
plothandlers								& \RRR{62}  \\ 
positioning									& \RRR{17-5-3} \\ 
profiler									& \RRR{64}   \\ 
quotes library 								& \RRR{17-10-4} \\
routing graph drawing library 				& \RRR{34} \\
shadows										& \RRR{66}   \\ 
shapes.gates.ee								& \\ 
shapes.gates.ee.IEC							& \\ 
shapes.gates.logic							& \\ 
shapes.gates.logic.IEC						& \\ 
shapes.gates.logic.US						& \\ 
spy											&  \RRR{68} \\ 
svg.path									&  \RRR{69} \\ 
through										&  \RRR{71} \\ 
topaths										&  \RRR{70} \\ 
trees graph drawing library					& \\
turtle										&  \RRR{73} \\ 
\hline
\end{tabular}  
%circuit.ee.IEC, 309
%circuits, 292
%circuits.ee, 308
%, 300
%circuits.logic.CDH, 301
%circuits.logic.IEC, 300
%circuits.logic.US, 301

%
%\bigskip
%\textbf{Autres modules}
%
%%
%\begin{tabular}{|c|c|l|}\hline 
%nom 			& voir page 				& documentation\footnotemark[1]  \\  \hline 
%pst-fr3d 		& \pageref{pst-fr3d}		& pst-fr3d.pdf		\\ 
%pst-slpe 		& \pageref{pst-slpe}		& pst-slpe.pdf		\\ 
%infix-RPN 		& \pageref{infix-RPN}		& pst-infixplot.pdf	\\
%pst-infixplot 	& \pageref{pst-infixplot} 	& pst-infixplot.pdf \\ 
%pst-eucl 		& \pageref{pst-eucl} 		& pst-eucl-doc.pdf 	\\
%animate 		& \pageref{anim} 			& animate.pdf 	\\
%pst-3dplot		& \pageref{3dplot} 			& pst-3dplot-doc 	\\
%\hline 
%\end{tabular} 
%
%\bigskip
%\textbf{Additifs }
%
%%
%\begin{tabular}{|c|l|}\hline 
%année						& documentation\footnotemark[1]  \\  \hline 
%2005 			& pst-news5.pdf	\\
%2008  			& pst-news08.pdf \\ 
%2010 			& pst-news10.pdf 	\\
%\hline 
%\end{tabular} 
% 
%
%\footnotetext[1]{Vous pouvez les trouver pour la distribution Texlive dans le répertoire :  \BS{}texlive\BS{}2011\BS{}tesmf-dist\BS{}doc\BS{}generic}
%
%\newpage
%%
%% \tableofcontents
%\renewcommand{\bibname}{Sources}
%
\label{sources}
%\input{bib}

\newpage

\begin{thebibliography}{99}
\bibitem{pgfmanual} pgfmanual.pdf  	\hspace{1cm}	version 3.0.1a \hspace{1cm} 	1161 pages 	\hspace{1cm}	\DGB
\bibitem{pgfplots} pgfplots.pdf 	\hspace{1cm}	version 1.80 \hspace{1cm} 	439 pages 	\hspace{1cm}	\DGB
\bibitem{tikstab} tkz-tab-screen.pdf 	\hspace{1cm}	version 1.1c \hspace{1cm} 	83 pages 	\hspace{1cm}	\DFR

\end{thebibliography}



\newpage 
\section{Index}

% \printindex 
% \setcounter{tocdepth}{4}
%  \tableofcontents
%  \setcounter{tocdepth}{5}
% \addtolength{\hoffset}{-1.5cm} 
% 
% \setlength{\topmargin}{0pt}
% \setlength{\headsep}{0pt}
% 
%  \newpage
% 
% %\section{Les figures de base}
% \SSCT{Les figures de base}{Basic figures}
% \input{tkz1}
% 
% \newpage
% 
% \input{tkz2}
% 
% \newpage
% %
% \input{tkz3}
% %
% 
% 
% \input{tkz3a}
% 
% \newpage
% 
% %\section{Insertion de petites images}
% \SSCT{Insertion de petites images}{Small pictures}
% \input{tkzpic}
% 
% \newpage
% 
% \input{tkzangles}
% 
% %%%%% % % %===================================
% 
% \newpage
% 
% %\section{Les coordonnées }
% \SSCT{Les coordonnées }{Coordinates}
%  
% %\subsection{Quadrillage}
\SbSSCT{Quadrillage}{Grid}


\begin{tabular}{|c|}\hline 
\tikz \draw(0,0) grid (2,2); 
\\ \hline 
\BS{draw} (0,0) \RDD{grid} (2,2); \RRR{14-8}
\\ \hline 
\end{tabular} 


\bigskip
\begin{tabular}{|c|c|c|c|} \hline 
\multicolumn{4}{|c|}{ \BS{draw} (0,0) grid  [\RDD{step}=.75cm] (0,0) grid (3,3);   }\\ 
\hline  
\begin{tikzpicture}
\draw[dotted](0,0) grid (3,3); 
%\draw[thick] (0,0) grid [step=1] (3,2);
\draw[red] (0,0) grid [step=.75cm] (3,3);
\end{tikzpicture}
&  
\begin{tikzpicture}
\draw[dotted](0,0) grid (3,3); 
%\draw[thick] (0,0) grid [step=1] (3,2);
\draw[red] (0,0) grid [xstep=.75cm] (3,3);
\end{tikzpicture}
&  
\begin{tikzpicture}
\draw[dotted](0,0) grid (3,3); 
%\draw[thick] (0,0) grid [step=1] (3,2);
\draw[red] (0,0) grid [ystep=.75cm] (3,3);
\end{tikzpicture}
&
\begin{tikzpicture}
\draw[dotted](0,0) grid (3,3); 
%\draw[thick] (0,0) grid [step=1] (3,2);
\draw[red] (0,0) grid [step=(45:1)] (3,3);
\end{tikzpicture}
\\ \hline 
step=.75cm & x step=.75cm & ystep=.75cm  & step=(45:1)
\\ \hline 
\end{tabular} 

\bigskip

\begin{tabular}{|c|c|} \hline 
 
\BS{draw}[red] (0,0) grid [\RDD{rotate}=45] (3,3);
&  
\BS{draw}[\RDD{help lines}] (0,0) grid  (3,3);
\\ \hline  
\begin{tikzpicture}
\draw[dotted](0,0) grid (3,3); 
%\draw[thick] (0,0) grid [step=1] (3,2);
\draw[red] (0,0) grid [rotate=45] (3,3);
\end{tikzpicture}
& 
\tikz \draw[help lines] (0,0) grid (3,3); \\ 
\hline 
\end{tabular} 



%\begin{tabular}{|c|c|c|c|c|} \hline 
%\multicolumn{5}{|c|}{ \BS{tikz} \BS{draw} [\RDD{step}=1mm] (0,0) grid (2,2);   }\\ 
%\hline  
%\tikz \draw[step=1mm] (0,0) grid (2,2);
%&  
%\tikz[rotate=30] \draw (0,0) grid (2,2);
%&  
%\tikz \draw (0,0) grid [xstep=.5] (2,2);
%&  
%\tikz \draw (0,0) grid [ystep=.5] (2,2);
%&
%\tikz \draw[help lines] (0,0) grid (2,2);
%\\ \hline  
%[\RDD{step}=1mm] & [\RDD{rotate}=30] & [\RDD{xstep}=.5] & [\RDD{ystep}=.5] & [\RDD{help lines}] \\ 
%\hline 
%\end{tabular} 

% \newpage 
% 
% \input{tkzcoord}
% %
% %%%%%==========================================================
%  
% \newpage
% %\section[Les n\oe uds]{Les n\oe uds }
% \SSCT{Les n\oe uds }{Nodes}
% 
% 
% %\subsection{Définition des  n\oe uds}
\SbSSCT{Définition des  n\oe uds}{Creation of nodes}
\tikzset{blue}

\begin{tabular}{|c | c | c | c |} \hline
\multicolumn{4}{|c|}{  \BS{draw} (1,1) node[\RDD{fill}=red!20] \AC{};   }\\ 
\hline 
\tikz \draw (0,0) grid (2,2) (1,1) node[fill=red!20] {};
&
\tikz \draw (0,0) grid (2,2) (1,1) node[fill=red!20,draw] {}; 
&
\tikz \draw (0,0) grid (2,2) (1,1) node[circle,fill=red!20] {};
&
\tikz \draw (0,0) grid (2,2) (1,1) node[circle,fill=red!20,draw] {};
\\  \hline
\dft
&
node[\RDD{draw}] 
&
 node[\RDD{circle}]  
&
 node[\RDD{circle},\RDD{draw}]
 \\  \hline
\end{tabular}
\bigskip

\begin{tabular}{|c | c | c | c |} \hline
\multicolumn{4}{|c|}{ \BSS{node} \RDD{at} (1,1) [fill=red!20] \AC{};   }\\ 
\hline 
 \begin{tikzpicture}
\draw (0,0) grid (2,2) ; 
\node at (1,1) [fill=red!20] {};
 \end{tikzpicture}
&
 \begin{tikzpicture}
\draw (0,0) grid (2,2) ; 
\node at (1,1) [draw] {};
 \end{tikzpicture}
&
 \begin{tikzpicture}
\draw (0,0) grid (2,2) ; 
\node at (1,1) [fill=red!20,circle] {};
 \end{tikzpicture}
&
 \begin{tikzpicture}
\draw (0,0) grid (2,2) ; 
\node at (1,1) [circle,draw] {};
 \end{tikzpicture}
\\  \hline
[fill=red!20]
&
[\RDD{draw}] 
&
[\RDD{circle},fill=red!20]
 &
[\RDD{circle},draw] 
 \\  \hline
\end{tabular}
\bigskip

\TFRGB{Autres types de n\oe uds voir page}{Other type of nodes see page} \pageref{noeudboite}



%-------------------------------------------------------------------------------
%\subsection{Liaisons}
\SbSSCT{Liaisons}{Links}
\label{liaisons}

\begin{tabular}{|c|c|c|} \hline  
\begin{tikzpicture}[blue]
\node[draw] (A) at (0,0) {A};
\node[draw] (B) at (1.5,1.5) {B};
\draw (A) -- (B);
\end{tikzpicture}
&  
\begin{tikzpicture}[blue]
\node[draw] (A) at (0,0) {A};
\node[draw] (B) at (1.5,1.5) {B};
\draw (A) |- (B);
\end{tikzpicture}
&  
\begin{tikzpicture}[blue]
\node[draw] (A) at (0,0) {A};
\node[draw] (B) at (1.5,1.5) {B};
\draw (A) -| (B);
\end{tikzpicture}
\\ \hline  
(A){\color{red} - -} (B) & (A) {\color{red}|-} (B) &  (A) {\color{red}-|} (B)
\\ \hline 
\begin{tikzpicture}[blue]
\node[draw] (A) at (0,0) {A};
\node[draw] (B) at (1.5,1.5) {B};
\draw (A) to [bend right] (B);
\end{tikzpicture}
&  
\begin{tikzpicture}[blue]
\node[draw] (A) at (0,0) {A};
\node[draw] (B) at (1.5,1.5) {B};
\draw (A) to [bend left] (B);
\end{tikzpicture}
&  
\begin{tikzpicture}[blue]
\node[draw] (A) at (0,0) {A};
\node[draw] (B) at (1.5,1.5) {B};
\draw (A) to[bend left=0] (B);
\end{tikzpicture}
\\ \hline  
(A) to [\RDD{bend right}] (B) & (A) to [\RDD{bend left}] (B) &  (A) to[\RDD{bend left}=0] (B)
\\ \hline 
\begin{tikzpicture}[blue]
\node[draw] (A) at (0,0) {A};
\node[draw] (B) at (1.5,1.5) {B};
\draw (A) to[bend left=120]  (B);
\end{tikzpicture}
&  
\begin{tikzpicture}[blue]
\node[draw] (A) at (0,0) {A};
\node[draw] (B) at (1.5,1.5) {B};
\draw (A) to[bend left=45] (B);
\end{tikzpicture}
&  
\begin{tikzpicture}[blue]
\node[draw] (A) at (0,0) {A};
\node[draw] (B) at (1.5,1.5) {B};
\draw (A) to[bend left=90] (B);
\end{tikzpicture}
\\ \hline  
(A)  to[\RDD{bend left}=120]  (B) & (A) to[\RDD{bend left}=45] (B) &  (A) to[\RDD{bend left}=90] (B)
\\ \hline 
\begin{tikzpicture}[blue]
\node[draw] (A) at (0,0) {A};
\node[draw] (B) at (1.5,1.5) {B};
\draw (A)  to[out=90]  (B);
\end{tikzpicture}
&  
\begin{tikzpicture}[blue]
\node[draw] (A) at (0,0) {A};
\node[draw] (B) at (1.5,1.5) {B};
\draw (A) to[out=30] (B);
\end{tikzpicture}
&  
\begin{tikzpicture}[blue]
\node[draw] (A) at (0,0) {A};
\node[draw] (B) at (1.5,1.5) {B};
\draw (A)  to[in=-90]  (B);
\end{tikzpicture}
\\ \hline  
(A)  to[\RDD{out}=90] (B) & (A) to[\RDD{out}=30]  (B) &  (A)  to[\RDD{in}=-90]  (B)
\\ \hline 
%\begin{tikzpicture}[blue]
%\node[draw] (A) at (0,0) {A};
%\node[draw] (B) at (2,2) {B};
%\draw (A)  to[in=90]  (B);
%\end{tikzpicture}
%&  
%\begin{tikzpicture}[blue]
%\node[draw] (A) at (0,0) {A};
%\node[draw] (B) at (2,2) {B};
%\draw (B) to[in=0,out=90]  (B);
%\end{tikzpicture}
%&  
%\begin{tikzpicture}[blue]
%\node[draw] (A) at (0,0) {A};
%\node[draw] (B) at (2,2) {B};
%%\draw (A)  to[out=45,in=-90]  (A);
%\draw (B) to[out=45,in=135] (B);
%\end{tikzpicture}
%\\ \hline  
%(A)  to[\RDD{in}=90] (B) & (B) to[bend left]  (B) & (B) to[out=45,in=135] (B)
%\\ \hline 
\end{tabular} 

\bigskip
\begin{tabular}{|c|c|c|} \hline  
\multicolumn{2}{|c|}{ \BS{draw} (A) .. controls +(right:2cm) and +(down:2cm) .. (B);  }\\ 
\hline  
\begin{tikzpicture}[blue]
\node[draw] (A) at (0,0) {A};
\node[draw] (B) at (2,2) {B};
\draw  (A) .. controls +(right:2cm) and +(down:2cm) .. (B);
\end{tikzpicture}
&
\begin{tikzpicture}[blue]
\node[draw] (A) at (0,0) {A};
\node[draw] (B) at (2,2) {B};
\draw  (A) .. controls +(up:1cm) and +(left:1cm) .. (B);
\end{tikzpicture}
\\ \hline 
controls +(right:2cm) and +(down:2cm)  &
controls +(up:1cm) and +(left:1cm)
\\ \hline 
\begin{tikzpicture}[blue]
\node[draw] (A) at (0,0) {A};
\node[draw] (B) at (2,2) {B};
\draw  (A) .. controls +(right:1cm) and +(right:2cm) .. (B);
\end{tikzpicture}
&
\begin{tikzpicture}[blue]
\node[draw] (A) at (0,0) {A};
\node[draw] (B) at (2,2) {B};
\draw  (A) .. controls +(up:1cm) and +(right:2cm) .. (B);
\end{tikzpicture}
\\ \hline 
controls +(right:1cm) and +(right:2cm)  &
controls +(up:1cm) and +(right:2cm) 
\\ \hline 
\begin{tikzpicture}[blue]
\node[draw] (A) at (0,0) {A};
\node[draw] (B) at (2,2) {B};
\draw  (A) .. controls +(120:2cm) and +(200:1cm) .. (B);
\end{tikzpicture}
 &
 \begin{tikzpicture}[blue]
 \node[draw] (A) at (0,0) {A};
 \node[draw] (B) at (2,2) {B};T
 \draw  (A) .. controls +(120:2cm) and +(200:1cm) .. (A);
 \end{tikzpicture}
\\  \hline  
controls +(120:2cm) and +(200:1cm) & controls +(120:2cm) and +(200:1cm) 
\\ \hline 
\begin{tikzpicture}[blue]
\node[draw] (A) at (0,0) {A};
\node[draw] (B) at (2,2) {B};
\node[draw] (C) at (0,1) {C};
\node[draw] (D) at (3,0) {D};
\draw  (A) .. controls +(C) and +(D) .. (B);
\end{tikzpicture}
&
\begin{tikzpicture}[blue]
\node[draw] (A) at (0,0) {A};
\node[draw] (B) at (2,2) {B};
\node[draw] (C) at (0,1) {C};
\node[draw] (D) at (3,0) {D};
\draw (A) .. controls +(D)  .. (B);
\end{tikzpicture}
\\ \hline 
controls +(C) and +(D) &
controls +(D) 
\\ \hline 
\end{tabular} 
 \bigskip
 
\begin{tabular}{|c|c|c|} \hline 
\multicolumn{3}{|l|}{ \BS{node}[draw] (A) at (0,0) \AC{A}  }\\

\multicolumn{3}{|l|}{ \BS{node}[draw] (B) at (2,2) \AC{B} \RDD{edge}  [->] (A);  }\\
\multicolumn{3}{|c|}{\RRR{17-12-1}}  \\
\hline 
 \begin{tikzpicture}
 \node[draw] (A) at (0,0) {A};
 \node[draw] (B) at (2,2) {B} edge [->] (A);
 \end{tikzpicture}
 &
 \begin{tikzpicture}
 \node[draw] (A) at (0,0) {A};
 \node[draw] (B) at (2,2) {B} edge [red]  (A);
 \end{tikzpicture}
 &
 \begin{tikzpicture}
 \node[draw] (A) at (0,0) {A};
 \node[draw] (B) at (2,2) {B} edge [dashed] (A);
 \end{tikzpicture}
\\ \hline 
[->] & [red]  & [dashed]
\\ \hline 
\end{tabular}

%---------------------------------------------------------------------------------
%\subsection{\'Etiquettes sur les n\oe uds}
\SbSSCT{\'Etiquettes sur les n\oe uds}{Node labels}

\begin{tabular}{|c|c|c|c|} \hline
\multicolumn{4}{|c|}{  \BS{fill}(0,0) circle (2pt) node[\RDD{above}] \AC{texte} ;   }\\ 
\hline 
  
\begin{tikzpicture} \draw[help lines] (-1,-1) grid (1,1) ;\fill (0,0) circle (2pt) node[above] {texte};\end{tikzpicture}
& 
\begin{tikzpicture} \draw[help lines] (-1,-1) grid (1,1) ;\fill (0,0) circle (2pt) node[below] {texte};\end{tikzpicture}
 &  
\begin{tikzpicture} \draw[help lines] (-1,-1) grid (1,1);\fill (0,0) circle (2pt) node[left] {texte};\end{tikzpicture}
 &  
\begin{tikzpicture} \draw[help lines] (-1,-1) grid (1,1); \fill (0,0) circle (2pt) node[right] {texte};\end{tikzpicture}
 \\  \hline 
 [\RDD{above}] & [\RDD{below}] & [\RDD{left}] &  [\RDD{right}]
 \\ \hline 
 \begin{tikzpicture} \draw[help lines] (-1,-1) grid (1,1) ;\fill (0,0) circle (2pt) node[above left] {texte};\end{tikzpicture}
 & 
 \begin{tikzpicture} \draw[help lines] (-1,-1) grid (1,1) ;\fill (0,0) circle (2pt) node[below left] {texte};\end{tikzpicture}
  &  
 \begin{tikzpicture} \draw[help lines] (-1,-1) grid (1,1);\fill (0,0) circle (2pt) node[above right] {texte};\end{tikzpicture}
  &  
 \begin{tikzpicture} \draw[help lines] (-1,-1) grid (1,1); \fill (0,0) circle (2pt) node[below right] {texte};\end{tikzpicture}
  \\  \hline 
  [\RDD{above left}] & [\RDD{below left}] & [\RDD{above right}] &  [\RDD{below right}]
  \\ \hline 
 \begin{tikzpicture} \draw[help lines] (-1,-1) grid (1,1) ;\fill (0,0) circle (2pt) node[anchor=south] {texte};\end{tikzpicture}
 & 
 \begin{tikzpicture} \draw[help lines] (-1,-1) grid (1,1) ;\fill (0,0) circle (2pt) node[anchor=west] {texte};\end{tikzpicture}
  &  
 \begin{tikzpicture} \draw[help lines] (-1,-1) grid (1,1);\fill (0,0) circle (2pt) node[anchor=north] {texte};\end{tikzpicture}
  &  
 \begin{tikzpicture} \draw[help lines] (-1,-1) grid (1,1); \fill (0,0) circle (2pt) node[anchor=east] {texte};\end{tikzpicture}
  \\  \hline 
  [\RDD{anchor=south}] & [\RDD{anchor=west}] & [\RDD{anchor=north}] & [\RDD{anchor=east                                                                                                                                                               }]
  \\ \hline 
 \begin{tikzpicture} \draw[help lines] (-1,-1) grid (1,1) ;\fill (0,0) circle (2pt) node[anchor=south east] {texte};\end{tikzpicture}
 & 
\begin{tikzpicture} \draw[help lines] (-1,-1) grid (1,1) ;\fill (0,0) circle (2pt) node[anchor=south west] {texte};\end{tikzpicture}
&  
\begin{tikzpicture} \draw[help lines] (-1,-1) grid (1,1);\fill (0,0) circle (2pt) node[anchor=north west] {texte};\end{tikzpicture}
&  
\begin{tikzpicture} \draw[help lines] (-1,-1) grid (1,1); \fill (0,0) circle (2pt) node[anchor=east] {texte};\end{tikzpicture}
\\  \hline 
[\RDD{anchor=south east}] & [\RDD{anchor=south west}] & [\RDD{anchor=north west}] & [\RDD{anchor=north east                                                                                                                                                              }]
  \\ \hline 
\end{tabular} 


\bigskip
\begin{tabular}{|c|c|c|c|} \hline
\multicolumn{4}{|c|}{  \BS{fill}(0,0) circle (2pt) node[\RDD{above}=.3cm] \AC{texte} ;   }\\ 
\hline 
  
\begin{tikzpicture} \draw[help lines] (-1,-1) grid (1,1) ;\fill (0,0) circle (2pt) node[above=.3cm] {texte};\end{tikzpicture}
& 
\begin{tikzpicture} \draw[help lines] (-1,-1) grid (1,1) ;\fill (0,0) circle (2pt) node[below=.3cm] {texte};\end{tikzpicture}
 &  
\begin{tikzpicture} \draw[help lines] (-1,-1) grid (1,1);\fill (0,0) circle (2pt) node[left=.3cm] {texte};\end{tikzpicture}
 &  
\begin{tikzpicture} \draw[help lines] (-1,-1) grid (1,1); \fill (0,0) circle (2pt) node[right=.3cm] {texte};\end{tikzpicture}
 \\  \hline 
 [\RDD{above}=.3cm] & [\RDD{below}=.3cm] & [\RDD{left}=.3cm] &  [\RDD{right}=.3cm]]
 \\ \hline 
\begin{tikzpicture} \draw[help lines] (-1,-1) grid (1,1) ;\fill (0,0) circle (2pt) node[above left=.3cm] {texte};\end{tikzpicture}
& 
\begin{tikzpicture} \draw[help lines] (-1,-1) grid (1,1) ;\fill (0,0) circle (2pt) node[below left=.3cm] {texte};\end{tikzpicture}
 &  
\begin{tikzpicture} \draw[help lines] (-1,-1) grid (1,1);\fill (0,0) circle (2pt) node[above right=.3cm] {texte};\end{tikzpicture}
 &  
\begin{tikzpicture} \draw[help lines] (-1,-1) grid (1,1); \fill (0,0) circle (2pt) node[below right=.3cm] {texte};\end{tikzpicture}
 \\  \hline 
 [\RDD{above left}=.3cm] & [\RDD{below left}=.3cm] & [\RDD{above right}=.3cm] &  [\RDD{below right}=.3cm]]
 \\ \hline 
 
 \end{tabular} 
 
%\begin{tikzpicture} \draw[help lines] (-1,-1) grid (1,1);\fill (0,0) circle (2pt) node[distance=.3cm] {texte};\end{tikzpicture} 
 
 \newpage
\selectlanguage{french}
 
 \begin{tabular}{|c|c|c|c|c|} \hline
 \multicolumn{5}{|l|}{ \BSS{shorthandoff}\AC{:} \footnotemark[1]  } \\
 \multicolumn{5}{|l|}{  \BS{node} [draw,\RDD{label}=right:texte] \AC{}   }\\
 \multicolumn{5}{|l|}{ \BSS{shorthandon}\AC{:} } \\ 
 \hline 
     \shorthandoff{:} 
 \tikz \node [draw,label=right:texte] {};
 \shorthandon{:}
 &
  \shorthandoff{:}
 \tikz \node [draw,label=left:texte] {};
 \shorthandon{:}
 &
  \shorthandoff{:}
 \tikz \node [draw,label=above:texte] {};
 \shorthandon{:}
 &
  \shorthandoff{:}
 \tikz \node [draw,label=below:texte] {};
 \shorthandon{:}
 &
  \shorthandoff{:}
 \tikz \node [draw,label=45:texte] {};
    \shorthandon{:}
   \\ \hline
  label=right & label=left &  label=above & label=below & label=45
    \\ \hline 
 \end{tabular}
 \footnotetext[1]{\TFRGB{désactivation et ré-activation de \og : \fg  conflit entre les modules Tikz et Babel en français}{Only useful when the package babel is loaded with the frenchb option    }}
 
 \bigskip
  \begin{tabular}{|c|c|c|c|c|} \hline
  \BS{fill}(0,0) circle (2pt) node[below right=.3cm,draw,label=45:étiquette] \AC{texte} ;
      \\ \hline 
  
  \shorthandoff{:}
\begin{tikzpicture} \draw[help lines] (-1,-1) grid (2,1); \fill (0,0) circle (2pt) node[below right=.3cm,draw,label=45:étiquette] {texte};\end{tikzpicture}
 \shorthandon{:}
 
    \\ \hline 
 \end{tabular}
\bigskip

 \shorthandoff{:}
 

 
\begin{tabular}{|c|c|c|} \hline
\multicolumn{3}{|c|}{  \BSS{shorthandoff}\AC{:} \BS{node}[circle,draw,blue,\RDD{pin}=texte] \AC{} ;   \BSS{shorthandon}\AC{:}  \footnotemark[1] }\\ 
\hline
\begin{tikzpicture} 
\node [circle,draw,blue,pin=texte] {};
\end{tikzpicture}
&
\begin{tikzpicture} 
\node [circle,draw,blue,pin=60:texte] {};
\end{tikzpicture}
&
\begin{tikzpicture} 
\node [circle,draw,blue,pin=right:texte] {};
\end{tikzpicture}
 \\ \hline
[circle,pin=texte] &   [circle,pin=60:texte] & [circle,pin=right:texte]
 \\ \hline 
\end{tabular}  

\bigskip
\begin{tabular}{|c|c|c|} \hline
\multicolumn{3}{|c|}{  \BS{tikz}[\RDD{pin position}=60] \BS{node} [circle,pin=texte] \AC{} ;   }\\ 
\hline 
\tikz[pin position=60] \node [circle,draw,blue,pin=texte] {};
&
\tikz[pin distance=0 cm] \node [circle,draw,blue,pin=60:texte] {};
&
\tikz[pin distance=2 cm] \node [circle,draw,blue,pin=60:texte,pin distance=0cm] {};
  \\ \hline
  [\RDD{pin position}=60] & [\RDD{pin distance}=0 cm] & [\RDD{pin distance}=2 cm]
    \\ \hline
  \dft{ : above} & \multicolumn{2}{|c|}{ \dft{ : 3 ex}}
      \\ \hline
\end{tabular}  

% % % % % % % % % % >>>>>>>>>> a voir : option edge <<<<<<<<<<<<<<<<<<<<<<<<<<<<<<<<<<<<<<

   \shorthandon{:} 
   
\selectlanguage{english}   
% >>>>>>>>>>>>>>>>>>>>>> A Voir : positioning librairy <<<<<<<<<<<<<<<<<<<<<<<<<<<<<<<<<<<<<<<

%\subsection{ N\oe uds  sur un chemin}
\SbSSCT{ N\oe uds  sur un chemin}{Nodes on a path}

\begin{tabular}{|c|c|c|} \hline
\multicolumn{3}{|c|}{  \BS{draw}(0,0) .. controls (1,2) and (2,-1) .. (4,0) node[\RDD{at end}] \AC{texte} ;   }\\ 
\hline 
\tikz \draw (0,0) .. controls (1,2) and (2,-1) .. (4,0) node[pos=0] {texte}; 
&
\tikz \draw (0,0) .. controls (1,2) and (2,-1) .. (4,0) node[pos=.33] {texte}; 
&
\tikz \draw (0,0) .. controls (1,2) and (2,-1) .. (4,0) node[at end] {texte}; 
  \\ \hline 
\RDD{pos}{\color{red}  =0} & \RDD{pos}{\color{red}  =.33} & \RDD{at end} (pos=1)
  \\ \hline 

\tikz \draw (0,0) .. controls (1,2) and (2,-1) .. (4,0) node[very near end] {texte}; 
&
\tikz \draw (0,0) .. controls (1,2) and (2,-1) .. (4,0) node[near end] {texte}; 
&
\tikz \draw (0,0) .. controls (1,2) and (2,-1) .. (4,0) node[midway] {texte}; 
  \\ \hline 
\RDD{very near end} (pos=0.875.) & \RDD{ near end} (pos=0.75) & \RDD{midway} (pos=0.5)
  \\ \hline 
  
\tikz \draw (0,0) .. controls (1,2) and (2,-1) .. (4,0) node[near start] {texte}; 
&
\tikz \draw (0,0) .. controls (1,2) and (2,-1) .. (4,0) node[very near start] {texte}; 
&
\tikz \draw (0,0) .. controls (1,2) and (2,-1) .. (4,0) node[at start] {texte};
\\ \hline 
\RDD{near start} (pos=0.25) & \RDD{very near start} (pos=0.125) & \RDD{at start} (pos=0)
  \\ \hline 
  
\end{tabular} 

\bigskip
\begin{tabular}{|c|c|c|} \hline
\multicolumn{3}{|c|}{  \BS{draw}(0,0) .. controls (1,2) and (2,1) .. (4,0) node[\RDD{sloped},midway] \AC{texte} ;   }\\ 
\hline 
\tikz \draw (0,0) .. controls (1,2) and (2,-1) .. (4,0) node[sloped,midway] {texte};
&
\tikz \draw (0,0) .. controls (1,2) and (2,-1) .. (4,0) node[above,midway] {texte};
&
\tikz \draw (0,0) .. controls (1,2) and (2,-1) .. (4,0) node[below,midway] {texte};
  \\ \hline
\RDD{sloped} & \RDD{above} &\RDD{below}
  \\ \hline
\end{tabular}
\bigskip

\begin{tabular}{|c|c|c|} \hline
\multicolumn{3}{|c|}{  \BS{draw}(0,0) .. controls (1,2) and (2,1) .. (5,0) node[\RDD{sloped},midway,allow upside down] \AC{texte} ;   }\\ 
\hline 
\tikz \draw (0,0) .. controls (1,2) and (2,-1) .. (4,0) node[sloped,midway,allow upside down] {texte};
&
\tikz \draw (0,0) .. controls (1,2) and (2,-1) .. (4,0) node[above,midway,allow upside down] {texte};
&
\tikz \draw (0,0) .. controls (1,2) and (2,-1) .. (4,0) node[below,midway,allow upside down] {texte};
  \\ \hline
\RDD{sloped} & \RDD{above} &\RDD{below}
  \\ \hline
\end{tabular}  


\begin{tabular}{|c|c|c|} \hline
\multicolumn{3}{|c|}{  \BS{draw}(A)  to [bend right]  node [\RDD{bend right}] \AC{texte} (B);   }\\ 
\hline 
\begin{tikzpicture} %[auto,bend right]
\node[draw] (A) at (0,0) {A};
\node[draw] (B) at (2,2) {B};
\draw (A) to [bend right] node [bend right] {texte} (B);
\end{tikzpicture}
&
\begin{tikzpicture} 
\node[draw] (A) at (0,0) {A};
\node[draw] (B) at (2,2) {B};
\draw (A) to [bend right] node [auto,bend right] {texte} (B);
\end{tikzpicture}
&
\begin{tikzpicture} 
\node[draw] (A) at (0,0) {A};
\node[draw] (B) at (2,2) {B};
\draw (A) to[bend right] node [auto,swap,bend right] {texte} (B);
\end{tikzpicture}
  \\ \hline
[bend right]  & [\RDD{auto},bend right] & [auto,\RDD{swap},bend right] 
  \\ \hline
\end{tabular}  

\SbSSCT{ N\oe uds  sur un \og edge\fg}{Nodes on an edge}

\begin{tabular}{|c|c|c|}\hline  
\multicolumn{3}{|c|}{  \BS{draw}(0,0) edge \BDD{["abc", ->]} (4,0);  }\\ 
\multicolumn{3}{|c|}{  \RRR{17-12-2} }\\ 
\hline 
\begin{tikzpicture}[blue] 
\useasboundingbox  (0,-.5) rectangle (4,.5); 
\draw (0,0) edge ["abc", ->] (4,0);
\end{tikzpicture}
&
\begin{tikzpicture}[blue] 
\useasboundingbox  (0,-.5) rectangle (4,.5); 
\draw (0,0) edge ["abc", near start] (4,0);
\end{tikzpicture}
&
\begin{tikzpicture}[blue] 
\useasboundingbox  (0,-.5) rectangle (4,.5); 
\draw (0,0) edge ["abc", style={auto=right}] (4,0);
\end{tikzpicture}
\\ \hline 
["abc", ->]
& 
["abc", near start] &  ["abc", style=\AC{auto=right}] 
\\ \hline  
\begin{tikzpicture}[blue] 
\useasboundingbox  (0,-.5) rectangle (4,.5); 
\draw (0,0) edge [font=\Large,"abc" ] (4,0);
\end{tikzpicture}
&
\begin{tikzpicture}[blue] 
\useasboundingbox  (0,-.5) rectangle (4,.5); 
\draw (0,0) edge ["abc" color=red ] (4,0);
\end{tikzpicture}
&
\begin{tikzpicture}[blue] 
\useasboundingbox  (0,-.5) rectangle (4,.5); 
 \draw (0,0) edge ["abc" '] (4,0);
\end{tikzpicture}
\\ \hline 
[font=\BS{Large},"abc" ] & ["abc" color=red ]
&["abc" ' ]
\\ \hline 

\begin{tikzpicture}[blue] 
\useasboundingbox  (0,-.5) rectangle (4,.75); 
\draw (0,0) edge ["abc" draw ] (4,0);
\end{tikzpicture}
&
\begin{tikzpicture}[blue] 
\useasboundingbox  (0,-.5) rectangle (4,.5); 
\draw (0,0) edge ["abc" inner sep=0pt ] (4,0);
\end{tikzpicture}
&
\begin{tikzpicture}[blue] 
\useasboundingbox  (0,-.5) rectangle (4,.5); 
\draw (0,0) edge ["abc" fill ,fill=yellow ] (4,0);
\end{tikzpicture}
\\ \hline
["abc" draw ]
&
["abc" inner sep=0pt ]
&
["abc" fill ,fill=yellow ]
\\ \hline
\end{tabular} 



\bigskip

\begin{tabular}{|c|} \hline  
\BS{draw}[every edge quotes/.style=\AC{fill=yellow}] (0,0) edge ["abc"] (4,0);
\\ \hline  
\begin{tikzpicture}[blue] 
\useasboundingbox  (0,-.5) rectangle (4,.5); 
 \draw[every edge quotes/.style={fill=yellow}] (0,0) edge ["abc"] (4,0);
\end{tikzpicture}
\\ \hline 
\end{tabular} 





% 
% \input{tkzfit}
% 
% %
% \newpage
% 
% %%%======================================================
% %\section[Constructions particulières]{Constructions particulières  }
% \SSCT{Constructions particulières  }{Transformations}
% %
% 
%\subsubsection{Transformations}

\begin{center}
\RRR{25-3}
\end{center}


\begin{tabular}{|c|c|c|c|} \hline 
\multicolumn{4}{|c|}{  \BS{draw}[\RDD{rotate},blue] (0,0)  rectangle  (2,2) ;   }\\ 
\hline  
\begin{tikzpicture}
\draw[dashed,red] (0,0) rectangle  (2,2) ; 
\draw[rotate=40,blue] (0,0) rectangle  (2,2) ;
\end{tikzpicture}
&  
\begin{tikzpicture}
\draw[dashed,red] (0,0) rectangle  (2,2) ; 
\draw[x=1cm,y=.5cm,blue] (0,0) rectangle  (2,2); 
\end{tikzpicture}
&  
\begin{tikzpicture}
\draw[dashed,red] (0,0) rectangle  (2,2) ; 
\draw[xslant=.75,blue] (0,0) rectangle  (2,2);  
\end{tikzpicture}
&
\begin{tikzpicture}
\draw[dashed,red] (0,0) rectangle  (2,2) ; 
\draw[yslant=.75,blue] (0,0) rectangle  (2,2);  
\end{tikzpicture}
\\ \hline  
\RDD{rotate}=40 & \RDD{x}=1cm,\RDD{y}=0.5cm & \RDD{xslant}=0.75 & \RDD{yslant}=0.75\\ 
\hline 
  
\begin{tikzpicture}
\draw[dashed,red] (0,0) rectangle  (2,2) ; 
\draw[scale=1.5,blue] (0,0) rectangle  (2,2) ; 
\end{tikzpicture}
&  
\begin{tikzpicture}
\draw[dashed,red] (0,0) rectangle  (2,2) ; 
\draw[scale=-1,y=.5cm,blue] (0,0) rectangle  (2,2); 
\end{tikzpicture}
&  
\begin{tikzpicture}
\draw[dashed,red] (0,0) rectangle  (2,2) ; 
\draw[xshift=.5cm,blue] (0,0) rectangle  (2,2); 
\end{tikzpicture}
&
\begin{tikzpicture}
\draw[dashed,red] (0,0) rectangle  (2,2) ; 
\draw[yshift=.5cm,blue] (0,0) rectangle  (2,2);  
\end{tikzpicture}
\\ \hline  
\RDD{scale}=1.5 & \RDD{scale}=-1 & \RDD{xshift}=0.5cm & \RDD{yshift}=0.5cm
\\ \hline 
\end{tabular} 

\bigskip

%==============================================
%\begin{tikzpicture}
%\draw[help lines] (0,0) grid (3,2);
%\draw (0,0) - - (1,1) - - (1,0);
%\draw[rotate=40,blue] (0,0) - - (1,1) - - (1,0); % rotation 40°
%\draw[rotate=-20,red] (0,0) - - (1,1) - - (1,0); % rotation -20°
%\end{tikzpicture}

%\tikz \draw[x=1cm,y=.5cm] (0,0) rectangle(2,2);

%\tikz \draw (0,0) rectangle (1,0.5) [xshift=2cm] (0,0) rectangle (1,0.5);

%\begin{tikzpicture}
%\draw[help lines] (0,0) grid (3,2);
%\draw (0,0) - - (1,1) - - (1,0);
%\draw[scale=2,blue] (0,0) - - (1,1) - - (1,0); % échelle 2
%\draw[scale=-1,red] (0,0) - - (1,1) - - (1,0); % échelle -1
%\end{tikzpicture}

%\begin{tikzpicture}
%\draw[help lines] (0,0) grid (3,2);
%\draw (0,0) - - (1,1) - - (1,0);
%\draw[xslant=2,blue] (0,0) - - (1,1) - - (1,0);
%\draw[xslant=-1,red] (0,0) - - (1,1) - - (1,0);
%\end{tikzpicture}

%\begin{tikzpicture}
%\draw[help lines] (0,0) grid (3,2);
%\draw (0,0) rectangle (1,0.5);
%\beginscope[xshift=2cm] % Décalage en X de 2cm
%\draw [red] (0,0) rectangle (1,0.5);
%\draw[yshift=1cm,blue] (0,0) rectangle (1,0.5);
%\draw[rotate=30,orange] (0,0) rectangle (1,0.5);
%\endscope
%\end{tikzpicture}
 
% 
% \newpage
% 
% %\section{Placer son dessin}
% \SSCT{Placer son dessin}{Placing the picture}
% %
% \input{tkzfig}
% 
% \newpage
% 
% \section{Scope}
% %
% \input{tkzscope} 
% 
% \newpage
% 
%  
% %\section{Position absolue sur une page}
% \SSCT{Position absolue sur une page}{Absolute position on a page}
% 
%  \input{tkzpage}
%  
% \newpage 
% 
% %\section{Arrière plan du dessin}
% \SSCT{Arrière plan du dessin}{Background}
% 
%  \input{tkzbackground}
% 
% \newpage 
% 
% %%\section{Placer des objets}
% %%
% %%%\input{plac}
% %%
% %
% %%\newpage
% %%
% %%%===================================================
% %\section{Créer ses couleurs}
% \SSCT{Créer ses couleurs}{Defining your own colors}
% 
%  \input{tkzcoul}
%  
% 
% \newpage
% 
% %\sectionCreate {Créer ses commandes}
% \SSCT{Créer ses commandes}{Create command}
% 
% \input{tkzcde}
% 
% 
% \newpage
% 
% %\section[Créer ses styles]{Créer ses styles}
% \SSCT{Créer ses styles}{Creating styles}
% 
% \input{tkzstyl}
% 
% %%%%%%%======================================================================
% 
% \newpage
% 
% %\section{Mettre du texte  en valeur}
% \SSCT{Mettre du texte  en valeur}{Text highlighting}
% 
% \label{ndbt}

\tikzset{blue}

%\subsection{Dans un n\oe ud de Tikz}
\SbSSCT{Dans un n\oe ud de Tikz}{In a TikZ node}
\label{noeudboite}

\begin{tabular}{|c | c | c | c |} \hline
\multicolumn{4}{|c|}{ \BS{tikz} \BS{draw} (0,0) grid (2,2) (1,1) node[fill=red!20,] \AC{texte};   }\\ 
\hline 
\tikz \draw (0,0) grid (2,2) (1,1) node[fill=red!20] {texte};
&
\tikz \draw (0,0) grid (2,2) (1,1) node[fill=red!20,draw] {texte}; 
&
\tikz \draw (0,0) grid (2,2) (1,1) node[circle,fill=red!20] {texte};
&
\tikz \draw (0,0) grid (2,2) (1,1) node[circle,fill=red!20,draw] {texte};
\\  \hline
node[fill=red!20] 
&
node[fill=red!20,\RDD{draw}] 
&
 node[fill=red!20,\RDD{circle}]  
&
 node[fill=red!20,\RDD{circle},\RDD{draw}]
 \\  \hline
\end{tabular}
\bigskip


\subsubsection{Options}
\begin{tabular}{|c | c | c | c |c |c |c |c |} \hline
\multicolumn{8}{|c|}{ \BS{tikz} \BS{draw} node[draw,\RDD{double},blue] \AC{texte};   }\\ 
\hline 

\tikz \draw  node[draw,double,blue] {texte};
&
\tikz \draw  node[draw,rounded corners,blue] {texte};
&
\tikz \draw  node[draw,ultra thick,blue] {texte};
&
\tikz \draw  node[draw,dashed,blue] {texte};
&
\tikz \draw  node[draw,red] {texte};
&
\tikz \draw  node[draw,rotate=45,blue] {texte};
&
\tikz \draw  node[draw,shading=radial,blue] {texte};
&
\tikz \draw  node[draw,blue,text=red] {texte};
\\ \hline
\RDD{double} & \RDD{rounded corners} &  ultra thick & dashed & red & rotate=45 & shading=radial & text=red 
\\ \hline
\end{tabular}
\bigskip


\begin{tabular}{|c | c | c | c |c |} \hline
\multicolumn{4}{|c|}{ \BS{tikz} \BS{draw}  node[draw,\RDD{inner sep}=0pt] \AC{texte};   }\\ 
\hline 
\tikz \draw  node[draw,inner sep=0pt,blue] {texte};
&
\tikz \draw node[draw,inner sep=1cm,blue] {texte};
&
\tikz \draw  node[draw,inner xsep=1cm,blue] {texte};
&
\tikz \draw  node[draw,inner ysep=1cm,blue] {texte};
\\ \hline
 \RDD{inner sep}=0pt & \RDD{inner sep}=1cm & \RDD{inner xsep}=1cm & \RDD{inner ysep}=1cm
\\ \hline
\multicolumn{4}{|c|}{ \dft{} : 0.3333em }\\ 
\hline 

\end{tabular}

\bigskip

\begin{tabular}{|c | c | c | c |} \hline
\multicolumn{4}{|l|}{ \BS{node} [fill=red!20,\RDD{outer sep}=1cm] (A) at (1,1) \AC{texte};   }\\ 
\multicolumn{4}{|l|}{ \BS{fill} (node cs:name=A,anchor=east) circle (3pt);  }\\ 
\multicolumn{4}{|l|}{ \BS{fill} (node cs:name=A,anchor=south) circle (3pt);  }\\ 
\hline 
\begin{tikzpicture}
\draw[help lines] (0,0) grid (3,2);
\node[fill=red!20,outer sep=1cm] (A) at (1,1) {texte};
\fill[red] (node cs:name=A,anchor=east) circle (3pt);
\fill[red] (node cs:name=A,anchor=south) circle (3pt);
\end{tikzpicture}
&
\begin{tikzpicture}
\draw[help lines] (0,0) grid (3,2);
\node[fill=red!20,outer sep=0pt] (A) at (1,1) {texte};
\fill[red] (node cs:name=A,anchor=east) circle (3pt);
\fill[red] (node cs:name=A,anchor=south) circle (3pt);
\end{tikzpicture}
&
\begin{tikzpicture}
\draw[help lines] (0,0) grid (3,2);
\node[fill=red!20,outer xsep=1cm] (A) at (1,1){texte};
\fill[red] (node cs:name=A,anchor=east) circle (3pt);
\fill[red] (node cs:name=A,anchor=south) circle (3pt);
\end{tikzpicture}
&
\begin{tikzpicture}
\draw[help lines] (0,0) grid (3,2);
\node[fill=red!20,outer ysep=1cm] (A) at (1,1) {texte};
\fill[red] (node cs:name=A,anchor=east) circle (3pt);
\fill[red] (node cs:name=A,anchor=south) circle (3pt);
\end{tikzpicture}
\\ \hline
 \RDD{outer sep}=1cm & \RDD{outer sep}=0pt & \RDD{outer xsep}=1cm & \RDD{outer ysep}=1cm
\\ \hline
\multicolumn{4}{|c|}{ \dft{} : 0.5\BS{pgflinewidth} }\\ 
\hline 
\end{tabular}
%----------------------------------------------------------------------------------
%\subsubsection{Taille minimale des noeuds}
\SbSbSSCT{Taille minimale des noeuds}{Minimum size}

\begin{tabular}{|c|c|} \hline  
\multicolumn{2}{|c|}{  \BS{draw}((0,0) node[fill=blue!20,\RDD{minimum height}=1.5cm,draw]  \AC{texte} ;   }\\ 
\hline 
\tikz \draw (0,0) node[fill=red!20,minimum height=1.5cm,draw] {texte};
&  
\tikz \draw (0,0) node[fill=red!20,minimum width=3cm,draw] {texte};

\\ \hline  

\RDD{minimum height}=1.5cm
&  
\RDD{minimum width}=3cm
\\ \hline  
\tikz \draw (0,0) node[fill=red!20,minimum size=1.5cm,draw] {texte};
&  
\tikz \draw (0,0) node[fill=red!20,minimum size=1.5cm,draw,circle] {texte};

\\ \hline 
\RDD{minimum size}=1.5cm,draw
&  
\RDD{minimum size}=1.5cm,circle

\\ \hline 
\end{tabular} 

\newpage
%-----------------------------------------------
%\subsection{Dans un n\oe ud à formes géométriques}
\SbSSCT{Dans un n\oe ud à formes géométriques}{Geometric Shapes nodes}

\label{lib-geom}
\label{formes}
%Insérer dans le préambule :

 \maboite{\BS{usetikzlibrary}\AC{shapes.geometric}}
 
 
\begin{center}
\RRR{67-3}
\end{center}
%\subsubsection{Formes disponibles}
\SbSbSSCT{Formes disponibles}{Available shapes}

\label{nd1}

\begin{tabular}{|c|c|c|c|} \hline  
\multicolumn{4}{|l|}{ 2 syntaxes :   }\\ 
\multicolumn{4}{|l|}{ \BS{tikz} \BS{node}[fill=green!20,\RDD{shape}=diamond,draw,blue] \AC{texte};   }\\ 
\multicolumn{4}{|l|}{ \BS{tikz} \BS{node}[fill=green!20,\RDD{diamond},draw] \AC{texte};   }\\ 
\hline 
\tikz  \node[fill=green!20,diamond,draw] {texte}; 
&  
\tikz  \node[fill=green!20,ellipse,draw] {texte};
&  
\tikz  \node[fill=green!20,trapezium, regular polygon sides=6,draw] {texte};
&
\tikz  \node[fill=green!20,semicircle,draw] {texte}; 
\\ \hline 
diamond & ellipse  & trapezium & semicircle
\\ \hline 
\tikz  \node[fill=green!20,star,draw] {texte};
&  
\tikz  \node[fill=green!20,regular polygon,draw] {texte};
&  
\tikz  \node[fill=green!20,isosceles triangle,draw] {texte};
&
\tikz  \node[fill=green!20,kite,draw] {texte};
\\ \hline 
star & regular polygon  & isosceles triangle & kite 
\\ \hline 
\tikz  \node[fill=green!20,dart,draw] {texte};
&
\tikz  \node[fill=green!20,circular sector,draw] {texte};
&
\tikz  \node[fill=green!20,cylinder,draw] {texte};
&

\\ \hline 
dart & circular sector & cylinder &
\\ \hline 
\end{tabular} 

%---------------------------------------------------------------------------------------
\subsubsection{Options}

\begin{tabular}{|c|c|c|} \hline
\multicolumn{3}{|c|}{  \BS{node} [trapezium,draw,\RDD{trapezium left angle}=90,draw,blue] \AC{texte};   }\\ 
\hline
\begin{tikzpicture}
\node[trapezium,draw,red,dashed] {texte};
\node[trapezium,draw,trapezium left angle=90,draw,blue] {texte};
\end{tikzpicture}
& 
\begin{tikzpicture}
\node[trapezium,draw,red,dashed] {texte};
\node[trapezium,draw,trapezium right angle=90,draw,blue] {texte};
\end{tikzpicture} 
& 
\begin{tikzpicture}
\node[trapezium,draw,red,dashed] {texte};
\node[trapezium,draw,trapezium angle=120,draw,blue] {texte};
\end{tikzpicture} 
\\ \hline
\RDD{trapezium left angle}=90  & \RDD{trapezium right angle}=90  & \RDD{trapezium  angle}=120 \\ 
\hline 
\begin{tikzpicture}
\node[trapezium,draw,red,dashed] {texte};
\node[trapezium,draw,minimum height=1.5cm,trapezium stretches=true,draw,blue] {texte};
\end{tikzpicture}
& 
\begin{tikzpicture}
\node[trapezium,draw,red,dashed] {texte};
\node[trapezium,draw,minimum height=1.5cm,trapezium stretches=false,draw,blue] {texte};
\end{tikzpicture} 
& 
\begin{tikzpicture}
\node[trapezium,draw,red,dashed] {texte};
\node[trapezium,draw,minimum width=3cm,trapezium stretches =false,draw,blue] {texte};
\end{tikzpicture} 

\\ \hline
minimum height=1.5cm & minimum height=1.5cm & minimum width=1.5cm \\
\RDD{trapezium stretches}=true & \RDD{trapezium stretches}=false & \RDD{trapezium stretches}  \\ 
\hline
%
%& 
%\begin{tikzpicture}
%\node[trapezium,draw,red,dashed] {texte};
%\node[trapezium,draw,minimum width=1.5cm,trapezium stretches body=false,draw,blue] {texte};
%\end{tikzpicture} 
%&
%\\
\end{tabular} 

%\tikz  \draw (-1,-1) grid (1,1) (0,0) node[fill=red!20,shape=trapezium,draw,minimum height=1.5cm,trapezium stretches=true] {texte};
%
%\tikz  \draw (-1,-1) grid (1,1) (0,0) node[fill=red!20,shape=trapezium,draw,minimum height=1.5cm,trapezium stretches=false] {texte};
%
%\tikz  \draw (-1,-1) grid (1,1) (0,0) node[fill=red!20,shape=trapezium,draw,minimum width=1.5cm,trapezium stretches] {texte};
%
%\tikz  \draw (-1,-1) grid (1,1) (0,0) node[fill=red!20,shape=trapezium,draw,minimum width=1.5cm,trapezium stretches body] {texte};


\bigskip
\begin{tabular}{|c|c|c|} \hline
\multicolumn{3}{|c|}{ \BS{tikz} \BS{node} [fill=green!20,star,\RDD{star points}=6,draw] \AC{texte};   }\\ 
\hline
\begin{tikzpicture}
\node[star,draw,red,dashed] {texte};
\node[star,star points=7,draw,blue] {texte};
\end{tikzpicture}
&  
\begin{tikzpicture}
\node[star,draw,red,dashed] {texte};
\node[star,star point height = 2cm,draw,blue] {texte};
\end{tikzpicture} 
&  
\begin{tikzpicture}
\node[star,draw,red,dashed] {texte};
\node[star,star point ratio = 3,draw,blue] {texte};
\end{tikzpicture} 
\\ \hline  
\RDD{star points}=7 & \RDD{star point height} = 2cm & \RDD{star point ratio} = 3 \\ \hline
\dft{5} & \dft.5cm &  \dft{1.5}\\ 
\hline 
\end{tabular} 
\bigskip

\begin{tabular}{|c|c|c|} \hline
\multicolumn{3}{|c|}{  \BS{node} [isosceles triangle,\RDD{isosceles triangle apex angle}=90,draw,blue] \AC{texte};   }\\ 
\multicolumn{3}{|c|}{  \BS{node} [regular polygon, \RDD{regular polygon sides}=6,draw,blue] \AC{texte};   }\\ 
\hline
\begin{tikzpicture}
\node[isosceles triangle,draw,red,dashed] {texte};
 \node[isosceles triangle,isosceles triangle apex angle=90,draw,blue] {texte};
\end{tikzpicture} 
& 
\begin{tikzpicture}
\node[isosceles triangle,draw,red,dashed] {texte};
 \node[isosceles triangle,isosceles triangle stretches=true,draw,blue] {texte};
\end{tikzpicture}
&  
\begin{tikzpicture}
\node[regular polygon,draw,red,dashed] {texte};
\node[regular polygon, regular polygon sides=6,draw,blue] {texte};
\end{tikzpicture} 
\\ \hline  
\RDD{isosceles triangle apex angle}=90 & \RDD{isosceles triangle stretches} & \RDD{regular polygon sides}=6 \\ 
\hline 
\end{tabular} 
\bigskip

\begin{tabular}{|c|c|c|} \hline 
\multicolumn{3}{|c|}{  \BS{node} [kite,\RDD{kite upper vertex angle}=90,draw,blue] \AC{texte};   }\\ 
\hline 
\begin{tikzpicture}
\node[red,kite,draw,dashed] {texte} ;
 \node[kite,kite upper vertex angle=90,draw,blue] {texte};
\end{tikzpicture} 
&  
\begin{tikzpicture}
\node[red,kite,draw,dashed] {texte} ;
 \node[kite,kite lower vertex angle=90,draw,blue] {texte};
\end{tikzpicture} 
&  
\begin{tikzpicture}
\node[red,kite,draw,dashed] {texte} ;
\node[kite,kite vertex angles=90,draw,blue] {texte};
\end{tikzpicture} 
\\ \hline  
\RDD{kite upper vertex angle}=90 & \RDD{kite lower vertex angle}=90 &\RDD{kite vertex angles}=90
\\ \hline 
initially 120 & initially 60 &  \\ 
\hline 
\end{tabular} 

\bigskip

\begin{tabular}{|c|c|c|} \hline
\multicolumn{3}{|c|}{  \BS{node} [dart,\RDD{dart tip angle}=90,draw,blue] \AC{texte};   }\\ 
\hline 
\begin{tikzpicture}
\node[dart,draw,red,dashed] {texte};
\node[dart,dart tip angle=90,draw,blue] {texte};
\end{tikzpicture} 
&  
\begin{tikzpicture}
\node[dart,draw,red,dashed] {texte};
\node[dart,dart tail angle=90,draw,blue] {texte};
\end{tikzpicture} 
&  
\begin{tikzpicture}
\node[,circular sector,draw,red,dashed] {texte};
\node[circular sector,circular sector angle=90,draw,blue] {texte};
\end{tikzpicture} 
\\ \hline  
\RDD{dart tip angle}=90 & \RDD{dart tail angle}=90  & \RDD{circular sector angle}=90
\\ \hline  
initially 45 & initially 135 & initially 60  \\ 
\hline 
\end{tabular} 

\bigskip

\begin{tabular}{|c|c|} \hline  
\multicolumn{2}{|c|}{  \BS{node} [cylinder,\RDD{aspect=2},draw,blue] \AC{texte};   }\\ 
\hline
\tikz  \node[cylinder,aspect=2,draw,blue] {texte};
& 
 \tikz  \node[cylinder,aspect=4,draw,blue] {texte};
\\ \hline 
\RDD{aspect}=2 & \RDD{aspect}=4 
\\ \hline
\tikz  \node[cylinder,cylinder uses custom fill, cylinder end fill=yellow,draw,blue] {texte};
&  
\tikz  \node[cylinder,cylinder uses custom fill, cylinder body fill=yellow,draw,blue] {texte};
\\ \hline
\RDD{cylinder uses custom fill}, & \RDD{cylinder uses custom fill}, \\ 
\RDD{cylinder end fill}=yellow & \RDD{cylinder body fill}=yellow  \\ 
\hline 
\end{tabular} 

%\subsection{Ratio hauteur/largeur}
\bigskip

\begin{tabular}{|c|c|c|c|} \hline 
\multicolumn{4}{|c|}{  \BS{draw}(0,0) node[\RDD{shape aspect}=1,diamond,draw]  \AC{texte} ;   }
\\ \hline
 
\tikz \draw (0,0) node[shape aspect=1,diamond,draw,blue] {texte};
&  
\tikz \draw (0,-2) node[shape aspect=2,diamond,draw,blue] {texte};
&
\tikz \draw (0,0) node[shape aspect=3,diamond,draw,blue] {texte};
&
\tikz \draw (0,0) node[shape aspect=4,diamond,draw,blue] {texte};
\\ \hline  
\RDD{shape aspect}=1
&  
\RDD{shape aspect}=2
&
\RDD{shape aspect}=3
&
\RDD{shape aspect}=4
\\ \hline 
\end{tabular} 


%==============================================================
\newpage
%\subsection{Dans un n\oe ud en forme de symboles}
\SbSSCT{Dans un n\oe ud en forme de symboles}{Symbol Shapes nodes}
\label{lib-symb}

\maboite{\BS{usetikzlibrary}\AC{shapes.symbols}}

\begin{center}
\RRR{67-4}
\end{center}

%\subsubsection{Formes disponibles}
\SbSbSSCT{Formes disponibles}{Available shapes}

\label{nd2}

\begin{tabular}{|c|c|c|} \hline  
\tikz  \node[fill=green!20,forbidden sign,draw] {texte};
&  
\tikz  \node[fill=green!20,magnifying glass,draw] {texte};
&  
\tikz  \node[fill=green!20,cloud,draw] {texte};
\\ \hline 
forbidden sign & magnifying glass & cloud
\\ \hline  
\tikz  \node[fill=green!20,starburst,draw] {texte};
&  
\tikz  \node[fill=green!20,signal,draw] {texte};

&  
\tikz  \node[fill=green!20,tape,draw] {texte};
\\ \hline 
starburst & signal & tape
\\ \hline 
\end{tabular} 
\bigskip

\subsubsection{Options}

\begin{tabular}{|c|c|c|} \hline  
\multicolumn{3}{|c|}{  \BS{node}[magnifying glass,\RDD{magnifying glass handle angle}=45,draw,blue]  \AC{texte} ;   }
\\ \hline
\tikz  \node[magnifying glass,magnifying glass handle angle=45,draw,blue] {texte};
&  
\tikz  \node[,magnifying glass,magnifying glass handle aspect=3,draw,blue] {texte};
& 
\tikz  \node[magnifying glass,line width=1ex,draw,blue] {texte};

\\ \hline  
\RDD{magnifying glass handle angle}=45 & \RDD{magnifying glass handle aspect}=3  & line width=1ex  
\\ \hline 
\dft{ : -45} & \dft{ : 1.5}& 
\\ \hline 
\end{tabular} 

\bigskip

\begin{tabular}{|c|c|c|c|} \hline 
\multicolumn{4}{|c|}{  \BS{node} [cloud,\RDD{cloud puffs}=5,draw,blue] \AC{texte};   }\\ 
\hline 
\begin{tikzpicture}
\node[cloud,draw,red,dashed] {texte};
\node[cloud,cloud puffs=5,draw,blue] {texte};
\end{tikzpicture} 
&  
\begin{tikzpicture}
\node[cloud,draw,red,dashed] {texte};
\node[cloud,cloud puff arc=270,draw,blue] {texte};
\end{tikzpicture} 
&  
\begin{tikzpicture}
\node[cloud,draw,red,dashed] {texte};
\node[cloud,cloud ignores aspect=true,draw,blue] {texte};
\end{tikzpicture} 
&
\begin{tikzpicture}
\node[cloud,draw,red,dashed] {texte};
\node[cloud,cloud ignores aspect=false,draw,blue] {texte};
\end{tikzpicture} 
\\ \hline  
\RDD{cloud puffs}=5 & \RDD{cloud puff arc}=270 & \RDD{cloud ignores aspect}=false & \RDD{cloud ignores aspect}=true  \\ 
\hline 
\dft :  10 & \dft :  135 &\multicolumn{2}{|c|}{ \dft :  true } \\ \hline
\end{tabular} 

\bigskip

\begin{tabular}{|c|c|c|c|} \hline 
\multicolumn{4}{|c|}{  \BS{node} [starburst,\RDD{starburst points}=5,draw,blue] \AC{texte};   }\\ 
\hline  
\tikz  \node[starburst,starburst points=5,draw,blue] {texte};
&  
\tikz  \node[starburst,starburst point height=1cm,draw,blue] {texte};
&  
\tikz  \node[starburst,random starburst=50,draw,blue] {texte};
&
\tikz  \node[,starburst,random starburst=0,draw,blue] {texte};
\\ \hline  
\RDD{starburst points}=5 & \RDD{starburst point height}=1cm & \RDD{random starburst}=50 & \RDD{random starburst}=0  \\ 
\hline 
\end{tabular} 

\bigskip


\begin{tabular}{|c|c|c|} \hline 
\multicolumn{3}{|c|}{  \BS{node} [signal,\RDD{signal pointer angle}=45,draw,blue] \AC{texte};   }\\ 
\hline 
\tikz  \node[signal,signal pointer angle=45,draw,blue] {texte};
&
\tikz  \node[signal,signal pointer angle=10,draw,blue] {texte};
&
\tikz  \node[signal,signal pointer angle=300,draw,blue] {texte};
\\ \hline 
\RDD{signal pointer angle}=45
&
signal pointer angle=10
&
signal pointer angle=300
\\ \hline 
\multicolumn{3}{|c|}{  \dft{ : signal pointer angle= 90}  }
\\  \hline 

\end{tabular} 
\bigskip

\begin{tabular}{|c|c|c|c|c|} \hline 
\multicolumn{4}{|c|}{  \BS{node} [signal,\RDD{signal to}=above,draw,blue] \AC{texte};   }
\\ \hline 
\tikz  \node[signal,signal to=above,draw,blue] {texte};
&  
\tikz  \node[signal,signal to=below,draw,blue] {texte};
&
\tikz  \node[signal,signal to=right,draw,blue] {texte};
&
\tikz  \node[signal,signal to=above,draw,blue] {texte};
\\ \hline  
  \RDD{signal to}=above  & \RDD{signal to}=below & \RDD{signal to}=right  & \RDD{signal to}=above \\ 
\hline 
\end{tabular} 
\bigskip

\begin{tabular}{|c|c|c|c|c|} \hline 
\multicolumn{4}{|c|}{ \BS{tikz} [signal to=nowhere] \BS{node} [signal,\RDD{signal from=above}=45,draw,blue] \AC{texte};   }\\ 
\hline 
\tikz [signal to=nowhere] \node[signal,signal from=above,draw,blue] {texte};
&  
\tikz [signal to=nowhere] \node[signal,signal from=below,draw,blue] {texte};
&
\tikz [signal to=nowhere] \node[signal,signal from=right,draw,blue] {texte};
&
\tikz [signal to=nowhere] \node[signal,signal from=above,draw,blue] {texte};
\\ \hline  
  \RDD{signal from}=above  & \RDD{signal from}=below & \RDD{signal from}=right  & \RDD{signal from}=above \\ 
\hline 
\end{tabular} 

\bigskip
\begin{tabular}{|c|c|c|c|} \hline
\multicolumn{2}{|c|}{ \tikz  \node[draw,signal, signal from=east , signal to=west,blue] at (0,0) {texte};}
&
\multicolumn{2}{|c|}{ \tikz  \node[draw,signal,signal from=south, signal to=north,blue] at (0,0) {texte};}
\\ \hline 
\multicolumn{2}{|c|}{ \RDD{signal from}=east , \RDD{signal to}=west}
&
\multicolumn{2}{|c|}{\RDD{signal from}=south, \RDD{signal to}=north}

\\ \hline 
\end{tabular}
\bigskip

\begin{tabular}{|c | c | c | c |} \hline
\multicolumn{3}{|c|}{ \BS{tikz} \BS{node}  [tape, draw,\RDD{tape bend top}=out and in] \AC{texte};   }\\ 
\hline  
\tikz \node [tape, draw,tape bend top=out and in,blue] {texte};
&
\tikz \node [tape, draw, tape bend bottom=out and in,blue] {texte};
&
\tikz \node [tape, draw, tape bend bottom=in and in,blue] {texte};
 \\  \hline
 \RDD{tape bend top}=out and in & \RDD{tape bend bottom}=out and in &  \RDD{tape bend bottom}=in and in 
  \\  \hline
 \tikz \node [tape, draw, tape bend top=none,blue] {texte};
 &
 \tikz \node [tape, draw,tape bend top=out and in,tape bend bottom=out and in,blue] {texte};
 &
  \tikz \node [tape, draw,tape bend top=in and out,tape bend bottom=in and out,blue] {texte};
  \\  \hline
 \RDD{tape bend top}=none & \RDD{tape bend bottom}=out and in 	&  \RDD{tape bend bottom}=in and out  \\
 					& \RDD{tape bend top}=out and in 		& \RDD{tape bend top}=in and out  \\
 					& & (\dft{} ) 
  \\  \hline 
\end{tabular}
\bigskip

\begin{tabular}{|c | c | c | c |} \hline
\BS{tikz} \BS{node} [tape, draw, \RDD{tape bend height}=1cm,blue] \AC{texte}; 
  \\  \hline 
\tikz \node [tape, draw, tape bend height=1cm,blue] {texte};

  \\  \hline 
\dft{ : tape bend height = 5pt}
  \\  \hline 
\end{tabular}
%=============================================================
\newpage
%\subsection{Dans un n\oe ud en forme de flèche}
\SbSSCT{Dans un n\oe ud en forme de flèche}{Arrow Shapes nodes}

\label{lib-arr}

\maboite{\BS{usetikzlibrary}\AC{shapes.arrows}}

\begin{center}
\RRR{67-5}
\end{center}
%\subsubsection{Formes disponibles}
\SbSbSSCT{Formes disponibles}{Available shapes}
\label{nd3}

\begin{tabular}{|c|c|c|} \hline  
\tikz \node[fill=green!20,single arrow,draw] {texte};
&  
\tikz  \node[fill=green!20,double arrow,draw] {texte};
&  
\tikz  \node[fill=green!20,arrow box,draw] {texte};
\\ \hline 
single arrow & double arrow & arrow box \\ 
\hline 
\end{tabular} 

\subsubsection{Options}

\begin{tabular}{|c|c|c|c|c|} \hline  
 \multicolumn{5}{|c|}{  \BS{node}[single arrow,draw,\RDD{single arrow tip angle}=45] \AC{texte};   }\\ 
  \multicolumn{5}{|c|}{  \BS{node}[single arrow,draw,\RDD{single arrow head extend}=.75cm] \AC{texte};   }\\
 \hline
\begin{tikzpicture}
 \node[single arrow,draw,red,dashed,text=black] {texte};
 \node[single arrow,draw,single arrow tip angle=45,blue] {texte};
\end{tikzpicture}
&
\begin{tikzpicture}
 \node[single arrow,draw,red,dashed,text=black] {texte};
\node[single arrow,draw,single arrow tip angle=120,blue] {texte};
\end{tikzpicture}
&
\begin{tikzpicture}
 \node[single arrow,draw,red,dashed,text=black] {texte};
 \node[single arrow,draw,single arrow head extend=.75cm,blue] {texte};
\end{tikzpicture}
&
\begin{tikzpicture}
 \node[single arrow,draw,red,dashed,text=black] {texte};
 \node[single arrow,draw,single arrow head extend=0cm,blue] {texte};
 \end{tikzpicture}
 &
 \begin{tikzpicture}
  \node[single arrow,draw,red,dashed,text=black] {texte};
  \node[single arrow,draw,single arrow head extend=-1mm,blue] {texte};
 \end{tikzpicture}

\\ \hline
angle=45 & angle=120 & extend=.75cm] & extend=0cm & extend=-1mm
\\ \hline 
\multicolumn{2}{|c|}{  \dft : single arrow tip angle= 90   }
&
\multicolumn{3}{|c|}{  \dft : single arrow head extend=0.5cm   }
\\ \hline 
\end{tabular} 
\bigskip


\begin{tabular}{|c|c|c|c|} \hline
 \multicolumn{4}{|c|}{  \BS{node}[minimum size=2cm,single arrow,draw,\RDD{single arrow head indent}=1cm,blue] \AC{texte};   }\\ 
 \hline   
\begin{tikzpicture}
 \node[minimum size=2cm,single arrow,draw,red,dashed,text=black] {texte};
\node[minimum size=2cm,single arrow,draw,single arrow head indent=1cm,blue] {texte};
\end{tikzpicture}
&
\begin{tikzpicture}
 \node[minimum size=2cm,single arrow,draw,red,dashed,text=black] {texte};
  \node[minimum size=2cm,single arrow,draw,single arrow head indent=10pt,blue] {texte};
  \end{tikzpicture}
&
\begin{tikzpicture}
 \node[minimum size=2cm,single arrow,draw,red,dashed,text=black] {texte};
  \node[minimum size=2cm,single arrow,draw,single arrow head indent=1ex,blue] {texte};
  \end{tikzpicture}
  &
  \begin{tikzpicture}
   \node[minimum size=2cm,single arrow,draw,red,dashed,text=black] {texte};
    \node[minimum size=2cm,single arrow,draw,single arrow head indent=-1ex,blue] {texte};
    \end{tikzpicture}
\\ \hline
indent=1cm & indent=10pt & indent=1ex & indent=-1ex
\\ \hline 
\end{tabular}
\bigskip

 



\begin{tabular}{|c|c|c|c|c|} \hline
 \multicolumn{5}{|c|}{  \BS{node}[minimum size=2cm,double arrow,draw,\RDD{double arrow tip angle}=45] \AC{texte};   }\\ 
  \multicolumn{5}{|c|}{  \BS{node}[minimum size=2cm,double arrow,draw,\RDD{double arrow head extend}=1ex] \AC{texte};   }\\
   \multicolumn{5}{|c|}{  \BS{node}[minimum size=2cm,double arrow,draw,\RDD{double arrow head indent}=1ex] \AC{texte};   }\\ 
 \hline  
\begin{tikzpicture}
\node[minimum size=2cm,double arrow,draw,red,dashed,text=black] {texte};
\node[minimum size=2cm,double arrow,draw,double arrow tip angle=45,blue] {texte};
\end{tikzpicture}
&
\begin{tikzpicture}
\node[minimum size=2cm,double arrow,draw,red,dashed,text=black] {texte};
\node[minimum size=2cm,double arrow,draw,double arrow tip angle=120,blue] {texte};
\end{tikzpicture}
&
\begin{tikzpicture}
 \node[minimum size=2cm,double arrow,draw,red,dashed,text=black] {texte};
 \node[minimum size=2cm,double arrow,draw,double arrow head extend=1ex,blue] {texte};
   \end{tikzpicture}
&
\begin{tikzpicture}
 \node[minimum size=2cm,double arrow,draw,red,dashed,text=black] {texte};
  \node[minimum size=2cm,double arrow,draw,double arrow head extend=0,blue] {texte};
    \end{tikzpicture}
&
\begin{tikzpicture}
 \node[minimum size=2cm,double arrow,draw,red,dashed,text=black] {texte};
  \node[,minimum size=2cm,double arrow,draw,double arrow head indent=1ex,blue] {texte};
    \end{tikzpicture}
\\ \hline 
angle=45 & angle=120 & extend=1ex & extend=0 & indent=1ex
\\ \hline
\end{tabular}

\bigskip

\begin{tabular}{|c|c|c|c|c|} \hline
\multicolumn{4}{|c|}{ \BS{node} [arrow box, draw, \RDD{arrow box arrows}=\AC{north:.25cm}] \AC{texte}; }\\ 
\hline 
\begin{tikzpicture}
\node[arrow box, draw,red,text=white,dashed] {texte};
\node[arrow box, draw, arrow box arrows={north:.25cm},blue] {texte};
\end{tikzpicture}
& 
\begin{tikzpicture}
\node[arrow box, draw,red,text=white,dashed] {texte};
\node[arrow box, draw, arrow box arrows={west:.25cm},blue] {texte};
\end{tikzpicture}
 &
 \begin{tikzpicture}
 \node[arrow box, draw,red,text=white,dashed] {texte};
 \node[arrow box, draw, arrow box arrows={south:.25cm},blue] {texte};
 \end{tikzpicture}
&
 \begin{tikzpicture}
 \node[arrow box, draw,red,text=white,dashed] {texte};
 \node[arrow box, draw, arrow box arrows={east:.25cm},blue] {texte};
 \end{tikzpicture}   
 \\ \hline
\AC{north:.25cm} & \AC{west:.25cm} & \AC{south:.25cm}& \AC{east:.25cm} 
\\ \hline
\multicolumn{4}{|c|}{  \dft{} : 0.5 cm}
 \\ \hline 
 \end{tabular}
 
 
 \bigskip
 
 \begin{tabular}{|c|c|} \hline
 \multicolumn{2}{|c|}{ \BS{node} [arrow box, draw, \RDD{arrow box tip angle}=45] \AC{texte}; }\\ 
 \hline 
  \begin{tikzpicture}
  \node[arrow box, draw,red,text=white,dashed] {texte};
  \node[arrow box, draw, arrow box tip angle=45,blue] {texte};
  \end{tikzpicture} 
  &
    \begin{tikzpicture}
   \node[arrow box, draw,red,text=white,dashed] {texte};
   \node[arrow box, draw, arrow box head extend=.25cm,blue] {texte};
   \end{tikzpicture}
\\ \hline  
\RDD{arrow box tip angle}=45 & \RDD{arrow box head extend}=.25cm
\\ \hline 
\dft : 90  & \dft : 0.125cm 
\\ \hline 
   \begin{tikzpicture}
   \node[arrow box, draw,red,text=white,dashed] {texte};
   \node[arrow box, draw, arrow box head indent=.25cm,blue] {texte};
   \end{tikzpicture} 
 &
    \begin{tikzpicture}
    \node[arrow box, draw,red,text=white,dashed] {texte};
    \node[arrow box, draw,arrow box shaft width=.25cm,blue] {texte};
    \end{tikzpicture} 
 \\ \hline 
\RDD{arrow box head indent}=.25cm  &  \RDD{arrow box shaft width}=.25cm
 \\ \hline  
 \dft{ : 0cm } &  \dft{ : 0.125cm }
 \\ \hline  
 \end{tabular}



\newpage
%-----------------------------------------------------------------------
%\subsection{Dans un n\oe ud en forme de bulle}
\SbSSCT{Dans un n\oe ud en forme de bulle}{Callout Shapes nodes}
\label{lib-call}

%insérer dans le préambule : 

 \maboite{\BS{usetikzlibrary}\AC{shapes.callouts}}
 
\begin{center}
\RRR{67-7}
\end{center}
%\subsubsection{Formes disponibles}
\SbSbSSCT{Formes disponibles}{Available shapes}

\begin{tabular}{|c|c|c|} \hline 
\tikz  \node[fill=green!20,ellipse callout,draw] {texte};
 &  
 \tikz  \node[fill=green!20,rectangle callout,draw] {texte};
  &  
  \tikz  \node[fill=green!20,cloud callout,draw] {texte};
 \\ \hline
 ellipse callout  &  rectangle callout  & cloud callout \\ 
\hline 
\end{tabular} 
%------------------------------------------------

\subsubsection{Options}


\begin{tabular}{|c | c | c | c |} \hline
\multicolumn{4}{|c|}{  \BS{node} [rectangle callout,draw,\RDD{callout absolute pointer}={(0,1)}] at (2,1) \AC{texte};   }\\ 
\hline 
\begin{tikzpicture} 
\draw [help lines] grid(3,3);
\node [rectangle callout,draw,blue, callout relative pointer={(0,1)}] at (2,1) {texte};
\end{tikzpicture}
&
\begin{tikzpicture} 
\draw [help lines] grid(3,3);
\node [ellipse callout,draw, callout relative pointer={(0,1)},blue] at (2,1) {texte};
\end{tikzpicture}
&
\begin{tikzpicture} 
\draw [help lines] grid(3,3);
\node [rectangle callout,draw,blue,callout absolute pointer={(0,1)}] at (2,1) {texte};
\end{tikzpicture}
&
\begin{tikzpicture} 
\draw [help lines] grid(3,3);
\node [ellipse callout,draw, callout absolute pointer={(0,1)},blue] at (2,1) {texte};
\end{tikzpicture}
 \\  \hline
\multicolumn{2}{|c|}{ \RDD{callout relative pointer}=\AC{(0,1)} } & 
\multicolumn{2}{|c|}{  \RDD{callout absolute pointer}=\AC{(0,1)} }
 \\  \hline 
 \begin{tikzpicture} 
 \draw [help lines] grid(3,3);
 \node [rectangle callout,draw, callout relative pointer={(0,1)},callout pointer shorten=.5cm,blue] at (2,1) {texte};
 \end{tikzpicture}
 &
  \begin{tikzpicture} 
  \draw [help lines] grid(3,3);
  \node [ellipse callout,draw, callout relative pointer={(0,1)},callout pointer shorten=.5cm,blue] at (2,1) {texte};
  \end{tikzpicture}
  &
 \begin{tikzpicture} 
 \draw [help lines] grid(3,3);
 \node [rectangle callout,draw, callout absolute pointer={(0,1)},callout pointer shorten=.5cm,blue] at (2,1) {texte};
 \end{tikzpicture}
  &
  \begin{tikzpicture} 
  \draw [help lines] grid(3,3);
  \node [ellipse callout,draw, callout absolute pointer={(0,1)},callout pointer shorten=.5cm,blue] at (2,1) {texte};
  \end{tikzpicture}
  \\  \hline
\multicolumn{4}{|c|}{ \RDD{callout pointer shorten}=.5cm} 
  \\  \hline 
\end{tabular}

%-------------------------------------------------------------

\bigskip
 


\bigskip
\begin{tabular}{|c | c | c | c |} \hline
\multicolumn{3}{|c|}{  \BS{node} [ellipse callout,draw,\RDD{callout pointer arc}=1] at (0,1.5) \AC{texte};   }\\ 
\hline
\begin{tikzpicture}
\node[ellipse callout,draw, callout pointer arc=1,blue] at (0,1.5) {texte};
\end{tikzpicture}
&
\begin{tikzpicture}
\node[ellipse callout,draw, callout pointer arc=30,blue] at (0,1.5) {texte};
\end{tikzpicture}
 &
\begin{tikzpicture}
\node[ellipse callout,draw, callout pointer arc=90,blue] at (0,1.5) {texte};
\end{tikzpicture}
  \\  \hline 
   callout pointer arc=1 & callout pointer arc=30 & callout pointer arc=90
  \\  \hline  
  \multicolumn{3}{|c|}{  \dft{ : callout pointer arc=15}}
 \\  \hline  
 \end{tabular}

\bigskip

\begin{tabular}{|c | c | c | c |} \hline
\multicolumn{3}{|c|}{  \BS{node}[draw,cloud callout, aspect=2.5] \AC{texte};   }\\ 
\hline 
 \begin{tikzpicture}
  \node[draw,cloud callout, dashed,red,text=black] {texte};
 \node[draw,cloud callout, cloud puffs=5,blue] {texte};
 \end{tikzpicture}
&
 \begin{tikzpicture}
 \node[draw,cloud callout, dashed,red,text=black] {texte};
 \node[draw,cloud callout, aspect=2.5,blue] {texte};
 \end{tikzpicture}
&
  \begin{tikzpicture}
  \node[draw,cloud callout, dashed,red,text=black] {texte};
  \node[draw,cloud callout,cloud puff arc=120,blue] {texte};
  \end{tikzpicture}
   \\  \hline 
cloud puffs=5 & aspect=2.5 &  cloud puff arc=120
\\  \hline 
 \end{tabular}

\bigskip

\begin{tabular}{|c | c | c | c |c |} \hline
\multicolumn{3}{|c|}{  \BS{node} [draw,cloud callout,\RDD{callout pointer start size}=.1] \AC{texte};   }\\ 
\hline 
  \begin{tikzpicture}
  \node[draw,cloud callout, dashed,red,text=black] {texte};
  \node[draw,cloud callout,callout pointer start size=.1,blue] {texte};
  \end{tikzpicture}
&
  \begin{tikzpicture}
  \node[draw,cloud callout, dashed,red,text=black] {texte};
  \node[draw,cloud callout,callout pointer start size=.8cm,blue] {texte};
  \end{tikzpicture}
&
  \begin{tikzpicture}
  \node[draw,cloud callout, dashed,red,text=black] {texte};
 \node[draw,cloud callout,callout pointer start size=1cm and 0.1cm,blue] {texte};
  \end{tikzpicture}
\\  \hline 
\RDD{callout pointer start size}=.1 &start size=.8cm & start size=20pt and 1pt
\\  \hline 
\multicolumn{3}{|c|}{  \dft{} : callout pointer start size =.2 of callout  }
\\ 
\hline 
  \begin{tikzpicture}
  \node[draw,cloud callout, dashed,red,text=black] {texte};
  \node[draw,cloud callout,callout pointer end size=5,blue] {texte};
  \end{tikzpicture}
&
  \begin{tikzpicture}
  \node[draw,cloud callout, dashed,red,text=black] {texte};
  \node[draw,cloud callout,callout pointer end size=.8cm,blue] {texte};
  \end{tikzpicture}
&
    \begin{tikzpicture}
    \node[draw,cloud callout, dashed,red,text=black] {texte};
    \node[draw,cloud callout,callout pointer segments=3,blue] {texte};
    \end{tikzpicture}
\\  \hline 
\RDD{callout pointer end size}=.5 & \RDD{callout pointer end size}=.8cm & \RDD{callout pointer segments}=3
\\  \hline 
\multicolumn{2}{|c|}{  \dft{} : callout pointer start size = .1 of callout  }
& \dft{} : segments=2
\\  \hline  

 \end{tabular}
 


%----------------------------------------------------------------------
\newpage

%\subsection{Dans un n\oe ud en diverses formes  diverses}

\SbSSCT{Dans un n\oe ud en diverses formes  diverses}{Miscellaneous Shapes nodes}

\label{lib-misc}

%insérer dans le préambule:

 \maboite{\BS{usetikzlibrary}\AC{shapes.misc}}
 
\begin{center}
\RRR{67-8}
\end{center}

%\subsubsection{formes disponibles}
\SbSbSSCT{Formes disponibles}{Available shapes}

\begin{tabular}{|c|c|c|c|} \hline  
\tikz  \node[fill=green!20,cross out,draw] {texte};
&  
\tikz  \node[fill=green!20,strike out,draw] {texte};
&  
\tikz  \node[fill=green!20,rounded rectangle,draw] {texte};
&  
\tikz  \node[fill=green!20,chamfered rectangle,draw] {texte};
\\ \hline  
cross out & strike out & rounded rectangle & chamfered rectangle \\ 
\hline 
\end{tabular} 


\subsubsection{Options}

\paragraph{Options \TFRGB{pour}{for} \og rounded rectangle \fg} :


%
\begin{tabular}{|c|c|c|c|c|} \hline
\multicolumn{5}{|c|}{  \BS{node} [draw, rounded rectangle,\RDD{rounded rectangle arc length}=270] \AC{texte};   }\\ 

\hline 

%\begin{tikzpicture}
\tikz \node[draw, rounded rectangle,rounded rectangle arc length=270,blue] {texte}; 
&
\tikz \node[draw, rounded rectangle,rounded rectangle arc length=180,blue]  {texte}; 
&
\tikz \node[draw, rounded rectangle,rounded rectangle arc length=120,blue] {texte}; 
&
\tikz \node[draw, rounded rectangle,rounded rectangle arc length=90,blue]  {texte}; 
&
\tikz \node[draw, rounded rectangle,rounded rectangle arc length=45,blue] {texte}; 
 \\ \hline 
270 & 180 & 120 & 90& 45 
\\ \hline 
%\end{tikzpicture}

\end{tabular} 

\bigskip


\begin{tabular}{|c|c|c|c|} \hline 
\multicolumn{4}{|c|}{  \BS{node} [draw, rounded rectangle,\RDD{rounded rectangle west arc}=concave] \AC{texte};   }\\ 
\multicolumn{4}{|c|}{  \BS{node} [draw, rounded rectangle,\RDD{rounded rectangle left arc}=concave] \AC{texte};   }\\ 
\hline 
\tikz \node[draw, rounded rectangle,rounded rectangle west arc=concave,blue] {texte}; 
&
\tikz \node[draw, rounded rectangle,rounded rectangle left arc=concave,blue] {texte}; 
&
\tikz \node[draw, rounded rectangle,rounded rectangle west arc=convex,blue] {texte}; 
&
\tikz \node[draw, rounded rectangle,rounded rectangle left arc=none,blue] {texte};
 \\\hline 
concave & convex & none 
 \\\hline 
\end{tabular} 

\bigskip

\begin{tabular}{|c|c|c|c|} \hline 
\multicolumn{3}{|c|}{  \BS{node} [draw, rounded rectangle,\RDD{rounded rectangle east arc}=concave] \AC{texte};   }\\ 
\multicolumn{3}{|c|}{  \BS{node} [draw, rounded rectangle,\RDD{rounded rectangle right arc}=concave] \AC{texte};   }\\ 

\hline 
\tikz \node[draw, rounded rectangle,rounded rectangle east arc=concave,blue] {texte}; 
&
\tikz \node[draw, rounded rectangle,rounded rectangle  east arc=convex,blue] {texte}; 
&
\tikz \node[draw, rounded rectangle,rounded rectangle right arc=none,blue] {texte};
 \\\hline 
concave & convex & none 
 \\\hline 
\end{tabular} 

\paragraph{Options  \TFRGB{pour}{for} \og chamfered rectangle \fg} :


\begin{tabular}{|c|c|c|c|} \hline 
\multicolumn{4}{|c|}{  \BS{node} [draw, chamfered rectangle,\RDD{chamfered rectangle angle}=30] \AC{texte};   }\\ 
\hline 
\tikz \node[draw, chamfered rectangle,chamfered rectangle angle=10,blue] {texte}; 
&
\tikz \node[draw, chamfered rectangle,chamfered rectangle angle=30,blue] {texte}; 
&
\tikz \node[draw,chamfered rectangle,chamfered rectangle angle=60,blue] {texte};
&
\tikz \node[draw,chamfered rectangle,chamfered rectangle angle=80,blue] {texte};
 \\ \hline 
10 & 30 & 60 & 80
\\ \hline 
\multicolumn{4}{|c|}{  \dft :  45 }
  \\\hline  

\end{tabular}

\bigskip

\begin{tabular}{|c|c|c|c|c|} \hline 
\multicolumn{5}{|c|}{  \BS{node} [draw, chamfered rectangle,\RDD{chamfered rectangle xsep}=10pt] \AC{texte};   }\\ 
\hline 
\tikz \node[draw, chamfered rectangle,chamfered rectangle xsep=0pt,blue] {texte}; 
&
\tikz \node[draw, chamfered rectangle,chamfered rectangle xsep=5pt,blue] {texte}; 
&
\tikz \node[draw, chamfered rectangle,chamfered rectangle xsep=10pt,blue] {texte}; 
&
\tikz \node[draw,chamfered rectangle,chamfered rectangle xsep=-10pt,blue] {texte};
&
\tikz \node[draw,chamfered rectangle,chamfered rectangle xsep=2cm,blue] {texte};
 \\\hline 
  xsep=0pt & xsep=5pt & xsep=10pt & xsep=-10pt  & xsep=2cm
  \\\hline  
\multicolumn{5}{|c|}{  \dft :  0.666ex }
  \\\hline   
\end{tabular}

\bigskip

\begin{tabular}{|c|c|c|c|c|} \hline 
\multicolumn{5}{|c|}{  \BS{node} [draw, chamfered rectangle,\RDD{chamfered rectangle ysep}=10pt] \AC{texte};   }\\ 
\hline 
\tikz \node[draw, chamfered rectangle,chamfered rectangle ysep=0pt,blue] {texte}; 
&
\tikz \node[draw, chamfered rectangle,chamfered rectangle ysep=5pt,blue] {texte}; 
&
\tikz \node[draw,chamfered rectangle,chamfered rectangle ysep=10pt,blue] {texte};
&
\tikz \node[draw,chamfered rectangle,chamfered rectangle ysep=-10pt,blue] {texte};
&
\tikz \node[draw,chamfered rectangle,chamfered rectangle ysep=1cm,blue] {texte};
 \\ \hline 
 ysep=0pt & ysep=5pt & ysep=10pt & ysep=-10pt & ysep=1cm
 \\\hline  
\end{tabular}

\bigskip

\begin{tabular}{|c|c|c|c|c|} \hline 
\multicolumn{5}{|c|}{  \BS{node} [draw, chamfered rectangle,\RDD{chamfered rectangle ysep}=10pt] \AC{texte};   }\\ 
\hline 
\tikz \node[draw, chamfered rectangle,chamfered rectangle sep=0pt,blue] {texte}; 
&
\tikz \node[draw, chamfered rectangle,chamfered rectangle sep=5pt,blue] {texte}; 
&
\tikz \node[draw, chamfered rectangle,chamfered rectangle sep=10pt,blue] {texte}; 

&
\tikz \node[draw, chamfered rectangle,chamfered rectangle sep=-10pt,blue] {texte}; 
&
\tikz \node[draw,chamfered rectangle,chamfered rectangle sep=1cm,blue] {texte};
 \\\hline 
 sep=0pt & sep=5pt & sep=10pt& sep=-10pt & sep=1cm
 \\\hline  
\end{tabular}

\bigskip

\begin{tabular}{|c|c|c|c|} \hline 
\multicolumn{3}{|c|}{  \BS{node} [draw, chamfered rectangle,\RDD{chamfered rectangle corners}=north west] \AC{texte};   }\\ 
\hline
\tikz \node[draw, chamfered rectangle,chamfered rectangle corners=north west,blue] {texte}; 
&
\tikz \node[draw, chamfered rectangle,chamfered rectangle corners={north east, south east},blue] {texte}; 
&
\tikz \node[draw,chamfered rectangle,chamfered rectangle corners={north east, south west},blue] {texte};
 \\ \hline 
 north west & \AC{north east, south east}  & \AC{north east, south west}
 \\ \hline 
\end{tabular}





%\begin{tikzpicture}
%\tikzset{every node/.style={chamfered rectangle, draw}}
%\node[chamfered rectangle corners=north west] {ghi};
%\node[chamfered rectangle corners={north east, south east}] at (1.5,0) {789};
%\end{tikzpicture}


%\begin{tikzpicture}
%\tikzset{every node/.style={chamfered rectangle, draw}}
%\node[chamfered rectangle xsep=2pt] {def};
%\node[chamfered rectangle xsep=2cm] at (1.5,0) {456};
%\end{tikzpicture}

%\begin{tikzpicture}
%\tikzset{every node/.style={chamfered rectangle, draw}}
%\node[chamfered rectangle angle=30] {abc};
%\node[chamfered rectangle angle=60] at (1.5,0) {123};
%\end{tikzpicture}

%\begin{tikzpicture}
%\matrix[row sep=5pt, every node/.style={draw, rounded rectangle}]{
%\node[rounded rectangle west arc=concave] {Concave}; \\
%\node[rounded rectangle west arc=convex] {Convex}; \\
%\node[rounded rectangle left arc=none] {None}; \\};
%\end{tikzpicture}
%\tikz  \draw (-1,-1) grid (1,1) (0,0) node[fill=red!20,diamond,draw,rounded corners] {texte};&
 
%------------------------------------------------------------------------------------------

\newpage
%\subsection{N\oe uds à plusieurs parties}
\SbSSCT{N\oe uds à plusieurs parties}{Shapes with Multiple Text Parts}

\label{lib-mult}

%insérer dans le préambule :

 \maboite{\BS{usetikzlibrary}\AC{shapes.multipart}}

\begin{center}
\RRR{67-6}
\end{center}



\begin{tabular}{|c|c|c|c|} \hline 
\multicolumn{4}{|c|}{  \BS{node} [\RDD{circle split},draw,fill=green!20]\AC{haut  \BSS{nodepart}\AC{lower} bas };   }\\ 
\hline 
 
\tikz  \node [circle split,draw,blue,fill=green!20] {haut  \nodepart{lower} bas }; % \filldraw[fill=red] (0,0) circle (3pt);

&  
\tikz  \node [circle solidus,draw,blue,fill=green!20]{haut  \nodepart{lower} bas };
&  
\tikz  \node [ellipse split,draw,blue,fill=green!20]{texte haut  \nodepart{lower} texte bas };
& 
\tikz  \node [rectangle split,draw,blue,fill=green!20]{haut  \nodepart{lower} bas}; 
%\tikz  \node [rectangle split ,draw,fill=green!20]{a\nodepart{two}b\nodepart{three}c\nodepart{four}d\nodepart{five}e};
\\ \hline 
\RDD{circle split} & \RDD{circle solidus} & \RDD{ellipse split} & \RDD{rectangle split} \\ 
\hline 
\end{tabular} 

 \bigskip
 
 \begin{tabular}{|c|c|}  \hline  
 \begin{tikzpicture} [baseline=0pt]%[every text node part/.style={text centered}]
 \node[rectangle split,rectangle split parts=5,draw,blue,fill=green!20] at(0,0)
 {texte 1
 \nodepart{second}
 texte 2
 \nodepart{four}
 texte 3};
 \end{tikzpicture}
&
\parbox[c]{10cm}{
 \BS{node}[rectangle split,\RDD{rectangle split parts}=5,\\
 draw] \\
 \AC{texte 1 \\
 \BSS{nodepart}\AC{second} texte 2 \\
 \BSS{nodepart}\AC{four} texte 3}; \\
 \\
\dft : rectangle split parts=4 }
 \\  \hline 
 \end{tabular} 
 
\bigskip

\begin{tabular}{|c|}\hline  
\BS{node} [rectangle split,rectangle split parts=3,\RDD{rectangle split horizontal},draw,blue] \\
\AC{texte1\BSS{nodepart}\AC{two}texte2\BSS{nodepart}\AC{three}texte3};
\\ \hline  
\tikz \node [rectangle split,rectangle split parts=3, rectangle split horizontal,draw,blue]
{texte 1\nodepart{two}texte 2\nodepart{three}texte 3}; 
\\ \hline 
\end{tabular} 
 
 \bigskip
 
% % % <<<<<<<<<<<<<<<<< A Voir rectangle split allocate boxes= >>>>>>>>>>>>>>>>>>>>>>>>>>>>>>>>

% \begin{tikzpicture} [baseline=0pt]%[every text node part/.style={text centered}]
% \node[rectangle split,draw,rectangle split parts=5,fill=green!20,rectangle split allocate boxes=3] at(0,0)
% {texte 1  \nodepart{second}  texte 2  \nodepart{four}  texte 3};
% \end{tikzpicture}
% 
 
\bigskip
 \begin{tabular}{|c|c|}  \hline  
\begin{tikzpicture}[baseline=0pt] %[every text node part/.style={align=center}]
\node[rectangle split, rectangle split parts=3, draw,blue, text width=2.75cm]
{texte 1
\nodepart{two}
texte 2a \\
texte 2b \\
texte 2c
\nodepart{three}
texte 3a \\
texte 3b};
\end{tikzpicture}
&
\parbox{8cm}{
 \BS{node}[rectangle split,\RDD{rectangle split parts}=5, draw] \\
 \AC{texte 1 \\
 \BSS{nodepart}\AC{second} texte 2a  \BS{}\BS{}texte 2b  \BS{}\BS{}  texte 2c \\
 \BSS{nodepart}\AC{three} texte 3a \BS{}\BS{} texte 3b }; \\
}
 \\  \hline 
 \end{tabular} 
\bigskip
%---------------------------------------------------------------------------------

 \begin{tabular}{|c|c|}  \hline  
 \multicolumn{2}{|c|}{  \BS{node}[rectangle split, draw,blue,minimum size = 2cm,\RDD{rectangle split draw splits}= true] } \\
  \multicolumn{2}{|c|}{ 
  \AC{texte 1 \BS{nodepart}\AC{two} texte 2 \BS{nodepart}\AC{three} texte 3 \BS{nodepart}\AC{four} texte 4};   }\\ 
 \hline 
\tikz \node[rectangle split, draw,blue,minimum size = 2cm,rectangle split draw splits= true] {texte 1 \nodepart{two} texte 2 \nodepart{three} texte 3 \nodepart{four} texte 4};
&
\tikz \node[rectangle split, draw,blue,minimum size = 2cm,rectangle split draw splits= false] {texte 1 \nodepart{two} texte 2 \nodepart{three} texte 3 \nodepart{four} texte 4};
 \\ \hline
 \RDD{rectangle split draw splits}= true & \RDD{rectangle split draw splits}= false \\
 \dft &
 \\ \hline 
 \end{tabular}
 
\bigskip

 \begin{tabular}{|c|c|}  \hline  
\multicolumn{2}{|c|}{  
\BS{node} [rectangle split,rectangle split parts=3,draw,\RDD{rectangle split ignore empty parts}=false] }\\
 \multicolumn{2}{|c|}{ \AC{texte 1 \BS{nodepart}\AC{second} \BS{nodepart}\AC{third}texte 3};} 
\\ \hline  
\begin{tikzpicture} 
\node[rectangle split,rectangle split parts=3,draw,blue,rectangle split ignore empty parts=false] {texte 1 \nodepart{second} \nodepart{third}texte 3};
\end{tikzpicture}
&
\begin{tikzpicture}
\node[rectangle split,rectangle split parts=3,draw,blue,rectangle split ignore empty parts] 
{texte 1 \nodepart{second} \nodepart{third}texte 3};
\end{tikzpicture}
 \\  \hline 
\RDD{rectangle split ignore empty parts}=false & \RDD{rectangle split ignore empty parts}=true 
\\ \hline
 \end{tabular}
 
\bigskip

 \begin{tabular}{|c|c|}  \hline  
\multicolumn{2}{|c|}{  
\BS{node} [rectangle split,rectangle split parts=3,draw,\RDD{rectangle split empty part depth}=1cm] }\\
 \multicolumn{2}{|c|}{ \AC{texte 1 \BS{nodepart}\AC{second} \BS{nodepart}\AC{third}texte 3};} 
\\ \hline 
\begin{tikzpicture} 
\node[rectangle split,rectangle split parts=3,draw,blue,rectangle split empty part depth=1cm] {texte 1 \nodepart{second} \nodepart{third}texte 3};
\end{tikzpicture}
&
\begin{tikzpicture} 
\node[rectangle split,rectangle split parts=3,draw,blue,text depth=1cm] {texte 1 \nodepart{second} \nodepart{third}texte 3};
\end{tikzpicture}
\\ \hline 
\RDD{rectangle split empty part depth}=1cm & \RDD{text depth}=1cm
\\ \hline
\dft : 0ex & \dft : 0ex
\\ \hline 
\begin{tikzpicture}
\node[rectangle split,rectangle split parts=3,draw,blue,rectangle split empty part  height=1cm] 
{texte 1 \nodepart{second} \nodepart{third}texte 3};
\end{tikzpicture}
&
\begin{tikzpicture}
\node[rectangle split,rectangle split parts=3,draw,blue,text height=1cm] 
{texte 1 \nodepart{second} \nodepart{third}texte 3};
\end{tikzpicture}
\\  \hline 
\RDD{rectangle split empty part height}=1cm & \RDD{text height}=1cm
\\ \hline
\dft : 1ex & \dft : 1ex
\\ \hline 
 \end{tabular}
 
\bigskip



 \begin{tabular}{|c|c|}  \hline 
 \multicolumn{2}{|c|}{ 
 \BS{node} [rectangle split,rectangle split parts=3,draw,\RDD{rectangle split empty part width}=1cm]   \AC{};  } 
 \\ \hline 
\begin{tikzpicture} 
\node[rectangle split,rectangle split parts=3,draw,blue,rectangle split empty part width=2cm]{}; % {texte 1 \nodepart{second} \nodepart{third}texte 3};
\end{tikzpicture}
%\rule{6cm}{0pt}
&
\begin{tikzpicture} 
\node[rectangle split,rectangle split parts=3,draw,blue]{}; % {texte 1 \nodepart{second} \nodepart{third}texte 3};
\end{tikzpicture}
\\  \hline 
 \RDD{rectangle split empty part width}=2cm  &  \dft : 1ex
\\ \hline
 \end{tabular} 
 
 \bigskip



% % % % <<<<<<<<<< A voir   /pgf/rectangle split use custom fill= (default true) <<<<<<<<<<<<<<<<<<<<<<<<<<<<
 
 

%--------------------------------------------------------------------------------------

 \begin{tabular}{|c|c|}  \hline 
 \tikz[baseline=0pt] \node[rectangle split, draw,blue,minimum size = 2cm,rectangle split part align={center, left,right}] {texte 1 \nodepart{two} texte 2 \nodepart{three} texte 3 \nodepart{four} texte 4};
&
\parbox{8cm}{
\BS{node}[rectangle split, draw,blue,minimum size = 2cm,\\
\RDD{rectangle split part align}=\AC{center, left,right}]\\
 \AC{texte 1 \BS{nodepart}\AC{two} texte 2  \\
 \BS{nodepart}\AC{three} texte 3  \BS{nodepart}\AC{four} texte 4};
}
\\ \hline
 \tikz[baseline=0pt] \node[rectangle split, draw,blue,minimum size = 2cm, rectangle split horizontal,rectangle split part align={center,base, top,bottom}] {texte 1 \nodepart{two} texte 2 \nodepart{three} texte 3 \nodepart{four} texte 4};
 &
 \parbox{8cm}{
 \BS{node}[rectangle split, draw,blue,minimum size = 2cm,\\
 rectangle split horizontal,\\
 \RDD{rectangle split part align}=\AC{center,base, top,bottom}]\\
  \AC{texte 1 \BS{nodepart}\AC{two} texte 2  \\
  \BS{nodepart}\AC{three} texte 3  \BS{nodepart}\AC{four} texte 4};
 }
 \\ \hline
 \end{tabular}
 
\bigskip
%--------------------------------------------------------------------

 \begin{tabular}{|c|c|}  \hline  
\tikz[baseline=0pt] \node[rectangle split, draw,blue, minimum width=1cm,rectangle split part fill={red, green,cyan}]{};
&
\parbox{12cm}{
\BS{node}[rectangle split, draw,blue, minimum width=1cm,\\
 \RDD{rectangle split part fill}=\AC{red, green,cyan}]\AC{};}
\\ \hline
\end{tabular} 

%--------------------------------------------
\newpage
%\subsection{Mise en forme du texte}
\SbSSCT{Mise en forme du texte}{Text attributes}

\subsubsection{Position}

\begin{center}
\RRR{17-4-3}
\end{center}

\begin{tabular}{|c|c|c|c|} \hline  
\multicolumn{4}{|l|}{ \BS{tikz} \BS{draw} (0,0) node[fill=blue!10,text width=2cm,\RDD{text justified}]   }\\ 

\multicolumn{4}{|l|}{ \AC{Ceci est une démonstration d'un texte  sur une largeur de 2cm};  }\\ 
\hline 
\tikz \draw (0,0) node[fill=blue!10,text width=2cm]
{Ceci est une démonstration d'un texte  sur une largeur de 2cm.};
&  
\tikz \draw (0,0) node[fill=blue!10,text width=2cm,text justified]
{Ceci est une démonstration d'un texte  sur une largeur de 2cm};
&  
\tikz \draw (0,0) node[fill=blue!10,text width=2cm,text centered]
{Ceci est une démonstration d'un texte  sur une largeur de 2cm .};
&  
\tikz \draw (0,0) node[fill=blue!10,text width=2cm,text ragged]
{Ceci est une démonstration d'un texte  sur une largeur de 2cm .};
\\  \hline  
\TFRGB{sans}{without} option & text justified & text centered & text ragged   
\\ \hline  
\tikz \draw (0,0) node[fill=blue!10,text width=2cm,text badly ragged]
{Ceci est une démonstration d'un texte  sur une largeur de 2cm.};
&  
\tikz \draw (0,0) node[fill=blue!10,text width=2cm,text badly centered]
{Ceci est une démonstration d'un texte  sur une largeur de 2cm .};
&
\tikz \draw (0,0) node[fill=blue!10,text width=2cm,align=center]
{Ceci est une démonstration d'un texte  sur une largeur de 2cm .};
&
\tikz \draw (0,0) node[fill=blue!10,text width=2cm,align=flush center]
{Ceci est une démonstration d'un texte  sur une largeur de 2cm .};
\\  \hline 
text badly ragged &  text badly centered &  align=center & align=flush center 
\\  \hline 
\tikz \draw (0,0) node[fill=blue!10,text width=2cm,align=justify]
{Ceci est une démonstration d'un texte  sur une largeur de 2cm .};
&
\tikz \draw (0,0) node[fill=blue!10,text width=2cm,align=flush right]
{Ceci est une démonstration d'un texte  sur une largeur de 2cm .};
&
\tikz \draw (0,0) node[fill=blue!10,text width=2cm,align=right]
{Ceci est une démonstration d'un texte  sur une largeur de 2cm .};
&
\tikz \draw (0,0) node[fill=blue!10,text width=2cm,align=flush left]
{Ceci est une démonstration d'un texte  sur une largeur de 2cm .};
\\ \hline 
 align=justify & align=flush right &  align=right & align=flush left
\\ \hline 

\end{tabular} 
\bigskip

%--------------------------------------------------------------
%\subsubsection{Couleur et fontes } 
\SbSbSSCT{Couleur et fontes }{Colors and Fonts}

\begin{tabular}{|c|c|c|c|c|c|} \hline  
\tikz \draw (0,0) node[text= red]{Texte.};
&
\tikz \draw (0,0) node[font=\itshape]{Texte.};
&
\tikz \draw (0,0) node[font=\slshape]{Texte.};
&
\tikz \draw (0,0) node[font=\scshape]{Texte.};
&
\tikz \draw (0,0) node[font=\upshape]{Texte.};
&
\tikz \draw (0,0) node[font=\bfseries]{Texte.};
\\ \hline 



[text= red] & [font=\BS{itshape}]  & [font=\BS{slshape}] & [font=\BS{scshape}] & [font=\BS{upshape}] & [font=\BS{bfseries}]
\\ \hline 
\end{tabular} 



\bigskip

%\subsubsection{Taille des fontes} 
\SbSbSSCT{Taille des fontes}{Font Sizes}

\begin{tabular}{|c|c|c|c|c|c|c|}\hline
\multicolumn{7}{|c|}{ \BS{tikz} \BS{draw} (0,0) node[\RDD{font}=\BS{tiny}]\AC{Texte.}   }
\\  \hline
\tikz \draw (0,0) node[font=\tiny]{Texte.};
&
\tikz \draw (0,0) node[font=\footnotesize]{Texte.};
&
\tikz \draw (0,0) node[font=\small]{Texte.};
&
\tikz \draw (0,0) node[font=\large]{Texte.};
&
\tikz \draw (0,0) node[font=\Large]{Texte.};
&
\tikz \draw (0,0) node[font=\huge]{Texte.};
&
\tikz \draw (0,0) node[font=\Huge]{Texte.};
\\ \hline \BS{tiny} & \BS{footnotesize}  & \BS{small} & \BS{large} & \BS{Large} & \BS{huge} & \BS{Huge} \\ 
\hline 
\end{tabular} 

\bigskip
\begin{center}
\RRR{17-4-4}
\end{center}

\begin{tabular}{|c|c|} \hline  
\tikz \draw (0,0) node[fill=blue!10,text height=1cm,draw]{Texte.};
&  
\tikz \draw (0,0) node[fill=blue!10,text depth=1cm,draw]{Texte.};
\\ \hline  
\RDD{text height}=1cm
&  
\RDD{text depth}=1cm
\\ \hline 
\end{tabular} 

%\subsection{Positions prédéfinies  sur un n\oe ud}
\SbSSCT{Positions prédéfinies  sur un n\oe ud}{Positions on a node}
\label{nomnoeud}

%\subsubsection{pour l'ensemble des n\oe uds}
\SbSbSSCT{pour l'ensemble des n\oe uds}{For all types of node}
\begin{center}
\RRR{17-5-1}
\end{center}

\begin{tabular}{|c|c|c|c|} \hline  
\begin{tikzpicture}
\node[rectangle,draw,minimum size=3cm] (A) at (1,1) {\Huge texte};
\fill[red] (node cs:name=A,anchor=north west) circle (3pt);
\end{tikzpicture}
&
\begin{tikzpicture}
\node[rectangle,draw,minimum size=3cm] (A) at (1,1) {\Huge texte};
\fill[red] (node cs:name=A,anchor=north) circle (3pt);
\end{tikzpicture}
&
\begin{tikzpicture}
\node[rectangle,draw,minimum size=3cm] (A) at (1,1) {\Huge texte};
\fill[red] (node cs:name=A,anchor=north east) circle (3pt);
\end{tikzpicture}
&
\begin{tikzpicture}
\node[rectangle,draw,minimum size=3cm] (A) at (1,1) {\Huge texte};
\fill[red] (node cs:name=A,anchor=text) circle (3pt);
\end{tikzpicture}
\\ \hline 
north west & north & north east & text
\\ \hline 
%---------------------------------------------------------------
\begin{tikzpicture}
\node[rectangle,draw,minimum size=3cm] (A) at (1,1) {\Huge texte};
\fill[red] (node cs:name=A,anchor= west) circle (3pt);
\end{tikzpicture}
&
\begin{tikzpicture}
\node[rectangle,draw,minimum size=3cm] (A) at (1,1) {\Huge texte};
\fill[red] (node cs:name=A,anchor=mid  west) circle (3pt);
\end{tikzpicture}
&
\begin{tikzpicture}
\node[rectangle,draw,minimum size=3cm] (A) at (1,1) {\Huge texte};
\fill[red] (node cs:name=A,anchor= base west) circle (3pt);
\end{tikzpicture}
&
\begin{tikzpicture}
\node[rectangle,draw,minimum size=3cm] (A) at (1,1) {\Huge texte};
\fill[red] (node cs:name=A,anchor= base) circle (3pt);
\end{tikzpicture}
\\ \hline 
west & mid west & base west &  base
\\ \hline
%------------------------------------------------------------ 
\begin{tikzpicture}
\node[rectangle,draw,minimum size=3cm] (A) at (1,1) {\Huge texte};
\fill[red] (node cs:name=A,anchor=east) circle (3pt);
\end{tikzpicture}
&
\begin{tikzpicture}
\node[rectangle,draw,minimum size=3cm] (A) at (1,1) {\Huge texte};
\fill[red] (node cs:name=A,anchor=mid east) circle (3pt);
\end{tikzpicture}
&
\begin{tikzpicture}
\node[rectangle,draw,minimum size=3cm] (A) at (1,1) {\Huge texte};
\fill[red] (node cs:name=A,anchor=base east) circle (3pt);
\end{tikzpicture}
&
\begin{tikzpicture}
\node[rectangle,draw,minimum size=3cm] (A) at (1,1) {\Huge texte};
\fill[red] (node cs:name=A,anchor= mid) circle (3pt);
\end{tikzpicture}
\\ \hline 
east & mid esat & base east & mid
\\ \hline 
%--------------------------------------
\begin{tikzpicture}
\node[rectangle,draw,minimum size=3cm] (A) at (1,1) {\Huge texte};
\fill[red] (node cs:name=A,anchor= south east) circle (3pt);
\end{tikzpicture}
&
\begin{tikzpicture}
\node[rectangle,draw,minimum size=3cm] (A) at (1,1) {\Huge texte};
\fill[red] (node cs:name=A,anchor= south) circle (3pt);
\end{tikzpicture}
&
\begin{tikzpicture}                                       
\node[rectangle,draw,minimum size=3cm] (A) at (1,1) {\Huge texte};
\fill[red] (node cs:name=A,anchor= south west) circle (3pt);
\end{tikzpicture}
&
\begin{tikzpicture}
\node[rectangle,draw,minimum size=3cm] (A) at (1,1) {\Huge texte};
\fill[red] (node cs:name=A,anchor=center ) circle (3pt);
\end{tikzpicture}
\\ \hline 
south east & south & south west & center
\\ \hline
%------------------------------------------------------------------------- 
\begin{tikzpicture}
\node[rectangle,draw,minimum size=3cm] (A) at (1,1) {\Huge texte};
\fill[red] (node cs:name=A,anchor=0) circle (3pt);
\end{tikzpicture}
&
\begin{tikzpicture}
\node[rectangle,draw,minimum size=3cm] (A) at (1,1) {\Huge texte};
\fill[red] (node cs:name=A,anchor=120) circle (3pt);
\end{tikzpicture}
&
\begin{tikzpicture}
\node[rectangle,draw,minimum size=3cm] (A) at (1,1) {\Huge texte};
\fill[red] (node cs:name=A,anchor=-60) circle (3pt);
\end{tikzpicture}
&
%\begin{tikzpicture}
%\node[rectangle,draw,minimum size=3cm] (A) at (1,1) {\Huge texte};
%\fill[red] (node cs:name=A,anchor=text) circle (3pt);
%\end{tikzpicture}

\\ \hline 
0 & 120 & -60 & %text  
\\ \hline 
\end{tabular}
 
%\subsubsection{spécifique à un n\oe ud}
\SbSbSSCT{spécifique à un n\oe ud}{Specific to a node}

\TFRGB{Dans une prochaine version !}{In a future version}







 

% 
% %============\newpage
% 
% \section{Decorations}
% 
%  \input{tkzdeco}
% % 
% % ======================================================================
% \newpage
% 
% %\section{Insertion images dans un environnement TikZ}
% \SSCT{Insertion images dans un environnement TikZ}{Pictures in a TikZ picture}
% 
% \input{tkzimage}
% 
% 
% %%
% %%>>>> \section[Mettre des objets en cadre]{Mettre des objets en cadre }
% %%
% %
% %%
% %%\newpage
% %%>>>>> \section[Mettre des objets en bouton]{Mettre des objets en bouton }
% %
% %
% %%%%%%=============================================================
% %
% 
% %\section{Des lignes et liaisons spéciales}
% %\subsection[Trait à main levé]{Trait à main levée }
% \SSCT{Trait à main levée }{Freehand drawing}
% 
% \input{tkzalea}
% 
% %% >>>> \subsection{Tracer avec des symboles}
% %
% %%
% %%\newpage
% 
% %%>>>>> \subsection[Les bobines]{Les bobines \cite{pst-user} \cite{pst-coil}}
% %%
% 
% %%\newpage
% %%
% %%>>>> \subsection[Les accolades]{Les accolades }
% %%
% %
% %%%%%======================================================================
% %%\section{Des remplissages spéciaux}
% %%\subsection{Des gradients de couleurs}
% %%
% %%\subsubsection[Module pst-grad]{Module pst-grad \cite{pst-user} \cite{pst-grad}}
% %
% %%%
% %%\newpage
% %%\subsubsection[Module pst-slpe]{Module pst-slpe  \cite{pst-slpe}}
% %%
% %
% %%
% %%\newpage
% %%\subsection[Remplissage par des motifs]{Remplissage par des motifs \cite{pst-fill}}
% %%
% %
% %%
% %%\subsection[Remplissage par des points aléatoires]{Remplissage par des points aléatoires \cite{pst-add}}
% %
% %%\newpage
% %%
% %%% ========================================================================
% %%\section[Effets spéciaux avec du texte ]{Effets spéciaux avec du texte  \cite{pst-user}  \cite{pst-text}}
% %
% %
% %\newpage
% %%% % % % %======================================================================
% %\section[Créer un graphe]{Créer un graphe }
% \SSCT{Créer un graphe }{Creating Graphs}
% 
% 
%\subsection{Graphe avec Tikz}
\SbSSCT{Graphe avec TikZ}{Graph with TikZ}
%\subsubsection{Graphe à partir d'une liste de points}
\SbSbSSCT{Graphe à partir d'une liste de points}{From a list of points}
\label{plot}

\begin{tabular}{|c | } \hline
\BS{tikz} \BS{draw} plot \RDD{coordinates} \AC{(0,0) (1,1) (2,0) (3,1) (4,1) (5,2)}; \\ 
\hline
\tikz \draw plot coordinates {(0,0) (1,1) (2,0) (3,1) (4,1) (5,2)};
\\ \hline
\end{tabular}

%\subsubsection{Graphe à partir partir d'un fichier de données}
\SbSbSSCT{Graphe à partir partir d'un fichier de données}{From a data file}

\begin{tabular}{|c | c | c | c |} \hline
\multicolumn{4}{|c|}{ \BS{tikz} \BS{draw}  plot[mark=x] \RDD{file} \AC{table.dat} ;   }\\ 
\hline
%\draw plot[mark=x] file {table.dat};
& 
\tikz \draw plot[mark=x,smooth] file {table.dat};
&
\tikz \draw plot[mark=x,smooth,tension=.2] file {table.dat};
&
\tikz \draw plot[mark=x,smooth,tension=1] file {table.dat};
\\ \hline
[mark=x] & [mark=x,\RDD{smooth}] & [mark=x,smooth,\RDD{tension}=.2] & [mark=x,smooth,\RDD{tension}=1]
\\ \hline
\multicolumn{4}{|c|}{ \dft : tension= 0:55}
\\ \hline
\end{tabular}

\bigskip


\begin{tabular}{|c  c |} \hline
\multicolumn{2}{|c|}{\TFRGB{Contenu du fichier}{content of the file} table.dat}
\\ \hline
0.0 & 0.3 \\
0.3 & 0.6 \\
0.6 & 0.9 \\
0.9 & 1.5  \\
1.2 & 0.6  \\
1.5 & 1.2  \\
1.8 & 1.5  \\
2.1 & 2.0 \\
2.4 & 3.0 \\
\hline
\end{tabular}

\bigskip

%\subsubsection{Les types de graphes}
\SbSbSSCT{Les types de graphes}{Graph types}

\begin{tabular}{|c | c | c | c |} \hline
\multicolumn{4}{|c|}{ \BS{tikz} \BS{draw}  plot[mark=*,\RDD{const plot}] file \AC{table.dat} ;   }\\ 
\hline
\tikz \draw plot[mark=*,const plot] file {table.dat};
&

\tikz \draw plot[const plot mark left,mark=*] file {table.dat};
&
\tikz \draw plot[const plot mark right,mark=*] file {table.dat};
&
\tikz \draw plot[jump mark left, mark=*] file {table.dat};
\\ \hline
\RDD{const plot} & \RDD{const plot mark left} & \RDD{const plot mark right} & \RDD{jump mark left}
\\ \hline
\tikz \draw plot[jump mark right, mark=*] file {table.dat};
&
\tikz \draw plot[ycomb,thin,mark=*] file {table.dat};
&
\tikz \draw plot[xcomb,mark=*] file {table.dat};
&
\tikz \draw plot[only marks,mark=*] file {table.dat};
\\ \hline
\RDD{jump mark right} & \RDD{ycomb} & \RDD{xcomb} & \RDD{only marks}
\\ \hline
\end{tabular}

\bigskip
\begin{tabular}{|c | c | c |c |} \hline
%\begin{tikzpicture}
%\draw[help lines] (-2,-3) grid (2,2);
\tikz  \draw plot[polar comb,mark=*]coordinates {(0:1) (60:0.5) (120:1.5) (180:3) (240:.5) (300:1) (0:1)};
%\draw[line width=1pt,color=red] plot coordinates (0:1cm)(60:0.5)(120:1.5)(180:1)(240:3)(300:1)(0:1cm);
%\end{tikzpicture}
\\ \hline
\BS{tikz}  \BS{draw} plot[\RDD{polar comb},mark=*]coordinates \\
\AC{(0:1) (60:0.5) (120:1.5) (180:3) (240:.5) (300:1) (0:1)};
\\ \hline
\end{tabular}

\bigskip

\begin{tabular}{|c | c | c |c |} \hline
\multicolumn{4}{|c|}{ \BS{tikz} \BS{draw}  plot[\RDD{ybar}] file \AC{table.dat} ;   }\\ 
\hline
\tikz \draw plot[ybar] file {table.dat};
&
\tikz \draw plot[ybar interval] file {table.dat};
&
\tikz \draw plot[ybar interval,x=2cm] file {table.dat};
&
\tikz \draw plot[ybar interval,y=.5cm] file {table.dat};
\\ \hline
[\RDD{ybar}] & [\RDD{ybar interval}] & [ybar interval,\RDD{x}=2cm] & [ybar interval,\RDD{y}=.5cm]
\\ \hline
\end{tabular}

\bigskip
 \begin{tabular}{|c|c|}  \hline 
\begin{tikzpicture}[baseline=0pt]
\draw[red,fill=cyan,ybar,bar width=.5cm]plot coordinates{(0,1) (1,1.2) (2,.6) (3,.7) (4,.9)};
\draw[blue,fill=green,ybar,bar width=.5cm,bar shift=.3cm]plot coordinates{(0,1.2) (1,1.3) (2,.5) (3,.2) (4,.5)};
\end{tikzpicture}
&
\parbox[c]{10cm}{
\BS{begin}\AC{tikzpicture} \\
\BS{draw}[red,fill=cyan,ybar,bar width=.5cm] \\
\rule{1cm}{.0pt} plot coordinates \AC{(0,1) (1,1.2) (2,.6) (3,.7) (4,.9)}; \\
\BS{draw}[blue,fill=green,ybar,bar width=.5cm,\RDD{bar shift}=.3cm] \\
\rule{1cm}{.0pt} plot coordinates \AC{(0,1.2) (1,1.3) (2,.5) (3,.2) (4,.5)}; \\
\BS{end}\AC{tikzpicture} }
 \\  \hline 
 \end{tabular} 

\bigskip

\begin{tabular}{|c | c | c | c |c |} \hline
\multicolumn{4}{|c|}{ \BS{tikz} \BS{draw}  plot[xbar interval] file \AC{table.dat} ;   }\\ 
\hline
\tikz \draw[blue] plot[xbar] coordinates{(2,0) (3,1) (1,2) (2,3)};
&
\tikz \draw[blue] plot[xbar interval]  coordinates {(2,0) (3,1) (1,2) (2,3)};
&
\tikz \draw[blue] plot[xbar interval,x=.5cm]  coordinates {(2,0) (3,1) (1,2) (2,3)};
&
\tikz \draw[blue] plot[xbar interval,y=.5cm]  coordinates {(2,0) (3,1) (1,2) (2,3)};
%&
%\tikz \draw[blue!20] plot[xbar interval,x=.5cm,y=.5cm]  coordinates {(2,0) (3,1) (1,2) (2,3)};
\\ \hline
[\RDD{xbar}] & [\RDD{xbar interval}] & [xbar interval,\RDD{x}=.5cm] & [xbar interval,\RDD{y}=.5cm] 
\\ \hline
\end{tabular}

\newpage
%--------------------------------------------------------------
%\subsubsection{Graphe à partir d'une fonction}
\SbSbSSCT{Graphe à partir d'une fonction}{Graph of a function}


\begin{tabular}{|c | c | c | } \hline
\multicolumn{3}{|c|}{  \BS{draw}  [color=red] plot (\BS{x},\BS{x});   }\\ 
\hline
\begin{tikzpicture}[domain=0:4,ultra thick]
%\draw[very thin,color=gray] (-0.1,-1.1) grid (4.1,4.1);
\draw[->,blue,ultra thick] (-.1,0) -- (4.5,0);
\draw[->,blue,ultra thick] (0,-1.1) -- (0,04);
\draw[color=red] plot (\x,\x);
\end{tikzpicture} 
&
\begin{tikzpicture}[domain=0:6.28,ultra thick,x=0.7cm]
%\draw[very thin,color=gray] (-0.1,-2.1) grid (4.1,2.1);
\draw[->,blue,ultra thick] (-.1,0) -- (7,0);
\draw[->,blue,ultra thick] (0,-2.5) -- (0,2.5);
\draw[color=red] plot  (\x,{sin(\x r)});
\end{tikzpicture} 
&
\begin{tikzpicture}[domain=0:360,x=0.3,ultra thick]
%\draw[very thin,color=gray] (-0.1,-2.1) grid (4.1,2.1);
\draw[->,blue,ultra thick] (-.1,0) -- (370,0);
\draw[->,blue,ultra thick] (0,-2.5) -- (0,2.5);
\draw[color=red] plot (\x,{sin(\x)});
\end{tikzpicture} 
\\ \hline
(\BS{x},\BS{x}) &  (\BS{x},\AC{sin(\BS{x} r)}) & (\BS{x},\AC{sin(\BS{x})}) \\
& x en radian & x en degré
\\ \hline
\end{tabular}

Options 

\begin{tabular}{|c | c |} \hline
\multicolumn{2}{|l|}{ \BS{draw}[color=red,dashed] plot(\BS{x},\AC{sin(\BS{x} r)});}  \\
\multicolumn{2}{|l|}{ \BS{draw}[color=blue,\RDD{samples}=5,mark=*,ultra thick] plot(\BS{x},\AC{sin(\BS{x} r)});} 
\\ \hline
\begin{tikzpicture}[domain=0:6.28]
\draw[very thin,color=gray] (-0.1,-1.1) grid (6.28,1.1);
\draw[color=red,dashed] plot  (\x,{sin(\x r)});
\draw[color=blue,samples=5,mark=*,ultra thick] plot  (\x,{sin(\x r)});
\end{tikzpicture} 
&
\begin{tikzpicture}
\draw[very thin,color=gray] (-0.1,-1.1) grid (6.28,1.1);
\draw[color=red,dashed,domain=0:6.28] plot  (\x,{sin(\x r)});
\draw[color=blue,domain=0:4,ultra thick] plot  (\x,{sin(\x r)});
\end{tikzpicture} 
  \\ \hline
[color=blue,\RDD{samples}=5,mark=*] & [color=blue,\RDD{domain}=0:4]
\\ \hline
\begin{tikzpicture}
\draw[very thin,color=gray] (-0.1,-1.1) grid (6.28,1.1);
\draw[color=red,dashed,domain=0:6.28] plot  (\x,{sin(\x r)});
\draw[color=blue,domain=1:5,ultra thick] plot  (\x,{sin(\x r)});
\end{tikzpicture} 
&
\begin{tikzpicture}[domain=0:6.28]
\draw[very thin,color=gray] (-0.1,-1.1) grid (6.28,1.1);
\draw[color=red,dashed] plot  (\x,{sin(\x r)});
\draw[color=blue,samples at={1,2,4,5},mark=*,ultra thick] plot  (\x,{sin(\x r)});
\end{tikzpicture} 
\\ \hline
[color=blue,\RDD{domain}=1:5] & [color=blue,\RDD{samples at}=\AC{1,2,4,5},mark=*]
\\ \hline
\end{tabular}


%-------------------------------------------------------------------------
%\subsubsection{Fonctions paramétriques}
\SbSbSSCT{Fonctions paramétriques}{Parametric function}


\begin{tabular}{|c | c |} \hline
\multicolumn{2}{|l|}{  \BS{draw}[domain=-3.141:3.141,smooth,variable=\BS{t}]plot (\AC{sin(\BS{t} r)},\AC{sin(2 *\BS{t} r)});} \\
\multicolumn{2}{|l|}{  \BS{draw}[domain=0:720,smooth,variable=\BS{t}]plot (\AC{sin(\BS{t})},\BS{t}/360,\AC{cos(\BS{t})});} 
\\ \hline

\tikz \draw[domain=-3.141:3.141,smooth,variable=\t,ultra thick]plot ({sin(\t r)},{sin(2*\t r)});
&
\tikz \draw[domain=0:720,smooth,variable=\t,ultra thick] plot ({sin(\t)},\t/360,{cos(\t)});
\\ \hline
(\AC{sin(\BS{t} r)},\AC{sin(2 *\BS{t} r)}) & (\AC{sin(\BS{t})},\BS{t}/360,\AC{cos(\BS{t})})
\\ \hline
\end{tabular} 
%\tikz \draw plot[mark=x,mark repeat=3,smooth] file {plots/pgfmanual-sine.table};
 

%\subsection{Marques}
\SbSSCT{Marques}{Marks}

%\subsubsection{Marques avec Tikz}
\SbSbSSCT{Marques avec TikZ}{Marks with TikZ}

\begin{tabular}{|c | c | c | c |} \hline
\tikz \draw plot[mark=+,mark size=5pt] coordinates {(0,0) (1,1) (2,0)};
&
\tikz \draw plot[mark=x,mark size=5pt] coordinates {(0,0) (1,1) (2,0) };
&
\tikz \draw plot[mark=*,mark size=5pt] coordinates {(0,0) (1,1) (2,0)};
&
\tikz \draw plot[mark=ball,mark size=5pt] coordinates {(0,0) (1,1) (2,0)};
\\ \hline
mark=+ & mark=x & mark=* & mark=ball
\\ \hline
\end{tabular}

\bigskip

\begin{tabular}{|c | c |} \hline
\begin{tikzpicture}[domain=0:6.28]
\draw[very thin,color=gray] (-0.1,-1.1) grid (6.28,1.1);
\draw[color=red,dashed,mark=+] plot  (\x,{sin(\x r)});
\draw[color=blue,mark repeat=3,mark=*] plot  (\x,{sin(\x r)});
\end{tikzpicture} 
&
\begin{tikzpicture}[domain=0:6.28]
\draw[very thin,color=gray] (-0.1,-1.1) grid (6.28,1.1);
\draw[color=red,dashed,mark=+] plot  (\x,{sin(\x r)});
\draw[color=blue,mark repeat=3,mark phase=5,mark=*] plot  (\x,{sin(\x r)});
\end{tikzpicture} 
\\ \hline
[color=blue,\RDD{mark repeat}=3,mark=*] & [color=blue,mark repeat=3,\RDD{mark phase}=5,mark=*]
\\ \hline
\begin{tikzpicture}[domain=0:6.28]
\draw[very thin,color=gray] (-0.1,-1.1) grid (6.28,1.1);
\draw[color=red,dashed,mark=+] plot  (\x,{sin(\x r)});
\draw[color=blue,mark indices={1,4,...,15,17,20},mark=*] plot  (\x,{sin(\x r)});
\end{tikzpicture} 
&
\begin{tikzpicture}[domain=0:6.28]
\draw[very thin,color=gray] (-0.1,-1.1) grid (6.28,1.1);
\draw[color=red,dashed,mark=+] plot  (\x,{sin(\x r)});
\draw[color=blue,mark size=5pt,mark=*] plot  (\x,{sin(\x r)});
\end{tikzpicture} 
\\ \hline
[color=blue,\RDD{mark indices}={1,4,...,15,17,20},mark=*] & [color=blue,\RDD{mark size}=5pt,mark=*]
\\ \hline
\begin{tikzpicture}[domain=0:6.28]
\draw[very thin,color=gray] (-0.1,-1.1) grid (6.28,1.1);
%\draw[color=red,dashed,mark=*] plot  (\x,{sin(\x r)});
\draw[color=blue,mark size=5pt,mark options={color=magenta},mark=+] plot  (\x,{sin(\x r)});
\end{tikzpicture}
&
\begin{tikzpicture}[domain=0:6.28]
\draw[very thin,color=gray] (-0.1,-1.1) grid (6.28,1.1);
%\draw[color=red,dashed,mark=*] plot  (\x,{sin(\x r)});
\draw[color=blue,mark size=5pt,mark options={rotate=10},mark=+] plot  (\x,{sin(\x r)});
\end{tikzpicture}
\\ \hline
\RDD{mark options}=\AC{color=magenta},mark=+ & \RDD{mark options}=\AC{rotate=10},mark=+
\\ \hline
\end{tabular}
 

%\subsubsection{Marques personnalisées avec text mark}
\SbSbSSCT{Marques personnalisées avec text mark}{Marks with text mark}

\begin{tabular}{|c | c | c |} \hline
\multicolumn{3}{|l|}{ \BS{draw}[\RDD{mark=text},\RDD{text mark}=A,mark size=5pt] coordinates \AC{(0,0) (1,1) (2,0)};} 
\\ \hline
\tikz \draw plot[mark=text,text mark=A,mark size=5pt] coordinates {(0,0) (1,1) (2,0)};
&
\tikz \draw plot[mark=text,text mark=Texte,mark size=5pt] coordinates {(0,0) (1,1) (2,0)};
&
\begin{tikzpicture}
\draw[white]  (-1,0)-- (-1,1.5);
 \draw plot[mark=text,text mark=\DFR,mark size=5pt] coordinates {(0,0) (1,1) (2,0)};
\end{tikzpicture} 
\\ \hline
\RDD{text mark}=A &  \RDD{text mark}=Texte & \RDD{text mark}=\BS{DFR} \pageref{DFR} 
\\ \hline 
\multicolumn{3}{|c|}{ 
\begin{tikzpicture}
\draw[white]  (-1,0)-- (-1,1.5);
\draw plot[mark=text,text mark={\includegraphics[width=.5cm]{tiger}} ,mark size=5pt] coordinates {(0,0) (1,1) (2,0)};  
\end{tikzpicture} }
\\ \hline  
\multicolumn{3}{|c|}{ \RDD{text mark}=\AC{\BS{includegraphics}[width=.5cm]\AC{tiger}} }
\\ \hline   
\end{tabular}


\newpage
%\subsubsection{Marques avec l'extension plotmarks }
\SbSbSSCT{Marques avec l'extension plotmarks }{Marks with plotmarks library}

\label{plotmarks}

%Insérer dans le préambule :

 \maboite{\BS{usetikzlibrary}\AC{plotmarks}}
 
\begin{center}
\RRR{63}
\end{center}

\begin{tabular}{|c | c | c | c |} \hline
\tikz \draw plot[mark=-,mark size=5pt] coordinates {(0,0) (1,1) (2,0)};
& 
\tikz \draw plot[mark=|,mark size=5pt] coordinates {(0,0) (1,1) (2,0)};
 &
\tikz \draw plot[mark=o,mark size=5pt] coordinates {(0,0) (1,1) (2,0)};
 &
\tikz \draw plot[mark=asterisk,mark size=5pt] coordinates {(0,0) (1,1) (2,0)};
\\ \hline 
mark=- & mark=| & mark=o &mark=asterisk
\\ \hline
\tikz \draw plot[mark=star,mark size=5pt] coordinates {(0,0) (1,1) (2,0)};
&
\tikz \draw plot[mark=10-pointed star,mark size=5pt] coordinates {(0,0) (1,1) (2,0)};
&
\tikz \draw plot[mark=oplus,mark size=5pt] coordinates {(0,0) (1,1) (2,0)};
&
\tikz \draw plot[mark=oplus*,mark size=5pt] coordinates {(0,0) (1,1) (2,0)};
\\ \hline
mark=star & mark=10-pointed star & mark=oplus & mark=oplus*
\\ \hline
 
\tikz \draw plot[mark=otimes,mark size=5pt] coordinates {(0,0) (1,1) (2,0)};
&
\tikz \draw plot[mark=otimes*,mark size=5pt] coordinates {(0,0) (1,1) (2,0)};
&
\tikz \draw plot[mark=square,mark size=5pt] coordinates {(0,0) (1,1) (2,0)};
&
\tikz \draw plot[mark=square*,mark size=5pt] coordinates {(0,0) (1,1) (2,0)};
\\ \hline
 mark=otimes & mark=otimes* & mark=square & mark=square*
  \\ \hline
  
\tikz \draw plot[mark=triangle,mark size=5pt] coordinates {(0,0) (1,1) (2,0)};
& 
\tikz \draw plot[mark=triangle*,mark size=5pt] coordinates {(0,0) (1,1) (2,0)};
& 
\tikz \draw plot[mark=diamond,mark size=5pt]  coordinates {(0,0) (1,1) (2,0)};
 &
\tikz \draw plot[mark=diamond*,mark size=5pt] coordinates {(0,0) (1,1) (2,0)};
\\ \hline 
mark=triangle & mark=triangle* & mark=diamond & mark=diamond*
\\ \hline 

\tikz \draw plot[mark=halfdiamond*,mark size=5pt] coordinates {(0,0) (1,1) (2,0)};
 &
\tikz \draw plot[mark=halfsquare*,mark size=5pt] coordinates {(0,0) (1,1) (2,0)};
 &
\tikz \draw plot[mark=halfsquare right*,mark size=5pt] coordinates {(0,0) (1,1) (2,0)};
 &
\tikz \draw plot[mark=halfsquare left*,mark size=5pt] coordinates {(0,0) (1,1) (2,0)};
\\ \hline 
mark=halfdiamond* & mark=halfsquare* & mark=halfsquare right* & mark=halfsquare left*
\\ \hline 

\tikz \draw plot[mark=pentagon,mark size=5pt] coordinates {(0,0) (1,1) (2,0)};
 &
\tikz \draw plot[mark=pentagon*,mark size=5pt] coordinates {(0,0) (1,1) (2,0)};
 &
\tikz \draw plot[mark=Mercedes star,mark size=5pt] coordinates {(0,0) (1,1) (2,0)};
 &
\tikz \draw plot[mark=Mercedes star flipped,mark size=5pt] coordinates {(0,0) (1,1) (2,0)};
 \\ \hline
 mark=pentagon & mark=pentagon* & mark=Mercedes star & mark=Mercedes star flipped
 \\ \hline 
 
\tikz \draw plot[mark=halfcircle,mark size=5pt] coordinates {(0,0) (1,1) (2,0)};
 &
\tikz \draw plot[mark=halfcircle*,mark size=5pt] coordinates {(0,0) (1,1) (2,0)};
& 
\tikz \draw plot[mark=heart,mark size=5pt] coordinates {(0,0) (1,1) (2,0)};
 &
\tikz \draw plot[mark=text,mark size=5pt] coordinates {(0,0) (1,1) (2,0)};
 \\ \hline
 mark=halfcircle & mark=halfcircle* & mark=heart & mark=text
  \\ \hline
\end{tabular}

\bigskip

\begin{tabular}{|c | c | c | c |} \hline
\multicolumn{4}{|l|}{ \BS{draw}[mark=halfcircle,\RDD{mark color}=red,mark size=5pt] coordinates \AC{(0,0) (1,1) (2,0)};} 
\\ \hline
\tikz \draw plot[mark=halfcircle,mark color=red,mark size=5pt] coordinates {(0,0) (1,1) (2,0)};
&
\tikz \draw plot[mark=halfcircle*,mark color=red,mark size=5pt] coordinates {(0,0) (1,1) (2,0)};
&
\tikz \draw plot[mark=halfdiamond*,mark color=red,mark size=5pt] coordinates {(0,0) (1,1) (2,0)};
&
\tikz \draw plot[mark=halfsquare*,mark color=red,mark size=5pt] coordinates {(0,0) (1,1) (2,0)};
  \\ \hline
  mark=halfcircle & mark=halfcircle* & mark=halfdiamond* & mark=halfsquare*
   \\ \hline 
\end{tabular}



% \subsection{Graphes avec Gnuplot}
\SbSSCT{Graphes avec Gnuplot}{Graph with Gnuplot}
 
 \begin{tabular}{|l| } \hline
%\begin{tikzpicture}[domain=0:6.28]
%%\draw[very thin,color=gray] (-0.1,-1.1) grid (7.1,1.1);
%%\draw[->,ultra thick,blue] (-0.2,0) -- (7,0) node[right] {$x$};
%%\draw[->,ultra thick,blue] (0,-1.2) -- (0,1.2) node[above] {$f(x)$};
%%\draw[color=red] plot[id=x] function{x} node[right] {$f(x) =x$};
%\draw[color=red] plot[id=sin] function{sin(x)} ;
%%\draw[color=orange] plot[id=exp] function{0.05*exp(x)} node[right] {$f(x) = \frac{1}{20} \mathrm e^x$};
%\end{tikzpicture}
\BS{draw}[color=red] plot[\RDD{id}=sin] function\AC{sin(x)} ;
   \\ \hline
\\
==> plot[id=sin] \TFRGB{crée le fichier}{create the file} \og sin.gnuplot \fg \\
==>  \TFRGB{Ouvrir le fichier}{Open the file} \og sin.gnuplot \fg \TFRGB{avec le programme gnuplot pour créer le fichier}{with the program gnuplot : creation of the file }   \og sin.table \fg\\
==> \TFRGB{Utiliser le fichier de données} {Use the datafile }
 \og sin.table  \fg   \\ \hline 
\end{tabular}
% 
% \newpage
% 
% \SSCT{Créer un graphe avec pgfplot}{Creation of a graph with pgfplots}
% 
% \input{tkzgraph2} % <<<<<<<<<<<<<<<<<<<<<<<<<<<<<
% %
% %\subsection{Ccùourbes 3D}
% \SSCT{Courbes 3D}{3D graph}
% 
% \input{tkzgraph3D} % très lourd à compiler
% 
% %
% %%%
% %%%\essais{pstgraph2ess.tex}
% %%\newpage
% %%\section[Créer un graphe d'après une équation]{Créer un graphe d'après une équation  \cite{pst-user} \cite{pst-plot}}
% %%
% %
% %%%
% %%%\essais{pstgraph3ess.tex} 
% %%\newpage
% %% \section[Des outils pour les graphes]{Des outils pour les graphes \cite{pst-add} }
% %% 
% %
% %%
% %%\newpage
% %% \section[Créer un graphe en camembert]{Créer un graphe en camembert \cite{pst-add} }
% %% 
% %%\input{chart} % camembert
% 
% \newpage
% 
% %\section{Les Tableaux de variation}
% \SSCT{Les Tableaux de variation}{Table of a function variation }
% 
% \input{tkztab}
% 
% \newpage
% %%%=============================================
% %\section{Les répétitions}
% \SSCT{Les répétitions}{Repetitions}
% 
% 
% \input{tkzrep1}  % OK
% 
% %%\subsection[Commande multido]{Commande multido \cite{pst-user} \cite{multido} }
% %%
% %
% %%
% %%\essais{pstrep2ess.tex}
% %%
% %%\subsection[Commande psforeach]{Commande psforeach \cite{pst-news10} }
% %
% %
% %%
% %%\newpage
% %%% % % %======================================================================
% %%\section[La géométrie]{La géométrie  \cite{pst-eucl} }
% %%
% %%Utilisation du module \textbf{pst-eucl} \label{pst-eucl}(consultez le fichier\textbf{ pst-eucl-doc.pdf} )
% %%
% %%
% %%\psset{fillcolor=yellow,linecolor=blue,dotscale=2}
% %%\subsection{\'Elements de base}
% %%
% %
% %%
% %%\subsection[Transformations géométriques]{Transformations géométriques \cite{pst-eucl} }
% %%
% %%
% %
% %%
% %%
% %%\subsection[Constructions particulières en géométrie ]{Constructions particulières en géométrie }
% %%
% %
% %%
% %%\subsection[Intersections]{Intersections  }
% %%
% %
% %%
% %%%--------------------------------------------------------------
% %%\section[Les vecteurs]{Les vecteurs  }
% %%
% %
% %%%==============================================================
% \newpage
%  
% %\section[Les diagrammes arborescents ]{Les diagrammes arborescents }
% \SSCT{Les diagrammes arborescents }{Tree diagram}
% 
% 
% \input{tkztree}
% 
% %%%==============================================================
% \newpage
% 
% %\section[Les animations ]{Les animations }
% \SSCT{Les animations }{Animate a TikZ picture}
% 
% 
% \input{tkzanim}
% %%
% %%\newpage
% %%
% %%\section[Créer un dessin en 3D]{Créer un dessin en 3D  }
% %
% 
% %%\subsection{Les objets en 3D}
% %%
% 
% %%\newpage
% %%\subsection[Créer un graphe en 3D]{Créer un graphe en 3D } 
% %%
% 
% \newpage
% %%=======================================================================
% %\section{Les modules étudiés dans ce document}
% \SSCT{Les modules étudiés dans ce document}{Packages studied in this document}
% 
% 
\TFRGB{module de base TikZ}{Basic TikZ package} : 

\maboite{\BS{usepackage}\AC{tikz} }

%\bigskip
\bigskip
\textbf{\TFRGB{Autres modules}{Other packages}}

%
\begin{tabular}{|c|c|l c|}\hline 
\TFRGB{nom}{name} 			& \TFRGB{voir page} 				& documentation\footnotemark[1] & \\  \hline 

animate 		& \pageref{anim} 			& animate.pdf 			& \DGB\\
tkz-tab  		& \pageref{tab} 			& tkz-tab-screen.pdf 	& \DFR \\
\hline 
\end{tabular} 
\bigskip

\textbf{\TFRGB{Compléments optionnels}{Optional library} :}

\begin{tabular}{|l|c|l|}\hline 
\TFRGB{nom}{name} 				& \TFRGB{voir page}{see page}						& \TFRGB{A insérer dans le préambule}{Load package}\\ \hline 
angles				& \pageref{lib-angles}			&  \BS{usetikzlibrary}\AC{angles}
\\
arrows.meta				& \pageref{lib-arrows.meta}			&  \BS{usetikzlibrary}\AC{arrows.meta}
\\
bending				& \pageref{lib-bending}			&  \BS{usetikzlibrary}\AC{bending}
\\
backgrounds			& \pageref{lib-bkgd} 			&  \BS{usetikzlibrary}\AC{backgrounds}
\\
calc				& \pageref{lib-calc}			&  \BS{usetikzlibrary}\AC{calc}
\\
circuits.ee.IEC				& \pageref{lib-ee}			&  \BS{usetikzlibrary}\AC{circuits.ee.IEC}
\\
 
fit & \pageref{lib-fit} 	& \BS{usetikzlibrary}\AC{fit} 
\\
decorations.footprints & \pageref{lib-footprints} 	& \BS{usetikzlibrary}\AC{decorations.footprints} 
\\
decorations.fractals & \pageref{lib-fractals} 		& \BS{usetikzlibrary}\AC{decorations.fractals} 
\\
decorations.markings & \pageref{lib-mark} 			& \BS{usetikzlibrary}\AC{decorations.markings} 
\\
decorations.pathmorphing  & \pageref{lib-morph}		& \BS{usetikzlibrary}\AC{decorations.pathmorphing}
\\
decorations.pathreplacing & \pageref{lib-replac}	& \BS{usetikzlibrary}\AC{decorations.pathreplacing} 
\\
decorations.shapes & \pageref{lib-shapes} 			& \BS{usetikzlibrary}\AC{decorations.shapes} 
\\
decorations.text & \pageref{lib-text} 				& \BS{usetikzlibrary}\AC{decorations.text} 
\\

fadings 			& \pageref{lib-fadings}			&  \BS{usetikzlibrary}\AC{fadings }
\\
intersections		& \pageref{lib-intersections}	&  \BS{usetikzlibrary}\AC{intersections}
\\
patterns			& \pageref{lib-patterns}		&  \BS{usetikzlibrary}\AC{patterns}
\\
plotmarks			& \pageref{plotmarks} 			&  \BS{usetikzlibrary}\AC{plotmarks}
\\ 
scopes				& \pageref{lib-scopes}			&  \BS{usetikzlibrary}\AC{scopes}
\\
shadings			& \pageref{lib-shadings}		&  \BS{usetikzlibrary}\AC{shadings}
\\
shapes.arrows		& \pageref{lib-arr}				&\BS{usetikzlibrary}\AC{shapes.arrows} 
\\shapes.callouts		& \pageref{lib-call}			& \BS{usetikzlibrary}\AC{shapes.callouts} 
\\
shapes.geometric	& \pageref{lib-geom} 			& \BS{usetikzlibrary}\AC{shapes.geometric}
\\  
shapes.misc			& \pageref{lib-misc} 			& \BS{usetikzlibrary}\AC{shapes.misc} 
\\
shapes.multipart	& \pageref{lib-mult} 			& \BS{usetikzlibrary}\AC{shapes.multipart} 
\\
shapes.symbols		& \pageref{lib-symb}			& \BS{usetikzlibrary}\AC{shapes.symbols} 
\\
trees				& \pageref{lib-trees}			&  \BS{usetikzlibrary}\AC{trees}
\\ 

\hline
 \end{tabular} 


\bigskip



\begin{tabular}{|l|c|}\hline
\multicolumn{2}{|c|}{ \TFRGB{dans une prochaine mise à jour}{In a a future update } }
\\ \hline
automata									& \RRR{41} \\
babel										& \RRR{42} \\
calendar									& \RRR{45} \\
chains										& \RRR{46} \\ 
%circuits.ee									& \RRR{47-4} \\ 
circuits.logic								& \RRR{47-3} \\ 
circular graph drawing library 				& \RRR{32} \\
curvilinear library 						& \RRR{103-4-7} \\
datavisualization library					& \RRR{75} \\
datavisualization.formats.functions library & \RRR{76-4} \\
datavisualization.polar library 			& \RRR{80}  \\
 er 										& \RRR{49}  \\
examples graph drawing library 				& \RRR{35-8} \\ 
external 									& \RRR{50}  \\  
%fit 										& \RRR{52} \\ 
fixedpointarithmetic 						& \RRR{53} \\ 
folding 									& \RRR{59} \\
force graph drawing library 				& \RRR{31}  \\
fpu											& \RRR{54}  \\
graph.standard library 						& \RRR{19-10}\\
graphdrawing library 						& \RRR{27} \\
graphs library 								& \RRR{19} \\ 
layered graph drawing library 				& \RRR{30}  \\
lindenmayersystems							& \RRR{55}  \\ 
matrix										& \RRR{57}  \\ 
mindmap										& \RRR{58} \\ 
petri										& \RRR{61}  \\ 
phylogenetics graph drawing library 		& \RRR{33} \\
plothandlers								& \RRR{62}  \\ 
positioning									& \RRR{17-5-3} \\ 
profiler									& \RRR{64}   \\ 
quotes library 								& \RRR{17-10-4} \\
routing graph drawing library 				& \RRR{34} \\
shadows										& \RRR{66}   \\ 
shapes.gates.ee								& \\ 
shapes.gates.ee.IEC							& \\ 
shapes.gates.logic							& \\ 
shapes.gates.logic.IEC						& \\ 
shapes.gates.logic.US						& \\ 
spy											&  \RRR{68} \\ 
svg.path									&  \RRR{69} \\ 
through										&  \RRR{71} \\ 
topaths										&  \RRR{70} \\ 
trees graph drawing library					& \\
turtle										&  \RRR{73} \\ 
\hline
\end{tabular}  
%circuit.ee.IEC, 309
%circuits, 292
%circuits.ee, 308
%, 300
%circuits.logic.CDH, 301
%circuits.logic.IEC, 300
%circuits.logic.US, 301

%
%\bigskip
%\textbf{Autres modules}
%
%%
%\begin{tabular}{|c|c|l|}\hline 
%nom 			& voir page 				& documentation\footnotemark[1]  \\  \hline 
%pst-fr3d 		& \pageref{pst-fr3d}		& pst-fr3d.pdf		\\ 
%pst-slpe 		& \pageref{pst-slpe}		& pst-slpe.pdf		\\ 
%infix-RPN 		& \pageref{infix-RPN}		& pst-infixplot.pdf	\\
%pst-infixplot 	& \pageref{pst-infixplot} 	& pst-infixplot.pdf \\ 
%pst-eucl 		& \pageref{pst-eucl} 		& pst-eucl-doc.pdf 	\\
%animate 		& \pageref{anim} 			& animate.pdf 	\\
%pst-3dplot		& \pageref{3dplot} 			& pst-3dplot-doc 	\\
%\hline 
%\end{tabular} 
%
%\bigskip
%\textbf{Additifs }
%
%%
%\begin{tabular}{|c|l|}\hline 
%année						& documentation\footnotemark[1]  \\  \hline 
%2005 			& pst-news5.pdf	\\
%2008  			& pst-news08.pdf \\ 
%2010 			& pst-news10.pdf 	\\
%\hline 
%\end{tabular} 
% 
%
%\footnotetext[1]{Vous pouvez les trouver pour la distribution Texlive dans le répertoire :  \BS{}texlive\BS{}2011\BS{}tesmf-dist\BS{}doc\BS{}generic}
%
%\newpage
%%
%% \tableofcontents
%\renewcommand{\bibname}{Sources}
%
\label{sources}
%\input{bib}

\newpage

\begin{thebibliography}{99}
\bibitem{pgfmanual} pgfmanual.pdf  	\hspace{1cm}	version 3.0.1a \hspace{1cm} 	1161 pages 	\hspace{1cm}	\DGB
\bibitem{pgfplots} pgfplots.pdf 	\hspace{1cm}	version 1.80 \hspace{1cm} 	439 pages 	\hspace{1cm}	\DGB
\bibitem{tikstab} tkz-tab-screen.pdf 	\hspace{1cm}	version 1.1c \hspace{1cm} 	83 pages 	\hspace{1cm}	\DFR

\end{thebibliography}
% 
% 
% 
% \newpage 
 \section{Index}
 
  \printindex 
%  
  
\end{document}

%\input{tkztitre}
%
%%==========================================================
%
%\setcounter{tocdepth}{4}
% \tableofcontents
% \setcounter{tocdepth}{5}
%\addtolength{\hoffset}{-1.5cm} 
%
%\setlength{\topmargin}{0pt}
%\setlength{\headsep}{0pt}
%
% \newpage
%
%\section{Les figures de base}
%
%\input{tkz1}
%
%\newpage
%
%\input{tkz2}
%
%\newpage
%%
%\input{tkz3}
%%
%
%
%\input{tkz3a}
%
%\newpage
%
%\section{Insertion de petites images}
%
%\input{tkzpic}
%
%\newpage
%
%\input{tkzangles}
%
%%%%%% % % %===================================
%
%\newpage
%
%\section{Les coordonnées }
% 
%%\subsection{Quadrillage}
\SbSSCT{Quadrillage}{Grid}


\begin{tabular}{|c|}\hline 
\tikz \draw(0,0) grid (2,2); 
\\ \hline 
\BS{draw} (0,0) \RDD{grid} (2,2); \RRR{14-8}
\\ \hline 
\end{tabular} 


\bigskip
\begin{tabular}{|c|c|c|c|} \hline 
\multicolumn{4}{|c|}{ \BS{draw} (0,0) grid  [\RDD{step}=.75cm] (0,0) grid (3,3);   }\\ 
\hline  
\begin{tikzpicture}
\draw[dotted](0,0) grid (3,3); 
%\draw[thick] (0,0) grid [step=1] (3,2);
\draw[red] (0,0) grid [step=.75cm] (3,3);
\end{tikzpicture}
&  
\begin{tikzpicture}
\draw[dotted](0,0) grid (3,3); 
%\draw[thick] (0,0) grid [step=1] (3,2);
\draw[red] (0,0) grid [xstep=.75cm] (3,3);
\end{tikzpicture}
&  
\begin{tikzpicture}
\draw[dotted](0,0) grid (3,3); 
%\draw[thick] (0,0) grid [step=1] (3,2);
\draw[red] (0,0) grid [ystep=.75cm] (3,3);
\end{tikzpicture}
&
\begin{tikzpicture}
\draw[dotted](0,0) grid (3,3); 
%\draw[thick] (0,0) grid [step=1] (3,2);
\draw[red] (0,0) grid [step=(45:1)] (3,3);
\end{tikzpicture}
\\ \hline 
step=.75cm & x step=.75cm & ystep=.75cm  & step=(45:1)
\\ \hline 
\end{tabular} 

\bigskip

\begin{tabular}{|c|c|} \hline 
 
\BS{draw}[red] (0,0) grid [\RDD{rotate}=45] (3,3);
&  
\BS{draw}[\RDD{help lines}] (0,0) grid  (3,3);
\\ \hline  
\begin{tikzpicture}
\draw[dotted](0,0) grid (3,3); 
%\draw[thick] (0,0) grid [step=1] (3,2);
\draw[red] (0,0) grid [rotate=45] (3,3);
\end{tikzpicture}
& 
\tikz \draw[help lines] (0,0) grid (3,3); \\ 
\hline 
\end{tabular} 



%\begin{tabular}{|c|c|c|c|c|} \hline 
%\multicolumn{5}{|c|}{ \BS{tikz} \BS{draw} [\RDD{step}=1mm] (0,0) grid (2,2);   }\\ 
%\hline  
%\tikz \draw[step=1mm] (0,0) grid (2,2);
%&  
%\tikz[rotate=30] \draw (0,0) grid (2,2);
%&  
%\tikz \draw (0,0) grid [xstep=.5] (2,2);
%&  
%\tikz \draw (0,0) grid [ystep=.5] (2,2);
%&
%\tikz \draw[help lines] (0,0) grid (2,2);
%\\ \hline  
%[\RDD{step}=1mm] & [\RDD{rotate}=30] & [\RDD{xstep}=.5] & [\RDD{ystep}=.5] & [\RDD{help lines}] \\ 
%\hline 
%\end{tabular} 

%\newpage 
%
%\input{tkzcoord}
%%
%%%%%%==========================================================
% 
%\newpage
%\section[Les n\oe uds]{Les n\oe uds }
%
%%\subsection{Définition des  n\oe uds}
\SbSSCT{Définition des  n\oe uds}{Creation of nodes}
\tikzset{blue}

\begin{tabular}{|c | c | c | c |} \hline
\multicolumn{4}{|c|}{  \BS{draw} (1,1) node[\RDD{fill}=red!20] \AC{};   }\\ 
\hline 
\tikz \draw (0,0) grid (2,2) (1,1) node[fill=red!20] {};
&
\tikz \draw (0,0) grid (2,2) (1,1) node[fill=red!20,draw] {}; 
&
\tikz \draw (0,0) grid (2,2) (1,1) node[circle,fill=red!20] {};
&
\tikz \draw (0,0) grid (2,2) (1,1) node[circle,fill=red!20,draw] {};
\\  \hline
\dft
&
node[\RDD{draw}] 
&
 node[\RDD{circle}]  
&
 node[\RDD{circle},\RDD{draw}]
 \\  \hline
\end{tabular}
\bigskip

\begin{tabular}{|c | c | c | c |} \hline
\multicolumn{4}{|c|}{ \BSS{node} \RDD{at} (1,1) [fill=red!20] \AC{};   }\\ 
\hline 
 \begin{tikzpicture}
\draw (0,0) grid (2,2) ; 
\node at (1,1) [fill=red!20] {};
 \end{tikzpicture}
&
 \begin{tikzpicture}
\draw (0,0) grid (2,2) ; 
\node at (1,1) [draw] {};
 \end{tikzpicture}
&
 \begin{tikzpicture}
\draw (0,0) grid (2,2) ; 
\node at (1,1) [fill=red!20,circle] {};
 \end{tikzpicture}
&
 \begin{tikzpicture}
\draw (0,0) grid (2,2) ; 
\node at (1,1) [circle,draw] {};
 \end{tikzpicture}
\\  \hline
[fill=red!20]
&
[\RDD{draw}] 
&
[\RDD{circle},fill=red!20]
 &
[\RDD{circle},draw] 
 \\  \hline
\end{tabular}
\bigskip

\TFRGB{Autres types de n\oe uds voir page}{Other type of nodes see page} \pageref{noeudboite}



%-------------------------------------------------------------------------------
%\subsection{Liaisons}
\SbSSCT{Liaisons}{Links}
\label{liaisons}

\begin{tabular}{|c|c|c|} \hline  
\begin{tikzpicture}[blue]
\node[draw] (A) at (0,0) {A};
\node[draw] (B) at (1.5,1.5) {B};
\draw (A) -- (B);
\end{tikzpicture}
&  
\begin{tikzpicture}[blue]
\node[draw] (A) at (0,0) {A};
\node[draw] (B) at (1.5,1.5) {B};
\draw (A) |- (B);
\end{tikzpicture}
&  
\begin{tikzpicture}[blue]
\node[draw] (A) at (0,0) {A};
\node[draw] (B) at (1.5,1.5) {B};
\draw (A) -| (B);
\end{tikzpicture}
\\ \hline  
(A){\color{red} - -} (B) & (A) {\color{red}|-} (B) &  (A) {\color{red}-|} (B)
\\ \hline 
\begin{tikzpicture}[blue]
\node[draw] (A) at (0,0) {A};
\node[draw] (B) at (1.5,1.5) {B};
\draw (A) to [bend right] (B);
\end{tikzpicture}
&  
\begin{tikzpicture}[blue]
\node[draw] (A) at (0,0) {A};
\node[draw] (B) at (1.5,1.5) {B};
\draw (A) to [bend left] (B);
\end{tikzpicture}
&  
\begin{tikzpicture}[blue]
\node[draw] (A) at (0,0) {A};
\node[draw] (B) at (1.5,1.5) {B};
\draw (A) to[bend left=0] (B);
\end{tikzpicture}
\\ \hline  
(A) to [\RDD{bend right}] (B) & (A) to [\RDD{bend left}] (B) &  (A) to[\RDD{bend left}=0] (B)
\\ \hline 
\begin{tikzpicture}[blue]
\node[draw] (A) at (0,0) {A};
\node[draw] (B) at (1.5,1.5) {B};
\draw (A) to[bend left=120]  (B);
\end{tikzpicture}
&  
\begin{tikzpicture}[blue]
\node[draw] (A) at (0,0) {A};
\node[draw] (B) at (1.5,1.5) {B};
\draw (A) to[bend left=45] (B);
\end{tikzpicture}
&  
\begin{tikzpicture}[blue]
\node[draw] (A) at (0,0) {A};
\node[draw] (B) at (1.5,1.5) {B};
\draw (A) to[bend left=90] (B);
\end{tikzpicture}
\\ \hline  
(A)  to[\RDD{bend left}=120]  (B) & (A) to[\RDD{bend left}=45] (B) &  (A) to[\RDD{bend left}=90] (B)
\\ \hline 
\begin{tikzpicture}[blue]
\node[draw] (A) at (0,0) {A};
\node[draw] (B) at (1.5,1.5) {B};
\draw (A)  to[out=90]  (B);
\end{tikzpicture}
&  
\begin{tikzpicture}[blue]
\node[draw] (A) at (0,0) {A};
\node[draw] (B) at (1.5,1.5) {B};
\draw (A) to[out=30] (B);
\end{tikzpicture}
&  
\begin{tikzpicture}[blue]
\node[draw] (A) at (0,0) {A};
\node[draw] (B) at (1.5,1.5) {B};
\draw (A)  to[in=-90]  (B);
\end{tikzpicture}
\\ \hline  
(A)  to[\RDD{out}=90] (B) & (A) to[\RDD{out}=30]  (B) &  (A)  to[\RDD{in}=-90]  (B)
\\ \hline 
%\begin{tikzpicture}[blue]
%\node[draw] (A) at (0,0) {A};
%\node[draw] (B) at (2,2) {B};
%\draw (A)  to[in=90]  (B);
%\end{tikzpicture}
%&  
%\begin{tikzpicture}[blue]
%\node[draw] (A) at (0,0) {A};
%\node[draw] (B) at (2,2) {B};
%\draw (B) to[in=0,out=90]  (B);
%\end{tikzpicture}
%&  
%\begin{tikzpicture}[blue]
%\node[draw] (A) at (0,0) {A};
%\node[draw] (B) at (2,2) {B};
%%\draw (A)  to[out=45,in=-90]  (A);
%\draw (B) to[out=45,in=135] (B);
%\end{tikzpicture}
%\\ \hline  
%(A)  to[\RDD{in}=90] (B) & (B) to[bend left]  (B) & (B) to[out=45,in=135] (B)
%\\ \hline 
\end{tabular} 

\bigskip
\begin{tabular}{|c|c|c|} \hline  
\multicolumn{2}{|c|}{ \BS{draw} (A) .. controls +(right:2cm) and +(down:2cm) .. (B);  }\\ 
\hline  
\begin{tikzpicture}[blue]
\node[draw] (A) at (0,0) {A};
\node[draw] (B) at (2,2) {B};
\draw  (A) .. controls +(right:2cm) and +(down:2cm) .. (B);
\end{tikzpicture}
&
\begin{tikzpicture}[blue]
\node[draw] (A) at (0,0) {A};
\node[draw] (B) at (2,2) {B};
\draw  (A) .. controls +(up:1cm) and +(left:1cm) .. (B);
\end{tikzpicture}
\\ \hline 
controls +(right:2cm) and +(down:2cm)  &
controls +(up:1cm) and +(left:1cm)
\\ \hline 
\begin{tikzpicture}[blue]
\node[draw] (A) at (0,0) {A};
\node[draw] (B) at (2,2) {B};
\draw  (A) .. controls +(right:1cm) and +(right:2cm) .. (B);
\end{tikzpicture}
&
\begin{tikzpicture}[blue]
\node[draw] (A) at (0,0) {A};
\node[draw] (B) at (2,2) {B};
\draw  (A) .. controls +(up:1cm) and +(right:2cm) .. (B);
\end{tikzpicture}
\\ \hline 
controls +(right:1cm) and +(right:2cm)  &
controls +(up:1cm) and +(right:2cm) 
\\ \hline 
\begin{tikzpicture}[blue]
\node[draw] (A) at (0,0) {A};
\node[draw] (B) at (2,2) {B};
\draw  (A) .. controls +(120:2cm) and +(200:1cm) .. (B);
\end{tikzpicture}
 &
 \begin{tikzpicture}[blue]
 \node[draw] (A) at (0,0) {A};
 \node[draw] (B) at (2,2) {B};T
 \draw  (A) .. controls +(120:2cm) and +(200:1cm) .. (A);
 \end{tikzpicture}
\\  \hline  
controls +(120:2cm) and +(200:1cm) & controls +(120:2cm) and +(200:1cm) 
\\ \hline 
\begin{tikzpicture}[blue]
\node[draw] (A) at (0,0) {A};
\node[draw] (B) at (2,2) {B};
\node[draw] (C) at (0,1) {C};
\node[draw] (D) at (3,0) {D};
\draw  (A) .. controls +(C) and +(D) .. (B);
\end{tikzpicture}
&
\begin{tikzpicture}[blue]
\node[draw] (A) at (0,0) {A};
\node[draw] (B) at (2,2) {B};
\node[draw] (C) at (0,1) {C};
\node[draw] (D) at (3,0) {D};
\draw (A) .. controls +(D)  .. (B);
\end{tikzpicture}
\\ \hline 
controls +(C) and +(D) &
controls +(D) 
\\ \hline 
\end{tabular} 
 \bigskip
 
\begin{tabular}{|c|c|c|} \hline 
\multicolumn{3}{|l|}{ \BS{node}[draw] (A) at (0,0) \AC{A}  }\\

\multicolumn{3}{|l|}{ \BS{node}[draw] (B) at (2,2) \AC{B} \RDD{edge}  [->] (A);  }\\
\multicolumn{3}{|c|}{\RRR{17-12-1}}  \\
\hline 
 \begin{tikzpicture}
 \node[draw] (A) at (0,0) {A};
 \node[draw] (B) at (2,2) {B} edge [->] (A);
 \end{tikzpicture}
 &
 \begin{tikzpicture}
 \node[draw] (A) at (0,0) {A};
 \node[draw] (B) at (2,2) {B} edge [red]  (A);
 \end{tikzpicture}
 &
 \begin{tikzpicture}
 \node[draw] (A) at (0,0) {A};
 \node[draw] (B) at (2,2) {B} edge [dashed] (A);
 \end{tikzpicture}
\\ \hline 
[->] & [red]  & [dashed]
\\ \hline 
\end{tabular}

%---------------------------------------------------------------------------------
%\subsection{\'Etiquettes sur les n\oe uds}
\SbSSCT{\'Etiquettes sur les n\oe uds}{Node labels}

\begin{tabular}{|c|c|c|c|} \hline
\multicolumn{4}{|c|}{  \BS{fill}(0,0) circle (2pt) node[\RDD{above}] \AC{texte} ;   }\\ 
\hline 
  
\begin{tikzpicture} \draw[help lines] (-1,-1) grid (1,1) ;\fill (0,0) circle (2pt) node[above] {texte};\end{tikzpicture}
& 
\begin{tikzpicture} \draw[help lines] (-1,-1) grid (1,1) ;\fill (0,0) circle (2pt) node[below] {texte};\end{tikzpicture}
 &  
\begin{tikzpicture} \draw[help lines] (-1,-1) grid (1,1);\fill (0,0) circle (2pt) node[left] {texte};\end{tikzpicture}
 &  
\begin{tikzpicture} \draw[help lines] (-1,-1) grid (1,1); \fill (0,0) circle (2pt) node[right] {texte};\end{tikzpicture}
 \\  \hline 
 [\RDD{above}] & [\RDD{below}] & [\RDD{left}] &  [\RDD{right}]
 \\ \hline 
 \begin{tikzpicture} \draw[help lines] (-1,-1) grid (1,1) ;\fill (0,0) circle (2pt) node[above left] {texte};\end{tikzpicture}
 & 
 \begin{tikzpicture} \draw[help lines] (-1,-1) grid (1,1) ;\fill (0,0) circle (2pt) node[below left] {texte};\end{tikzpicture}
  &  
 \begin{tikzpicture} \draw[help lines] (-1,-1) grid (1,1);\fill (0,0) circle (2pt) node[above right] {texte};\end{tikzpicture}
  &  
 \begin{tikzpicture} \draw[help lines] (-1,-1) grid (1,1); \fill (0,0) circle (2pt) node[below right] {texte};\end{tikzpicture}
  \\  \hline 
  [\RDD{above left}] & [\RDD{below left}] & [\RDD{above right}] &  [\RDD{below right}]
  \\ \hline 
 \begin{tikzpicture} \draw[help lines] (-1,-1) grid (1,1) ;\fill (0,0) circle (2pt) node[anchor=south] {texte};\end{tikzpicture}
 & 
 \begin{tikzpicture} \draw[help lines] (-1,-1) grid (1,1) ;\fill (0,0) circle (2pt) node[anchor=west] {texte};\end{tikzpicture}
  &  
 \begin{tikzpicture} \draw[help lines] (-1,-1) grid (1,1);\fill (0,0) circle (2pt) node[anchor=north] {texte};\end{tikzpicture}
  &  
 \begin{tikzpicture} \draw[help lines] (-1,-1) grid (1,1); \fill (0,0) circle (2pt) node[anchor=east] {texte};\end{tikzpicture}
  \\  \hline 
  [\RDD{anchor=south}] & [\RDD{anchor=west}] & [\RDD{anchor=north}] & [\RDD{anchor=east                                                                                                                                                               }]
  \\ \hline 
 \begin{tikzpicture} \draw[help lines] (-1,-1) grid (1,1) ;\fill (0,0) circle (2pt) node[anchor=south east] {texte};\end{tikzpicture}
 & 
\begin{tikzpicture} \draw[help lines] (-1,-1) grid (1,1) ;\fill (0,0) circle (2pt) node[anchor=south west] {texte};\end{tikzpicture}
&  
\begin{tikzpicture} \draw[help lines] (-1,-1) grid (1,1);\fill (0,0) circle (2pt) node[anchor=north west] {texte};\end{tikzpicture}
&  
\begin{tikzpicture} \draw[help lines] (-1,-1) grid (1,1); \fill (0,0) circle (2pt) node[anchor=east] {texte};\end{tikzpicture}
\\  \hline 
[\RDD{anchor=south east}] & [\RDD{anchor=south west}] & [\RDD{anchor=north west}] & [\RDD{anchor=north east                                                                                                                                                              }]
  \\ \hline 
\end{tabular} 


\bigskip
\begin{tabular}{|c|c|c|c|} \hline
\multicolumn{4}{|c|}{  \BS{fill}(0,0) circle (2pt) node[\RDD{above}=.3cm] \AC{texte} ;   }\\ 
\hline 
  
\begin{tikzpicture} \draw[help lines] (-1,-1) grid (1,1) ;\fill (0,0) circle (2pt) node[above=.3cm] {texte};\end{tikzpicture}
& 
\begin{tikzpicture} \draw[help lines] (-1,-1) grid (1,1) ;\fill (0,0) circle (2pt) node[below=.3cm] {texte};\end{tikzpicture}
 &  
\begin{tikzpicture} \draw[help lines] (-1,-1) grid (1,1);\fill (0,0) circle (2pt) node[left=.3cm] {texte};\end{tikzpicture}
 &  
\begin{tikzpicture} \draw[help lines] (-1,-1) grid (1,1); \fill (0,0) circle (2pt) node[right=.3cm] {texte};\end{tikzpicture}
 \\  \hline 
 [\RDD{above}=.3cm] & [\RDD{below}=.3cm] & [\RDD{left}=.3cm] &  [\RDD{right}=.3cm]]
 \\ \hline 
\begin{tikzpicture} \draw[help lines] (-1,-1) grid (1,1) ;\fill (0,0) circle (2pt) node[above left=.3cm] {texte};\end{tikzpicture}
& 
\begin{tikzpicture} \draw[help lines] (-1,-1) grid (1,1) ;\fill (0,0) circle (2pt) node[below left=.3cm] {texte};\end{tikzpicture}
 &  
\begin{tikzpicture} \draw[help lines] (-1,-1) grid (1,1);\fill (0,0) circle (2pt) node[above right=.3cm] {texte};\end{tikzpicture}
 &  
\begin{tikzpicture} \draw[help lines] (-1,-1) grid (1,1); \fill (0,0) circle (2pt) node[below right=.3cm] {texte};\end{tikzpicture}
 \\  \hline 
 [\RDD{above left}=.3cm] & [\RDD{below left}=.3cm] & [\RDD{above right}=.3cm] &  [\RDD{below right}=.3cm]]
 \\ \hline 
 
 \end{tabular} 
 
%\begin{tikzpicture} \draw[help lines] (-1,-1) grid (1,1);\fill (0,0) circle (2pt) node[distance=.3cm] {texte};\end{tikzpicture} 
 
 \newpage
\selectlanguage{french}
 
 \begin{tabular}{|c|c|c|c|c|} \hline
 \multicolumn{5}{|l|}{ \BSS{shorthandoff}\AC{:} \footnotemark[1]  } \\
 \multicolumn{5}{|l|}{  \BS{node} [draw,\RDD{label}=right:texte] \AC{}   }\\
 \multicolumn{5}{|l|}{ \BSS{shorthandon}\AC{:} } \\ 
 \hline 
     \shorthandoff{:} 
 \tikz \node [draw,label=right:texte] {};
 \shorthandon{:}
 &
  \shorthandoff{:}
 \tikz \node [draw,label=left:texte] {};
 \shorthandon{:}
 &
  \shorthandoff{:}
 \tikz \node [draw,label=above:texte] {};
 \shorthandon{:}
 &
  \shorthandoff{:}
 \tikz \node [draw,label=below:texte] {};
 \shorthandon{:}
 &
  \shorthandoff{:}
 \tikz \node [draw,label=45:texte] {};
    \shorthandon{:}
   \\ \hline
  label=right & label=left &  label=above & label=below & label=45
    \\ \hline 
 \end{tabular}
 \footnotetext[1]{\TFRGB{désactivation et ré-activation de \og : \fg  conflit entre les modules Tikz et Babel en français}{Only useful when the package babel is loaded with the frenchb option    }}
 
 \bigskip
  \begin{tabular}{|c|c|c|c|c|} \hline
  \BS{fill}(0,0) circle (2pt) node[below right=.3cm,draw,label=45:étiquette] \AC{texte} ;
      \\ \hline 
  
  \shorthandoff{:}
\begin{tikzpicture} \draw[help lines] (-1,-1) grid (2,1); \fill (0,0) circle (2pt) node[below right=.3cm,draw,label=45:étiquette] {texte};\end{tikzpicture}
 \shorthandon{:}
 
    \\ \hline 
 \end{tabular}
\bigskip

 \shorthandoff{:}
 

 
\begin{tabular}{|c|c|c|} \hline
\multicolumn{3}{|c|}{  \BSS{shorthandoff}\AC{:} \BS{node}[circle,draw,blue,\RDD{pin}=texte] \AC{} ;   \BSS{shorthandon}\AC{:}  \footnotemark[1] }\\ 
\hline
\begin{tikzpicture} 
\node [circle,draw,blue,pin=texte] {};
\end{tikzpicture}
&
\begin{tikzpicture} 
\node [circle,draw,blue,pin=60:texte] {};
\end{tikzpicture}
&
\begin{tikzpicture} 
\node [circle,draw,blue,pin=right:texte] {};
\end{tikzpicture}
 \\ \hline
[circle,pin=texte] &   [circle,pin=60:texte] & [circle,pin=right:texte]
 \\ \hline 
\end{tabular}  

\bigskip
\begin{tabular}{|c|c|c|} \hline
\multicolumn{3}{|c|}{  \BS{tikz}[\RDD{pin position}=60] \BS{node} [circle,pin=texte] \AC{} ;   }\\ 
\hline 
\tikz[pin position=60] \node [circle,draw,blue,pin=texte] {};
&
\tikz[pin distance=0 cm] \node [circle,draw,blue,pin=60:texte] {};
&
\tikz[pin distance=2 cm] \node [circle,draw,blue,pin=60:texte,pin distance=0cm] {};
  \\ \hline
  [\RDD{pin position}=60] & [\RDD{pin distance}=0 cm] & [\RDD{pin distance}=2 cm]
    \\ \hline
  \dft{ : above} & \multicolumn{2}{|c|}{ \dft{ : 3 ex}}
      \\ \hline
\end{tabular}  

% % % % % % % % % % >>>>>>>>>> a voir : option edge <<<<<<<<<<<<<<<<<<<<<<<<<<<<<<<<<<<<<<

   \shorthandon{:} 
   
\selectlanguage{english}   
% >>>>>>>>>>>>>>>>>>>>>> A Voir : positioning librairy <<<<<<<<<<<<<<<<<<<<<<<<<<<<<<<<<<<<<<<

%\subsection{ N\oe uds  sur un chemin}
\SbSSCT{ N\oe uds  sur un chemin}{Nodes on a path}

\begin{tabular}{|c|c|c|} \hline
\multicolumn{3}{|c|}{  \BS{draw}(0,0) .. controls (1,2) and (2,-1) .. (4,0) node[\RDD{at end}] \AC{texte} ;   }\\ 
\hline 
\tikz \draw (0,0) .. controls (1,2) and (2,-1) .. (4,0) node[pos=0] {texte}; 
&
\tikz \draw (0,0) .. controls (1,2) and (2,-1) .. (4,0) node[pos=.33] {texte}; 
&
\tikz \draw (0,0) .. controls (1,2) and (2,-1) .. (4,0) node[at end] {texte}; 
  \\ \hline 
\RDD{pos}{\color{red}  =0} & \RDD{pos}{\color{red}  =.33} & \RDD{at end} (pos=1)
  \\ \hline 

\tikz \draw (0,0) .. controls (1,2) and (2,-1) .. (4,0) node[very near end] {texte}; 
&
\tikz \draw (0,0) .. controls (1,2) and (2,-1) .. (4,0) node[near end] {texte}; 
&
\tikz \draw (0,0) .. controls (1,2) and (2,-1) .. (4,0) node[midway] {texte}; 
  \\ \hline 
\RDD{very near end} (pos=0.875.) & \RDD{ near end} (pos=0.75) & \RDD{midway} (pos=0.5)
  \\ \hline 
  
\tikz \draw (0,0) .. controls (1,2) and (2,-1) .. (4,0) node[near start] {texte}; 
&
\tikz \draw (0,0) .. controls (1,2) and (2,-1) .. (4,0) node[very near start] {texte}; 
&
\tikz \draw (0,0) .. controls (1,2) and (2,-1) .. (4,0) node[at start] {texte};
\\ \hline 
\RDD{near start} (pos=0.25) & \RDD{very near start} (pos=0.125) & \RDD{at start} (pos=0)
  \\ \hline 
  
\end{tabular} 

\bigskip
\begin{tabular}{|c|c|c|} \hline
\multicolumn{3}{|c|}{  \BS{draw}(0,0) .. controls (1,2) and (2,1) .. (4,0) node[\RDD{sloped},midway] \AC{texte} ;   }\\ 
\hline 
\tikz \draw (0,0) .. controls (1,2) and (2,-1) .. (4,0) node[sloped,midway] {texte};
&
\tikz \draw (0,0) .. controls (1,2) and (2,-1) .. (4,0) node[above,midway] {texte};
&
\tikz \draw (0,0) .. controls (1,2) and (2,-1) .. (4,0) node[below,midway] {texte};
  \\ \hline
\RDD{sloped} & \RDD{above} &\RDD{below}
  \\ \hline
\end{tabular}
\bigskip

\begin{tabular}{|c|c|c|} \hline
\multicolumn{3}{|c|}{  \BS{draw}(0,0) .. controls (1,2) and (2,1) .. (5,0) node[\RDD{sloped},midway,allow upside down] \AC{texte} ;   }\\ 
\hline 
\tikz \draw (0,0) .. controls (1,2) and (2,-1) .. (4,0) node[sloped,midway,allow upside down] {texte};
&
\tikz \draw (0,0) .. controls (1,2) and (2,-1) .. (4,0) node[above,midway,allow upside down] {texte};
&
\tikz \draw (0,0) .. controls (1,2) and (2,-1) .. (4,0) node[below,midway,allow upside down] {texte};
  \\ \hline
\RDD{sloped} & \RDD{above} &\RDD{below}
  \\ \hline
\end{tabular}  


\begin{tabular}{|c|c|c|} \hline
\multicolumn{3}{|c|}{  \BS{draw}(A)  to [bend right]  node [\RDD{bend right}] \AC{texte} (B);   }\\ 
\hline 
\begin{tikzpicture} %[auto,bend right]
\node[draw] (A) at (0,0) {A};
\node[draw] (B) at (2,2) {B};
\draw (A) to [bend right] node [bend right] {texte} (B);
\end{tikzpicture}
&
\begin{tikzpicture} 
\node[draw] (A) at (0,0) {A};
\node[draw] (B) at (2,2) {B};
\draw (A) to [bend right] node [auto,bend right] {texte} (B);
\end{tikzpicture}
&
\begin{tikzpicture} 
\node[draw] (A) at (0,0) {A};
\node[draw] (B) at (2,2) {B};
\draw (A) to[bend right] node [auto,swap,bend right] {texte} (B);
\end{tikzpicture}
  \\ \hline
[bend right]  & [\RDD{auto},bend right] & [auto,\RDD{swap},bend right] 
  \\ \hline
\end{tabular}  

\SbSSCT{ N\oe uds  sur un \og edge\fg}{Nodes on an edge}

\begin{tabular}{|c|c|c|}\hline  
\multicolumn{3}{|c|}{  \BS{draw}(0,0) edge \BDD{["abc", ->]} (4,0);  }\\ 
\multicolumn{3}{|c|}{  \RRR{17-12-2} }\\ 
\hline 
\begin{tikzpicture}[blue] 
\useasboundingbox  (0,-.5) rectangle (4,.5); 
\draw (0,0) edge ["abc", ->] (4,0);
\end{tikzpicture}
&
\begin{tikzpicture}[blue] 
\useasboundingbox  (0,-.5) rectangle (4,.5); 
\draw (0,0) edge ["abc", near start] (4,0);
\end{tikzpicture}
&
\begin{tikzpicture}[blue] 
\useasboundingbox  (0,-.5) rectangle (4,.5); 
\draw (0,0) edge ["abc", style={auto=right}] (4,0);
\end{tikzpicture}
\\ \hline 
["abc", ->]
& 
["abc", near start] &  ["abc", style=\AC{auto=right}] 
\\ \hline  
\begin{tikzpicture}[blue] 
\useasboundingbox  (0,-.5) rectangle (4,.5); 
\draw (0,0) edge [font=\Large,"abc" ] (4,0);
\end{tikzpicture}
&
\begin{tikzpicture}[blue] 
\useasboundingbox  (0,-.5) rectangle (4,.5); 
\draw (0,0) edge ["abc" color=red ] (4,0);
\end{tikzpicture}
&
\begin{tikzpicture}[blue] 
\useasboundingbox  (0,-.5) rectangle (4,.5); 
 \draw (0,0) edge ["abc" '] (4,0);
\end{tikzpicture}
\\ \hline 
[font=\BS{Large},"abc" ] & ["abc" color=red ]
&["abc" ' ]
\\ \hline 

\begin{tikzpicture}[blue] 
\useasboundingbox  (0,-.5) rectangle (4,.75); 
\draw (0,0) edge ["abc" draw ] (4,0);
\end{tikzpicture}
&
\begin{tikzpicture}[blue] 
\useasboundingbox  (0,-.5) rectangle (4,.5); 
\draw (0,0) edge ["abc" inner sep=0pt ] (4,0);
\end{tikzpicture}
&
\begin{tikzpicture}[blue] 
\useasboundingbox  (0,-.5) rectangle (4,.5); 
\draw (0,0) edge ["abc" fill ,fill=yellow ] (4,0);
\end{tikzpicture}
\\ \hline
["abc" draw ]
&
["abc" inner sep=0pt ]
&
["abc" fill ,fill=yellow ]
\\ \hline
\end{tabular} 



\bigskip

\begin{tabular}{|c|} \hline  
\BS{draw}[every edge quotes/.style=\AC{fill=yellow}] (0,0) edge ["abc"] (4,0);
\\ \hline  
\begin{tikzpicture}[blue] 
\useasboundingbox  (0,-.5) rectangle (4,.5); 
 \draw[every edge quotes/.style={fill=yellow}] (0,0) edge ["abc"] (4,0);
\end{tikzpicture}
\\ \hline 
\end{tabular} 





%%
%\newpage
%
%%%%======================================================
%\section[Constructions particulières]{Constructions particulières  }
%%
%
%\subsubsection{Transformations}

\begin{center}
\RRR{25-3}
\end{center}


\begin{tabular}{|c|c|c|c|} \hline 
\multicolumn{4}{|c|}{  \BS{draw}[\RDD{rotate},blue] (0,0)  rectangle  (2,2) ;   }\\ 
\hline  
\begin{tikzpicture}
\draw[dashed,red] (0,0) rectangle  (2,2) ; 
\draw[rotate=40,blue] (0,0) rectangle  (2,2) ;
\end{tikzpicture}
&  
\begin{tikzpicture}
\draw[dashed,red] (0,0) rectangle  (2,2) ; 
\draw[x=1cm,y=.5cm,blue] (0,0) rectangle  (2,2); 
\end{tikzpicture}
&  
\begin{tikzpicture}
\draw[dashed,red] (0,0) rectangle  (2,2) ; 
\draw[xslant=.75,blue] (0,0) rectangle  (2,2);  
\end{tikzpicture}
&
\begin{tikzpicture}
\draw[dashed,red] (0,0) rectangle  (2,2) ; 
\draw[yslant=.75,blue] (0,0) rectangle  (2,2);  
\end{tikzpicture}
\\ \hline  
\RDD{rotate}=40 & \RDD{x}=1cm,\RDD{y}=0.5cm & \RDD{xslant}=0.75 & \RDD{yslant}=0.75\\ 
\hline 
  
\begin{tikzpicture}
\draw[dashed,red] (0,0) rectangle  (2,2) ; 
\draw[scale=1.5,blue] (0,0) rectangle  (2,2) ; 
\end{tikzpicture}
&  
\begin{tikzpicture}
\draw[dashed,red] (0,0) rectangle  (2,2) ; 
\draw[scale=-1,y=.5cm,blue] (0,0) rectangle  (2,2); 
\end{tikzpicture}
&  
\begin{tikzpicture}
\draw[dashed,red] (0,0) rectangle  (2,2) ; 
\draw[xshift=.5cm,blue] (0,0) rectangle  (2,2); 
\end{tikzpicture}
&
\begin{tikzpicture}
\draw[dashed,red] (0,0) rectangle  (2,2) ; 
\draw[yshift=.5cm,blue] (0,0) rectangle  (2,2);  
\end{tikzpicture}
\\ \hline  
\RDD{scale}=1.5 & \RDD{scale}=-1 & \RDD{xshift}=0.5cm & \RDD{yshift}=0.5cm
\\ \hline 
\end{tabular} 

\bigskip

%==============================================
%\begin{tikzpicture}
%\draw[help lines] (0,0) grid (3,2);
%\draw (0,0) - - (1,1) - - (1,0);
%\draw[rotate=40,blue] (0,0) - - (1,1) - - (1,0); % rotation 40°
%\draw[rotate=-20,red] (0,0) - - (1,1) - - (1,0); % rotation -20°
%\end{tikzpicture}

%\tikz \draw[x=1cm,y=.5cm] (0,0) rectangle(2,2);

%\tikz \draw (0,0) rectangle (1,0.5) [xshift=2cm] (0,0) rectangle (1,0.5);

%\begin{tikzpicture}
%\draw[help lines] (0,0) grid (3,2);
%\draw (0,0) - - (1,1) - - (1,0);
%\draw[scale=2,blue] (0,0) - - (1,1) - - (1,0); % échelle 2
%\draw[scale=-1,red] (0,0) - - (1,1) - - (1,0); % échelle -1
%\end{tikzpicture}

%\begin{tikzpicture}
%\draw[help lines] (0,0) grid (3,2);
%\draw (0,0) - - (1,1) - - (1,0);
%\draw[xslant=2,blue] (0,0) - - (1,1) - - (1,0);
%\draw[xslant=-1,red] (0,0) - - (1,1) - - (1,0);
%\end{tikzpicture}

%\begin{tikzpicture}
%\draw[help lines] (0,0) grid (3,2);
%\draw (0,0) rectangle (1,0.5);
%\beginscope[xshift=2cm] % Décalage en X de 2cm
%\draw [red] (0,0) rectangle (1,0.5);
%\draw[yshift=1cm,blue] (0,0) rectangle (1,0.5);
%\draw[rotate=30,orange] (0,0) rectangle (1,0.5);
%\endscope
%\end{tikzpicture}
 
%
%\newpage
%
%\section{Placer son dessin}
%%
%\input{tkzfig}
%
%\newpage
%
%\section{Scope}
%%
%\input{tkzscope} 
%à
%\newpage
%
%\section{Position absolue sur une page}
%
% \input{tkzpage}
% 
%\newpage 
%
%\section{Arrière plan du dessin}
%
% \input{tkzbackground}
%
%\newpage 
%
%%%\section{Placer des objets}
%%%
%%%%\input{plac}
%%%
%%
%%%\newpage
%%%
%%%%===================================================
%\section{Créer ses couleurs}
%
%
% \input{tkzcoul}
% 
%
%\newpage
%
%\section{Créer ses commandes}
%
%\input{tkzcde}
%
%
%\newpage
%
%\section[Créer ses styles]{Créer ses styles}
%
%\input{tkzstyl}
%
%%%%%%%%======================================================================
%
%\newpage
%
%\section{Mettre du texte  en valeur}
%
%
%\label{ndbt}

\tikzset{blue}

%\subsection{Dans un n\oe ud de Tikz}
\SbSSCT{Dans un n\oe ud de Tikz}{In a TikZ node}
\label{noeudboite}

\begin{tabular}{|c | c | c | c |} \hline
\multicolumn{4}{|c|}{ \BS{tikz} \BS{draw} (0,0) grid (2,2) (1,1) node[fill=red!20,] \AC{texte};   }\\ 
\hline 
\tikz \draw (0,0) grid (2,2) (1,1) node[fill=red!20] {texte};
&
\tikz \draw (0,0) grid (2,2) (1,1) node[fill=red!20,draw] {texte}; 
&
\tikz \draw (0,0) grid (2,2) (1,1) node[circle,fill=red!20] {texte};
&
\tikz \draw (0,0) grid (2,2) (1,1) node[circle,fill=red!20,draw] {texte};
\\  \hline
node[fill=red!20] 
&
node[fill=red!20,\RDD{draw}] 
&
 node[fill=red!20,\RDD{circle}]  
&
 node[fill=red!20,\RDD{circle},\RDD{draw}]
 \\  \hline
\end{tabular}
\bigskip


\subsubsection{Options}
\begin{tabular}{|c | c | c | c |c |c |c |c |} \hline
\multicolumn{8}{|c|}{ \BS{tikz} \BS{draw} node[draw,\RDD{double},blue] \AC{texte};   }\\ 
\hline 

\tikz \draw  node[draw,double,blue] {texte};
&
\tikz \draw  node[draw,rounded corners,blue] {texte};
&
\tikz \draw  node[draw,ultra thick,blue] {texte};
&
\tikz \draw  node[draw,dashed,blue] {texte};
&
\tikz \draw  node[draw,red] {texte};
&
\tikz \draw  node[draw,rotate=45,blue] {texte};
&
\tikz \draw  node[draw,shading=radial,blue] {texte};
&
\tikz \draw  node[draw,blue,text=red] {texte};
\\ \hline
\RDD{double} & \RDD{rounded corners} &  ultra thick & dashed & red & rotate=45 & shading=radial & text=red 
\\ \hline
\end{tabular}
\bigskip


\begin{tabular}{|c | c | c | c |c |} \hline
\multicolumn{4}{|c|}{ \BS{tikz} \BS{draw}  node[draw,\RDD{inner sep}=0pt] \AC{texte};   }\\ 
\hline 
\tikz \draw  node[draw,inner sep=0pt,blue] {texte};
&
\tikz \draw node[draw,inner sep=1cm,blue] {texte};
&
\tikz \draw  node[draw,inner xsep=1cm,blue] {texte};
&
\tikz \draw  node[draw,inner ysep=1cm,blue] {texte};
\\ \hline
 \RDD{inner sep}=0pt & \RDD{inner sep}=1cm & \RDD{inner xsep}=1cm & \RDD{inner ysep}=1cm
\\ \hline
\multicolumn{4}{|c|}{ \dft{} : 0.3333em }\\ 
\hline 

\end{tabular}

\bigskip

\begin{tabular}{|c | c | c | c |} \hline
\multicolumn{4}{|l|}{ \BS{node} [fill=red!20,\RDD{outer sep}=1cm] (A) at (1,1) \AC{texte};   }\\ 
\multicolumn{4}{|l|}{ \BS{fill} (node cs:name=A,anchor=east) circle (3pt);  }\\ 
\multicolumn{4}{|l|}{ \BS{fill} (node cs:name=A,anchor=south) circle (3pt);  }\\ 
\hline 
\begin{tikzpicture}
\draw[help lines] (0,0) grid (3,2);
\node[fill=red!20,outer sep=1cm] (A) at (1,1) {texte};
\fill[red] (node cs:name=A,anchor=east) circle (3pt);
\fill[red] (node cs:name=A,anchor=south) circle (3pt);
\end{tikzpicture}
&
\begin{tikzpicture}
\draw[help lines] (0,0) grid (3,2);
\node[fill=red!20,outer sep=0pt] (A) at (1,1) {texte};
\fill[red] (node cs:name=A,anchor=east) circle (3pt);
\fill[red] (node cs:name=A,anchor=south) circle (3pt);
\end{tikzpicture}
&
\begin{tikzpicture}
\draw[help lines] (0,0) grid (3,2);
\node[fill=red!20,outer xsep=1cm] (A) at (1,1){texte};
\fill[red] (node cs:name=A,anchor=east) circle (3pt);
\fill[red] (node cs:name=A,anchor=south) circle (3pt);
\end{tikzpicture}
&
\begin{tikzpicture}
\draw[help lines] (0,0) grid (3,2);
\node[fill=red!20,outer ysep=1cm] (A) at (1,1) {texte};
\fill[red] (node cs:name=A,anchor=east) circle (3pt);
\fill[red] (node cs:name=A,anchor=south) circle (3pt);
\end{tikzpicture}
\\ \hline
 \RDD{outer sep}=1cm & \RDD{outer sep}=0pt & \RDD{outer xsep}=1cm & \RDD{outer ysep}=1cm
\\ \hline
\multicolumn{4}{|c|}{ \dft{} : 0.5\BS{pgflinewidth} }\\ 
\hline 
\end{tabular}
%----------------------------------------------------------------------------------
%\subsubsection{Taille minimale des noeuds}
\SbSbSSCT{Taille minimale des noeuds}{Minimum size}

\begin{tabular}{|c|c|} \hline  
\multicolumn{2}{|c|}{  \BS{draw}((0,0) node[fill=blue!20,\RDD{minimum height}=1.5cm,draw]  \AC{texte} ;   }\\ 
\hline 
\tikz \draw (0,0) node[fill=red!20,minimum height=1.5cm,draw] {texte};
&  
\tikz \draw (0,0) node[fill=red!20,minimum width=3cm,draw] {texte};

\\ \hline  

\RDD{minimum height}=1.5cm
&  
\RDD{minimum width}=3cm
\\ \hline  
\tikz \draw (0,0) node[fill=red!20,minimum size=1.5cm,draw] {texte};
&  
\tikz \draw (0,0) node[fill=red!20,minimum size=1.5cm,draw,circle] {texte};

\\ \hline 
\RDD{minimum size}=1.5cm,draw
&  
\RDD{minimum size}=1.5cm,circle

\\ \hline 
\end{tabular} 

\newpage
%-----------------------------------------------
%\subsection{Dans un n\oe ud à formes géométriques}
\SbSSCT{Dans un n\oe ud à formes géométriques}{Geometric Shapes nodes}

\label{lib-geom}
\label{formes}
%Insérer dans le préambule :

 \maboite{\BS{usetikzlibrary}\AC{shapes.geometric}}
 
 
\begin{center}
\RRR{67-3}
\end{center}
%\subsubsection{Formes disponibles}
\SbSbSSCT{Formes disponibles}{Available shapes}

\label{nd1}

\begin{tabular}{|c|c|c|c|} \hline  
\multicolumn{4}{|l|}{ 2 syntaxes :   }\\ 
\multicolumn{4}{|l|}{ \BS{tikz} \BS{node}[fill=green!20,\RDD{shape}=diamond,draw,blue] \AC{texte};   }\\ 
\multicolumn{4}{|l|}{ \BS{tikz} \BS{node}[fill=green!20,\RDD{diamond},draw] \AC{texte};   }\\ 
\hline 
\tikz  \node[fill=green!20,diamond,draw] {texte}; 
&  
\tikz  \node[fill=green!20,ellipse,draw] {texte};
&  
\tikz  \node[fill=green!20,trapezium, regular polygon sides=6,draw] {texte};
&
\tikz  \node[fill=green!20,semicircle,draw] {texte}; 
\\ \hline 
diamond & ellipse  & trapezium & semicircle
\\ \hline 
\tikz  \node[fill=green!20,star,draw] {texte};
&  
\tikz  \node[fill=green!20,regular polygon,draw] {texte};
&  
\tikz  \node[fill=green!20,isosceles triangle,draw] {texte};
&
\tikz  \node[fill=green!20,kite,draw] {texte};
\\ \hline 
star & regular polygon  & isosceles triangle & kite 
\\ \hline 
\tikz  \node[fill=green!20,dart,draw] {texte};
&
\tikz  \node[fill=green!20,circular sector,draw] {texte};
&
\tikz  \node[fill=green!20,cylinder,draw] {texte};
&

\\ \hline 
dart & circular sector & cylinder &
\\ \hline 
\end{tabular} 

%---------------------------------------------------------------------------------------
\subsubsection{Options}

\begin{tabular}{|c|c|c|} \hline
\multicolumn{3}{|c|}{  \BS{node} [trapezium,draw,\RDD{trapezium left angle}=90,draw,blue] \AC{texte};   }\\ 
\hline
\begin{tikzpicture}
\node[trapezium,draw,red,dashed] {texte};
\node[trapezium,draw,trapezium left angle=90,draw,blue] {texte};
\end{tikzpicture}
& 
\begin{tikzpicture}
\node[trapezium,draw,red,dashed] {texte};
\node[trapezium,draw,trapezium right angle=90,draw,blue] {texte};
\end{tikzpicture} 
& 
\begin{tikzpicture}
\node[trapezium,draw,red,dashed] {texte};
\node[trapezium,draw,trapezium angle=120,draw,blue] {texte};
\end{tikzpicture} 
\\ \hline
\RDD{trapezium left angle}=90  & \RDD{trapezium right angle}=90  & \RDD{trapezium  angle}=120 \\ 
\hline 
\begin{tikzpicture}
\node[trapezium,draw,red,dashed] {texte};
\node[trapezium,draw,minimum height=1.5cm,trapezium stretches=true,draw,blue] {texte};
\end{tikzpicture}
& 
\begin{tikzpicture}
\node[trapezium,draw,red,dashed] {texte};
\node[trapezium,draw,minimum height=1.5cm,trapezium stretches=false,draw,blue] {texte};
\end{tikzpicture} 
& 
\begin{tikzpicture}
\node[trapezium,draw,red,dashed] {texte};
\node[trapezium,draw,minimum width=3cm,trapezium stretches =false,draw,blue] {texte};
\end{tikzpicture} 

\\ \hline
minimum height=1.5cm & minimum height=1.5cm & minimum width=1.5cm \\
\RDD{trapezium stretches}=true & \RDD{trapezium stretches}=false & \RDD{trapezium stretches}  \\ 
\hline
%
%& 
%\begin{tikzpicture}
%\node[trapezium,draw,red,dashed] {texte};
%\node[trapezium,draw,minimum width=1.5cm,trapezium stretches body=false,draw,blue] {texte};
%\end{tikzpicture} 
%&
%\\
\end{tabular} 

%\tikz  \draw (-1,-1) grid (1,1) (0,0) node[fill=red!20,shape=trapezium,draw,minimum height=1.5cm,trapezium stretches=true] {texte};
%
%\tikz  \draw (-1,-1) grid (1,1) (0,0) node[fill=red!20,shape=trapezium,draw,minimum height=1.5cm,trapezium stretches=false] {texte};
%
%\tikz  \draw (-1,-1) grid (1,1) (0,0) node[fill=red!20,shape=trapezium,draw,minimum width=1.5cm,trapezium stretches] {texte};
%
%\tikz  \draw (-1,-1) grid (1,1) (0,0) node[fill=red!20,shape=trapezium,draw,minimum width=1.5cm,trapezium stretches body] {texte};


\bigskip
\begin{tabular}{|c|c|c|} \hline
\multicolumn{3}{|c|}{ \BS{tikz} \BS{node} [fill=green!20,star,\RDD{star points}=6,draw] \AC{texte};   }\\ 
\hline
\begin{tikzpicture}
\node[star,draw,red,dashed] {texte};
\node[star,star points=7,draw,blue] {texte};
\end{tikzpicture}
&  
\begin{tikzpicture}
\node[star,draw,red,dashed] {texte};
\node[star,star point height = 2cm,draw,blue] {texte};
\end{tikzpicture} 
&  
\begin{tikzpicture}
\node[star,draw,red,dashed] {texte};
\node[star,star point ratio = 3,draw,blue] {texte};
\end{tikzpicture} 
\\ \hline  
\RDD{star points}=7 & \RDD{star point height} = 2cm & \RDD{star point ratio} = 3 \\ \hline
\dft{5} & \dft.5cm &  \dft{1.5}\\ 
\hline 
\end{tabular} 
\bigskip

\begin{tabular}{|c|c|c|} \hline
\multicolumn{3}{|c|}{  \BS{node} [isosceles triangle,\RDD{isosceles triangle apex angle}=90,draw,blue] \AC{texte};   }\\ 
\multicolumn{3}{|c|}{  \BS{node} [regular polygon, \RDD{regular polygon sides}=6,draw,blue] \AC{texte};   }\\ 
\hline
\begin{tikzpicture}
\node[isosceles triangle,draw,red,dashed] {texte};
 \node[isosceles triangle,isosceles triangle apex angle=90,draw,blue] {texte};
\end{tikzpicture} 
& 
\begin{tikzpicture}
\node[isosceles triangle,draw,red,dashed] {texte};
 \node[isosceles triangle,isosceles triangle stretches=true,draw,blue] {texte};
\end{tikzpicture}
&  
\begin{tikzpicture}
\node[regular polygon,draw,red,dashed] {texte};
\node[regular polygon, regular polygon sides=6,draw,blue] {texte};
\end{tikzpicture} 
\\ \hline  
\RDD{isosceles triangle apex angle}=90 & \RDD{isosceles triangle stretches} & \RDD{regular polygon sides}=6 \\ 
\hline 
\end{tabular} 
\bigskip

\begin{tabular}{|c|c|c|} \hline 
\multicolumn{3}{|c|}{  \BS{node} [kite,\RDD{kite upper vertex angle}=90,draw,blue] \AC{texte};   }\\ 
\hline 
\begin{tikzpicture}
\node[red,kite,draw,dashed] {texte} ;
 \node[kite,kite upper vertex angle=90,draw,blue] {texte};
\end{tikzpicture} 
&  
\begin{tikzpicture}
\node[red,kite,draw,dashed] {texte} ;
 \node[kite,kite lower vertex angle=90,draw,blue] {texte};
\end{tikzpicture} 
&  
\begin{tikzpicture}
\node[red,kite,draw,dashed] {texte} ;
\node[kite,kite vertex angles=90,draw,blue] {texte};
\end{tikzpicture} 
\\ \hline  
\RDD{kite upper vertex angle}=90 & \RDD{kite lower vertex angle}=90 &\RDD{kite vertex angles}=90
\\ \hline 
initially 120 & initially 60 &  \\ 
\hline 
\end{tabular} 

\bigskip

\begin{tabular}{|c|c|c|} \hline
\multicolumn{3}{|c|}{  \BS{node} [dart,\RDD{dart tip angle}=90,draw,blue] \AC{texte};   }\\ 
\hline 
\begin{tikzpicture}
\node[dart,draw,red,dashed] {texte};
\node[dart,dart tip angle=90,draw,blue] {texte};
\end{tikzpicture} 
&  
\begin{tikzpicture}
\node[dart,draw,red,dashed] {texte};
\node[dart,dart tail angle=90,draw,blue] {texte};
\end{tikzpicture} 
&  
\begin{tikzpicture}
\node[,circular sector,draw,red,dashed] {texte};
\node[circular sector,circular sector angle=90,draw,blue] {texte};
\end{tikzpicture} 
\\ \hline  
\RDD{dart tip angle}=90 & \RDD{dart tail angle}=90  & \RDD{circular sector angle}=90
\\ \hline  
initially 45 & initially 135 & initially 60  \\ 
\hline 
\end{tabular} 

\bigskip

\begin{tabular}{|c|c|} \hline  
\multicolumn{2}{|c|}{  \BS{node} [cylinder,\RDD{aspect=2},draw,blue] \AC{texte};   }\\ 
\hline
\tikz  \node[cylinder,aspect=2,draw,blue] {texte};
& 
 \tikz  \node[cylinder,aspect=4,draw,blue] {texte};
\\ \hline 
\RDD{aspect}=2 & \RDD{aspect}=4 
\\ \hline
\tikz  \node[cylinder,cylinder uses custom fill, cylinder end fill=yellow,draw,blue] {texte};
&  
\tikz  \node[cylinder,cylinder uses custom fill, cylinder body fill=yellow,draw,blue] {texte};
\\ \hline
\RDD{cylinder uses custom fill}, & \RDD{cylinder uses custom fill}, \\ 
\RDD{cylinder end fill}=yellow & \RDD{cylinder body fill}=yellow  \\ 
\hline 
\end{tabular} 

%\subsection{Ratio hauteur/largeur}
\bigskip

\begin{tabular}{|c|c|c|c|} \hline 
\multicolumn{4}{|c|}{  \BS{draw}(0,0) node[\RDD{shape aspect}=1,diamond,draw]  \AC{texte} ;   }
\\ \hline
 
\tikz \draw (0,0) node[shape aspect=1,diamond,draw,blue] {texte};
&  
\tikz \draw (0,-2) node[shape aspect=2,diamond,draw,blue] {texte};
&
\tikz \draw (0,0) node[shape aspect=3,diamond,draw,blue] {texte};
&
\tikz \draw (0,0) node[shape aspect=4,diamond,draw,blue] {texte};
\\ \hline  
\RDD{shape aspect}=1
&  
\RDD{shape aspect}=2
&
\RDD{shape aspect}=3
&
\RDD{shape aspect}=4
\\ \hline 
\end{tabular} 


%==============================================================
\newpage
%\subsection{Dans un n\oe ud en forme de symboles}
\SbSSCT{Dans un n\oe ud en forme de symboles}{Symbol Shapes nodes}
\label{lib-symb}

\maboite{\BS{usetikzlibrary}\AC{shapes.symbols}}

\begin{center}
\RRR{67-4}
\end{center}

%\subsubsection{Formes disponibles}
\SbSbSSCT{Formes disponibles}{Available shapes}

\label{nd2}

\begin{tabular}{|c|c|c|} \hline  
\tikz  \node[fill=green!20,forbidden sign,draw] {texte};
&  
\tikz  \node[fill=green!20,magnifying glass,draw] {texte};
&  
\tikz  \node[fill=green!20,cloud,draw] {texte};
\\ \hline 
forbidden sign & magnifying glass & cloud
\\ \hline  
\tikz  \node[fill=green!20,starburst,draw] {texte};
&  
\tikz  \node[fill=green!20,signal,draw] {texte};

&  
\tikz  \node[fill=green!20,tape,draw] {texte};
\\ \hline 
starburst & signal & tape
\\ \hline 
\end{tabular} 
\bigskip

\subsubsection{Options}

\begin{tabular}{|c|c|c|} \hline  
\multicolumn{3}{|c|}{  \BS{node}[magnifying glass,\RDD{magnifying glass handle angle}=45,draw,blue]  \AC{texte} ;   }
\\ \hline
\tikz  \node[magnifying glass,magnifying glass handle angle=45,draw,blue] {texte};
&  
\tikz  \node[,magnifying glass,magnifying glass handle aspect=3,draw,blue] {texte};
& 
\tikz  \node[magnifying glass,line width=1ex,draw,blue] {texte};

\\ \hline  
\RDD{magnifying glass handle angle}=45 & \RDD{magnifying glass handle aspect}=3  & line width=1ex  
\\ \hline 
\dft{ : -45} & \dft{ : 1.5}& 
\\ \hline 
\end{tabular} 

\bigskip

\begin{tabular}{|c|c|c|c|} \hline 
\multicolumn{4}{|c|}{  \BS{node} [cloud,\RDD{cloud puffs}=5,draw,blue] \AC{texte};   }\\ 
\hline 
\begin{tikzpicture}
\node[cloud,draw,red,dashed] {texte};
\node[cloud,cloud puffs=5,draw,blue] {texte};
\end{tikzpicture} 
&  
\begin{tikzpicture}
\node[cloud,draw,red,dashed] {texte};
\node[cloud,cloud puff arc=270,draw,blue] {texte};
\end{tikzpicture} 
&  
\begin{tikzpicture}
\node[cloud,draw,red,dashed] {texte};
\node[cloud,cloud ignores aspect=true,draw,blue] {texte};
\end{tikzpicture} 
&
\begin{tikzpicture}
\node[cloud,draw,red,dashed] {texte};
\node[cloud,cloud ignores aspect=false,draw,blue] {texte};
\end{tikzpicture} 
\\ \hline  
\RDD{cloud puffs}=5 & \RDD{cloud puff arc}=270 & \RDD{cloud ignores aspect}=false & \RDD{cloud ignores aspect}=true  \\ 
\hline 
\dft :  10 & \dft :  135 &\multicolumn{2}{|c|}{ \dft :  true } \\ \hline
\end{tabular} 

\bigskip

\begin{tabular}{|c|c|c|c|} \hline 
\multicolumn{4}{|c|}{  \BS{node} [starburst,\RDD{starburst points}=5,draw,blue] \AC{texte};   }\\ 
\hline  
\tikz  \node[starburst,starburst points=5,draw,blue] {texte};
&  
\tikz  \node[starburst,starburst point height=1cm,draw,blue] {texte};
&  
\tikz  \node[starburst,random starburst=50,draw,blue] {texte};
&
\tikz  \node[,starburst,random starburst=0,draw,blue] {texte};
\\ \hline  
\RDD{starburst points}=5 & \RDD{starburst point height}=1cm & \RDD{random starburst}=50 & \RDD{random starburst}=0  \\ 
\hline 
\end{tabular} 

\bigskip


\begin{tabular}{|c|c|c|} \hline 
\multicolumn{3}{|c|}{  \BS{node} [signal,\RDD{signal pointer angle}=45,draw,blue] \AC{texte};   }\\ 
\hline 
\tikz  \node[signal,signal pointer angle=45,draw,blue] {texte};
&
\tikz  \node[signal,signal pointer angle=10,draw,blue] {texte};
&
\tikz  \node[signal,signal pointer angle=300,draw,blue] {texte};
\\ \hline 
\RDD{signal pointer angle}=45
&
signal pointer angle=10
&
signal pointer angle=300
\\ \hline 
\multicolumn{3}{|c|}{  \dft{ : signal pointer angle= 90}  }
\\  \hline 

\end{tabular} 
\bigskip

\begin{tabular}{|c|c|c|c|c|} \hline 
\multicolumn{4}{|c|}{  \BS{node} [signal,\RDD{signal to}=above,draw,blue] \AC{texte};   }
\\ \hline 
\tikz  \node[signal,signal to=above,draw,blue] {texte};
&  
\tikz  \node[signal,signal to=below,draw,blue] {texte};
&
\tikz  \node[signal,signal to=right,draw,blue] {texte};
&
\tikz  \node[signal,signal to=above,draw,blue] {texte};
\\ \hline  
  \RDD{signal to}=above  & \RDD{signal to}=below & \RDD{signal to}=right  & \RDD{signal to}=above \\ 
\hline 
\end{tabular} 
\bigskip

\begin{tabular}{|c|c|c|c|c|} \hline 
\multicolumn{4}{|c|}{ \BS{tikz} [signal to=nowhere] \BS{node} [signal,\RDD{signal from=above}=45,draw,blue] \AC{texte};   }\\ 
\hline 
\tikz [signal to=nowhere] \node[signal,signal from=above,draw,blue] {texte};
&  
\tikz [signal to=nowhere] \node[signal,signal from=below,draw,blue] {texte};
&
\tikz [signal to=nowhere] \node[signal,signal from=right,draw,blue] {texte};
&
\tikz [signal to=nowhere] \node[signal,signal from=above,draw,blue] {texte};
\\ \hline  
  \RDD{signal from}=above  & \RDD{signal from}=below & \RDD{signal from}=right  & \RDD{signal from}=above \\ 
\hline 
\end{tabular} 

\bigskip
\begin{tabular}{|c|c|c|c|} \hline
\multicolumn{2}{|c|}{ \tikz  \node[draw,signal, signal from=east , signal to=west,blue] at (0,0) {texte};}
&
\multicolumn{2}{|c|}{ \tikz  \node[draw,signal,signal from=south, signal to=north,blue] at (0,0) {texte};}
\\ \hline 
\multicolumn{2}{|c|}{ \RDD{signal from}=east , \RDD{signal to}=west}
&
\multicolumn{2}{|c|}{\RDD{signal from}=south, \RDD{signal to}=north}

\\ \hline 
\end{tabular}
\bigskip

\begin{tabular}{|c | c | c | c |} \hline
\multicolumn{3}{|c|}{ \BS{tikz} \BS{node}  [tape, draw,\RDD{tape bend top}=out and in] \AC{texte};   }\\ 
\hline  
\tikz \node [tape, draw,tape bend top=out and in,blue] {texte};
&
\tikz \node [tape, draw, tape bend bottom=out and in,blue] {texte};
&
\tikz \node [tape, draw, tape bend bottom=in and in,blue] {texte};
 \\  \hline
 \RDD{tape bend top}=out and in & \RDD{tape bend bottom}=out and in &  \RDD{tape bend bottom}=in and in 
  \\  \hline
 \tikz \node [tape, draw, tape bend top=none,blue] {texte};
 &
 \tikz \node [tape, draw,tape bend top=out and in,tape bend bottom=out and in,blue] {texte};
 &
  \tikz \node [tape, draw,tape bend top=in and out,tape bend bottom=in and out,blue] {texte};
  \\  \hline
 \RDD{tape bend top}=none & \RDD{tape bend bottom}=out and in 	&  \RDD{tape bend bottom}=in and out  \\
 					& \RDD{tape bend top}=out and in 		& \RDD{tape bend top}=in and out  \\
 					& & (\dft{} ) 
  \\  \hline 
\end{tabular}
\bigskip

\begin{tabular}{|c | c | c | c |} \hline
\BS{tikz} \BS{node} [tape, draw, \RDD{tape bend height}=1cm,blue] \AC{texte}; 
  \\  \hline 
\tikz \node [tape, draw, tape bend height=1cm,blue] {texte};

  \\  \hline 
\dft{ : tape bend height = 5pt}
  \\  \hline 
\end{tabular}
%=============================================================
\newpage
%\subsection{Dans un n\oe ud en forme de flèche}
\SbSSCT{Dans un n\oe ud en forme de flèche}{Arrow Shapes nodes}

\label{lib-arr}

\maboite{\BS{usetikzlibrary}\AC{shapes.arrows}}

\begin{center}
\RRR{67-5}
\end{center}
%\subsubsection{Formes disponibles}
\SbSbSSCT{Formes disponibles}{Available shapes}
\label{nd3}

\begin{tabular}{|c|c|c|} \hline  
\tikz \node[fill=green!20,single arrow,draw] {texte};
&  
\tikz  \node[fill=green!20,double arrow,draw] {texte};
&  
\tikz  \node[fill=green!20,arrow box,draw] {texte};
\\ \hline 
single arrow & double arrow & arrow box \\ 
\hline 
\end{tabular} 

\subsubsection{Options}

\begin{tabular}{|c|c|c|c|c|} \hline  
 \multicolumn{5}{|c|}{  \BS{node}[single arrow,draw,\RDD{single arrow tip angle}=45] \AC{texte};   }\\ 
  \multicolumn{5}{|c|}{  \BS{node}[single arrow,draw,\RDD{single arrow head extend}=.75cm] \AC{texte};   }\\
 \hline
\begin{tikzpicture}
 \node[single arrow,draw,red,dashed,text=black] {texte};
 \node[single arrow,draw,single arrow tip angle=45,blue] {texte};
\end{tikzpicture}
&
\begin{tikzpicture}
 \node[single arrow,draw,red,dashed,text=black] {texte};
\node[single arrow,draw,single arrow tip angle=120,blue] {texte};
\end{tikzpicture}
&
\begin{tikzpicture}
 \node[single arrow,draw,red,dashed,text=black] {texte};
 \node[single arrow,draw,single arrow head extend=.75cm,blue] {texte};
\end{tikzpicture}
&
\begin{tikzpicture}
 \node[single arrow,draw,red,dashed,text=black] {texte};
 \node[single arrow,draw,single arrow head extend=0cm,blue] {texte};
 \end{tikzpicture}
 &
 \begin{tikzpicture}
  \node[single arrow,draw,red,dashed,text=black] {texte};
  \node[single arrow,draw,single arrow head extend=-1mm,blue] {texte};
 \end{tikzpicture}

\\ \hline
angle=45 & angle=120 & extend=.75cm] & extend=0cm & extend=-1mm
\\ \hline 
\multicolumn{2}{|c|}{  \dft : single arrow tip angle= 90   }
&
\multicolumn{3}{|c|}{  \dft : single arrow head extend=0.5cm   }
\\ \hline 
\end{tabular} 
\bigskip


\begin{tabular}{|c|c|c|c|} \hline
 \multicolumn{4}{|c|}{  \BS{node}[minimum size=2cm,single arrow,draw,\RDD{single arrow head indent}=1cm,blue] \AC{texte};   }\\ 
 \hline   
\begin{tikzpicture}
 \node[minimum size=2cm,single arrow,draw,red,dashed,text=black] {texte};
\node[minimum size=2cm,single arrow,draw,single arrow head indent=1cm,blue] {texte};
\end{tikzpicture}
&
\begin{tikzpicture}
 \node[minimum size=2cm,single arrow,draw,red,dashed,text=black] {texte};
  \node[minimum size=2cm,single arrow,draw,single arrow head indent=10pt,blue] {texte};
  \end{tikzpicture}
&
\begin{tikzpicture}
 \node[minimum size=2cm,single arrow,draw,red,dashed,text=black] {texte};
  \node[minimum size=2cm,single arrow,draw,single arrow head indent=1ex,blue] {texte};
  \end{tikzpicture}
  &
  \begin{tikzpicture}
   \node[minimum size=2cm,single arrow,draw,red,dashed,text=black] {texte};
    \node[minimum size=2cm,single arrow,draw,single arrow head indent=-1ex,blue] {texte};
    \end{tikzpicture}
\\ \hline
indent=1cm & indent=10pt & indent=1ex & indent=-1ex
\\ \hline 
\end{tabular}
\bigskip

 



\begin{tabular}{|c|c|c|c|c|} \hline
 \multicolumn{5}{|c|}{  \BS{node}[minimum size=2cm,double arrow,draw,\RDD{double arrow tip angle}=45] \AC{texte};   }\\ 
  \multicolumn{5}{|c|}{  \BS{node}[minimum size=2cm,double arrow,draw,\RDD{double arrow head extend}=1ex] \AC{texte};   }\\
   \multicolumn{5}{|c|}{  \BS{node}[minimum size=2cm,double arrow,draw,\RDD{double arrow head indent}=1ex] \AC{texte};   }\\ 
 \hline  
\begin{tikzpicture}
\node[minimum size=2cm,double arrow,draw,red,dashed,text=black] {texte};
\node[minimum size=2cm,double arrow,draw,double arrow tip angle=45,blue] {texte};
\end{tikzpicture}
&
\begin{tikzpicture}
\node[minimum size=2cm,double arrow,draw,red,dashed,text=black] {texte};
\node[minimum size=2cm,double arrow,draw,double arrow tip angle=120,blue] {texte};
\end{tikzpicture}
&
\begin{tikzpicture}
 \node[minimum size=2cm,double arrow,draw,red,dashed,text=black] {texte};
 \node[minimum size=2cm,double arrow,draw,double arrow head extend=1ex,blue] {texte};
   \end{tikzpicture}
&
\begin{tikzpicture}
 \node[minimum size=2cm,double arrow,draw,red,dashed,text=black] {texte};
  \node[minimum size=2cm,double arrow,draw,double arrow head extend=0,blue] {texte};
    \end{tikzpicture}
&
\begin{tikzpicture}
 \node[minimum size=2cm,double arrow,draw,red,dashed,text=black] {texte};
  \node[,minimum size=2cm,double arrow,draw,double arrow head indent=1ex,blue] {texte};
    \end{tikzpicture}
\\ \hline 
angle=45 & angle=120 & extend=1ex & extend=0 & indent=1ex
\\ \hline
\end{tabular}

\bigskip

\begin{tabular}{|c|c|c|c|c|} \hline
\multicolumn{4}{|c|}{ \BS{node} [arrow box, draw, \RDD{arrow box arrows}=\AC{north:.25cm}] \AC{texte}; }\\ 
\hline 
\begin{tikzpicture}
\node[arrow box, draw,red,text=white,dashed] {texte};
\node[arrow box, draw, arrow box arrows={north:.25cm},blue] {texte};
\end{tikzpicture}
& 
\begin{tikzpicture}
\node[arrow box, draw,red,text=white,dashed] {texte};
\node[arrow box, draw, arrow box arrows={west:.25cm},blue] {texte};
\end{tikzpicture}
 &
 \begin{tikzpicture}
 \node[arrow box, draw,red,text=white,dashed] {texte};
 \node[arrow box, draw, arrow box arrows={south:.25cm},blue] {texte};
 \end{tikzpicture}
&
 \begin{tikzpicture}
 \node[arrow box, draw,red,text=white,dashed] {texte};
 \node[arrow box, draw, arrow box arrows={east:.25cm},blue] {texte};
 \end{tikzpicture}   
 \\ \hline
\AC{north:.25cm} & \AC{west:.25cm} & \AC{south:.25cm}& \AC{east:.25cm} 
\\ \hline
\multicolumn{4}{|c|}{  \dft{} : 0.5 cm}
 \\ \hline 
 \end{tabular}
 
 
 \bigskip
 
 \begin{tabular}{|c|c|} \hline
 \multicolumn{2}{|c|}{ \BS{node} [arrow box, draw, \RDD{arrow box tip angle}=45] \AC{texte}; }\\ 
 \hline 
  \begin{tikzpicture}
  \node[arrow box, draw,red,text=white,dashed] {texte};
  \node[arrow box, draw, arrow box tip angle=45,blue] {texte};
  \end{tikzpicture} 
  &
    \begin{tikzpicture}
   \node[arrow box, draw,red,text=white,dashed] {texte};
   \node[arrow box, draw, arrow box head extend=.25cm,blue] {texte};
   \end{tikzpicture}
\\ \hline  
\RDD{arrow box tip angle}=45 & \RDD{arrow box head extend}=.25cm
\\ \hline 
\dft : 90  & \dft : 0.125cm 
\\ \hline 
   \begin{tikzpicture}
   \node[arrow box, draw,red,text=white,dashed] {texte};
   \node[arrow box, draw, arrow box head indent=.25cm,blue] {texte};
   \end{tikzpicture} 
 &
    \begin{tikzpicture}
    \node[arrow box, draw,red,text=white,dashed] {texte};
    \node[arrow box, draw,arrow box shaft width=.25cm,blue] {texte};
    \end{tikzpicture} 
 \\ \hline 
\RDD{arrow box head indent}=.25cm  &  \RDD{arrow box shaft width}=.25cm
 \\ \hline  
 \dft{ : 0cm } &  \dft{ : 0.125cm }
 \\ \hline  
 \end{tabular}



\newpage
%-----------------------------------------------------------------------
%\subsection{Dans un n\oe ud en forme de bulle}
\SbSSCT{Dans un n\oe ud en forme de bulle}{Callout Shapes nodes}
\label{lib-call}

%insérer dans le préambule : 

 \maboite{\BS{usetikzlibrary}\AC{shapes.callouts}}
 
\begin{center}
\RRR{67-7}
\end{center}
%\subsubsection{Formes disponibles}
\SbSbSSCT{Formes disponibles}{Available shapes}

\begin{tabular}{|c|c|c|} \hline 
\tikz  \node[fill=green!20,ellipse callout,draw] {texte};
 &  
 \tikz  \node[fill=green!20,rectangle callout,draw] {texte};
  &  
  \tikz  \node[fill=green!20,cloud callout,draw] {texte};
 \\ \hline
 ellipse callout  &  rectangle callout  & cloud callout \\ 
\hline 
\end{tabular} 
%------------------------------------------------

\subsubsection{Options}


\begin{tabular}{|c | c | c | c |} \hline
\multicolumn{4}{|c|}{  \BS{node} [rectangle callout,draw,\RDD{callout absolute pointer}={(0,1)}] at (2,1) \AC{texte};   }\\ 
\hline 
\begin{tikzpicture} 
\draw [help lines] grid(3,3);
\node [rectangle callout,draw,blue, callout relative pointer={(0,1)}] at (2,1) {texte};
\end{tikzpicture}
&
\begin{tikzpicture} 
\draw [help lines] grid(3,3);
\node [ellipse callout,draw, callout relative pointer={(0,1)},blue] at (2,1) {texte};
\end{tikzpicture}
&
\begin{tikzpicture} 
\draw [help lines] grid(3,3);
\node [rectangle callout,draw,blue,callout absolute pointer={(0,1)}] at (2,1) {texte};
\end{tikzpicture}
&
\begin{tikzpicture} 
\draw [help lines] grid(3,3);
\node [ellipse callout,draw, callout absolute pointer={(0,1)},blue] at (2,1) {texte};
\end{tikzpicture}
 \\  \hline
\multicolumn{2}{|c|}{ \RDD{callout relative pointer}=\AC{(0,1)} } & 
\multicolumn{2}{|c|}{  \RDD{callout absolute pointer}=\AC{(0,1)} }
 \\  \hline 
 \begin{tikzpicture} 
 \draw [help lines] grid(3,3);
 \node [rectangle callout,draw, callout relative pointer={(0,1)},callout pointer shorten=.5cm,blue] at (2,1) {texte};
 \end{tikzpicture}
 &
  \begin{tikzpicture} 
  \draw [help lines] grid(3,3);
  \node [ellipse callout,draw, callout relative pointer={(0,1)},callout pointer shorten=.5cm,blue] at (2,1) {texte};
  \end{tikzpicture}
  &
 \begin{tikzpicture} 
 \draw [help lines] grid(3,3);
 \node [rectangle callout,draw, callout absolute pointer={(0,1)},callout pointer shorten=.5cm,blue] at (2,1) {texte};
 \end{tikzpicture}
  &
  \begin{tikzpicture} 
  \draw [help lines] grid(3,3);
  \node [ellipse callout,draw, callout absolute pointer={(0,1)},callout pointer shorten=.5cm,blue] at (2,1) {texte};
  \end{tikzpicture}
  \\  \hline
\multicolumn{4}{|c|}{ \RDD{callout pointer shorten}=.5cm} 
  \\  \hline 
\end{tabular}

%-------------------------------------------------------------

\bigskip
 


\bigskip
\begin{tabular}{|c | c | c | c |} \hline
\multicolumn{3}{|c|}{  \BS{node} [ellipse callout,draw,\RDD{callout pointer arc}=1] at (0,1.5) \AC{texte};   }\\ 
\hline
\begin{tikzpicture}
\node[ellipse callout,draw, callout pointer arc=1,blue] at (0,1.5) {texte};
\end{tikzpicture}
&
\begin{tikzpicture}
\node[ellipse callout,draw, callout pointer arc=30,blue] at (0,1.5) {texte};
\end{tikzpicture}
 &
\begin{tikzpicture}
\node[ellipse callout,draw, callout pointer arc=90,blue] at (0,1.5) {texte};
\end{tikzpicture}
  \\  \hline 
   callout pointer arc=1 & callout pointer arc=30 & callout pointer arc=90
  \\  \hline  
  \multicolumn{3}{|c|}{  \dft{ : callout pointer arc=15}}
 \\  \hline  
 \end{tabular}

\bigskip

\begin{tabular}{|c | c | c | c |} \hline
\multicolumn{3}{|c|}{  \BS{node}[draw,cloud callout, aspect=2.5] \AC{texte};   }\\ 
\hline 
 \begin{tikzpicture}
  \node[draw,cloud callout, dashed,red,text=black] {texte};
 \node[draw,cloud callout, cloud puffs=5,blue] {texte};
 \end{tikzpicture}
&
 \begin{tikzpicture}
 \node[draw,cloud callout, dashed,red,text=black] {texte};
 \node[draw,cloud callout, aspect=2.5,blue] {texte};
 \end{tikzpicture}
&
  \begin{tikzpicture}
  \node[draw,cloud callout, dashed,red,text=black] {texte};
  \node[draw,cloud callout,cloud puff arc=120,blue] {texte};
  \end{tikzpicture}
   \\  \hline 
cloud puffs=5 & aspect=2.5 &  cloud puff arc=120
\\  \hline 
 \end{tabular}

\bigskip

\begin{tabular}{|c | c | c | c |c |} \hline
\multicolumn{3}{|c|}{  \BS{node} [draw,cloud callout,\RDD{callout pointer start size}=.1] \AC{texte};   }\\ 
\hline 
  \begin{tikzpicture}
  \node[draw,cloud callout, dashed,red,text=black] {texte};
  \node[draw,cloud callout,callout pointer start size=.1,blue] {texte};
  \end{tikzpicture}
&
  \begin{tikzpicture}
  \node[draw,cloud callout, dashed,red,text=black] {texte};
  \node[draw,cloud callout,callout pointer start size=.8cm,blue] {texte};
  \end{tikzpicture}
&
  \begin{tikzpicture}
  \node[draw,cloud callout, dashed,red,text=black] {texte};
 \node[draw,cloud callout,callout pointer start size=1cm and 0.1cm,blue] {texte};
  \end{tikzpicture}
\\  \hline 
\RDD{callout pointer start size}=.1 &start size=.8cm & start size=20pt and 1pt
\\  \hline 
\multicolumn{3}{|c|}{  \dft{} : callout pointer start size =.2 of callout  }
\\ 
\hline 
  \begin{tikzpicture}
  \node[draw,cloud callout, dashed,red,text=black] {texte};
  \node[draw,cloud callout,callout pointer end size=5,blue] {texte};
  \end{tikzpicture}
&
  \begin{tikzpicture}
  \node[draw,cloud callout, dashed,red,text=black] {texte};
  \node[draw,cloud callout,callout pointer end size=.8cm,blue] {texte};
  \end{tikzpicture}
&
    \begin{tikzpicture}
    \node[draw,cloud callout, dashed,red,text=black] {texte};
    \node[draw,cloud callout,callout pointer segments=3,blue] {texte};
    \end{tikzpicture}
\\  \hline 
\RDD{callout pointer end size}=.5 & \RDD{callout pointer end size}=.8cm & \RDD{callout pointer segments}=3
\\  \hline 
\multicolumn{2}{|c|}{  \dft{} : callout pointer start size = .1 of callout  }
& \dft{} : segments=2
\\  \hline  

 \end{tabular}
 


%----------------------------------------------------------------------
\newpage

%\subsection{Dans un n\oe ud en diverses formes  diverses}

\SbSSCT{Dans un n\oe ud en diverses formes  diverses}{Miscellaneous Shapes nodes}

\label{lib-misc}

%insérer dans le préambule:

 \maboite{\BS{usetikzlibrary}\AC{shapes.misc}}
 
\begin{center}
\RRR{67-8}
\end{center}

%\subsubsection{formes disponibles}
\SbSbSSCT{Formes disponibles}{Available shapes}

\begin{tabular}{|c|c|c|c|} \hline  
\tikz  \node[fill=green!20,cross out,draw] {texte};
&  
\tikz  \node[fill=green!20,strike out,draw] {texte};
&  
\tikz  \node[fill=green!20,rounded rectangle,draw] {texte};
&  
\tikz  \node[fill=green!20,chamfered rectangle,draw] {texte};
\\ \hline  
cross out & strike out & rounded rectangle & chamfered rectangle \\ 
\hline 
\end{tabular} 


\subsubsection{Options}

\paragraph{Options \TFRGB{pour}{for} \og rounded rectangle \fg} :


%
\begin{tabular}{|c|c|c|c|c|} \hline
\multicolumn{5}{|c|}{  \BS{node} [draw, rounded rectangle,\RDD{rounded rectangle arc length}=270] \AC{texte};   }\\ 

\hline 

%\begin{tikzpicture}
\tikz \node[draw, rounded rectangle,rounded rectangle arc length=270,blue] {texte}; 
&
\tikz \node[draw, rounded rectangle,rounded rectangle arc length=180,blue]  {texte}; 
&
\tikz \node[draw, rounded rectangle,rounded rectangle arc length=120,blue] {texte}; 
&
\tikz \node[draw, rounded rectangle,rounded rectangle arc length=90,blue]  {texte}; 
&
\tikz \node[draw, rounded rectangle,rounded rectangle arc length=45,blue] {texte}; 
 \\ \hline 
270 & 180 & 120 & 90& 45 
\\ \hline 
%\end{tikzpicture}

\end{tabular} 

\bigskip


\begin{tabular}{|c|c|c|c|} \hline 
\multicolumn{4}{|c|}{  \BS{node} [draw, rounded rectangle,\RDD{rounded rectangle west arc}=concave] \AC{texte};   }\\ 
\multicolumn{4}{|c|}{  \BS{node} [draw, rounded rectangle,\RDD{rounded rectangle left arc}=concave] \AC{texte};   }\\ 
\hline 
\tikz \node[draw, rounded rectangle,rounded rectangle west arc=concave,blue] {texte}; 
&
\tikz \node[draw, rounded rectangle,rounded rectangle left arc=concave,blue] {texte}; 
&
\tikz \node[draw, rounded rectangle,rounded rectangle west arc=convex,blue] {texte}; 
&
\tikz \node[draw, rounded rectangle,rounded rectangle left arc=none,blue] {texte};
 \\\hline 
concave & convex & none 
 \\\hline 
\end{tabular} 

\bigskip

\begin{tabular}{|c|c|c|c|} \hline 
\multicolumn{3}{|c|}{  \BS{node} [draw, rounded rectangle,\RDD{rounded rectangle east arc}=concave] \AC{texte};   }\\ 
\multicolumn{3}{|c|}{  \BS{node} [draw, rounded rectangle,\RDD{rounded rectangle right arc}=concave] \AC{texte};   }\\ 

\hline 
\tikz \node[draw, rounded rectangle,rounded rectangle east arc=concave,blue] {texte}; 
&
\tikz \node[draw, rounded rectangle,rounded rectangle  east arc=convex,blue] {texte}; 
&
\tikz \node[draw, rounded rectangle,rounded rectangle right arc=none,blue] {texte};
 \\\hline 
concave & convex & none 
 \\\hline 
\end{tabular} 

\paragraph{Options  \TFRGB{pour}{for} \og chamfered rectangle \fg} :


\begin{tabular}{|c|c|c|c|} \hline 
\multicolumn{4}{|c|}{  \BS{node} [draw, chamfered rectangle,\RDD{chamfered rectangle angle}=30] \AC{texte};   }\\ 
\hline 
\tikz \node[draw, chamfered rectangle,chamfered rectangle angle=10,blue] {texte}; 
&
\tikz \node[draw, chamfered rectangle,chamfered rectangle angle=30,blue] {texte}; 
&
\tikz \node[draw,chamfered rectangle,chamfered rectangle angle=60,blue] {texte};
&
\tikz \node[draw,chamfered rectangle,chamfered rectangle angle=80,blue] {texte};
 \\ \hline 
10 & 30 & 60 & 80
\\ \hline 
\multicolumn{4}{|c|}{  \dft :  45 }
  \\\hline  

\end{tabular}

\bigskip

\begin{tabular}{|c|c|c|c|c|} \hline 
\multicolumn{5}{|c|}{  \BS{node} [draw, chamfered rectangle,\RDD{chamfered rectangle xsep}=10pt] \AC{texte};   }\\ 
\hline 
\tikz \node[draw, chamfered rectangle,chamfered rectangle xsep=0pt,blue] {texte}; 
&
\tikz \node[draw, chamfered rectangle,chamfered rectangle xsep=5pt,blue] {texte}; 
&
\tikz \node[draw, chamfered rectangle,chamfered rectangle xsep=10pt,blue] {texte}; 
&
\tikz \node[draw,chamfered rectangle,chamfered rectangle xsep=-10pt,blue] {texte};
&
\tikz \node[draw,chamfered rectangle,chamfered rectangle xsep=2cm,blue] {texte};
 \\\hline 
  xsep=0pt & xsep=5pt & xsep=10pt & xsep=-10pt  & xsep=2cm
  \\\hline  
\multicolumn{5}{|c|}{  \dft :  0.666ex }
  \\\hline   
\end{tabular}

\bigskip

\begin{tabular}{|c|c|c|c|c|} \hline 
\multicolumn{5}{|c|}{  \BS{node} [draw, chamfered rectangle,\RDD{chamfered rectangle ysep}=10pt] \AC{texte};   }\\ 
\hline 
\tikz \node[draw, chamfered rectangle,chamfered rectangle ysep=0pt,blue] {texte}; 
&
\tikz \node[draw, chamfered rectangle,chamfered rectangle ysep=5pt,blue] {texte}; 
&
\tikz \node[draw,chamfered rectangle,chamfered rectangle ysep=10pt,blue] {texte};
&
\tikz \node[draw,chamfered rectangle,chamfered rectangle ysep=-10pt,blue] {texte};
&
\tikz \node[draw,chamfered rectangle,chamfered rectangle ysep=1cm,blue] {texte};
 \\ \hline 
 ysep=0pt & ysep=5pt & ysep=10pt & ysep=-10pt & ysep=1cm
 \\\hline  
\end{tabular}

\bigskip

\begin{tabular}{|c|c|c|c|c|} \hline 
\multicolumn{5}{|c|}{  \BS{node} [draw, chamfered rectangle,\RDD{chamfered rectangle ysep}=10pt] \AC{texte};   }\\ 
\hline 
\tikz \node[draw, chamfered rectangle,chamfered rectangle sep=0pt,blue] {texte}; 
&
\tikz \node[draw, chamfered rectangle,chamfered rectangle sep=5pt,blue] {texte}; 
&
\tikz \node[draw, chamfered rectangle,chamfered rectangle sep=10pt,blue] {texte}; 

&
\tikz \node[draw, chamfered rectangle,chamfered rectangle sep=-10pt,blue] {texte}; 
&
\tikz \node[draw,chamfered rectangle,chamfered rectangle sep=1cm,blue] {texte};
 \\\hline 
 sep=0pt & sep=5pt & sep=10pt& sep=-10pt & sep=1cm
 \\\hline  
\end{tabular}

\bigskip

\begin{tabular}{|c|c|c|c|} \hline 
\multicolumn{3}{|c|}{  \BS{node} [draw, chamfered rectangle,\RDD{chamfered rectangle corners}=north west] \AC{texte};   }\\ 
\hline
\tikz \node[draw, chamfered rectangle,chamfered rectangle corners=north west,blue] {texte}; 
&
\tikz \node[draw, chamfered rectangle,chamfered rectangle corners={north east, south east},blue] {texte}; 
&
\tikz \node[draw,chamfered rectangle,chamfered rectangle corners={north east, south west},blue] {texte};
 \\ \hline 
 north west & \AC{north east, south east}  & \AC{north east, south west}
 \\ \hline 
\end{tabular}





%\begin{tikzpicture}
%\tikzset{every node/.style={chamfered rectangle, draw}}
%\node[chamfered rectangle corners=north west] {ghi};
%\node[chamfered rectangle corners={north east, south east}] at (1.5,0) {789};
%\end{tikzpicture}


%\begin{tikzpicture}
%\tikzset{every node/.style={chamfered rectangle, draw}}
%\node[chamfered rectangle xsep=2pt] {def};
%\node[chamfered rectangle xsep=2cm] at (1.5,0) {456};
%\end{tikzpicture}

%\begin{tikzpicture}
%\tikzset{every node/.style={chamfered rectangle, draw}}
%\node[chamfered rectangle angle=30] {abc};
%\node[chamfered rectangle angle=60] at (1.5,0) {123};
%\end{tikzpicture}

%\begin{tikzpicture}
%\matrix[row sep=5pt, every node/.style={draw, rounded rectangle}]{
%\node[rounded rectangle west arc=concave] {Concave}; \\
%\node[rounded rectangle west arc=convex] {Convex}; \\
%\node[rounded rectangle left arc=none] {None}; \\};
%\end{tikzpicture}
%\tikz  \draw (-1,-1) grid (1,1) (0,0) node[fill=red!20,diamond,draw,rounded corners] {texte};&
 
%------------------------------------------------------------------------------------------

\newpage
%\subsection{N\oe uds à plusieurs parties}
\SbSSCT{N\oe uds à plusieurs parties}{Shapes with Multiple Text Parts}

\label{lib-mult}

%insérer dans le préambule :

 \maboite{\BS{usetikzlibrary}\AC{shapes.multipart}}

\begin{center}
\RRR{67-6}
\end{center}



\begin{tabular}{|c|c|c|c|} \hline 
\multicolumn{4}{|c|}{  \BS{node} [\RDD{circle split},draw,fill=green!20]\AC{haut  \BSS{nodepart}\AC{lower} bas };   }\\ 
\hline 
 
\tikz  \node [circle split,draw,blue,fill=green!20] {haut  \nodepart{lower} bas }; % \filldraw[fill=red] (0,0) circle (3pt);

&  
\tikz  \node [circle solidus,draw,blue,fill=green!20]{haut  \nodepart{lower} bas };
&  
\tikz  \node [ellipse split,draw,blue,fill=green!20]{texte haut  \nodepart{lower} texte bas };
& 
\tikz  \node [rectangle split,draw,blue,fill=green!20]{haut  \nodepart{lower} bas}; 
%\tikz  \node [rectangle split ,draw,fill=green!20]{a\nodepart{two}b\nodepart{three}c\nodepart{four}d\nodepart{five}e};
\\ \hline 
\RDD{circle split} & \RDD{circle solidus} & \RDD{ellipse split} & \RDD{rectangle split} \\ 
\hline 
\end{tabular} 

 \bigskip
 
 \begin{tabular}{|c|c|}  \hline  
 \begin{tikzpicture} [baseline=0pt]%[every text node part/.style={text centered}]
 \node[rectangle split,rectangle split parts=5,draw,blue,fill=green!20] at(0,0)
 {texte 1
 \nodepart{second}
 texte 2
 \nodepart{four}
 texte 3};
 \end{tikzpicture}
&
\parbox[c]{10cm}{
 \BS{node}[rectangle split,\RDD{rectangle split parts}=5,\\
 draw] \\
 \AC{texte 1 \\
 \BSS{nodepart}\AC{second} texte 2 \\
 \BSS{nodepart}\AC{four} texte 3}; \\
 \\
\dft : rectangle split parts=4 }
 \\  \hline 
 \end{tabular} 
 
\bigskip

\begin{tabular}{|c|}\hline  
\BS{node} [rectangle split,rectangle split parts=3,\RDD{rectangle split horizontal},draw,blue] \\
\AC{texte1\BSS{nodepart}\AC{two}texte2\BSS{nodepart}\AC{three}texte3};
\\ \hline  
\tikz \node [rectangle split,rectangle split parts=3, rectangle split horizontal,draw,blue]
{texte 1\nodepart{two}texte 2\nodepart{three}texte 3}; 
\\ \hline 
\end{tabular} 
 
 \bigskip
 
% % % <<<<<<<<<<<<<<<<< A Voir rectangle split allocate boxes= >>>>>>>>>>>>>>>>>>>>>>>>>>>>>>>>

% \begin{tikzpicture} [baseline=0pt]%[every text node part/.style={text centered}]
% \node[rectangle split,draw,rectangle split parts=5,fill=green!20,rectangle split allocate boxes=3] at(0,0)
% {texte 1  \nodepart{second}  texte 2  \nodepart{four}  texte 3};
% \end{tikzpicture}
% 
 
\bigskip
 \begin{tabular}{|c|c|}  \hline  
\begin{tikzpicture}[baseline=0pt] %[every text node part/.style={align=center}]
\node[rectangle split, rectangle split parts=3, draw,blue, text width=2.75cm]
{texte 1
\nodepart{two}
texte 2a \\
texte 2b \\
texte 2c
\nodepart{three}
texte 3a \\
texte 3b};
\end{tikzpicture}
&
\parbox{8cm}{
 \BS{node}[rectangle split,\RDD{rectangle split parts}=5, draw] \\
 \AC{texte 1 \\
 \BSS{nodepart}\AC{second} texte 2a  \BS{}\BS{}texte 2b  \BS{}\BS{}  texte 2c \\
 \BSS{nodepart}\AC{three} texte 3a \BS{}\BS{} texte 3b }; \\
}
 \\  \hline 
 \end{tabular} 
\bigskip
%---------------------------------------------------------------------------------

 \begin{tabular}{|c|c|}  \hline  
 \multicolumn{2}{|c|}{  \BS{node}[rectangle split, draw,blue,minimum size = 2cm,\RDD{rectangle split draw splits}= true] } \\
  \multicolumn{2}{|c|}{ 
  \AC{texte 1 \BS{nodepart}\AC{two} texte 2 \BS{nodepart}\AC{three} texte 3 \BS{nodepart}\AC{four} texte 4};   }\\ 
 \hline 
\tikz \node[rectangle split, draw,blue,minimum size = 2cm,rectangle split draw splits= true] {texte 1 \nodepart{two} texte 2 \nodepart{three} texte 3 \nodepart{four} texte 4};
&
\tikz \node[rectangle split, draw,blue,minimum size = 2cm,rectangle split draw splits= false] {texte 1 \nodepart{two} texte 2 \nodepart{three} texte 3 \nodepart{four} texte 4};
 \\ \hline
 \RDD{rectangle split draw splits}= true & \RDD{rectangle split draw splits}= false \\
 \dft &
 \\ \hline 
 \end{tabular}
 
\bigskip

 \begin{tabular}{|c|c|}  \hline  
\multicolumn{2}{|c|}{  
\BS{node} [rectangle split,rectangle split parts=3,draw,\RDD{rectangle split ignore empty parts}=false] }\\
 \multicolumn{2}{|c|}{ \AC{texte 1 \BS{nodepart}\AC{second} \BS{nodepart}\AC{third}texte 3};} 
\\ \hline  
\begin{tikzpicture} 
\node[rectangle split,rectangle split parts=3,draw,blue,rectangle split ignore empty parts=false] {texte 1 \nodepart{second} \nodepart{third}texte 3};
\end{tikzpicture}
&
\begin{tikzpicture}
\node[rectangle split,rectangle split parts=3,draw,blue,rectangle split ignore empty parts] 
{texte 1 \nodepart{second} \nodepart{third}texte 3};
\end{tikzpicture}
 \\  \hline 
\RDD{rectangle split ignore empty parts}=false & \RDD{rectangle split ignore empty parts}=true 
\\ \hline
 \end{tabular}
 
\bigskip

 \begin{tabular}{|c|c|}  \hline  
\multicolumn{2}{|c|}{  
\BS{node} [rectangle split,rectangle split parts=3,draw,\RDD{rectangle split empty part depth}=1cm] }\\
 \multicolumn{2}{|c|}{ \AC{texte 1 \BS{nodepart}\AC{second} \BS{nodepart}\AC{third}texte 3};} 
\\ \hline 
\begin{tikzpicture} 
\node[rectangle split,rectangle split parts=3,draw,blue,rectangle split empty part depth=1cm] {texte 1 \nodepart{second} \nodepart{third}texte 3};
\end{tikzpicture}
&
\begin{tikzpicture} 
\node[rectangle split,rectangle split parts=3,draw,blue,text depth=1cm] {texte 1 \nodepart{second} \nodepart{third}texte 3};
\end{tikzpicture}
\\ \hline 
\RDD{rectangle split empty part depth}=1cm & \RDD{text depth}=1cm
\\ \hline
\dft : 0ex & \dft : 0ex
\\ \hline 
\begin{tikzpicture}
\node[rectangle split,rectangle split parts=3,draw,blue,rectangle split empty part  height=1cm] 
{texte 1 \nodepart{second} \nodepart{third}texte 3};
\end{tikzpicture}
&
\begin{tikzpicture}
\node[rectangle split,rectangle split parts=3,draw,blue,text height=1cm] 
{texte 1 \nodepart{second} \nodepart{third}texte 3};
\end{tikzpicture}
\\  \hline 
\RDD{rectangle split empty part height}=1cm & \RDD{text height}=1cm
\\ \hline
\dft : 1ex & \dft : 1ex
\\ \hline 
 \end{tabular}
 
\bigskip



 \begin{tabular}{|c|c|}  \hline 
 \multicolumn{2}{|c|}{ 
 \BS{node} [rectangle split,rectangle split parts=3,draw,\RDD{rectangle split empty part width}=1cm]   \AC{};  } 
 \\ \hline 
\begin{tikzpicture} 
\node[rectangle split,rectangle split parts=3,draw,blue,rectangle split empty part width=2cm]{}; % {texte 1 \nodepart{second} \nodepart{third}texte 3};
\end{tikzpicture}
%\rule{6cm}{0pt}
&
\begin{tikzpicture} 
\node[rectangle split,rectangle split parts=3,draw,blue]{}; % {texte 1 \nodepart{second} \nodepart{third}texte 3};
\end{tikzpicture}
\\  \hline 
 \RDD{rectangle split empty part width}=2cm  &  \dft : 1ex
\\ \hline
 \end{tabular} 
 
 \bigskip



% % % % <<<<<<<<<< A voir   /pgf/rectangle split use custom fill= (default true) <<<<<<<<<<<<<<<<<<<<<<<<<<<<
 
 

%--------------------------------------------------------------------------------------

 \begin{tabular}{|c|c|}  \hline 
 \tikz[baseline=0pt] \node[rectangle split, draw,blue,minimum size = 2cm,rectangle split part align={center, left,right}] {texte 1 \nodepart{two} texte 2 \nodepart{three} texte 3 \nodepart{four} texte 4};
&
\parbox{8cm}{
\BS{node}[rectangle split, draw,blue,minimum size = 2cm,\\
\RDD{rectangle split part align}=\AC{center, left,right}]\\
 \AC{texte 1 \BS{nodepart}\AC{two} texte 2  \\
 \BS{nodepart}\AC{three} texte 3  \BS{nodepart}\AC{four} texte 4};
}
\\ \hline
 \tikz[baseline=0pt] \node[rectangle split, draw,blue,minimum size = 2cm, rectangle split horizontal,rectangle split part align={center,base, top,bottom}] {texte 1 \nodepart{two} texte 2 \nodepart{three} texte 3 \nodepart{four} texte 4};
 &
 \parbox{8cm}{
 \BS{node}[rectangle split, draw,blue,minimum size = 2cm,\\
 rectangle split horizontal,\\
 \RDD{rectangle split part align}=\AC{center,base, top,bottom}]\\
  \AC{texte 1 \BS{nodepart}\AC{two} texte 2  \\
  \BS{nodepart}\AC{three} texte 3  \BS{nodepart}\AC{four} texte 4};
 }
 \\ \hline
 \end{tabular}
 
\bigskip
%--------------------------------------------------------------------

 \begin{tabular}{|c|c|}  \hline  
\tikz[baseline=0pt] \node[rectangle split, draw,blue, minimum width=1cm,rectangle split part fill={red, green,cyan}]{};
&
\parbox{12cm}{
\BS{node}[rectangle split, draw,blue, minimum width=1cm,\\
 \RDD{rectangle split part fill}=\AC{red, green,cyan}]\AC{};}
\\ \hline
\end{tabular} 

%--------------------------------------------
\newpage
%\subsection{Mise en forme du texte}
\SbSSCT{Mise en forme du texte}{Text attributes}

\subsubsection{Position}

\begin{center}
\RRR{17-4-3}
\end{center}

\begin{tabular}{|c|c|c|c|} \hline  
\multicolumn{4}{|l|}{ \BS{tikz} \BS{draw} (0,0) node[fill=blue!10,text width=2cm,\RDD{text justified}]   }\\ 

\multicolumn{4}{|l|}{ \AC{Ceci est une démonstration d'un texte  sur une largeur de 2cm};  }\\ 
\hline 
\tikz \draw (0,0) node[fill=blue!10,text width=2cm]
{Ceci est une démonstration d'un texte  sur une largeur de 2cm.};
&  
\tikz \draw (0,0) node[fill=blue!10,text width=2cm,text justified]
{Ceci est une démonstration d'un texte  sur une largeur de 2cm};
&  
\tikz \draw (0,0) node[fill=blue!10,text width=2cm,text centered]
{Ceci est une démonstration d'un texte  sur une largeur de 2cm .};
&  
\tikz \draw (0,0) node[fill=blue!10,text width=2cm,text ragged]
{Ceci est une démonstration d'un texte  sur une largeur de 2cm .};
\\  \hline  
\TFRGB{sans}{without} option & text justified & text centered & text ragged   
\\ \hline  
\tikz \draw (0,0) node[fill=blue!10,text width=2cm,text badly ragged]
{Ceci est une démonstration d'un texte  sur une largeur de 2cm.};
&  
\tikz \draw (0,0) node[fill=blue!10,text width=2cm,text badly centered]
{Ceci est une démonstration d'un texte  sur une largeur de 2cm .};
&
\tikz \draw (0,0) node[fill=blue!10,text width=2cm,align=center]
{Ceci est une démonstration d'un texte  sur une largeur de 2cm .};
&
\tikz \draw (0,0) node[fill=blue!10,text width=2cm,align=flush center]
{Ceci est une démonstration d'un texte  sur une largeur de 2cm .};
\\  \hline 
text badly ragged &  text badly centered &  align=center & align=flush center 
\\  \hline 
\tikz \draw (0,0) node[fill=blue!10,text width=2cm,align=justify]
{Ceci est une démonstration d'un texte  sur une largeur de 2cm .};
&
\tikz \draw (0,0) node[fill=blue!10,text width=2cm,align=flush right]
{Ceci est une démonstration d'un texte  sur une largeur de 2cm .};
&
\tikz \draw (0,0) node[fill=blue!10,text width=2cm,align=right]
{Ceci est une démonstration d'un texte  sur une largeur de 2cm .};
&
\tikz \draw (0,0) node[fill=blue!10,text width=2cm,align=flush left]
{Ceci est une démonstration d'un texte  sur une largeur de 2cm .};
\\ \hline 
 align=justify & align=flush right &  align=right & align=flush left
\\ \hline 

\end{tabular} 
\bigskip

%--------------------------------------------------------------
%\subsubsection{Couleur et fontes } 
\SbSbSSCT{Couleur et fontes }{Colors and Fonts}

\begin{tabular}{|c|c|c|c|c|c|} \hline  
\tikz \draw (0,0) node[text= red]{Texte.};
&
\tikz \draw (0,0) node[font=\itshape]{Texte.};
&
\tikz \draw (0,0) node[font=\slshape]{Texte.};
&
\tikz \draw (0,0) node[font=\scshape]{Texte.};
&
\tikz \draw (0,0) node[font=\upshape]{Texte.};
&
\tikz \draw (0,0) node[font=\bfseries]{Texte.};
\\ \hline 



[text= red] & [font=\BS{itshape}]  & [font=\BS{slshape}] & [font=\BS{scshape}] & [font=\BS{upshape}] & [font=\BS{bfseries}]
\\ \hline 
\end{tabular} 



\bigskip

%\subsubsection{Taille des fontes} 
\SbSbSSCT{Taille des fontes}{Font Sizes}

\begin{tabular}{|c|c|c|c|c|c|c|}\hline
\multicolumn{7}{|c|}{ \BS{tikz} \BS{draw} (0,0) node[\RDD{font}=\BS{tiny}]\AC{Texte.}   }
\\  \hline
\tikz \draw (0,0) node[font=\tiny]{Texte.};
&
\tikz \draw (0,0) node[font=\footnotesize]{Texte.};
&
\tikz \draw (0,0) node[font=\small]{Texte.};
&
\tikz \draw (0,0) node[font=\large]{Texte.};
&
\tikz \draw (0,0) node[font=\Large]{Texte.};
&
\tikz \draw (0,0) node[font=\huge]{Texte.};
&
\tikz \draw (0,0) node[font=\Huge]{Texte.};
\\ \hline \BS{tiny} & \BS{footnotesize}  & \BS{small} & \BS{large} & \BS{Large} & \BS{huge} & \BS{Huge} \\ 
\hline 
\end{tabular} 

\bigskip
\begin{center}
\RRR{17-4-4}
\end{center}

\begin{tabular}{|c|c|} \hline  
\tikz \draw (0,0) node[fill=blue!10,text height=1cm,draw]{Texte.};
&  
\tikz \draw (0,0) node[fill=blue!10,text depth=1cm,draw]{Texte.};
\\ \hline  
\RDD{text height}=1cm
&  
\RDD{text depth}=1cm
\\ \hline 
\end{tabular} 

%\subsection{Positions prédéfinies  sur un n\oe ud}
\SbSSCT{Positions prédéfinies  sur un n\oe ud}{Positions on a node}
\label{nomnoeud}

%\subsubsection{pour l'ensemble des n\oe uds}
\SbSbSSCT{pour l'ensemble des n\oe uds}{For all types of node}
\begin{center}
\RRR{17-5-1}
\end{center}

\begin{tabular}{|c|c|c|c|} \hline  
\begin{tikzpicture}
\node[rectangle,draw,minimum size=3cm] (A) at (1,1) {\Huge texte};
\fill[red] (node cs:name=A,anchor=north west) circle (3pt);
\end{tikzpicture}
&
\begin{tikzpicture}
\node[rectangle,draw,minimum size=3cm] (A) at (1,1) {\Huge texte};
\fill[red] (node cs:name=A,anchor=north) circle (3pt);
\end{tikzpicture}
&
\begin{tikzpicture}
\node[rectangle,draw,minimum size=3cm] (A) at (1,1) {\Huge texte};
\fill[red] (node cs:name=A,anchor=north east) circle (3pt);
\end{tikzpicture}
&
\begin{tikzpicture}
\node[rectangle,draw,minimum size=3cm] (A) at (1,1) {\Huge texte};
\fill[red] (node cs:name=A,anchor=text) circle (3pt);
\end{tikzpicture}
\\ \hline 
north west & north & north east & text
\\ \hline 
%---------------------------------------------------------------
\begin{tikzpicture}
\node[rectangle,draw,minimum size=3cm] (A) at (1,1) {\Huge texte};
\fill[red] (node cs:name=A,anchor= west) circle (3pt);
\end{tikzpicture}
&
\begin{tikzpicture}
\node[rectangle,draw,minimum size=3cm] (A) at (1,1) {\Huge texte};
\fill[red] (node cs:name=A,anchor=mid  west) circle (3pt);
\end{tikzpicture}
&
\begin{tikzpicture}
\node[rectangle,draw,minimum size=3cm] (A) at (1,1) {\Huge texte};
\fill[red] (node cs:name=A,anchor= base west) circle (3pt);
\end{tikzpicture}
&
\begin{tikzpicture}
\node[rectangle,draw,minimum size=3cm] (A) at (1,1) {\Huge texte};
\fill[red] (node cs:name=A,anchor= base) circle (3pt);
\end{tikzpicture}
\\ \hline 
west & mid west & base west &  base
\\ \hline
%------------------------------------------------------------ 
\begin{tikzpicture}
\node[rectangle,draw,minimum size=3cm] (A) at (1,1) {\Huge texte};
\fill[red] (node cs:name=A,anchor=east) circle (3pt);
\end{tikzpicture}
&
\begin{tikzpicture}
\node[rectangle,draw,minimum size=3cm] (A) at (1,1) {\Huge texte};
\fill[red] (node cs:name=A,anchor=mid east) circle (3pt);
\end{tikzpicture}
&
\begin{tikzpicture}
\node[rectangle,draw,minimum size=3cm] (A) at (1,1) {\Huge texte};
\fill[red] (node cs:name=A,anchor=base east) circle (3pt);
\end{tikzpicture}
&
\begin{tikzpicture}
\node[rectangle,draw,minimum size=3cm] (A) at (1,1) {\Huge texte};
\fill[red] (node cs:name=A,anchor= mid) circle (3pt);
\end{tikzpicture}
\\ \hline 
east & mid esat & base east & mid
\\ \hline 
%--------------------------------------
\begin{tikzpicture}
\node[rectangle,draw,minimum size=3cm] (A) at (1,1) {\Huge texte};
\fill[red] (node cs:name=A,anchor= south east) circle (3pt);
\end{tikzpicture}
&
\begin{tikzpicture}
\node[rectangle,draw,minimum size=3cm] (A) at (1,1) {\Huge texte};
\fill[red] (node cs:name=A,anchor= south) circle (3pt);
\end{tikzpicture}
&
\begin{tikzpicture}                                       
\node[rectangle,draw,minimum size=3cm] (A) at (1,1) {\Huge texte};
\fill[red] (node cs:name=A,anchor= south west) circle (3pt);
\end{tikzpicture}
&
\begin{tikzpicture}
\node[rectangle,draw,minimum size=3cm] (A) at (1,1) {\Huge texte};
\fill[red] (node cs:name=A,anchor=center ) circle (3pt);
\end{tikzpicture}
\\ \hline 
south east & south & south west & center
\\ \hline
%------------------------------------------------------------------------- 
\begin{tikzpicture}
\node[rectangle,draw,minimum size=3cm] (A) at (1,1) {\Huge texte};
\fill[red] (node cs:name=A,anchor=0) circle (3pt);
\end{tikzpicture}
&
\begin{tikzpicture}
\node[rectangle,draw,minimum size=3cm] (A) at (1,1) {\Huge texte};
\fill[red] (node cs:name=A,anchor=120) circle (3pt);
\end{tikzpicture}
&
\begin{tikzpicture}
\node[rectangle,draw,minimum size=3cm] (A) at (1,1) {\Huge texte};
\fill[red] (node cs:name=A,anchor=-60) circle (3pt);
\end{tikzpicture}
&
%\begin{tikzpicture}
%\node[rectangle,draw,minimum size=3cm] (A) at (1,1) {\Huge texte};
%\fill[red] (node cs:name=A,anchor=text) circle (3pt);
%\end{tikzpicture}

\\ \hline 
0 & 120 & -60 & %text  
\\ \hline 
\end{tabular}
 
%\subsubsection{spécifique à un n\oe ud}
\SbSbSSCT{spécifique à un n\oe ud}{Specific to a node}

\TFRGB{Dans une prochaine version !}{In a future version}







 

%
%%============\newpage
%
%\section{Décorations}
%
% \input{tkzdeco}
%% 
%% ======================================================================
%\newpage
%
%\section{Insertion images dans un environnement TikZ}
%
%\input{tkzimage}
%
%
%%%
%%%>>>> \section[Mettre des objets en cadre]{Mettre des objets en cadre }
%%%
%%
%%%
%%%\newpage
%%%>>>>> \section[Mettre des objets en bouton]{Mettre des objets en bouton }
%%
%%
%%%%%%%=============================================================
%%
%
%\section{Des lignes et liaisons spéciales}
%\subsection[Trait à main levé]{Trait à main levée }
%
%\input{tkzalea}
%
%%% >>>> \subsection{Tracer avec des symboles}
%%
%%%
%%%\newpage
%
%%%>>>>> \subsection[Les bobines]{Les bobines \cite{pst-user} \cite{pst-coil}}
%%%
%
%%%\newpage
%%%
%%%>>>> \subsection[Les accolades]{Les accolades }
%%%
%%
%%%%%%======================================================================
%%%\section{Des remplissages spéciaux}
%%%\subsection{Des gradients de couleurs}
%%%
%%%\subsubsection[Module pst-grad]{Module pst-grad \cite{pst-user} \cite{pst-grad}}
%%
%%%%
%%%\newpage
%%%\subsubsection[Module pst-slpe]{Module pst-slpe  \cite{pst-slpe}}
%%%
%%
%%%
%%%\newpage
%%%\subsection[Remplissage par des motifs]{Remplissage par des motifs \cite{pst-fill}}
%%%
%%
%%%
%%%\subsection[Remplissage par des points aléatoires]{Remplissage par des points aléatoires \cite{pst-add}}
%%
%%%\newpage
%%%
%%%% ========================================================================
%%%\section[Effets spéciaux avec du texte ]{Effets spéciaux avec du texte  \cite{pst-user}  \cite{pst-text}}
%%
%%
%%\newpage
%%%% % % % %======================================================================
%\section[Créer un graphe]{Créer un graphe }
%
%
%
%\subsection{Graphe avec Tikz}
\SbSSCT{Graphe avec TikZ}{Graph with TikZ}
%\subsubsection{Graphe à partir d'une liste de points}
\SbSbSSCT{Graphe à partir d'une liste de points}{From a list of points}
\label{plot}

\begin{tabular}{|c | } \hline
\BS{tikz} \BS{draw} plot \RDD{coordinates} \AC{(0,0) (1,1) (2,0) (3,1) (4,1) (5,2)}; \\ 
\hline
\tikz \draw plot coordinates {(0,0) (1,1) (2,0) (3,1) (4,1) (5,2)};
\\ \hline
\end{tabular}

%\subsubsection{Graphe à partir partir d'un fichier de données}
\SbSbSSCT{Graphe à partir partir d'un fichier de données}{From a data file}

\begin{tabular}{|c | c | c | c |} \hline
\multicolumn{4}{|c|}{ \BS{tikz} \BS{draw}  plot[mark=x] \RDD{file} \AC{table.dat} ;   }\\ 
\hline
%\draw plot[mark=x] file {table.dat};
& 
\tikz \draw plot[mark=x,smooth] file {table.dat};
&
\tikz \draw plot[mark=x,smooth,tension=.2] file {table.dat};
&
\tikz \draw plot[mark=x,smooth,tension=1] file {table.dat};
\\ \hline
[mark=x] & [mark=x,\RDD{smooth}] & [mark=x,smooth,\RDD{tension}=.2] & [mark=x,smooth,\RDD{tension}=1]
\\ \hline
\multicolumn{4}{|c|}{ \dft : tension= 0:55}
\\ \hline
\end{tabular}

\bigskip


\begin{tabular}{|c  c |} \hline
\multicolumn{2}{|c|}{\TFRGB{Contenu du fichier}{content of the file} table.dat}
\\ \hline
0.0 & 0.3 \\
0.3 & 0.6 \\
0.6 & 0.9 \\
0.9 & 1.5  \\
1.2 & 0.6  \\
1.5 & 1.2  \\
1.8 & 1.5  \\
2.1 & 2.0 \\
2.4 & 3.0 \\
\hline
\end{tabular}

\bigskip

%\subsubsection{Les types de graphes}
\SbSbSSCT{Les types de graphes}{Graph types}

\begin{tabular}{|c | c | c | c |} \hline
\multicolumn{4}{|c|}{ \BS{tikz} \BS{draw}  plot[mark=*,\RDD{const plot}] file \AC{table.dat} ;   }\\ 
\hline
\tikz \draw plot[mark=*,const plot] file {table.dat};
&

\tikz \draw plot[const plot mark left,mark=*] file {table.dat};
&
\tikz \draw plot[const plot mark right,mark=*] file {table.dat};
&
\tikz \draw plot[jump mark left, mark=*] file {table.dat};
\\ \hline
\RDD{const plot} & \RDD{const plot mark left} & \RDD{const plot mark right} & \RDD{jump mark left}
\\ \hline
\tikz \draw plot[jump mark right, mark=*] file {table.dat};
&
\tikz \draw plot[ycomb,thin,mark=*] file {table.dat};
&
\tikz \draw plot[xcomb,mark=*] file {table.dat};
&
\tikz \draw plot[only marks,mark=*] file {table.dat};
\\ \hline
\RDD{jump mark right} & \RDD{ycomb} & \RDD{xcomb} & \RDD{only marks}
\\ \hline
\end{tabular}

\bigskip
\begin{tabular}{|c | c | c |c |} \hline
%\begin{tikzpicture}
%\draw[help lines] (-2,-3) grid (2,2);
\tikz  \draw plot[polar comb,mark=*]coordinates {(0:1) (60:0.5) (120:1.5) (180:3) (240:.5) (300:1) (0:1)};
%\draw[line width=1pt,color=red] plot coordinates (0:1cm)(60:0.5)(120:1.5)(180:1)(240:3)(300:1)(0:1cm);
%\end{tikzpicture}
\\ \hline
\BS{tikz}  \BS{draw} plot[\RDD{polar comb},mark=*]coordinates \\
\AC{(0:1) (60:0.5) (120:1.5) (180:3) (240:.5) (300:1) (0:1)};
\\ \hline
\end{tabular}

\bigskip

\begin{tabular}{|c | c | c |c |} \hline
\multicolumn{4}{|c|}{ \BS{tikz} \BS{draw}  plot[\RDD{ybar}] file \AC{table.dat} ;   }\\ 
\hline
\tikz \draw plot[ybar] file {table.dat};
&
\tikz \draw plot[ybar interval] file {table.dat};
&
\tikz \draw plot[ybar interval,x=2cm] file {table.dat};
&
\tikz \draw plot[ybar interval,y=.5cm] file {table.dat};
\\ \hline
[\RDD{ybar}] & [\RDD{ybar interval}] & [ybar interval,\RDD{x}=2cm] & [ybar interval,\RDD{y}=.5cm]
\\ \hline
\end{tabular}

\bigskip
 \begin{tabular}{|c|c|}  \hline 
\begin{tikzpicture}[baseline=0pt]
\draw[red,fill=cyan,ybar,bar width=.5cm]plot coordinates{(0,1) (1,1.2) (2,.6) (3,.7) (4,.9)};
\draw[blue,fill=green,ybar,bar width=.5cm,bar shift=.3cm]plot coordinates{(0,1.2) (1,1.3) (2,.5) (3,.2) (4,.5)};
\end{tikzpicture}
&
\parbox[c]{10cm}{
\BS{begin}\AC{tikzpicture} \\
\BS{draw}[red,fill=cyan,ybar,bar width=.5cm] \\
\rule{1cm}{.0pt} plot coordinates \AC{(0,1) (1,1.2) (2,.6) (3,.7) (4,.9)}; \\
\BS{draw}[blue,fill=green,ybar,bar width=.5cm,\RDD{bar shift}=.3cm] \\
\rule{1cm}{.0pt} plot coordinates \AC{(0,1.2) (1,1.3) (2,.5) (3,.2) (4,.5)}; \\
\BS{end}\AC{tikzpicture} }
 \\  \hline 
 \end{tabular} 

\bigskip

\begin{tabular}{|c | c | c | c |c |} \hline
\multicolumn{4}{|c|}{ \BS{tikz} \BS{draw}  plot[xbar interval] file \AC{table.dat} ;   }\\ 
\hline
\tikz \draw[blue] plot[xbar] coordinates{(2,0) (3,1) (1,2) (2,3)};
&
\tikz \draw[blue] plot[xbar interval]  coordinates {(2,0) (3,1) (1,2) (2,3)};
&
\tikz \draw[blue] plot[xbar interval,x=.5cm]  coordinates {(2,0) (3,1) (1,2) (2,3)};
&
\tikz \draw[blue] plot[xbar interval,y=.5cm]  coordinates {(2,0) (3,1) (1,2) (2,3)};
%&
%\tikz \draw[blue!20] plot[xbar interval,x=.5cm,y=.5cm]  coordinates {(2,0) (3,1) (1,2) (2,3)};
\\ \hline
[\RDD{xbar}] & [\RDD{xbar interval}] & [xbar interval,\RDD{x}=.5cm] & [xbar interval,\RDD{y}=.5cm] 
\\ \hline
\end{tabular}

\newpage
%--------------------------------------------------------------
%\subsubsection{Graphe à partir d'une fonction}
\SbSbSSCT{Graphe à partir d'une fonction}{Graph of a function}


\begin{tabular}{|c | c | c | } \hline
\multicolumn{3}{|c|}{  \BS{draw}  [color=red] plot (\BS{x},\BS{x});   }\\ 
\hline
\begin{tikzpicture}[domain=0:4,ultra thick]
%\draw[very thin,color=gray] (-0.1,-1.1) grid (4.1,4.1);
\draw[->,blue,ultra thick] (-.1,0) -- (4.5,0);
\draw[->,blue,ultra thick] (0,-1.1) -- (0,04);
\draw[color=red] plot (\x,\x);
\end{tikzpicture} 
&
\begin{tikzpicture}[domain=0:6.28,ultra thick,x=0.7cm]
%\draw[very thin,color=gray] (-0.1,-2.1) grid (4.1,2.1);
\draw[->,blue,ultra thick] (-.1,0) -- (7,0);
\draw[->,blue,ultra thick] (0,-2.5) -- (0,2.5);
\draw[color=red] plot  (\x,{sin(\x r)});
\end{tikzpicture} 
&
\begin{tikzpicture}[domain=0:360,x=0.3,ultra thick]
%\draw[very thin,color=gray] (-0.1,-2.1) grid (4.1,2.1);
\draw[->,blue,ultra thick] (-.1,0) -- (370,0);
\draw[->,blue,ultra thick] (0,-2.5) -- (0,2.5);
\draw[color=red] plot (\x,{sin(\x)});
\end{tikzpicture} 
\\ \hline
(\BS{x},\BS{x}) &  (\BS{x},\AC{sin(\BS{x} r)}) & (\BS{x},\AC{sin(\BS{x})}) \\
& x en radian & x en degré
\\ \hline
\end{tabular}

Options 

\begin{tabular}{|c | c |} \hline
\multicolumn{2}{|l|}{ \BS{draw}[color=red,dashed] plot(\BS{x},\AC{sin(\BS{x} r)});}  \\
\multicolumn{2}{|l|}{ \BS{draw}[color=blue,\RDD{samples}=5,mark=*,ultra thick] plot(\BS{x},\AC{sin(\BS{x} r)});} 
\\ \hline
\begin{tikzpicture}[domain=0:6.28]
\draw[very thin,color=gray] (-0.1,-1.1) grid (6.28,1.1);
\draw[color=red,dashed] plot  (\x,{sin(\x r)});
\draw[color=blue,samples=5,mark=*,ultra thick] plot  (\x,{sin(\x r)});
\end{tikzpicture} 
&
\begin{tikzpicture}
\draw[very thin,color=gray] (-0.1,-1.1) grid (6.28,1.1);
\draw[color=red,dashed,domain=0:6.28] plot  (\x,{sin(\x r)});
\draw[color=blue,domain=0:4,ultra thick] plot  (\x,{sin(\x r)});
\end{tikzpicture} 
  \\ \hline
[color=blue,\RDD{samples}=5,mark=*] & [color=blue,\RDD{domain}=0:4]
\\ \hline
\begin{tikzpicture}
\draw[very thin,color=gray] (-0.1,-1.1) grid (6.28,1.1);
\draw[color=red,dashed,domain=0:6.28] plot  (\x,{sin(\x r)});
\draw[color=blue,domain=1:5,ultra thick] plot  (\x,{sin(\x r)});
\end{tikzpicture} 
&
\begin{tikzpicture}[domain=0:6.28]
\draw[very thin,color=gray] (-0.1,-1.1) grid (6.28,1.1);
\draw[color=red,dashed] plot  (\x,{sin(\x r)});
\draw[color=blue,samples at={1,2,4,5},mark=*,ultra thick] plot  (\x,{sin(\x r)});
\end{tikzpicture} 
\\ \hline
[color=blue,\RDD{domain}=1:5] & [color=blue,\RDD{samples at}=\AC{1,2,4,5},mark=*]
\\ \hline
\end{tabular}


%-------------------------------------------------------------------------
%\subsubsection{Fonctions paramétriques}
\SbSbSSCT{Fonctions paramétriques}{Parametric function}


\begin{tabular}{|c | c |} \hline
\multicolumn{2}{|l|}{  \BS{draw}[domain=-3.141:3.141,smooth,variable=\BS{t}]plot (\AC{sin(\BS{t} r)},\AC{sin(2 *\BS{t} r)});} \\
\multicolumn{2}{|l|}{  \BS{draw}[domain=0:720,smooth,variable=\BS{t}]plot (\AC{sin(\BS{t})},\BS{t}/360,\AC{cos(\BS{t})});} 
\\ \hline

\tikz \draw[domain=-3.141:3.141,smooth,variable=\t,ultra thick]plot ({sin(\t r)},{sin(2*\t r)});
&
\tikz \draw[domain=0:720,smooth,variable=\t,ultra thick] plot ({sin(\t)},\t/360,{cos(\t)});
\\ \hline
(\AC{sin(\BS{t} r)},\AC{sin(2 *\BS{t} r)}) & (\AC{sin(\BS{t})},\BS{t}/360,\AC{cos(\BS{t})})
\\ \hline
\end{tabular} 
%\tikz \draw plot[mark=x,mark repeat=3,smooth] file {plots/pgfmanual-sine.table};
 

%\subsection{Marques}
\SbSSCT{Marques}{Marks}

%\subsubsection{Marques avec Tikz}
\SbSbSSCT{Marques avec TikZ}{Marks with TikZ}

\begin{tabular}{|c | c | c | c |} \hline
\tikz \draw plot[mark=+,mark size=5pt] coordinates {(0,0) (1,1) (2,0)};
&
\tikz \draw plot[mark=x,mark size=5pt] coordinates {(0,0) (1,1) (2,0) };
&
\tikz \draw plot[mark=*,mark size=5pt] coordinates {(0,0) (1,1) (2,0)};
&
\tikz \draw plot[mark=ball,mark size=5pt] coordinates {(0,0) (1,1) (2,0)};
\\ \hline
mark=+ & mark=x & mark=* & mark=ball
\\ \hline
\end{tabular}

\bigskip

\begin{tabular}{|c | c |} \hline
\begin{tikzpicture}[domain=0:6.28]
\draw[very thin,color=gray] (-0.1,-1.1) grid (6.28,1.1);
\draw[color=red,dashed,mark=+] plot  (\x,{sin(\x r)});
\draw[color=blue,mark repeat=3,mark=*] plot  (\x,{sin(\x r)});
\end{tikzpicture} 
&
\begin{tikzpicture}[domain=0:6.28]
\draw[very thin,color=gray] (-0.1,-1.1) grid (6.28,1.1);
\draw[color=red,dashed,mark=+] plot  (\x,{sin(\x r)});
\draw[color=blue,mark repeat=3,mark phase=5,mark=*] plot  (\x,{sin(\x r)});
\end{tikzpicture} 
\\ \hline
[color=blue,\RDD{mark repeat}=3,mark=*] & [color=blue,mark repeat=3,\RDD{mark phase}=5,mark=*]
\\ \hline
\begin{tikzpicture}[domain=0:6.28]
\draw[very thin,color=gray] (-0.1,-1.1) grid (6.28,1.1);
\draw[color=red,dashed,mark=+] plot  (\x,{sin(\x r)});
\draw[color=blue,mark indices={1,4,...,15,17,20},mark=*] plot  (\x,{sin(\x r)});
\end{tikzpicture} 
&
\begin{tikzpicture}[domain=0:6.28]
\draw[very thin,color=gray] (-0.1,-1.1) grid (6.28,1.1);
\draw[color=red,dashed,mark=+] plot  (\x,{sin(\x r)});
\draw[color=blue,mark size=5pt,mark=*] plot  (\x,{sin(\x r)});
\end{tikzpicture} 
\\ \hline
[color=blue,\RDD{mark indices}={1,4,...,15,17,20},mark=*] & [color=blue,\RDD{mark size}=5pt,mark=*]
\\ \hline
\begin{tikzpicture}[domain=0:6.28]
\draw[very thin,color=gray] (-0.1,-1.1) grid (6.28,1.1);
%\draw[color=red,dashed,mark=*] plot  (\x,{sin(\x r)});
\draw[color=blue,mark size=5pt,mark options={color=magenta},mark=+] plot  (\x,{sin(\x r)});
\end{tikzpicture}
&
\begin{tikzpicture}[domain=0:6.28]
\draw[very thin,color=gray] (-0.1,-1.1) grid (6.28,1.1);
%\draw[color=red,dashed,mark=*] plot  (\x,{sin(\x r)});
\draw[color=blue,mark size=5pt,mark options={rotate=10},mark=+] plot  (\x,{sin(\x r)});
\end{tikzpicture}
\\ \hline
\RDD{mark options}=\AC{color=magenta},mark=+ & \RDD{mark options}=\AC{rotate=10},mark=+
\\ \hline
\end{tabular}
 

%\subsubsection{Marques personnalisées avec text mark}
\SbSbSSCT{Marques personnalisées avec text mark}{Marks with text mark}

\begin{tabular}{|c | c | c |} \hline
\multicolumn{3}{|l|}{ \BS{draw}[\RDD{mark=text},\RDD{text mark}=A,mark size=5pt] coordinates \AC{(0,0) (1,1) (2,0)};} 
\\ \hline
\tikz \draw plot[mark=text,text mark=A,mark size=5pt] coordinates {(0,0) (1,1) (2,0)};
&
\tikz \draw plot[mark=text,text mark=Texte,mark size=5pt] coordinates {(0,0) (1,1) (2,0)};
&
\begin{tikzpicture}
\draw[white]  (-1,0)-- (-1,1.5);
 \draw plot[mark=text,text mark=\DFR,mark size=5pt] coordinates {(0,0) (1,1) (2,0)};
\end{tikzpicture} 
\\ \hline
\RDD{text mark}=A &  \RDD{text mark}=Texte & \RDD{text mark}=\BS{DFR} \pageref{DFR} 
\\ \hline 
\multicolumn{3}{|c|}{ 
\begin{tikzpicture}
\draw[white]  (-1,0)-- (-1,1.5);
\draw plot[mark=text,text mark={\includegraphics[width=.5cm]{tiger}} ,mark size=5pt] coordinates {(0,0) (1,1) (2,0)};  
\end{tikzpicture} }
\\ \hline  
\multicolumn{3}{|c|}{ \RDD{text mark}=\AC{\BS{includegraphics}[width=.5cm]\AC{tiger}} }
\\ \hline   
\end{tabular}


\newpage
%\subsubsection{Marques avec l'extension plotmarks }
\SbSbSSCT{Marques avec l'extension plotmarks }{Marks with plotmarks library}

\label{plotmarks}

%Insérer dans le préambule :

 \maboite{\BS{usetikzlibrary}\AC{plotmarks}}
 
\begin{center}
\RRR{63}
\end{center}

\begin{tabular}{|c | c | c | c |} \hline
\tikz \draw plot[mark=-,mark size=5pt] coordinates {(0,0) (1,1) (2,0)};
& 
\tikz \draw plot[mark=|,mark size=5pt] coordinates {(0,0) (1,1) (2,0)};
 &
\tikz \draw plot[mark=o,mark size=5pt] coordinates {(0,0) (1,1) (2,0)};
 &
\tikz \draw plot[mark=asterisk,mark size=5pt] coordinates {(0,0) (1,1) (2,0)};
\\ \hline 
mark=- & mark=| & mark=o &mark=asterisk
\\ \hline
\tikz \draw plot[mark=star,mark size=5pt] coordinates {(0,0) (1,1) (2,0)};
&
\tikz \draw plot[mark=10-pointed star,mark size=5pt] coordinates {(0,0) (1,1) (2,0)};
&
\tikz \draw plot[mark=oplus,mark size=5pt] coordinates {(0,0) (1,1) (2,0)};
&
\tikz \draw plot[mark=oplus*,mark size=5pt] coordinates {(0,0) (1,1) (2,0)};
\\ \hline
mark=star & mark=10-pointed star & mark=oplus & mark=oplus*
\\ \hline
 
\tikz \draw plot[mark=otimes,mark size=5pt] coordinates {(0,0) (1,1) (2,0)};
&
\tikz \draw plot[mark=otimes*,mark size=5pt] coordinates {(0,0) (1,1) (2,0)};
&
\tikz \draw plot[mark=square,mark size=5pt] coordinates {(0,0) (1,1) (2,0)};
&
\tikz \draw plot[mark=square*,mark size=5pt] coordinates {(0,0) (1,1) (2,0)};
\\ \hline
 mark=otimes & mark=otimes* & mark=square & mark=square*
  \\ \hline
  
\tikz \draw plot[mark=triangle,mark size=5pt] coordinates {(0,0) (1,1) (2,0)};
& 
\tikz \draw plot[mark=triangle*,mark size=5pt] coordinates {(0,0) (1,1) (2,0)};
& 
\tikz \draw plot[mark=diamond,mark size=5pt]  coordinates {(0,0) (1,1) (2,0)};
 &
\tikz \draw plot[mark=diamond*,mark size=5pt] coordinates {(0,0) (1,1) (2,0)};
\\ \hline 
mark=triangle & mark=triangle* & mark=diamond & mark=diamond*
\\ \hline 

\tikz \draw plot[mark=halfdiamond*,mark size=5pt] coordinates {(0,0) (1,1) (2,0)};
 &
\tikz \draw plot[mark=halfsquare*,mark size=5pt] coordinates {(0,0) (1,1) (2,0)};
 &
\tikz \draw plot[mark=halfsquare right*,mark size=5pt] coordinates {(0,0) (1,1) (2,0)};
 &
\tikz \draw plot[mark=halfsquare left*,mark size=5pt] coordinates {(0,0) (1,1) (2,0)};
\\ \hline 
mark=halfdiamond* & mark=halfsquare* & mark=halfsquare right* & mark=halfsquare left*
\\ \hline 

\tikz \draw plot[mark=pentagon,mark size=5pt] coordinates {(0,0) (1,1) (2,0)};
 &
\tikz \draw plot[mark=pentagon*,mark size=5pt] coordinates {(0,0) (1,1) (2,0)};
 &
\tikz \draw plot[mark=Mercedes star,mark size=5pt] coordinates {(0,0) (1,1) (2,0)};
 &
\tikz \draw plot[mark=Mercedes star flipped,mark size=5pt] coordinates {(0,0) (1,1) (2,0)};
 \\ \hline
 mark=pentagon & mark=pentagon* & mark=Mercedes star & mark=Mercedes star flipped
 \\ \hline 
 
\tikz \draw plot[mark=halfcircle,mark size=5pt] coordinates {(0,0) (1,1) (2,0)};
 &
\tikz \draw plot[mark=halfcircle*,mark size=5pt] coordinates {(0,0) (1,1) (2,0)};
& 
\tikz \draw plot[mark=heart,mark size=5pt] coordinates {(0,0) (1,1) (2,0)};
 &
\tikz \draw plot[mark=text,mark size=5pt] coordinates {(0,0) (1,1) (2,0)};
 \\ \hline
 mark=halfcircle & mark=halfcircle* & mark=heart & mark=text
  \\ \hline
\end{tabular}

\bigskip

\begin{tabular}{|c | c | c | c |} \hline
\multicolumn{4}{|l|}{ \BS{draw}[mark=halfcircle,\RDD{mark color}=red,mark size=5pt] coordinates \AC{(0,0) (1,1) (2,0)};} 
\\ \hline
\tikz \draw plot[mark=halfcircle,mark color=red,mark size=5pt] coordinates {(0,0) (1,1) (2,0)};
&
\tikz \draw plot[mark=halfcircle*,mark color=red,mark size=5pt] coordinates {(0,0) (1,1) (2,0)};
&
\tikz \draw plot[mark=halfdiamond*,mark color=red,mark size=5pt] coordinates {(0,0) (1,1) (2,0)};
&
\tikz \draw plot[mark=halfsquare*,mark color=red,mark size=5pt] coordinates {(0,0) (1,1) (2,0)};
  \\ \hline
  mark=halfcircle & mark=halfcircle* & mark=halfdiamond* & mark=halfsquare*
   \\ \hline 
\end{tabular}



% \subsection{Graphes avec Gnuplot}
\SbSSCT{Graphes avec Gnuplot}{Graph with Gnuplot}
 
 \begin{tabular}{|l| } \hline
%\begin{tikzpicture}[domain=0:6.28]
%%\draw[very thin,color=gray] (-0.1,-1.1) grid (7.1,1.1);
%%\draw[->,ultra thick,blue] (-0.2,0) -- (7,0) node[right] {$x$};
%%\draw[->,ultra thick,blue] (0,-1.2) -- (0,1.2) node[above] {$f(x)$};
%%\draw[color=red] plot[id=x] function{x} node[right] {$f(x) =x$};
%\draw[color=red] plot[id=sin] function{sin(x)} ;
%%\draw[color=orange] plot[id=exp] function{0.05*exp(x)} node[right] {$f(x) = \frac{1}{20} \mathrm e^x$};
%\end{tikzpicture}
\BS{draw}[color=red] plot[\RDD{id}=sin] function\AC{sin(x)} ;
   \\ \hline
\\
==> plot[id=sin] \TFRGB{crée le fichier}{create the file} \og sin.gnuplot \fg \\
==>  \TFRGB{Ouvrir le fichier}{Open the file} \og sin.gnuplot \fg \TFRGB{avec le programme gnuplot pour créer le fichier}{with the program gnuplot : creation of the file }   \og sin.table \fg\\
==> \TFRGB{Utiliser le fichier de données} {Use the datafile }
 \og sin.table  \fg   \\ \hline 
\end{tabular}
%
%\newpage
%
%\section[Créer un graphe avec pgfplot]{Créer un graphe avec pgfplot \cite{pgfplots}}
%
%\input{tkzgraph2} % <<<<<<<<<<<<<<<<<<<<<<<<<<<<<
%%
%\subsection{Courbes 3D}
%
%\input{tkzgraph3D} % très lourd à compiler
%
%%
%%%%
%%%%\essais{pstgraph2ess.tex}
%%%\newpage
%%%\section[Créer un graphe d'après une équation]{Créer un graphe d'après une équation  \cite{pst-user} \cite{pst-plot}}
%%%
%%
%%%%
%%%%\essais{pstgraph3ess.tex} 
%%%\newpage
%%% \section[Des outils pour les graphes]{Des outils pour les graphes \cite{pst-add} }
%%% 
%%
%%%
%%%\newpage
%%% \section[Créer un graphe en camembert]{Créer un graphe en camembert \cite{pst-add} }
%%% 
%%%\input{chart} % camembert
%
%\newpage
%
%%%%=============================================
%\section{Les Tableaux de variation}
%
%\input{tkztab}
%
%\newpage
%%%%=============================================
%\section{Les répétitions}
%
%\input{tkzrep1}  % OK
%
%%%\subsection[Commande multido]{Commande multido \cite{pst-user} \cite{multido} }
%%%
%%
%%%
%%%\essais{pstrep2ess.tex}
%%%
%%%\subsection[Commande psforeach]{Commande psforeach \cite{pst-news10} }
%%
%%
%%%
%%%\newpage
%%%% % % %======================================================================
%%%\section[La géométrie]{La géométrie  \cite{pst-eucl} }
%%%
%%%Utilisation du module \textbf{pst-eucl} \label{pst-eucl}(consultez le fichier\textbf{ pst-eucl-doc.pdf} )
%%%
%%%
%%%\psset{fillcolor=yellow,linecolor=blue,dotscale=2}
%%%\subsection{\'Elements de base}
%%%
%%
%%%
%%%\subsection[Transformations géométriques]{Transformations géométriques \cite{pst-eucl} }
%%%
%%%
%%
%%%
%%%
%%%\subsection[Constructions particulières en géométrie ]{Constructions particulières en géométrie }
%%%
%%
%%%
%%%\subsection[Intersections]{Intersections  }
%%%
%%
%%%
%%%%--------------------------------------------------------------
%%%\section[Les vecteurs]{Les vecteurs  }
%%%
%%
%%%%==============================================================
%\newpage
% 
%\section[Les diagrammes arborescents ]{Les diagrammes arborescents }
%
%\input{tkztree}
%
%%%%==============================================================
%\newpage
%
%\section[Les animations ]{Les animations }
%
%\input{tkzanim}
%%%
%%%\newpage
%%%
%%%\section[Créer un dessin en 3D]{Créer un dessin en 3D  }
%%
%
%%%\subsection{Les objets en 3D}
%%%
%
%%%\newpage
%%%\subsection[Créer un graphe en 3D]{Créer un graphe en 3D } 
%%%
%
%\newpage
%%%=======================================================================
%\section{Les modules étudiés dans ce document}
%
%
\TFRGB{module de base TikZ}{Basic TikZ package} : 

\maboite{\BS{usepackage}\AC{tikz} }

%\bigskip
\bigskip
\textbf{\TFRGB{Autres modules}{Other packages}}

%
\begin{tabular}{|c|c|l c|}\hline 
\TFRGB{nom}{name} 			& \TFRGB{voir page} 				& documentation\footnotemark[1] & \\  \hline 

animate 		& \pageref{anim} 			& animate.pdf 			& \DGB\\
tkz-tab  		& \pageref{tab} 			& tkz-tab-screen.pdf 	& \DFR \\
\hline 
\end{tabular} 
\bigskip

\textbf{\TFRGB{Compléments optionnels}{Optional library} :}

\begin{tabular}{|l|c|l|}\hline 
\TFRGB{nom}{name} 				& \TFRGB{voir page}{see page}						& \TFRGB{A insérer dans le préambule}{Load package}\\ \hline 
angles				& \pageref{lib-angles}			&  \BS{usetikzlibrary}\AC{angles}
\\
arrows.meta				& \pageref{lib-arrows.meta}			&  \BS{usetikzlibrary}\AC{arrows.meta}
\\
bending				& \pageref{lib-bending}			&  \BS{usetikzlibrary}\AC{bending}
\\
backgrounds			& \pageref{lib-bkgd} 			&  \BS{usetikzlibrary}\AC{backgrounds}
\\
calc				& \pageref{lib-calc}			&  \BS{usetikzlibrary}\AC{calc}
\\
circuits.ee.IEC				& \pageref{lib-ee}			&  \BS{usetikzlibrary}\AC{circuits.ee.IEC}
\\
 
fit & \pageref{lib-fit} 	& \BS{usetikzlibrary}\AC{fit} 
\\
decorations.footprints & \pageref{lib-footprints} 	& \BS{usetikzlibrary}\AC{decorations.footprints} 
\\
decorations.fractals & \pageref{lib-fractals} 		& \BS{usetikzlibrary}\AC{decorations.fractals} 
\\
decorations.markings & \pageref{lib-mark} 			& \BS{usetikzlibrary}\AC{decorations.markings} 
\\
decorations.pathmorphing  & \pageref{lib-morph}		& \BS{usetikzlibrary}\AC{decorations.pathmorphing}
\\
decorations.pathreplacing & \pageref{lib-replac}	& \BS{usetikzlibrary}\AC{decorations.pathreplacing} 
\\
decorations.shapes & \pageref{lib-shapes} 			& \BS{usetikzlibrary}\AC{decorations.shapes} 
\\
decorations.text & \pageref{lib-text} 				& \BS{usetikzlibrary}\AC{decorations.text} 
\\

fadings 			& \pageref{lib-fadings}			&  \BS{usetikzlibrary}\AC{fadings }
\\
intersections		& \pageref{lib-intersections}	&  \BS{usetikzlibrary}\AC{intersections}
\\
patterns			& \pageref{lib-patterns}		&  \BS{usetikzlibrary}\AC{patterns}
\\
plotmarks			& \pageref{plotmarks} 			&  \BS{usetikzlibrary}\AC{plotmarks}
\\ 
scopes				& \pageref{lib-scopes}			&  \BS{usetikzlibrary}\AC{scopes}
\\
shadings			& \pageref{lib-shadings}		&  \BS{usetikzlibrary}\AC{shadings}
\\
shapes.arrows		& \pageref{lib-arr}				&\BS{usetikzlibrary}\AC{shapes.arrows} 
\\shapes.callouts		& \pageref{lib-call}			& \BS{usetikzlibrary}\AC{shapes.callouts} 
\\
shapes.geometric	& \pageref{lib-geom} 			& \BS{usetikzlibrary}\AC{shapes.geometric}
\\  
shapes.misc			& \pageref{lib-misc} 			& \BS{usetikzlibrary}\AC{shapes.misc} 
\\
shapes.multipart	& \pageref{lib-mult} 			& \BS{usetikzlibrary}\AC{shapes.multipart} 
\\
shapes.symbols		& \pageref{lib-symb}			& \BS{usetikzlibrary}\AC{shapes.symbols} 
\\
trees				& \pageref{lib-trees}			&  \BS{usetikzlibrary}\AC{trees}
\\ 

\hline
 \end{tabular} 


\bigskip



\begin{tabular}{|l|c|}\hline
\multicolumn{2}{|c|}{ \TFRGB{dans une prochaine mise à jour}{In a a future update } }
\\ \hline
automata									& \RRR{41} \\
babel										& \RRR{42} \\
calendar									& \RRR{45} \\
chains										& \RRR{46} \\ 
%circuits.ee									& \RRR{47-4} \\ 
circuits.logic								& \RRR{47-3} \\ 
circular graph drawing library 				& \RRR{32} \\
curvilinear library 						& \RRR{103-4-7} \\
datavisualization library					& \RRR{75} \\
datavisualization.formats.functions library & \RRR{76-4} \\
datavisualization.polar library 			& \RRR{80}  \\
 er 										& \RRR{49}  \\
examples graph drawing library 				& \RRR{35-8} \\ 
external 									& \RRR{50}  \\  
%fit 										& \RRR{52} \\ 
fixedpointarithmetic 						& \RRR{53} \\ 
folding 									& \RRR{59} \\
force graph drawing library 				& \RRR{31}  \\
fpu											& \RRR{54}  \\
graph.standard library 						& \RRR{19-10}\\
graphdrawing library 						& \RRR{27} \\
graphs library 								& \RRR{19} \\ 
layered graph drawing library 				& \RRR{30}  \\
lindenmayersystems							& \RRR{55}  \\ 
matrix										& \RRR{57}  \\ 
mindmap										& \RRR{58} \\ 
petri										& \RRR{61}  \\ 
phylogenetics graph drawing library 		& \RRR{33} \\
plothandlers								& \RRR{62}  \\ 
positioning									& \RRR{17-5-3} \\ 
profiler									& \RRR{64}   \\ 
quotes library 								& \RRR{17-10-4} \\
routing graph drawing library 				& \RRR{34} \\
shadows										& \RRR{66}   \\ 
shapes.gates.ee								& \\ 
shapes.gates.ee.IEC							& \\ 
shapes.gates.logic							& \\ 
shapes.gates.logic.IEC						& \\ 
shapes.gates.logic.US						& \\ 
spy											&  \RRR{68} \\ 
svg.path									&  \RRR{69} \\ 
through										&  \RRR{71} \\ 
topaths										&  \RRR{70} \\ 
trees graph drawing library					& \\
turtle										&  \RRR{73} \\ 
\hline
\end{tabular}  
%circuit.ee.IEC, 309
%circuits, 292
%circuits.ee, 308
%, 300
%circuits.logic.CDH, 301
%circuits.logic.IEC, 300
%circuits.logic.US, 301

%
%\bigskip
%\textbf{Autres modules}
%
%%
%\begin{tabular}{|c|c|l|}\hline 
%nom 			& voir page 				& documentation\footnotemark[1]  \\  \hline 
%pst-fr3d 		& \pageref{pst-fr3d}		& pst-fr3d.pdf		\\ 
%pst-slpe 		& \pageref{pst-slpe}		& pst-slpe.pdf		\\ 
%infix-RPN 		& \pageref{infix-RPN}		& pst-infixplot.pdf	\\
%pst-infixplot 	& \pageref{pst-infixplot} 	& pst-infixplot.pdf \\ 
%pst-eucl 		& \pageref{pst-eucl} 		& pst-eucl-doc.pdf 	\\
%animate 		& \pageref{anim} 			& animate.pdf 	\\
%pst-3dplot		& \pageref{3dplot} 			& pst-3dplot-doc 	\\
%\hline 
%\end{tabular} 
%
%\bigskip
%\textbf{Additifs }
%
%%
%\begin{tabular}{|c|l|}\hline 
%année						& documentation\footnotemark[1]  \\  \hline 
%2005 			& pst-news5.pdf	\\
%2008  			& pst-news08.pdf \\ 
%2010 			& pst-news10.pdf 	\\
%\hline 
%\end{tabular} 
% 
%
%\footnotetext[1]{Vous pouvez les trouver pour la distribution Texlive dans le répertoire :  \BS{}texlive\BS{}2011\BS{}tesmf-dist\BS{}doc\BS{}generic}
%
%\newpage
%%
%% \tableofcontents
%\renewcommand{\bibname}{Sources}
%
\label{sources}
%\input{bib}

\newpage

\begin{thebibliography}{99}
\bibitem{pgfmanual} pgfmanual.pdf  	\hspace{1cm}	version 3.0.1a \hspace{1cm} 	1161 pages 	\hspace{1cm}	\DGB
\bibitem{pgfplots} pgfplots.pdf 	\hspace{1cm}	version 1.80 \hspace{1cm} 	439 pages 	\hspace{1cm}	\DGB
\bibitem{tikstab} tkz-tab-screen.pdf 	\hspace{1cm}	version 1.1c \hspace{1cm} 	83 pages 	\hspace{1cm}	\DFR

\end{thebibliography}
%
%
%
%\newpage 
%\section{Index}
%
% \printindex 
\end{document}