%\section{Circuit électrique}
\label{lib-ee}
%Insérer dans le préambule :

 \maboite{\BS{usepackage}\AC{circuits.ee.IEC}}
 
 
%\subsection{Symboles}
\SbSSCT{Symboles}{Symbols}

\begin{center}
\RRR{47-4}
\end{center}

\begin{tabular}{|c|c|c|c|} \hline 
\TFRGB{sur un noeud}{On a node} & \TFRGB{sur un chemin}{On a path}
\\   \hline
\begin{tikzpicture}[blue,line width=2pt]
\useasboundingbox (-.2,-.2) grid (2.2,1.2);
\draw [help lines] (0,0) grid (2,1);
\node[circuit ee IEC] at (1,.5) [resistor] {} ; 
\end{tikzpicture}
&
\begin{tikzpicture}[blue,line width=2pt]
\useasboundingbox (-.2,-.2) grid (2.2,1.2);
\draw [help lines] (0,0) grid (2,1);
\draw [circuit ee IEC] (0,.5) to [resistor] (2,.5) ; 
\end{tikzpicture} 
\\   \hline
\BS{node}  \rouge{[circuit ee IEC]}  at (1,0.5) \rouge{ to [resistor] } \AC{} ;
&
\BS{draw}   \rouge{[circuit ee IEC]}(0,0.5) \rouge{ to [resistor] } (2,.5) ;
\\   \hline   
\end{tabular} 

\bigskip

\begin{tabular}{|c|c|c|c|} \hline
\multicolumn{4}{|c|}{\textbf{\TFRGB{Composants de base}{Basic Elements}}  }\\ 
\hline  
\multicolumn{4}{|c|}{\BS{draw}  [circuit ee IEC] (0,.5) to [\RDD{resistor}] (2,.5) ;  }\\ 
\hline
\multicolumn{4}{|c|}{\RRR{47-4-3} }
\\ \hline   
\begin{tikzpicture}[blue]
\useasboundingbox (-.5,0) rectangle (2.5,1);
\draw [circuit ee IEC] (0,0.5) to [resistor] (2,0.5) ; 
\end{tikzpicture} 
&
\begin{tikzpicture}[blue]
\useasboundingbox (-.5,0) rectangle (2.5,1);
\draw [circuit ee IEC] (0,.5) to [inductor] (2,.5) ; 
\end{tikzpicture}
&
\begin{tikzpicture}[blue]
\useasboundingbox (-.5,0) rectangle (2.5,1);
\draw [circuit ee IEC] (0,.5) to [capacitor] (2,.5) ; 
\end{tikzpicture}
&
\begin{tikzpicture}[blue]
\useasboundingbox (-.5,0) rectangle (2.5,1);
\draw [circuit ee IEC] (0,.5) to [battery] (2,.5) ; 
\end{tikzpicture} 
\\   \hline 
[\RDD{resistor}] &[\RDD{inductor}] & [\RDD{capacitor}] &[\RDD{battery}]
\\   \hline 
\begin{tikzpicture}[blue]
\useasboundingbox (-.5,0) rectangle (2.5,1);
\draw [circuit ee IEC] (0,.5) to [bulb] (2,.5) ; 
\end{tikzpicture}
&
\begin{tikzpicture}[blue]
\useasboundingbox (-.5,0) rectangle (2.5,1);
\draw [circuit ee IEC] (0,.5) to [current source] (2,.5) ; 
\end{tikzpicture}
&
\begin{tikzpicture}[blue]
\useasboundingbox(-.5,0) rectangle (2.5,1);
\draw [circuit ee IEC] (0,.5) to [voltage source] (2,.5) ; 
\end{tikzpicture}; 
&
\begin{tikzpicture}[blue]
\useasboundingbox (-.5,0) rectangle (2.5,1);
\draw [circuit ee IEC] (0,.5) to [ground] (2,.5) ; 
\end{tikzpicture} 
\\   \hline
[\RDD{bulb}] &[\RDD{current source}] & [\RDD{voltage source}] &[\RDD{ground}]
\\   \hline 
\multicolumn{4}{|c|}{\RRR{47-4-4} }
\\ \hline  
\begin{tikzpicture}[blue]
\useasboundingbox (-.5,0) rectangle (2.5,1);
\draw [circuit ee IEC] (0,.5) to [diode] (2,.5) ; 
\end{tikzpicture} 
&
\begin{tikzpicture}[blue]
\useasboundingbox (-.5,0) rectangle (2.5,1);
\draw [circuit ee IEC] (0,.5) to [Zener diode]  (2,.5) ; 
\end{tikzpicture} 
&
\begin{tikzpicture}[blue]
\useasboundingbox (-.5,0) rectangle (2.5,1);
\draw [circuit ee IEC] (0,.5) to [Schottky diode]  (2,.5) ; 
\end{tikzpicture}
&
\begin{tikzpicture}[blue]
\useasboundingbox(-.5,0) rectangle (2.5,1);
\draw [circuit ee IEC] (0,.5) to [tunnel diode]  (2,.5) ; 
\end{tikzpicture} 
\\   \hline 
[\RDD{diode}] &[\RDD{Zener diode}] & [\RDD{Schottky diode}] &[\RDD{tunnel diode}]
\\   \hline
\begin{tikzpicture}[blue]
\useasboundingbox (-.5,0) rectangle (2.5,1); 
\draw [circuit ee IEC] (0,.5) to [backward diode]  (2,.5) ; 
\end{tikzpicture}  
&
\begin{tikzpicture}[blue]
\useasboundingbox (-.5,0) rectangle (2.5,1);
\draw [circuit ee IEC] (0,.5) to [breakdown diode]  (2,.5) ; 
\end{tikzpicture} 
&
& 
\\   \hline 
[\RDD{backward diode}] &[\RDD{breakdown diode}] &  &
\\   \hline 
\multicolumn{4}{|c|}{\RRR{47-4-5} }
\\ \hline 
\begin{tikzpicture}[blue]
\useasboundingbox (-.5,0) rectangle (2.5,1);
\draw [circuit ee IEC] (0,.5) to [contact] (2,.5) ; 
\end{tikzpicture} 
&
\begin{tikzpicture}[blue]
\useasboundingbox (-.5,0) rectangle (2.5,1); 
\draw [circuit ee IEC] (0,.5) to [make contact]  (2,.5) ; 
\end{tikzpicture}
&
\begin{tikzpicture}[blue]
\useasboundingbox (-.5,0) rectangle (2.5,1);
\draw [circuit ee IEC] (0,.5) to [break contact] (2,.5) ; 
\end{tikzpicture}
&

\\   \hline 
[\RDD{contact}] &[\RDD{make contact}] & [\RDD{break contact}] & 
\\   \hline 
\end{tabular}

\bigskip


%\subsection{Alternate appearance}

\begin{tabular}{|c|c|c|} \hline
\multicolumn{3}{|c|}{\textbf{\TFRGB{Autre apparence}{Alternate appearance }}  }
\\ \hline   
\multicolumn{3}{|l|}{\BS{draw} [circuit ee IEC,\rouge{set resistor graphic=var resistor IEC graphic} ]  }\\
\multicolumn{3}{|l|}{(0,0.5) to [resistor] (2,0.5) ;  }\\ 
\hline 
\begin{tikzpicture}[blue]
\useasboundingbox  (-.5,0) rectangle (2.5,1); 
\draw [circuit ee IEC,set resistor graphic=var resistor IEC graphic ] (0,.5) to [resistor] (2,.5) ; 
\end{tikzpicture}
&
\begin{tikzpicture}[blue]
\useasboundingbox  (-.5,0) rectangle (2.5,1); 
\draw [circuit ee IEC,set inductor graphic=var inductor IEC graphic ] (0,.5) to [inductor] (2,.5) ; 
\end{tikzpicture} 
&  
\begin{tikzpicture}[blue]
\useasboundingbox (-.5,0) rectangle (2.5,1);
\draw [circuit ee IEC,set diode graphic=var diode IEC graphic ] (0,.5) to [diode] (2,.5) ; 
\end{tikzpicture}
\\   \hline 
resistor & inductor &  diode 
\\   \hline
\begin{tikzpicture}[blue]
\useasboundingbox  (-.5,0) rectangle (2.5,1); 
\draw [circuit ee IEC,set Zener diode graphic=var Zener diode IEC graphic ] (0,.5) to [Zener diode] (2,.5) ; 
\end{tikzpicture}
&
\begin{tikzpicture}[blue]
\useasboundingbox  (-.5,0) rectangle (2.5,1);
\draw [circuit ee IEC,set Schottky diode graphic=var Schottky diode IEC graphic ] (0,.5) to [Schottky diode] (2,.5) ; 
\end{tikzpicture}
&
\begin{tikzpicture}[blue]
\useasboundingbox  (-.5,0) rectangle (2.5,1);
\draw [circuit ee IEC,set tunnel diode graphic=var tunnel diode IEC graphic ] (0,.5) to [tunnel diode] (2,.5) ; 
\end{tikzpicture}

\\   \hline 
Zener diode & Schottky diode & tunnel diode 
\\   \hline
 
\begin{tikzpicture}[blue]
\useasboundingbox  (-.5,0) rectangle (2.5,1); 
\draw [circuit ee IEC,set backward diode graphic=var backward diode IEC graphic ] (0,.5) to [backward diode] (2,.5) ; 
\end{tikzpicture}
&
\begin{tikzpicture}[blue]
\useasboundingbox  (-.5,0) rectangle (2.5,1);
\draw [circuit ee IEC,set breakdown diode graphic=var breakdown diode IEC graphic ] (0,.5) to [breakdown diode] (2,.5) ; 
\end{tikzpicture}
&
\begin{tikzpicture}[blue]
\useasboundingbox  (-.5,0) rectangle (2.5,1);
\draw [circuit ee IEC,set make contact graphic=var make contact IEC graphic ] (0,.5) to [make contact] (2,.5) ; 
\end{tikzpicture}
\\   \hline 
backward diode & breakdown diode & make contact 
\\   \hline 
\end{tabular}


\bigskip

%\subsection{Symbol Size}
%\begin{center}
%\RRR{47-2-1}
%\end{center}


\begin{tabular}{|c|c|c|c|c|} \hline
\multicolumn{5}{|c|}{ \textbf{\TFRGB{Taille des symboles}{Symbol Size}}}\\
\multicolumn{5}{|c|}{\RRR{47-2-1}}
\\ \hline 
\multicolumn{5}{|c|}{\BS{draw}  [circuit ee IEC]  (0,0.5) to [diode,\RDD{large circuit symbols}] (2,0.5) ;  }\\ 
\hline 
\begin{tikzpicture}[blue]
\useasboundingbox  (0,0) rectangle (2,1); 
\draw [circuit ee IEC] (0,.5) to [diode,huge circuit symbols] (2,.5) ; 
\end{tikzpicture}
&
\begin{tikzpicture}[blue]
\useasboundingbox  (0,0) rectangle (2,1); 
\draw [circuit ee IEC](0,.5) to [diode,large circuit symbols] (2,.5) ; 
\end{tikzpicture}
&
\begin{tikzpicture}[blue]
\useasboundingbox  (0,0) rectangle (2,1); 
\draw [circuit ee IEC](0,.5) to [diode,medium circuit symbols] (2,.5) ; 
\end{tikzpicture}
&
\begin{tikzpicture}[blue]
\useasboundingbox  (0,0) rectangle (2,1); 
\draw [circuit ee IEC] (0,.5) to [diode,small circuit symbols] (2,.5) ; 
\end{tikzpicture}
&
\begin{tikzpicture}[blue]
\useasboundingbox  (0,0) rectangle (2,1); 
\draw [circuit ee IEC] (0,.5) to [diode,tiny circuit symbols] (2,.5) ; 
\end{tikzpicture}
\\   \hline
\RDD{huge circuit symbols} & \RDD{large circuit symbols} & \RDD{medium circuit symbols} & \RDD{small circuit symbols} & \RDD{tiny circuit symbols}
\\   \hline
(10pt) & (8pt) & (7pt)& (6pt)& (5pt)
\\   \hline
\end{tabular}


\bigskip\begin{tabular}{|c|c|c|} \hline 
\multicolumn{3}{|c|}{\BS{draw}  [circuit ee IEC,\RDD{circuit symbol unit}=14pt] (0,0.5) to [diode] (2,0.5) ;  }\\ 
\hline  
\begin{tikzpicture}[blue]
\useasboundingbox  (0,0) rectangle (2,1); 
\draw [circuit ee IEC,circuit symbol unit=14pt] (0,.5) to [diode] (2,.5) ;
\end{tikzpicture}
&  
\begin{tikzpicture}[blue]
\useasboundingbox  (0,0) rectangle (2,1); 
\draw [circuit ee IEC,{circuit symbol size=width 3 height 1}] (0,.5) to [diode] (2,.5) ;
\end{tikzpicture}
&  
\begin{tikzpicture}[blue]
\useasboundingbox  (0,0) rectangle (2,1); 
\node[circuit ee IEC,circuit symbol size=width 1 height 5]at (1,.5) [diode] {} ; 
\end{tikzpicture}
\\ \hline  
\RDD{circuit symbol unit}=14pt & \RDD{circuit symbol size}=width 3 height 1  & \RDD{circuit symbol size}=width 1 height 5 \\
&
\multicolumn{2}{|c|}{ \DW{} }
\\ \hline 
\end{tabular} 

%\subsection{Declaring New Symbols}
\bigskip

\begin{tabular}{|c|c|c|c|} \hline 
\multicolumn{4}{|c|}{\textbf{\TFRGB{Création de nouveaux symboles}{Declaring New Symbols }}  }
\\ % \hline
\multicolumn{4}{|c|}{\RRR{47-2-2}  }
\\ \hline  
\begin{tikzpicture}
[baseline=0pt,circuit declare symbol=xxx,
set xxx graphic={draw,shape=rectangle,minimum size=5mm}]
\useasboundingbox (0,0) rectangle (3.2,1);
\node [xxx] at (0.5,0.5) {};
\draw[circuit ee IEC]  (1,0.5) to [xxx] (3,0.5) ;
\end{tikzpicture}
& 
\multicolumn{3}{|c|}{ 
\parbox[B]{10cm}{
\BS{begin}\AC{tikzpicture}
[\RDD{circuit declare symbol}={\color{blue} xxx},  \\
\rouge{set xxx graphic}=\AC{draw,\rouge{shape}=rectangle,minimum size=5mm}]  \\
\BS{node} [{\color{blue} xxx}] at (.5,.5) {}; \\
\BS{draw}[circuit ee IEC]  (1,.5) to [{\color{blue} xxx}] (3,.5) ; \\
\BS{end}\AC{tikzpicture}} } 
\\ \hline  
  
\begin{tikzpicture}
[circuit declare symbol=xxx,
set xxx graphic={draw,shape=circle,minimum size=5mm}] 
\useasboundingbox (0,0) rectangle (3.2,1);
\node [xxx] at (.5,.5) {};
\draw[circuit ee IEC]  (1,.5) to [xxx] (3,.5) ;
\end{tikzpicture}
&  
\begin{tikzpicture}
[circuit declare symbol=xxx,
set xxx graphic={draw,shape=dart,minimum size=5mm}] 
\useasboundingbox (0,0) rectangle (3.2,1);
\node [xxx] at (.5,.5) {};
\draw[circuit ee IEC]  (1,.5) to [xxx] (3,.5) ;
\end{tikzpicture}
&  
\begin{tikzpicture}
[circuit declare symbol=xxx,
set xxx graphic={draw,shape=star,minimum size=5mm}] 
\useasboundingbox (0,0) rectangle (3.2,1);
\node [xxx] at (.5,.5) {};
\draw[circuit ee IEC]  (1,.5) to [xxx] (3,.5) ;
\end{tikzpicture}
&  
\begin{tikzpicture}
[circuit declare symbol=xxx,
set xxx graphic={draw,shape=forbidden sign,minimum size=5mm}] 
\useasboundingbox (0,0) rectangle (3.2,1);
\node [xxx] at (.5,.5) {};
\draw[circuit ee IEC]  (1,.5) to [xxx] (3,.5) ;
\end{tikzpicture}
\\ \hline 
\RDD{shape}=circle & \RDD{shape}=dart & \RDD{shape}=star &  \RDD{shape}=forbidden sign \\ 
\hline 
\multicolumn{4}{|c|}{ \textbf{voir les \og different shape libraries \fg }{see the different shape libraries } }
\\ \hline 
\end{tabular} 

\bigskip
%\subsection{placement}


\begin{tabular}{|c|} \hline 
\textbf{\TFRGB{Placement des symboles sur un chemin}{Placement of symbol on a path }} 
\\ \hline  
\BS{draw} [circuit ee IEC] (0,0.5) to  [contact=\AC{\RDD{at start}},make contact=\AC{\RDD{very near start}},voltage source=\AC{\RDD{near start}},\\ resistor,
bulb=\AC{\RDD{near end}},
bulb=\AC{\RDD{very near end}},contact=\AC{\RDD{at end}}] (12,0.5) ;
\\ \hline  
\begin{tikzpicture}[blue]
\useasboundingbox (-.5,0) rectangle (12.5,1);
\draw [circuit ee IEC] (0,.5) to  [contact={at start},make contact={very near start},voltage source={near start},resistor,
bulb={near end},bulb={very near end},contact={at end}] (12,.5) ; 
\end{tikzpicture}
\\ \hline 
%\end{tabular}
%
%\bigskip
%\begin{tabular}{|c|} \hline  
\BS{draw} [circuit ee IEC] (0,0.5) to  [contact=\{ \RDDD{pos=0}{} \},make contact=\{\rouge{pos=0.2}{}\},voltage source=\{\RDDD{pos=0.3}{} \},    \\ 
resistor=\{ \RDDD{pos=0.5}{} \},
bulb=\{\RDDD{pos=0.75} {} \},contact=\{\RDD{pos} \rouge{=1}{} \}] (12,0.5) ;
\\ \hline  
\begin{tikzpicture}[blue]
\useasboundingbox (-.5,0) rectangle (12.5,1);
\draw [circuit ee IEC] (0,.5) to  [contact={pos=0},make contact={pos=0.2},voltage source={pos=0.3},resistor={pos=0.5},
bulb={pos=0.75},contact={pos=1}] (12,.5) ; 
\end{tikzpicture}
\\ \hline 
\end{tabular}

%===================
%\subsection{Orientation}

\bigskip

\begin{tabular}{|c|c|c|c|} \hline
\multicolumn{4}{|c|}{\textbf{\TFRGB{Orientation des symboles}{Symbol orientation }}  } \\
\multicolumn{4}{|c|}{ \RRR{47-2-3} }
\\ \hline   
\multicolumn{4}{|c|}{\BS{node}  [circuit ee IEC] at (1,.5) [diode,\RDD{point up}] \AC{} ;  }\\ 
\hline 
\begin{tikzpicture}[blue]
\useasboundingbox  (-.5,0) rectangle (2.5,1);
\node[circuit ee IEC] at (1,.5) [diode,point up] {} ; 
\end{tikzpicture}
&
\begin{tikzpicture}[blue]
\useasboundingbox  (-.5,0) rectangle (2.5,1);
\node[circuit ee IEC] at (1,.5) [diode,point down] {} ; 
\end{tikzpicture}
&
\begin{tikzpicture}[blue]
\useasboundingbox  (-.5,0) rectangle (2.5,1);
\node[circuit ee IEC] at (1,.5) [diode,point left] {} ; 
\end{tikzpicture}
&
\begin{tikzpicture}[blue]
\useasboundingbox  (-.5,0) rectangle (2.5,1);
\node[circuit ee IEC] at (1,.5) [diode,point right] {} ; 
\end{tikzpicture}
\\   \hline 
[diode,\RDD{point up}] & 
[diode,\RDD{point down}] & [diode,\RDD{point left}] & [diode,\RDD{point right}]
\\   \hline
\end{tabular}

\bigskip
\begin{tabular}{|c|c|} \hline 
\multicolumn{2}{|c|}{\textbf{\TFRGB{Orientation automatique}{Automatic orientation }}  } 

\\ \hline    
\begin{tikzpicture}[baseline=0pt,blue]
\useasboundingbox (-.5,-1.5) rectangle (2.5,1.5);
\draw [circuit ee IEC] (0,0) to  [voltage source] (1,1) to [resistor] (2,0) to 
[bulb] (1,-1) to [diode]  (0,0) ; 
\end{tikzpicture} 
 &  
\parbox[c]{10cm}{\BS{draw} [circuit ee IEC] (0,0) \\ to  [voltage source] (1,1) \\ to [resistor] (2,0) \\to 
[bulb] (1,-1) \\to [diode]  (0,0) ;} 
 \\ \hline 
\end{tabular} 

\bigskip

%===================
%\subsection{Current Directions}
\subsection{Annotations}
%\begin{center}
%\RRR{47-4-2}
%\end{center}

\noindent

\begin{tabular}{|c|c|} \hline 
\multicolumn{2}{|c|}{ \textbf{\TFRGB{Sens du courant}{Indicating Current Directions}}  } \\
\multicolumn{2}{|c|}{\RRR{47-4-2} }
\\ \hline 
\multicolumn{2}{|c|}{\BS{draw}  [circuit ee IEC] (0,0.5) to [\RDD{current direction}]  (2,0.5) ;  }
\\ \hline 
\begin{tikzpicture}[blue]
\useasboundingbox  (-1,0) rectangle (3,1); 
\draw [circuit ee IEC] (0,.5) to[current direction] (2,.5); 
\end{tikzpicture}
& 
\begin{tikzpicture}[blue]
\useasboundingbox  (-1,0) rectangle (3,1);
\draw [circuit ee IEC] (0,.5) to[current direction'] (2,.5); 
\end{tikzpicture} 
\\   \hline
[\RDD{current direction}]  & [\RDD{current direction' } ]

\\   \hline

\end{tabular}

\bigskip

%\subsection{units}
%\begin{center}
%\RRR{47-4-6}
%\end{center}

 
\begin{tabular}{|c|c|c|c|c|} \hline
\multicolumn{5}{|c|}{\textbf{ \TFRGB{Unités disponibles}{Units available}}  } \\
\multicolumn{5}{|c|}{\RRR{47-4-6}   }
\\ \hline   
\multicolumn{5}{|c|}{\BS{node}  [draw,circuit ee IEC] at(1,.5) [\RDD{ampere}=5] \AC{}  }\\ 
\hline 
\begin{tikzpicture}[blue]
\draw[help lines,white] (-.5,0) rectangle (2.5,1);
\node [draw,circuit ee IEC] at(1,0.5) [ampere=5] {} ;  
\end{tikzpicture}
&
\begin{tikzpicture}[blue]
\draw[help lines,white] (-.5,0) rectangle (2.5,1);
\node [draw,circuit ee IEC] at(1,0.5) [volt=5] {} ;  
\end{tikzpicture}
&
\begin{tikzpicture}[blue]
\draw[help lines,white] (-.5,0) rectangle (2.5,1);
\node [draw,circuit ee IEC] at(1,0.5) [ohm=5] {} ;  
\end{tikzpicture}

&
\begin{tikzpicture}[blue]
\draw[help lines,white] (-.5,0) rectangle (2.5,1);
\node [draw,circuit ee IEC] at(1,.5) [siemens=5] {} ;  
\end{tikzpicture}
&
\begin{tikzpicture}[blue]
\draw[help lines,white] (-.5,0) rectangle (2.5,1); 
\node [draw,circuit ee IEC] at(1,.5) [henry=5] {} ;  
\end{tikzpicture}
\\   \hline
[\RDD{ampere}=5] & [\RDD{volt}=5] & [\RDD{ohm}=5]  \DW{} & [\RDD{siemens}=5] & [\RDD{henry}=5]
\\   \hline
\begin{tikzpicture}[blue]
\draw[help lines,white] (-.5,0) rectangle (2.5,1);
\node [draw,circuit ee IEC] at(1,.5) [farad=5] {} ;  
\end{tikzpicture}
&
\begin{tikzpicture}[blue]
\draw[help lines,white] (-.5,0) rectangle (2.5,1); 
\node [draw,circuit ee IEC] at(1,.5) [coulomb=5] {} ;  
\end{tikzpicture}
&
\begin{tikzpicture}[blue]
\draw[help lines,white] (-.5,0) rectangle (2.5,1);
\node [draw,circuit ee IEC] at(1,.5) [voltampere=5] {} ;  
\end{tikzpicture}
&
\begin{tikzpicture}[blue]
\draw[help lines,white] (-.5,0) rectangle (2.5,1); 
\node [draw,circuit ee IEC] at(1,.5) [watt=5] {} ;  
\end{tikzpicture}
&
\begin{tikzpicture}[blue]
\draw[help lines,white] (-.5,0) rectangle (2.5,1);
\node [draw,circuit ee IEC] at(1,.5) [hertz=5] {} ;  
\end{tikzpicture}
\\   \hline
[\RDD{farad}=5] & [\RDD{coulomb}=5] &  [\RDD{voltampere}=5] & [\RDD{watt}=5] & [\RDD{hertz}=5] 
\\   \hline
\begin{tikzpicture}[blue]
\draw[help lines,white] (-.5,0) rectangle (2.5,1); 
\node [draw,circuit ee IEC] at(1,.5)[ampere=5k]{} ;  
\end{tikzpicture}
&
\begin{tikzpicture}[blue]
\draw[help lines,white] (-.5,0) rectangle (2.5,1);
\node [draw,circuit ee IEC] at(1,.5) [ampere=5m] {} ;  
\end{tikzpicture}
&
\begin{tikzpicture}[blue]
\draw[help lines,white] (-.5,0) rectangle (2.5,1);
\node [draw,circuit ee IEC] at(1,.5) [ampere=5\mu] {} ;  
\end{tikzpicture}
&
\begin{tikzpicture}[blue]
\draw[help lines,white] (-.5,0) rectangle (2.5,1); 
\node [draw,circuit ee IEC] at(1,.5) [watt=5k] {} ;  
\end{tikzpicture}
&
\begin{tikzpicture}[blue]
\draw[help lines,white] (-.5,0) rectangle (2.5,1);
\node [draw,circuit ee IEC] at(1,.5) [watt=5M] {} ;  
\end{tikzpicture}
\\   \hline
[ampere=\rouge{5k}] & [ampere=\rouge{5m}] &  [ampere=\rouge{5\BS{mu}}] & [watt=\rouge{5k}] & [watt=\rouge{5M}] 
\\   \hline
\end{tabular}

\bigskip

%\subsection{créer sa propre unité }

\begin{tabular}{|c|} \hline 
\textbf{ \TFRGB{créer sa propre unité}{Declare unit}} \\  
\RRR{47-2-4}
\\ \hline   
\BS{tikz}[circuit ee IEC,\RDD{circuit declare unit}=\AC{{\color{blue} xxx}}\AC{ Unit}] \\
\BS{draw} (0,0) to[resistor=\AC{{\color{blue} xxx}' sloped=3}] (3,2) to [resistor=\AC{{\color{blue} xxx}= 10\BS{mu}}] (5,2) to [resistor=\AC{{\color{blue} xxx}= 10M}] (7,0);
\\ \hline  
%\begin{tikzpicture}[circuit ee IEC,circuit declare unit={XXX}{ Unit}]
\tikz[circuit ee IEC,circuit declare unit={XXX}{ Unit}]
\draw (0,0) to[resistor={XXX' sloped=3}] (3,2) to [resistor={XXX= 10\mu}] (5,2) to [resistor={XXX= 10M}] (7,0);
%\end{tikzpicture}
\\ \hline 
\end{tabular} 

\bigskip
%\subsection{Annotations}

%\begin{center}
%\RRR{47-4-7}
%\end{center}

\begin{tabular}{|c|c|c|c|} \hline 
\multicolumn{4}{|c|}{ \textbf{\TFRGB{Annotations}{Annotations}}  } \\
\multicolumn{4}{|c|}{\RRR{47-4-7}}
\\ \hline 
\multicolumn{4}{|c|}{\BS{draw}  [circuit ee IEC] (0,0.5) to [resistor=\RDD{light emitting}] (2,0.5) ;  }\\ 
\hline 
\begin{tikzpicture}[blue]
\useasboundingbox (-.5,0) rectangle (2.5,1); 
\draw [circuit ee IEC] (0,.5) to [resistor=light emitting] (2,.5) ; 
\end{tikzpicture} 
&
\begin{tikzpicture}[blue]
\useasboundingbox  (-.5,0) rectangle (2.5,1); 
\draw [circuit ee IEC] (0,.5) to [resistor=light dependent]  (2,.5) ; 
\end{tikzpicture} 
&
\begin{tikzpicture}[blue]
\useasboundingbox (-.5,0) rectangle (2.5,1); 
\draw [circuit ee IEC] (0,.5) to [resistor=direction info]  (2,.5) ;  
\end{tikzpicture}  
&
\begin{tikzpicture}[blue]
\useasboundingbox  (-.5,0) rectangle (2.5,1);
\draw [circuit ee IEC] (0,.5) to [resistor=adjustable]  (2,.5) ;  
\end{tikzpicture} 
\\   \hline 
[resistor=\RDD{light emitting}] & [resistor=\RDD{light dependent}]  & [resistor=\RDD{direction info}]  & [resistor=\RDD{adjustable}]
\\   \hline 
\begin{tikzpicture}[blue]
\useasboundingbox  (-.5,0) rectangle (2.5,1); 
\draw [circuit ee IEC] (0,.5) to [diode=light emitting] (2,.5) ; 
\end{tikzpicture} 
&
\begin{tikzpicture}[blue]
\useasboundingbox  (-.5,0) rectangle (2.5,1);
\draw [circuit ee IEC] (0,.5) to [diode=light dependent]  (2,.5) ; 
\end{tikzpicture} 
&
\begin{tikzpicture}[blue]
\useasboundingbox  (-.5,0) rectangle (2.5,1);
\draw [circuit ee IEC] (0,.5) to [diode=direction info]  (2,.5) ;  
\end{tikzpicture}  
&
\begin{tikzpicture}[blue]
\useasboundingbox  (-.5,0) rectangle (2.5,1);
\draw [circuit ee IEC] (0,.5) to [diode=adjustable]  (2,.5) ;  
\end{tikzpicture} 
\\   \hline 
[diode=\RDD{light emitting}] & [diode=\RDD{light dependent}]  & [diode=\RDD{direction info}]  & [diode=\RDD{adjustable}]
\\   \hline 
\begin{tikzpicture}[blue]
\useasboundingbox  (-.5,0) rectangle (2.5,1);
\draw [circuit ee IEC] (0,.5) to [diode=light emitting'] (2,.5) ; 
\end{tikzpicture} 
&
\begin{tikzpicture}[blue]
\useasboundingbox  (-.5,0) rectangle (2.5,1); 
\draw [circuit ee IEC] (0,.5) to [diode=light dependent']  (2,.5) ; 
\end{tikzpicture} 
&
\begin{tikzpicture}[blue]
\useasboundingbox  (-.5,0) rectangle (2.5,1);
\draw [circuit ee IEC] (0,.5) to [diode=direction info']  (2,.5) ;  
\end{tikzpicture}  
&
\begin{tikzpicture}[blue]
\draw[help lines,white] (-.5,0) rectangle (2.5,1);
\draw [circuit ee IEC] (0,.5) to [diode=adjustable']  (2,.5) ;  
\end{tikzpicture} 
\\   \hline 
[diode=\RDD{light emitting'}] & [diode=\RDD{light dependent'}]  & [diode=\RDD{direction info'}]  & [diode=\RDD{adjustable'}]
\\   \hline 
\end{tabular}



\bigskip

%\subsection{Position}
%
%\begin{center}
%\RRR{47-2-4}
%\end{center}

\begin{tabular}{|c|c|} \hline 
\multicolumn{2}{|c|}{ \textbf{\TFRGB{Position des unités}{Units position}}  } \\
\multicolumn{2}{|c|}{  \RRR{47-2-4}}
\\ \hline 
\multicolumn{2}{|c|}{\BS{draw} [circuit ee IEC] (0,0) to [capacitor=\AC{farad=5\BS{mu}}]  (2,2)  ;  }\\ 
\hline 
\begin{tikzpicture}[blue]
\useasboundingbox   (0,-.5) rectangle (2,2.5); 
\draw [circuit ee IEC] (0,0) to [capacitor={farad=5\mu}]  (2,2) ; 
\end{tikzpicture}
&
\begin{tikzpicture}[blue]
\useasboundingbox   (0,-.5) rectangle (2,2.5);  
\draw [circuit ee IEC] (0,0) to [capacitor={farad'=5\mu}]  (2,2) ; 
\end{tikzpicture}
\\ \hline
[capacitor=\rouge{\AC{farad=5\BS{mu}}}] & [capacitor=\rouge{\AC{farad'=5\BS{mu}}}]
\\ \hline
\begin{tikzpicture}[blue]
\useasboundingbox   (0,-.5) rectangle (2,2.5); 
\draw [circuit ee IEC] (0,0) to [capacitor={farad sloped=5\mu}]  (2,2) ; 
\end{tikzpicture}
&
\begin{tikzpicture}[blue]
\useasboundingbox   (0,-.5) rectangle (2,2.5); 
\draw [circuit ee IEC] (0,0) to [capacitor={farad' sloped=5\mu}]  (2,2) ; 
\end{tikzpicture}
\\   \hline
[capacitor=\rouge{\AC{farad sloped=5\BS{mu}}}] & [capacitor=\rouge{\AC{farad' sloped=5\BS{mu}}}]
\\   \hline
\end{tabular}

\bigskip
%\subsection{Info Labels}

\begin{tabular}{|c|c|c|} \hline 
\multicolumn{3}{|c|}{ \textbf{ \TFRGB{Informations}{Info Labels}}  }\\ 
\multicolumn{3}{|c|}{ \RRR{47-2-4}  }\\ 
\hline 

\multicolumn{3}{|c|}{\BS{draw} [circuit ee IEC] (0,0.5) to [diode=\AC{light emitting=\AC{\RDD{info}=D1}}] (2,0.5) ;  }\\ 
\hline 

\begin{tikzpicture}[blue]
\useasboundingbox   (0,-1) rectangle (2,2); 
\draw [circuit ee IEC] (0,.5) to [diode={light emitting={info=D1}} ]  (2,.5) ; 
\end{tikzpicture}
&
\begin{tikzpicture}[blue]
\useasboundingbox   (0,-1) rectangle (2,2); 
\draw [circuit ee IEC] (0,.5) to [diode={light emitting={info'=D2}}]  (2,.5) ; 
\end{tikzpicture}
&
\begin{tikzpicture}[blue]
\useasboundingbox   (0,-1) rectangle (2,2);
\draw [circuit ee IEC] (0,.5) to [diode={light emitting,info'=D3}] (2,.5) ; 
\end{tikzpicture}
\\   \hline 
[diode=\AC{light emitting=\AC{\RDD{info}=D1}} ] & 
[diode=\AC{light emitting=\AC{\RDD{info'}=D2}} ] &
[diode=\AC{light emitting,\RDD{info'}=D3}]
\\   \hline
\end{tabular}

\bigskip

\begin{tabular}{|c|c|} \hline  
\TFRGB{sur un noeud}{On a node} & \TFRGB{sur un chemin}{On a path}

\\ \hline  
\begin{tikzpicture}[blue]
\node[circuit ee IEC] at (1,.5) [resistor,info=$3\Omega$,info'=R1] {} ; 
\end{tikzpicture}
&  
\begin{tikzpicture}[blue]
\useasboundingbox (-.2,-.2) grid (2.2,1.2);
\draw [circuit ee IEC] (0,.5) to [resistor={info=$3\Omega$,info'=R1}] (2,.5) ; 
\end{tikzpicture}
\\ \hline 
[resistor,\RDD{info}=\$3\BS{Omega}\$,\RDD{info'}=R1] &[resistor=\AC{\RDD{info}=\$3\BS{Omega}\$,\RDD{info'}=R1}] 
\\ \hline 
\end{tabular}

\bigskip

\begin{tabular}{|c|c|} \hline  
\begin{tikzpicture}[blue,circuit ee IEC]
\node [resistor,info=center:$3\Omega$] {};
\end{tikzpicture}
&  
\begin{tikzpicture}[blue,circuit ee IEC]
\node [resistor,point up,info=center:$3\Omega$] {};
%\node [resistor,point up,info=center:$R_1$] ;
\end{tikzpicture}
\\ \hline [resistor,point up,info=\RDD{center}:\$3\BS{Omega}\$] & 
[resistor,point up,info=\RDD{center}:\$3\BS{Omega}\$] \\ 
\hline 
\end{tabular} 


\bigskip



\begin{tabular}{|c|c||c|c|}  \hline 
\multicolumn{2}{|c||}{ \BS{node}  [voltage source,\RDD{direction info}=\AC{volt=10}] \AC{}  } & \multicolumn{2}{|c|}{ \BS{node}  [voltage source,\RDD{direction info'}=\AC{volt=10}] \AC{}}
\\ \hline 
\begin{tikzpicture}[blue,circuit ee IEC]
\useasboundingbox (-2,-1.2) grid (2,1.2);
\node  [voltage source,direction info={volt=10}] {};
\end{tikzpicture}
&  
\begin{tikzpicture}[blue,circuit ee IEC]
\useasboundingbox (-2,-1.2) grid (2,1.2);
\node  [voltage source,direction info={->,volt'=10}] {};
\end{tikzpicture}
&  
\begin{tikzpicture}[blue,circuit ee IEC]
\useasboundingbox (-2,-1.2) grid (2,1.2);
\node [voltage source,direction info'={volt=10}] {};
\end{tikzpicture}
&  
\begin{tikzpicture}[blue,circuit ee IEC]
\useasboundingbox (-2,-1.2) grid (2,1.2);
\node [voltage source,direction info'={volt'=10}] {};
\end{tikzpicture}
\\ \hline 
\AC{volt=10} &   \AC{volt'=10} & \AC{volt=10}& \AC{volt'=10}  \\
\TFRGB{ou}{or} \AC{->,volt=10} &  \TFRGB{ou}{or}  \AC{->,volt'=10} &\TFRGB{ou}{or} \AC{->,volt=10} & \TFRGB{ou}{or}  \AC{->,volt'=10} 
\\ \hline 
\begin{tikzpicture}[blue,circuit ee IEC]
\useasboundingbox (-2,-1.2) grid (2,1.2);
\node [voltage source,direction info={<-,volt=10}] {};
\end{tikzpicture} 
&  
\begin{tikzpicture}[blue,circuit ee IEC]
\useasboundingbox (-2,-1.2) grid (2,1.2);
\node   [voltage source,direction info'={<-,volt=10}] {};
\end{tikzpicture}
&  
\begin{tikzpicture}[blue,circuit ee IEC]
\useasboundingbox (-2,-1.2) grid (2,1.2);
\node   [voltage source,direction info'={<-,volt=10}] {};
\end{tikzpicture}
&  
\begin{tikzpicture}[blue,circuit ee IEC]
\useasboundingbox (-2,-1.2) grid (2,1.2);
\node   [voltage source,direction info'={<-,volt'=10}] {};
\end{tikzpicture}
\\ \hline 
\AC{<-,volt=10} & \AC{<-,volt=10} & \AC{<-,volt=10} & \AC{<-,volt'=10}\\ 
\hline 
\end{tabular} 


%===================
\bigskip

\begin{tabular}{|c|c|}\hline 
\multicolumn{2}{|c|}{  \textbf{\TFRGB{Créer sa propre annotation}{Declare annotation}} } \\
\multicolumn{2}{|c|}{ \RRR{47-2-5} }
\\ \hline  
\tikzset{circuit declare annotation={XXX}{9pt}
{ (-0.5cm,0.5cm) edge[to path={- -(0pt,2pt) - - (8pt,8pt)}] ()} }

\tikz[blue,circuit ee IEC]
\draw (0,0) to [resistor={XXX}] (3,0);
&  
\parbox[b]{12cm}{
\BS{tikzset}\AC{circuit \RDD{declare annotation}=\AC{\blll{XXX}}\AC{9pt} \\
\hspace{1cm}  \AC{ (-0.5cm,0.5cm) edge[to path=\AC{- -(0pt,2pt) - - (8pt,8pt)}] ()} }  \\
\BS{tikz}[blue,circuit ee IEC]
\BS{draw} (0,0) to [resistor={\blll{XXX}}] (3,0);
}
\\ \hline  
\tikzset{circuit declare annotation={xxx}{9pt}
{ (-.5cm,.5cm) edge[to path={--(0pt,2pt) -- (8pt,8pt)}] ()} }

\tikz[circuit ee IEC]
\draw (0,0) to [resistor={xxx={info=abc}}] (3,0);
&  
\parbox[b]{12cm}{
\BS{tikzset}\AC{circuit declare annotation=\AC{xxx}\{ \rouge{9pt} \} \}\\
\hspace{1cm}  \AC{ (-0.5cm,0.5cm) edge[to path=\AC{- -(0pt,2pt) - - (8pt,8pt)}] ()} }  \\
\BS{tikz}[blue,circuit ee IEC]
\BS{draw} (0,0) to [resistor=\AC{xxx\blll{=\AC{info=abc}}}] (3,0);
}
\\ \hline 
\tikzset{circuit declare annotation={xxx}{1cm}
{ (-.5,.5) edge[to path={--(0pt,2pt) -- (8pt,8pt)}] ()} }


\tikz[circuit ee IEC]
\draw (0,0) to [resistor={xxx={info=abc}}] (3,0);
&  
\parbox[b]{12cm}{
\BS{tikzset}\{circuit declare annotation=\AC{xxx}\{\rouge{1cm} \} \} \\
\hspace{1cm}  \AC{ (-0.5,0.5) edge[to path=\AC{- -(0pt,2pt) - - (8pt,8pt)}] ()} \}  \\
\BS{tikz}[blue,circuit ee IEC]
\BS{draw} (0,0) to [resistor=\AC{xxx\blll{=\AC{info=abc}}}] (3,0);
}
\\ \hline 
\end{tabular}


\bigskip
%47.2.6 Theming Symbols

\begin{tabular}{|c|}\hline 
\textbf{ \TFRGB{Style des symboles}{Theming Symbols
}}\\
 \RRR{47-2-6} 
 
\\  \hline 
\BS{draw}[\RDD{circuit symbol lines/.style}=\AC{draw,red,very thick}] (0,0) \\to [capacitor=\AC{near start},resistor,
make contact=\AC{near end}] (5,0);
\\ \hline  
\begin{tikzpicture}[blue,circuit ee IEC]
\useasboundingbox  (-1,-1) rectangle (6,1);
\draw[circuit symbol lines/.style={draw,red,very thick}] (0,0) to [capacitor={near start},resistor,
make contact={near end}] (5,0);
\end{tikzpicture}
 \\ \hline 

\hline  
\BS{draw}[\RDD{circuit symbol wires/.style}=\AC{draw,red,very thick}] (0,0) \\to [capacitor=\AC{near start},resistor,
make contact=\AC{near end}] (5,0);
\\ \hline  
\begin{tikzpicture}[blue,circuit ee IEC]
\useasboundingbox  (-1,-1) rectangle (6,1);
\draw[circuit symbol wires/.style={draw,red,very thick}] (0,0) to [capacitor={near start},resistor,
make contact={near end}] (5,0);
\end{tikzpicture}
 \\ \hline 

\hline  
\BS{draw}[\RDD{circuit symbol open/.style}=\AC{thick,draw,red,fill=yellow}] (0,0) \\to [capacitor=\AC{near start},resistor,
make contact=\AC{near end}] (5,0);
\\ \hline  
\begin{tikzpicture}[blue,circuit ee IEC]
\useasboundingbox  (-1,-1) rectangle (6,1);
\draw[circuit symbol open/.style={thick,draw,red,fill=yellow}] (0,0) to [capacitor={near start},resistor,
make contact={near end}] (5,0);
\end{tikzpicture}
 \\ \hline 
\end{tabular}

\bigskip

\begin{tabular}{|c|c|} \hline 
\multicolumn{2}{|l|}{\BS{tikz}[blue,circuit ee IEC,\RDD{every info/.style}=red] }\\
\multicolumn{2}{|l|}{\BS{draw} (0,0) to[resistor=\AC{info=\AC{\$3\BS{Omega}\$},info'=\AC{\$R\_1\$}}] (3,0) }\\
\multicolumn{2}{|l|}{to[resistor=\AC{info=\{\$4\BS{Omega}\$\},info'=\AC{\$R\_2\$}}] (3,2); }\\ 
\hline  
 
\tikz[blue,circuit ee IEC,every info/.style=red]
\draw (0,0) to[resistor={info={$3\Omega$},info'={$R_1$}}] (3,0)
to[resistor={info={$4\Omega$},info'={$R_2$}}] (3,2);

&  
\tikz[blue,circuit ee IEC,every info/.style={font=\tiny}]
\draw (0,0) to[resistor={info={$3\Omega$},info'={$R_1$}}] (3,0)
to[resistor={info={$4\Omega$},info'={$R_2$}}] (3,2);
\\ \hline  
\RDD{every info/.style}=red
&  
\RDD{every info/.style}=\AC{font=\BS{tiny}}
\\ \hline 
\end{tabular} 





\bigskip
%\subsection{Example}

\SbSSCT{Exemple}{Example}

\begin{tabular}{|c|c|} \hline
\multicolumn{2}{|c|}{ \textbf{\TFRGB{3 méthodes pour le même schéma}{3 methods for the same circuit}}  }
\\ \hline 
\begin{tikzpicture}[blue,circuit ee IEC,baseline=0pt]
\useasboundingbox (-1,-1) rectangle (3,3); 
\draw (0,0) to [voltage source={direction info={->,volt=10}}]  (0,2)   to [resistor={info=center:$3 k\Omega$}] (2,2)   to [diode=light emitting]  ( 2,0)  to [make contact]  (0,0);
\end{tikzpicture}
&
\parbox[b]{10cm}{
\BS{begin}\AC{tikzpicture}[blue,circuit ee IEC] \\ 
\BS{draw} (0,0) \\
to [voltage source=\AC{direction info=\AC{->,volt=10}}]  (0,2) \\ 
to [resistor=\AC{info=center:\$3 k\BS{Omega}\$}] (2,2) \\ 
to [diode=light emitting]  ( 2,0) \\ 
to [make contact]  (0,0); \\
\BS{end}\AC{tikzpicture}

}
\\   \hline
\begin{tikzpicture}[blue,circuit ee IEC,baseline=0pt]
\useasboundingbox  (-1,-1) rectangle (3,3); 
\draw (0,0) to [voltage source={direction info={->,volt=10}}]  ++(up:2)   
to [resistor={info=center:$ 3 k\Omega$}] ++(right:2)  
 to [diode=light emitting]  ++(down:2)  
 to [make contact]  ++(left:2) ;
%\node at (0,1) [volt=10,point up] {};
%\node at (1,2) [ohm=10k] {}; 
\end{tikzpicture}
&
\parbox[b]{10cm}{
\BS{begin}\AC{tikzpicture}[blue,circuit ee IEC] \\ 
\BS{draw} (0,0) to [voltage source=\AC{direction info=\AC{->,volt=10}}] ++(up:2) \\ 
to [resistor=\AC{info=center:\$ 3 k\BS{Omega}\$}] ++(right:2) \\ 
to [diode=light emitting]  ++(down:2) \\ 
to [make contact]  ++(left:2) ; \\
%\BS{node} at (0,1) [volt=10,point up] \AC{}; \\
%\BS{node} at (1,2) [ohm=10k] \AC{}; \\
\BS{end}\AC{tikzpicture}

}
\\   \hline
\begin{tikzpicture}[blue,circuit ee IEC]
\useasboundingbox  (-1,-1) rectangle (3,3); 
\node (A) at (0,1) [voltage source,point up,volt=10]{} ;
\node  (B) at (1,2) [resistor,ohm=10k] {};
\node(C) at (2,1)  [diode=light emitting,point down] {} ;
\node (D) at  ( 1,0)   [make contact]  {}; 
\draw (A) |- (B) -| (C) |- (D) -|  (A); 
\end{tikzpicture}
&
\parbox[b]{10cm}{
\BS{begin}\AC{tikzpicture}[blue,circuit ee IEC] \\
\BS{node} (A) at (0,1) [voltage source,point up,volt=10]\AC{}; \\
\BS{node} (B) at (1,2) [resistor,ohm=10k] \AC{}; \\
\BS{node} (C) at (2,1)  [diode=light emitting,point down] \AC{} ; \\
\BS{node} (D) at  ( 1,0)   [make contact]  \AC{}; \\
\BS{draw} (A) |- (B) -| (C) |- (D) -|  (A); \\
\BS{end}\AC{tikzpicture}
}
\\   \hline
\end{tabular}
 









