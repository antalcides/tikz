
\TFRGB{Voir aussi le module  de géométrie page}{See also the package of geometry on page} \pageref{geom}

\SbSSCT{Création de n\oe uds multiples}{Multiples nodes creation}

\psset{linewidth=1pt,tickcolor=black!10}
\psset{unit=1cm,fillcolor=white,fillstyle=none,linecolor=blue}

\begin{tabular}{|l|l|} \hline  
\begin{psgraph}[axesstyle=none,xticksize=0 4,yticksize=0 4,subticks=0](0,0)(4,4){3cm}{3cm} 
\pnodes(3,1){A}(2,3){B} (1,2){C}
\psdots[dotstyle=*,linecolor=blue](A) \nput{45}{A}{A} 
\psdots[dotstyle=*,linecolor=blue](B) \nput{45}{B}{B} 
\psdots[dotstyle=*,linecolor=blue](C) \nput{135}{C}{C} 
\psline (A) (B) (C)
 \end{psgraph}
& 
\begin{psgraph}[axesstyle=none,xticksize=0 4,yticksize=0 4,subticks=0](0,0)(4,4){3cm}{3cm} 
\pnodes{A} (3,1) (2,3) (1,2)
\psdots[dotstyle=*,linecolor=blue](A0) \nput{45}{A0}{A0} 
\psdots[dotstyle=*,linecolor=blue](A1) \nput{45}{A1}{A1} 
\psdots[dotstyle=*,linecolor=blue](A2) \nput{135}{A2}{A2} 
\psline (A0) (A1) (A2)
 \end{psgraph}
\\ \hline 

\BSS{pnodes}(3,1)\AC{A}(2,3)\AC{B} (1,2)\AC{C} \BSI{pnodes}{pst-node} 
&
\BSS{pnodes}\AC{A} (3,1)(2,3) (1,2)  \\
\BS{psline} (A) (B) (C)
&  
\BS{psline} (A0) (A1) (A2)
\\ \hline 
\end{tabular} 

%-----------------------------------------------------------------------------
\SbSSCT{Positionement calculé de n\oe uds}{Node positions calculated }

\SbSbSSCT{Positions relatives avec psLNode}{Relative position width psLNode}

\begin{tabular}{|c|c|}\hline  
\begin{psgraph}[axesstyle=none,xticksize=0 2cm,yticksize=0 6,subticks=0](0,0)(6,2){6cm}{2cm}
\pnode(1,1){B}\pnode(5,1){C}
  \psLNode(B)(C){0.75}{A}
\psdots[dotstyle=*,linecolor=red](A)  \nput{45}{A}{A}%
\psdots[dotstyle=*,linecolor=blue](B) \nput{45}{B}{B} 
\psdots[dotstyle=*,linecolor=blue](C) \nput{45}{C}{C}%
 \end{psgraph}
&  
\begin{psgraph}[axesstyle=none,xticksize=0 2cm,yticksize=0 6cm,subticks=0](0,0)(6,2){6cm}{2cm}
\pnode(1,1){B}\pnode(5,1){C}
  \psLNode(B)(C){-0.25}{A}
\psdots[dotstyle=*,linecolor=red](A)  \nput{45}{A}{A}%
\psdots[dotstyle=*,linecolor=blue](B) \nput{45}{B}{B} 
\psdots[dotstyle=*,linecolor=blue](C) \nput{45}{C}{C}%
 \end{psgraph} 
\\ \hline  
\BSS{psLNode}(B)(C)\AC{{\red 0.75}}\AC{A} \BSI{pslNode}{pst-node} 
&  
\BSS{psLNode}(B)(C)\AC{{\red -0.25}}\AC{A}
\\ \hline 
\end{tabular}

\SbSbSSCT{Positions relatives avec midAB}{Relative position width midAB}

\begin{tabular}{|c|} \hline  
\begin{psgraph}[axesstyle=none,xticksize=0 2cm,yticksize=0 6cm,subticks=0](0,0)(6,2){6cm}{2cm}
\pnode(1,1){B}\pnode(5,1){C}
\midAB(B)(C){A}
\psdots[dotstyle=*,linecolor=red](A)  \nput{45}{A}{A}%
\psdots[dotstyle=*,linecolor=blue](B) \nput{45}{B}{B} 
\psdots[dotstyle=*,linecolor=blue](C) \nput{45}{C}{C}%
 \end{psgraph}
\\ \hline  
\BSS{midAB}(B)(C)\AC{A} \BSI{midAB}{pst-node} 
\\ \hline 
\end{tabular}

%-------------------------------
\SbSbSSCT{Positions  avec psLDNode}{Position width psLDNode}

\begin{tabular}{|c|} \hline  
\begin{psgraph}[axesstyle=none,xticksize=0 2cm,yticksize=0 6cm,subticks=0](0,0)(6,2){6cm}{2cm}
\pnode(1,1){B}\pnode(5,1){C}
\psLDNode(B)(C){1cm}{A} 
\psdots[dotstyle=*,linecolor=red](A)  \nput{45}{A}{A}%
\psdots[dotstyle=*,linecolor=blue](B) \nput{45}{B}{B} 
\psdots[dotstyle=*,linecolor=blue](C) \nput{45}{C}{C}%
\end{psgraph} 
\\ \hline  
\BSS{psLDNode}(B)(C)\AC{{\red 1cm}}\AC{A} \BSI{psLNDode}{pst-node}  
\\ \hline 
\end{tabular} 

%-------------------------------------

\SbSbSSCT{Positions relatives avec psLCNode}{Relative position width psLCNode}

\begin{tabular}{|c|c|c|} \hline  
\begin{psgraph}[axesstyle=none,xticksize=0 5,yticksize=0 5,subticks=0](0,0)(5,5){3cm}{3cm}
\psline{-D>}(4,1)
\psline{-D>}(1,2)
\psLCNode(4,1){1}(1,2){1}{A}
\psdots[dotstyle=*,linecolor=red](A) \nput{135}{A}{A} 
\psline[linestyle=dashed](4,1)(A)
\psline[linestyle=dashed](1,2)(A)
\end{psgraph}
&  
\begin{psgraph}[axesstyle=none,xticksize=0 5,yticksize=0 5,subticks=0](0,0)(5,5){3cm}{3cm}
\psline{-D>}(4,1)
\psline{-D>}(1,2)
\psLCNode(4,1){.5}(1,2){1}{A}
\psdots[dotstyle=*,linecolor=red](A) \nput{135}{A}{A} 
\psline[linestyle=dashed](2,.5)(A)
\psline[linestyle=dashed](1,2)(A)
\end{psgraph}
&  
\begin{psgraph}[axesstyle=none,xticksize=0 5,yticksize=0 6,subticks=0](0,0)(6,5){3cm}{3cm}
\psline[linestyle=dashed](1.5,3)
\psline{-D>}(4,1)
\psline{-D>}(1,2)
\psLCNode(4,1){1}(1,2){1.5}{A}
\psdots[dotstyle=*,linecolor=red](A) \nput{135}{A}{A} 
\psline[linestyle=dashed](4,1)(A)
\psline[linestyle=dashed](1.5,3)(A)
\end{psgraph}
\\ \hline  
\BSS{psLCNode}(4,1){\red \AC{1}}(1,2){\red \AC{1}}\AC{A} \BSI{psLCNode}{pst-node} & 
\BSS{psLCNode}(4,1){\red \AC{.5}}(1,2){\red \AC{1}}\AC{A} &
\BSS{psLCNode}(4,1){\red \AC{1}}(1,2){\red \AC{1.5}}\AC{A}  \\ 
\hline 
\end{tabular} 

%-----------------------------------

\SbSbSSCT{Positions relatives avec psLCNodeVar}{Relative position width psLCNodeVar}

\begin{tabular}{|c|c|} \hline  
\begin{psgraph}[axesstyle=none,xticksize=0 5,yticksize=0 4,subticks=0](0,0)(5,4){3cm}{3cm}
\psline{-D>}(4,1)
\psline{-D>}(1,2)
\psLCNodeVar(4,1)(1,2)(1,1){A}
\psdots[dotstyle=*,linecolor=red](A) \nput{135}{A}{A} 
\psline[linestyle=dashed](4,1)(A)
\psline[linestyle=dashed](1,2)(A)
\end{psgraph}
&
\begin{psgraph}[axesstyle=none,xticksize=0 4,yticksize=0 5,subticks=0](0,0)(5,4){3cm}{3cm}
\psline{-D>}(4,1)
\psline{-D>}(1,2)
\psLCNodeVar(4,1)(1,2)(0.5,1){A}
\psdots[dotstyle=*,linecolor=red](A) \nput{135}{A}{A} 
\psline[linestyle=dashed](2,.5)(A)
\psline[linestyle=dashed](1,2)(A)
\end{psgraph}
\\ \hline 
\BSS{psLCNodeVar}(4,1)(1,2)({\red 1,1})\AC{A} \BSI{psLCNodeVar}{pst-node}  &
\BSS{psLCNodeVar}(4,1)(1,2)({\red 0.5,1})\AC{A} 
\\ \hline  
\end{tabular} 

%----------------------------------

\SbSbSSCT{Positions relatives avec rhombus}{Relative position width rhombus}

\begin{tabular}{|c|c|} \hline  
\begin{psgraph}[axesstyle=none,xticksize=0 6,yticksize=0 6,subticks=0](0,0)(6,6){3cm}{3cm}
\pnodes(5,2){A}(2,4){B} 
\uput[-135](B){B}
\uput[-45](A){A}
\psdots[dotstyle=*,linecolor=blue](B)
\psdots[dotstyle=*,linecolor=blue](A)
\rhombus{2}(A)(B){C}{D}
\psdots[dotstyle=*,linecolor=red](C)
\psdots[dotstyle=*,linecolor=red](D)
\uput[45](C){C}
\uput[45](D){D}
\psline[linestyle=dotted ](A)(B)
\psline[linestyle=dotted ](A)(C)
\psline[linestyle=dotted ](A)(D)
\psline[linestyle=dotted ](B)(C)
\psline[linestyle=dotted ](B)(D)
\psarc[linestyle=dashed,linewidth=1pt](A){2}{80}{210}
\end{psgraph}
&  
\begin{psgraph}[axesstyle=none,xticksize=0 6,yticksize=0 6,subticks=0](0,0)(6,6){3cm}{3cm}
\pnodes(5,2){A}(2,4){B} 
\uput[-135](B){B}
\uput[-45](A){A}
\psdots[dotstyle=*,linecolor=blue](B)
\psdots[dotstyle=*,linecolor=blue](A)
\rhombus{3}(A)(B){X}{Y}
\psdots[dotstyle=*,linecolor=red](X)
\psdots[dotstyle=*,linecolor=red](Y)
\psline[linestyle=dotted ](A)(B)
\psline[linestyle=dotted ](A)(X)
\psline[linestyle=dotted ](A)(Y)
\psline[linestyle=dotted ](B)(X)
\psline[linestyle=dotted ](B)(Y)
\uput[45](X){X}
\uput[45](Y){Y}
\psline[linestyle=dotted ](A)(B)
\psarc[linestyle=dashed,linewidth=1pt](A){3}{80}{210}
\end{psgraph}
\\ \hline  
\BSS{rhombus}\AC{{\red 2}}(A)(B)\AC{C}\AC{D} \BSI{rhombus}{pst-node} 
&  
\BSS{rhombus}\AC{{\red 3}}(A)(B)\AC{X}\AC{Y}
\\ \hline 
\end{tabular}
 
%-----------------------------------

\SbSbSSCT{Positions relatives avec psRelNodeVar}{Relative position width psRelNodeVar}

\begin{tabular}{|c|c|c|} \hline  
\begin{psgraph}[axesstyle=none,xticksize=0 5,yticksize=0 5,subticks=0](0,0)(5,5){3cm}{3cm}
 \pnode(0,1){B}
 \pnode(2,1){C}
\psRelNodeVar(B)(C)(1;45){A}
 \psline(B)(C)
\psline[linestyle=dotted](B)(A)
\psline[linestyle=dotted](C)(A)
\psarc[linestyle=dashed](B){2}{-20}{60}
\uput[-45](B){B}
\uput[45](A){A}
\uput[-45](C){C}
\psdots[dotstyle=*,linecolor=blue](B)
\psdots[dotstyle=*,linecolor=red](A)
\psdots[dotstyle=*,linecolor=blue](C)
 \end{psgraph}
&  
\begin{psgraph}[axesstyle=none,xticksize=0 5,yticksize=0 5,subticks=0](0,0)(5,5){3cm}{3cm}
\pnode(0,1){B}
\pnode(2,1){C}
\psRelNodeVar(B)(C)(2;45){A}
\psline(B)(C)
\psline[linestyle=dotted](B)(A)
\psarc[linestyle=dashed](B){4}{-20}{60}
\uput[-45](B){B}
\uput[45](A){A}
\uput[-45](C){C}
\psdots[dotstyle=*,linecolor=blue](B)
\psdots[dotstyle=*,linecolor=red](A)
\psdots[dotstyle=*,linecolor=blue](C)
 \end{psgraph}
&  
\begin{psgraph}[axesstyle=none,xticksize=0 5,yticksize=0 5,subticks=0](0,0)(5,5){3cm}{3cm}
\pnode(0,1){B}
\pnode(2,1){C}
\psRelNodeVar(B)(C)(2;30){A}
\psline(B)(C)
\psline[linestyle=dotted](B)(A)
\psarc[linestyle=dashed](B){4}{-20}{60}
\uput[-45](B){B}
\uput[45](A){A}
\uput[-45](C){C}
\psdots[dotstyle=*,linecolor=blue](B)
\psdots[dotstyle=*,linecolor=red](A)
\psdots[dotstyle=*,linecolor=blue](C)
\end{psgraph}
\\ \hline
\BSS{psRelNodeVar}(B)(C)({\red 1;45})\AC{A} \BSI{psRelNodeVar}{pst-node} 
&
\BSS{psRelNodeVar}(B)(C)({\red 2;45})\AC{A}
&  
\BSS{psRelNodeVar}(B)(C)({\red 2;30})\AC{A}
\\ \hline 
\end{tabular} 

%-----------------------------------

\SbSbSSCT{Positions relatives avec AplusB}{Relative position width AplusB} 

\begin{tabular}{|c|} \hline  
\begin{psgraph}[axesstyle=none,xticksize=0 4,yticksize=0 4,subticks=0](0,0)(4,4){3cm}{3cm}
\pnode(1,2){B}\pnode(2,1){C}
\pnode(0,0){O}
\AplusB(B)(C){A}
\psdots[dotstyle=*,linecolor=blue](B)
\psdots[dotstyle=*,linecolor=blue](C)
\psdots[dotstyle=*,linecolor=red](A)
\nput{45}{A}{A}%
\nput{135}{B}{B} 
\nput{135}{C}{C}%
\ncline {O} {B}
\ncline{O} {C}
\ncline {O} {B}
\ncline[linestyle=dashed]{->} {B} {A}
\ncline[linestyle=dashed]{->} {C} {A}
 \end{psgraph}
\\ \hline  
\BSS{AplusB}(B)(C)\AC{A} \BSI{AplusB}{pst-node} 
\\ \hline 
\end{tabular} 

%-----------------------------------

\SbSbSSCT{Positions relatives avec AtoB}{Relative position width AtoB} 

\begin{tabular}{|c|}\hline 
\begin{psgraph}[axesstyle=none,xticksize=-2 3,yticksize=-1 3,subticks=0](0,0)(-1,-2)(3,3){4cm}{4cm}
\pnode(1,2){B}\pnode(2,1){C}
\pnode(0,0){O}
\AtoB(B)(C){A}
\psdots[dotstyle=*,linecolor=blue](B)
\psdots[dotstyle=*,linecolor=blue](C)
\psdots[dotstyle=*,linecolor=red](A)
\nput{45}{A}{A}%
\nput{45}{B}{B} 
\nput{45}{C}{C}%
\ncline {O} {B}
\ncline {O} {C}
\ncline {O} {B}
\ncline[linestyle=dashed]{->} {B} {C}
\ncline[linestyle=dashed]{->} {O} {A}
\end{psgraph}
\\ \hline  
\BSS{AtoB}(B)(C)\AC{A} \BSI{AtoB}{pst-node} 
\\ \hline 
\end{tabular} 



%==================================================

\SbSSCT{N\oe ud sur une courbe}{Node on a curve} 

\SbSbSSCT{N\oe ud sur une courbe avec fnpnode  }{Node on a curve with fnpnode } 

\begin{tabular}{|l|} \hline  
\begin{psgraph}[axesstyle=frame,xticksize=-1.5 1.5cm,yticksize=0 8cm,subticks=0](0,0)(0,-1.5)(13,1.5){8cm}{3cm}
\psset{algebraic}
\psplot[plotpoints=200,linewidth=2pt]{0}{13}{sin(x)}
\fnpnode{2}{sin(x)}{A}
\fnpnode{10}{sin(x)}{B}
\psdots[dotstyle=*,linecolor=red](A) \uput[-135](A){A}
\psdots[dotstyle=*,linecolor=red](B) \uput[45](B){B}
\psline[linestyle=dashed] (A) (B)
 \end{psgraph}
\\ \hline  
\BS{psplot}[plotpoints=200,linewidth=2pt]{0}\AC{13}{sin(x)}\\
\BSS{fnpnode}\AC{2}\AC{sin(x)}\AC{A} \BSI{fnpnode}{pst-node} \\
\BSS{fnpnode}\AC{10}\AC{sin(x)}\AC{B}\\
\BS{psline}[linestyle=dashed] (A) (B)
\\ \hline 
\end{tabular} 

%--------------------------------------

\SbSbSSCT{N\oe uds sur une courbe avec fnpnodes  }{Nodes on a curve with fnpnodes } 

\begin{tabular}{|l|} \hline  
\begin{psgraph}[axesstyle=none,xticksize=-1.5 1.5cm,yticksize=0 8cm,subticks=0](0,0)(0,-1.5)(13,1.5){8cm}{3cm}
\psset{algebraic}
\psplot[plotpoints=200,linewidth=2pt]{0}{13}{sin(x)}
\fnpnodes[plotpoints=14]{0}{13}{sin(x)}{A}
\psdots[dotstyle=*,linecolor=red](A3)
\psdots[dotstyle=*,linecolor=red](A4)
\psdots[dotstyle=*,linecolor=red](A5)
\psdots[dotstyle=*,linecolor=red](A6)
\psdots[dotstyle=*,linecolor=red](A7)
\psdots[dotstyle=*,linecolor=red](A8)
\psdots[dotstyle=*,linecolor=red](A9)
\psdots[dotstyle=*,linecolor=red](A11)
\psdots[dotstyle=*,linecolor=red](A12)
\psdots[dotstyle=*,linecolor=red](A13)
\psdots[dotstyle=*,linecolor=red](A14)
\psdots[dotstyle=*,linecolor=red](A1)
\psdots[dotstyle=*,linecolor=red](A2) \uput[45](A2){A2}
\psdots[dotstyle=*,linecolor=red](A10) \uput[45](A10){A10}
\psline[linestyle=dashed]  (A2) (A10)
 \end{psgraph}
\\ \hline  
\BS{psplot}[plotpoints=200,linewidth=2pt]{0}\AC{13}{sin(x)} \\
\BSS{fnpnodes}[plotpoints=14]\AC{0}\AC{13}\AC{sin(x)}\AC{A} \BSI{fnpnodes}{pst-node}  \\
\BS{psline}[linestyle=dashed]  (A2) (A10)
\\ \hline 
\end{tabular} 
 
 
%--------------------------------------

\SbSbSSCT{N\oe ud sur une courbe paramétrique avec curvepnode  }{Node on a parametric curve with curvepnode } 

\begin{tabular}{|l|} \hline  
\begin{psgraph}[axesstyle=none,xticksize=-1.2 1.2 ,yticksize=-1.2 1.2 , subticks=0](0,0)(-1.2,-1.2)(1.2,1.2){4cm}{4cm}
\psset{algebraic}
\parametricplot[plotpoints=200,linewidth=2pt]{0}{6.28}{sin(t)|sin(2*t)}
\curvepnode{2}{sin(t)|sin(2*t)}{A}
\curvepnode{6}{sin(t)|sin(2*t)}{B}
\psdots[dotstyle=*,linecolor=red](A)  \uput[-45](A){A}
\psdots[dotstyle=*,linecolor=red](B)  \uput[135](B){B}
\psline[linestyle=dashed] (A) (B)
\end{psgraph}
\\ \hline  
\BS{parametricplot}[plotpoints=200]\AC{0}\AC{6.28}\AC{sin(t)|sin(2*t)}\\
\BSS{curvepnode}\AC{2}\AC{sin(t)|sin(2*t)}\AC{A} \BSI{curvepnode}{pst-node} \\
\BSS{curvepnode}\AC{6}\AC{sin(t)|sin(2*t)}\AC{B}\\
\BS{psline}[linestyle=dashed] (A) (B)
\\ \hline 
\end{tabular} 

\bigskip 

\paragraph{Création automatique d'un n\oe ud pour la tangente} :

\begin{tabular}{|c|c|} \hline  
\begin{psgraph}[axesstyle=none,xticksize=-1.5 1.5 ,yticksize=-1.5 1.5 , subticks=0](0,0)(-1.5,-1.5)(1.5,1.5){4cm}{4cm}
\psset{algebraic}
\parametricplot[plotpoints=200]{0}{6.28}{sin(t)|sin(2*t)}
\curvepnode{2}{sin(t)|sin(2*t)}{A}
\psxline[linewidth=2pt,linecolor=red]{<->}(A){-(Atang)}{(Atang)}
\uput[-45](A){A} 
\psline[linewidth=2pt,linecolor=red] (0,0) (Atang)
\psdots[dotstyle=*,linecolor=blue](Atang)
\psdots[dotstyle=*,linecolor=blue](A) 
\uput[-90](Atang){Atang}
\end{psgraph}
&  
\begin{psgraph}[axesstyle=none,xticksize=-1.5 1.5 ,yticksize=-1.5 1.5 , subticks=0](0,0)(-1.5,-1.5)(1.5,1.5){4cm}{4cm}
\psset{algebraic}
\parametricplot[plotpoints=200]{0}{6.28}{sin(t)|sin(2*t)}
\curvepnode{4}{sin(t)|sin(2*t)}{X}
\psxline[linewidth=2pt,linecolor=red]{<->}(X){-1.5(Xtang)}{0.5(Xtang)}
\uput[-90](X){X} 
\psline[linewidth=2pt,linecolor=red] (0,0) (Xtang)
\psdots[dotstyle=*,linecolor=blue](Xtang)
\psdots[dotstyle=*,linecolor=blue](X) \uput[-90](Xtang){Xtang}
\end{psgraph}
\\ \hline  
\BS{curvepnode}\AC{2}\AC{sin(t)|sin(2*t)}\AC{{\red A}}
&  
\BS{curvepnode}\AC{4}\AC{sin(t)|sin(2*t)}\AC{{\red X} }
\\
\BS{psxline}\AC{<->}(X)\AC{-({\red Atang})}\AC{({\red Atang})}
&
\BS{psxline}\AC{<->}(A)\AC{-0.5({\red Xtang})}\AC{1.5({\red Xtang})}
\\ \hline
\end{tabular} 


%--------------------------------------

\SbSbSSCT{N\oe uds sur une courbe paramétrique avec curvepnodes  }{Nodes on a parametric curve with curvepnodes }

\begin{tabular}{|c|c|} \hline 
\multicolumn{2}{|c|}{ \BSS{curvepnodes}[plotpoints=100]\AC{1}\AC{5}\AC{sin(t)|sin(2*t)}\AC{A} \BSI{curvepnodes}{pst-node}  } \\ \hline 
\begin{psgraph}[axesstyle=none,xticksize=-1.2 1.2 ,yticksize=-1.2 1.2 , subticks=0](0,0)(-1.2,-1.2)(1.2,1.2){4cm}{4cm}
\psset{algebraic}
\parametricplot[plotpoints=200,linecolor=red]{1}{5}{sin(t)|sin(2*t)}
\curvepnodes[plotpoints=100]{1}{5}{sin(t)|sin(2*t)}{A}
\cnodeput[fillstyle=solid](A20){A}{20}
\cnodeput[fillstyle=solid](A80){B}{80}
\psline[linestyle=dotted] (A) (B)
\end{psgraph}
&  
\begin{psgraph}[axesstyle=none,xticksize=-1.5 1.5 ,yticksize=-1.5 1.5 , subticks=0](0,0)(-1.5,-1.5)(1.5,1.5){4cm}{4cm}
\psset{algebraic}
\parametricplot[plotpoints=200,linecolor=red]{1}{5}{sin(t)|sin(2*t)}
\curvepnodes[plotpoints=100]{1}{5}{sin(t)|sin(2*t)}{A}
\psline[linewidth=2pt] (A20) (A80)
\end{psgraph}
\\ \hline  
\BS{cnodeput}(A20)\AC{A}\AC{20}

&  
\BS{psline}[linewidth=2pt] (A20) (A80)
\\
\BS{cnodeput}(A80)\AC{B}\AC{80}
&

\\ \hline 
\end{tabular} 

%--------------------------------------

\SbSSCT{Lignes relatives }{ Relative line }

\SbSbSSCT{Lignes relatives avec psRelNode  }{Relative line width psRelNode }

\begin{tabular}{|l|l|} \hline  
\begin{psgraph}[axesstyle=none,xticksize=0 5,yticksize=0 6,subticks=0](0,0)(0,0)(6,5){3cm}{2.5cm} 
\psline[linewidth=4pt](1,1)(3,2)
\psRelLine[linecolor=red](1,1)(3,2){2}{A}
\psdots[dotstyle=*,linecolor=red](A)  \nput{45}{A}{A}%
\psRelLine[linecolor=green,angle=30](1,1)(3,2){2}{B}
\psdots[dotstyle=*,linecolor=green](B)  \nput{0}{B}{B}%
\psdots[dotstyle=*,](1,1)
\psdots[dotstyle=*,](3,2)
\end{psgraph}
&  
\begin{psgraph}[axesstyle=none,xticksize=0 5,yticksize=0 6,subticks=0](0,0)(0,0)(6,5){3cm}{2.5cm} 
%\psline[linewidth=3pt](1,1)(3,2)
\pnode(1,1){B}\pnode(3,2){C}
\psRelNode[linecolor=red,angle=30](B)(C){2}{A}
\psdots[dotstyle=*,linecolor=red](A)  \nput{-45}{A}{A}%
\psdots[dotstyle=*,linecolor=blue](B) \nput{135}{B}{B} 
\psdots[dotstyle=*,linecolor=blue](C) \nput{45}{C}{C}%
\psline[linewidth=2pt](B)(C)
\psline[linewidth=2pt,linecolor=red](B)(A)
\end{psgraph}
\\ \hline  
\BSS{psRelLine}(1,1)(3,2)\AC{2}\AC{{\red A}}
\BSI{psRelLine}{pst-node}   \BSI{psRelLine}{pstricks-add} 
&  

\\
\BS{}psRelLine[{\red angle=30}](0,0)(2,1)\AC{2}\AC{{\red B}} 
&
\BSS{psRelNode}[linecolor=red,angle=30](B)(C)\AC{2}\AC{{\red A}}   \BSI{psRelNode}{pst-node}  \BSI{psRelNode}{pstricks-add} 
\\ \hline 
\end{tabular} 

\bigskip

\paragraph{Paramètre trueAngle} :
 
\begin{tabular}{|l|}  \hline  
\begin{psgraph}[axesstyle=none,xticksize=0 5,yticksize=0 4,subticks=0](0,0)(0,0)(4,6){6cm}{3cm} 
\psline[linewidth=3pt,linecolor=blue](1,1)(3,1)
\psRelLine[linecolor=red,angle=45](1,1)(3,1){1}{A}
\psRelLine[linecolor=red,angle=45,trueAngle](1,1)(3,1){1}{B}
\psdots[dotstyle=*,linecolor=red](A)   \nput{90}{A}{A}%
\psdots[dotstyle=*,linecolor=red](B)   \nput{90}{B}{B}%
\end{psgraph} 
\\  \hline  
\BS{psRelLine}[angle=45](1,1)(3,1)\AC{1}\AC{{\red A}} 
\\
\BS{psRelLine}[angle=45,\RDD{trueAngle}](1,1)(3,1)\AC{1}\AC{{\red B}} 
\\  \hline 
\end{tabular} 

%--------------------------------------

 
\SbSbSSCT{Lignes relatives avec psRelLineVar }{Relative line width psRelLineVar }

\begin{tabular}{|c|c|} \hline  
\begin{psgraph}[axesstyle=none,xticksize=0 6,yticksize=0 6,subticks=0](0,0)(6,6){3cm}{3cm}
\pnodes(4,1){A}(2,3){B} (1,2){C}
\uput[-135](B){B}
\uput[-45](A){A}
\psdots[dotstyle=*,linecolor=blue](B)
\psdots[dotstyle=*,linecolor=blue](A)
\psRelLineVar[linecolor=red](B)(A)(1;90){X}
\psarc[linestyle=dashed](B){2.828}{-60}{60}
\uput[45](X){X}
\psdots[dotstyle=*,linecolor=red](X) 
\psline(A)(B)
\end{psgraph}
&
\begin{psgraph}[axesstyle=none,xticksize=0 6,yticksize=0 6,subticks=0](0,0)(6,6){3cm}{3cm}
\pnodes(4,1){A}(2,3){B} (1,2){C}
\uput[-135](B){B}
\uput[-45](A){A}
\psdots[dotstyle=*,linecolor=blue](B)
\psdots[dotstyle=*,linecolor=blue](A)
\psRelLineVar[linecolor=red](B)(A)(0.5;135){Y} 
\uput[45](Y){Y}
\psarc[linestyle=dashed](B){1.414}{-70}{120}
\psdots[dotstyle=*,linecolor=red](Y)
\psline(A)(B)
\end{psgraph}
\\ \hline 
\BSS{psRelLineVar}(B)(A)({\red 1;90})\AC{X}  \BSI{psRelLineVar}{pst-node} &
\BSS{psRelLineVar}(B)(A)({\red 0.5;135})\AC{Y} 
\\ \hline 
\end{tabular} 
 
%--------------------------------------

 
\SbSbSSCT{Ligne par une série de points  avec psnline  }{ Line from seval points width psnline  }

\begin{tabular}{|l|l|} \hline  
\begin{psgraph}[axesstyle=none,xticksize=0 3cm,yticksize=0 5cm,subticks=0](0,0)(5,3){5cm}{3cm}
\pnodes{A}(1,1)(3,0.5)(4,2)(2,3)(1,2)

\psdots[dotstyle=*,linecolor=blue](A0) \uput[-135](A0){A0}
\psdots[dotstyle=*,linecolor=blue](A1) \uput[-45](A1){A1}
\psdots[dotstyle=*,linecolor=blue](A2) \uput[-45](A2){A2}
\psdots[dotstyle=*,linecolor=blue](A3) \uput[-90](A3){A3}
\psdots[dotstyle=*,linecolor=blue](A4) \uput[-45](A4){A4}
\psnline[linecolor=red](0,3){A}
\end{psgraph}
 &
\begin{psgraph}[axesstyle=none,xticksize=0 3cm,yticksize=0 5cm,subticks=0](0,0)(5,3){5cm}{3cm}
\pnodes{A}(1,1)(3,0.5)(4,2)(2,3)(1,2)
\psdots[dotstyle=*,linecolor=blue](A0) \uput[-135](A0){A0}
\psdots[dotstyle=*,linecolor=blue](A1) \uput[-45](A1){A1}
\psdots[dotstyle=*,linecolor=blue](A2) \uput[-45](A2){A2}
\psdots[dotstyle=*,linecolor=blue](A3) \uput[-90](A3){A3}
\psdots[dotstyle=*,linecolor=blue](A4) \uput[-45](A4){A4}
\psnline[linecolor=red](2,2){A}
\end{psgraph}
\\ \hline  
 \BS{pnodes}\AC{A}(1,1)(3,0.5)(4,2)(2,3)(1,1) 
 &
  \BS{pnodes}\AC{A}(1,1)(3,0.5)(4,2)(2,3)(1,1) 
 \\
 \BSS{psnline}({ \red 0,3})\AC{A} \BSI{psnline}{pst-node} 
 &
  \BSS{psnline}({ \red 2,2})\AC{A}
\\ \hline 
\end{tabular} 

%--------------------------------------

 
\SbSbSSCT{Courbe par une série de points  avec psncurve }{ Curve from seval points width psncurve }


\begin{tabular}{|l|l|} \hline  
\begin{psgraph}[axesstyle=none,xticksize=0 3cm,yticksize=0 5cm,subticks=0](0,0)(5,3){5cm}{3cm}
\pnodes{A}(1,1)(3,0.5)(4,2)(2,3)(1,2)

\psdots[dotstyle=*,linecolor=blue](A0) \uput[-135](A0){A0}
\psdots[dotstyle=*,linecolor=blue](A1) \uput[-45](A1){A1}
\psdots[dotstyle=*,linecolor=blue](A2) \uput[-45](A2){A2}
\psdots[dotstyle=*,linecolor=blue](A3) \uput[-90](A3){A3}
\psdots[dotstyle=*,linecolor=blue](A4) \uput[-45](A4){A4}
\psncurve[linecolor=red](0,3){A}
\end{psgraph}
 &
\begin{psgraph}[axesstyle=none,xticksize=0 3cm,yticksize=0 5cm,subticks=0](0,0)(5,3){5cm}{3cm}
\pnodes{A}(1,1)(3,0.5)(4,2)(2,3)(1,2)
\psdots[dotstyle=*,linecolor=blue](A0) \uput[-135](A0){A0}
\psdots[dotstyle=*,linecolor=blue](A1) \uput[-45](A1){A1}
\psdots[dotstyle=*,linecolor=blue](A2) \uput[-45](A2){A2}
\psdots[dotstyle=*,linecolor=blue](A3) \uput[-90](A3){A3}
\psdots[dotstyle=*,linecolor=blue](A4) \uput[-45](A4){A4}
\psncurve[linecolor=red](2,2){A}
\end{psgraph}
\\ \hline  
 \BS{pnodes}\AC{A}(1,1)(3,0.5)(4,2)(2,3)(1,1) 
 &
  \BS{pnodes}\AC{A}(1,1)(3,0.5)(4,2)(2,3)(1,1) 
 \\
 \BSS{psncurve}({\red 0,3})\AC{A} \BSI{psncurve}{pst-node} 
 &
  \BSS{psncurve}({\red 2,2})\AC{A}
\\ \hline 
\end{tabular}

%--------------------------------------
 
\SbSbSSCT{ligne par pas succesifs  avec psrline }{ Line by succesive step  width psrline }

\begin{tabular}{|c|} \hline  
\begin{psgraph}[axesstyle=none,xticksize=0 2cm,yticksize=0 5cm,subticks=0](0,0)(5,2){5cm}{2cm}
 \psrline[linewidth=2pt,linecolor=red](0,.5)(1,1)(1,-1)(2,1)
\psline[linestyle=dashed,arrowscale=2]{->}(0,.5)(1,.5)(1,1.5)\psline[linestyle=dashed,arrowscale=2]{->}(1,1.5)(2,1.5)(2,0.5)
\psline[linestyle=dashed,arrowscale=2]{->}(2,.5)(4,.5)(4,1.5)
 \end{psgraph}
\\ \hline  
\BSS{psrline}(0,0.5)(1,1)(1,-1)(2,1) \BSI{psrline}{pst-node} 

\\ \hline 
\end{tabular} 

%--------------------------------------
 
\SbSbSSCT{Lignes par rapport à un point avec psxline }{ Lines relative at a point  width psxline }

\begin{tabular}{|c|c|} \hline  
\begin{psgraph}[axesstyle=none,xticksize=0 6,yticksize=0 6,subticks=0](0,0)(6,6){4cm}{4cm}
\pnodes(3,1){A}(2,3){B} (1,2){C}
\uput[-45](B){B}
 \psxline[linecolor=red]{<->}(B){1,2}{3,1}
\psdots[dotstyle=*,dotscale=2](B)
 \psline (B)(3,5)
 \psline (B)(5,4) 
\end{psgraph}
&  
\begin{psgraph}[axesstyle=none,xticksize=0 6,yticksize=0 6,subticks=0](0,0)(6,6){4cm}{4cm}
\pnodes(3,1){A}(2,3){B} (1,2){C}
\uput[-45](A){A}
\uput[-45](B){B}
\uput[-45](C){C}
 \psxline[linecolor=red]{<->}(B){A}{C}
\psdots[dotstyle=*,linecolor=blue](A)
\psdots[dotstyle=*,linecolor=blue](B)
\psdots[dotstyle=*,linecolor=blue](C)
 \psline(0,0)(C)
  \psline (0,0)(A)
 \psline (B)(3,5)
 \psline (B)(5,4) 
\end{psgraph}
\\ \hline  
\BSS{psxline}\AC{<->}(B)\AC{1,2}\AC{3,1} \BSI{psxline}{pst-node} 
&  
\BSS{psxline}\AC{<->}(B)\AC{A}\AC{C}
\\ \hline 
\end{tabular}

\psset{unit=1}

 %-----------------------------------------------------------
\SbSSCT{Lignes parallèles et leur noeud final}{Parallel lines and their endpoint} 

\TFRGB{ Syntaxe : \\
\BS{}psParallelLine(Point 1)(point 2 )(point 3)\AC{longueur}\AC{nom extrémité} }{Syntax : 
\BS{}psParallelLine(Point 1)(point 2 )(point 3)\AC{lengh}\AC{end name}}

\bigskip


\begin{tabular}{|l|} \hline  
\begin{psgraph}[axesstyle=none,xticksize=0 5,yticksize=0 6,subticks=0](0,0)(0,0)(6,5){3cm}{2.5cm} 
\psline[linewidth=3pt,linecolor=blue](2,1)(4,2)
\psParallelLine[linecolor=red](2,1)(4,2)(1,2){2}{A}
\psdots[dotstyle=*,linecolor=red](A)   \nput{45}{A}{A}%
\psParallelLine[linecolor=green](2,1)(4,2)(1,3){1}{B} 
\psdots[dotstyle=*,linecolor=green](B)   \nput{45}{B}{B}%
\psParallelLine[linecolor=cyan](2,1)(4,2)(1,4){.5}{C}
\psdots[dotstyle=*,linecolor=cyan](C)   \nput{45}{C}{C}%
\end{psgraph} 
\\ \hline  
\BSS{psParallelLine}(2,1)(4,2))(1,2)\AC{{\red 2}}\AC{{\red A}}  \BSI{psParallelLine}{pstricks-add} 
\\
\BS{psParallelLine}(2,1)(4,2)(1,3)\AC{{\green 1}}\AC{{\green B}} 
\\
\BS{psParallelLine}(2,1)(4,2)(1,4)\AC{{\cyan 0.5}}\AC{{\cyan C}}
\\ \hline 
\end{tabular} 

%---------------------------

\SbSSCT{Lignes perpendiculaires une droite}{Perpendiculars to a lines}

\begin{tabular}{|l|} \hline  
\begin{psgraph}[axesstyle=none,xticksize=0 5,yticksize=0 6,subticks=0](0,0)(0,0)(6,5){3cm}{2.5cm} 
\psline(5,5)(3,0)
\psPline[linecolor=red,arrowscale=2]{->}(0,3)(5,5)(3,0)
\psPline[linecolor=green,arrowscale=2](1,4)(5,5)(3,0)
\end{psgraph} 
\\ \hline  
\BS{}psline(5,5)(3,0)
\\
\BSS{psPline}[linecolor=red]\AC{->}(0,3)(5,5)(3,0)
\\ 
\BS{}psPline[linecolor=green](1,4)(5,5)(3,0)
\\ \hline 
\end{tabular} 

%---------------------------------------

\subsection{Vecteur normal }

\begin{tabular}{|c|c|} \hline  
\begin{psgraph}[axesstyle=none,xticksize=0 3cm,yticksize=-2 4cm,subticks=0](0,0)(-2,0)(4,3){6cm}{3cm}
\pnode(2,1){B}
\normalvec(B){A}
\psline[linecolor=red]{->}(0,0) (A)
\psline{->}(0,0) (B)
\nput{90}{B}{B} 
\nput{90}{A}{A}%
\end{psgraph}
&
\begin{psgraph}[axesstyle=none,xticksize=-1.5 1.5 ,yticksize=-1.5 1.5 , subticks=0](0,0)(-1.5,-1.5)(1.5,1.5){2.5cm}{2.5cm}
\psset{algebraic}
\parametricplot[plotpoints=200]{0}{6.28}{sin(t)|sin(2*t)}
\curvepnode{2}{sin(t)|sin(2*t)}{A}
\psxline[linewidth=2pt,linestyle=dashed]{<->}(A){-(Atang)}{(Atang)}
\uput[-45](A){A}
\normalvec(Atang){B} 
\psdots[dotstyle=*,linecolor=blue](A) 
\psxline[linewidth=2pt,linecolor=red]{->}(A){}{-2(B)}
\end{psgraph}
\\ \hline  
\BSS{normalvec}(B)\AC{A} \BSI{normalvec}{pst-node} 
&
\BSS{normalvec}(Atang)\AC{B}
\\
&
\BS{psxline}\AC{->}(A)\AC{} \AC{-2(B)}
\\ \hline 
\end{tabular}

%-----------------------------------------

\SbSSCT{Tangentes}{Tangents}

\SbSbSSCT{Tangentes à un cercle par rapport à un point}{Tangent lines of a circle}

\begin{tabular}{|l|} \hline 
\begin{psgraph}[axesstyle=none,xticksize=-2 2,yticksize=-4 2,subticks=0](0,0)(-4,-2)(5,2){7cm}{3cm}  
\pscircle(03, 0){1}
\psdots[dotstyle=*,linecolor=blue,dotscale=2](-3,0)
\psCircleTangents(-3, 0)(3, 0){1}
\pcline[nodesep=-1cm,linestyle=dashed](-3, 0)(CircleT1)
\pcline[nodesep=-1cm,linestyle=dashed](-3, 0)(CircleT2)
\psdots[dotstyle=*,linecolor=red,dotscale=2](CircleT1)
\psdots[dotstyle=*,linecolor=red,dotscale=2](CircleT2)
\nput{-90}{CircleT1}{CircleT1}%
\nput{90}{CircleT2}{CircleT2}%
\psdots[dotstyle=*](3,0)
\psrline[linestyle=dotted]{->}(3,0)(1;45)
\end{psgraph}
\\ \hline  
\BS{pscircle}(3,0)\AC{1} \\
\BSS{psCircleTangents}(-3,0)(3,0)\AC{1}  \BSI{psCircleTangents}{pstricks-add} \\
\BS{pcline}[nodesep=-1cm,linecolor=blue](-3,0)({\red CircleT1}) \\
\BS{nput}\AC{-90}\AC{{\red CircleT1}}\AC{CircleT1}%
\\ \hline 
\end{tabular} 
 

%------------------------------------------------------------------------------
\SbSbSSCT{Tangentes à une ellipse par rapport à un point}{Tangent lines of an ellipse}

\begin{tabular}{|l|} \hline  
 
\begin{psgraph}[axesstyle=none,xticksize=-2 2,yticksize=-5 6,subticks=0](0,0)(-5,-2)(6,2){7cm}{3cm}  
\psellipse(3, 0)(2, 1)
\psdots[dotstyle=*,linecolor=blue,dotscale=2](-3,0)
\psEllipseTangents(3, 0)(2, 1)(-3,0)
\pcline[nodesep=-1cm,linestyle=dashed](-3, 0)(EllipseT1)
\pcline[nodesep=-1cm,linestyle=dashed](-3, 0)(EllipseT2)
\psdots[dotstyle=*,linecolor=red,dotscale=2](EllipseT1)
\psdots[dotstyle=*,linecolor=red,dotscale=2](EllipseT2)
\nput{-90}{EllipseT1}{EllipseT1}%
\nput{90}{EllipseT2}{EllipseT2}%
\psline[linestyle=dotted]{<->}(1,0)(3,0)
\psline[linestyle=dotted]{<->}(3,0)(3,1)
\end{psgraph}
\\ \hline  
\BS{}psellipse(3,0)(2,1) \\
\BSS{psEllipseTangents}(3,0)(2,1)(-3,0) \BSI{psEllipseTangents}{pstricks-add} \\
\BS{}pcline[nodesep=-1cm](-3,0)({\red EllipseT1}) \\
\BS{}nput\AC{90}\AC{{\red EllipseT1}}\AC{EllipseT1}%
\\ \hline 
\end{tabular} 

 
 %-----------------------------------------------------------\BSS{psCircleTangents}(-3,0)(3,0)\AC{1}-------------------------------------
\SbSbSSCT{Tangentes à deux cercles}{Tangent lines of circles}
 \psset{unit=.5cm}


\begin{tabular}{|l|} \hline  
\begin{psgraph}[axesstyle=none,xticksize=-3 3,yticksize=-4 14,subticks=0](0,0)(-4,-3)(14,3){9cm}{3cm}  
\pscircle(-1, 0){2}
\pscircle(5, 0){1}
\psCircleTangents(-1, 0){2}(5, 0){1}
\pcline[nodesep=-1cm,linestyle=dashed](CircleTO1)(CircleTO2)
\pcline[nodesep=-1cm,linestyle=dashed](CircleTO3)(CircleTO4)
\pcline[nodesep=-1cm,linestyle=dashed](CircleTO3)(CircleTC2)
\pcline[nodesep=-1cm,linestyle=dashed](CircleTO2)(CircleTC2)
\psdots[dotstyle=*,linecolor=red,dotscale=2](CircleTC2)
\psdots[dotstyle=*,linecolor=red,dotscale=2](CircleTO1)
\psdots[dotstyle=*,linecolor=red,dotscale=2](CircleTO2)
\psdots[dotstyle=*,linecolor=red,dotscale=2](CircleTO3) 
\psdots[dotstyle=*,linecolor=red,dotscale=2](CircleTO4)
\nput{90}{CircleTC2}{CircleTC2}% 
\nput{45}{CircleTO1}{CircleTO1}% 
\nput{45}{CircleTO2}{CircleTO2}% 
\nput{-45}{CircleTO3}{CircleTO3}% 
\nput{-45}{CircleTO4}{CircleTO4}% 
\end{psgraph}
\\ \hline  
\BSS{psCircleTangents}(-1, 0)\AC{2}(5,0)\AC{1} \\

\BS{}psdots[dotstyle=*,linecolor=red,dotscale=2]({\red CircleTC2})

\\ \hline  
\end{tabular} 
 

 
\bigskip
 \psset{xunit=1cm}


\begin{tabular}{|l|} \hline  
\begin{psgraph}[axesstyle=none,xticksize=-3 3,yticksize=-3 6,subticks=0](0,0)(-3,-3)(6,3){9cm}{3cm} 
\pscircle(-1, 0){2}
\pscircle(5, 0){1}
\psCircleTangents(-1, 0){2}(5, 0){1}
\pcline[nodesep=-1cm,linestyle=dashed](CircleTI1)(CircleTI2)
\pcline[nodesep=-1cm,linestyle=dashed](CircleTI3)(CircleTI4)
\psdots[dotstyle=*,linecolor=red,dotscale=2](CircleTC1)
\psdots[dotstyle=*,linecolor=red,dotscale=2](CircleTI1)
\psdots[dotstyle=*,linecolor=red,dotscale=2](CircleTI2)
\psdots[dotstyle=*,linecolor=red,dotscale=2](CircleTI3)
\psdots[dotstyle=*,linecolor=red,dotscale=2](CircleTI4)
\nput{90}{CircleTC1}{CircleTC1}% 
\nput{45}{CircleTI1}{CircleTI1}% 
\nput{-135}{CircleTI2}{CircleTI2}% 
\nput{-45}{CircleTI3}{CircleTI3}% 
\nput{135}{CircleTI4}{CircleTI4}% 
\end{psgraph}
\\ \hline  
\BS{}psdots[dotstyle=*,linecolor=red,dotscale=2]({\red CircleTC1}) \\

\BS{}nput\AC{90}\AC{{\red CircleTC1}}\AC{CircleTC1}% 
\\ \hline 
\end{tabular}  
 

\psset{unit=1cm,fillcolor=white,fillstyle=none,linecolor=blue}

%------------------------------------------------------------
\subsection{Intersections}

\SbSbSSCT{Point d'intersection avec psIntersectionPoint}{Intersection point of two lines width psIntersectionPoint}
\TFRGB{Syntaxe : \\
\BS{}psIntersectionPoint(point 1)(point 2)(point 3)(point 4)\AC{nom}}{Syntax : \BS{}psIntersectionPoint(point 1)(point 2)(point 3)(point 4)\AC{name}}

\bigskip
\psset{unit=1cm}


\begin{tabular}{|c|} \hline  
\begin{psgraph}[axesstyle=none,xticksize=-1 4,yticksize=0 5,subticks=0](0,0)(0,-1)(5,4){3cm}{3cm} 
\psdots[dotstyle=*,dotsize=6pt](1,3)
\psdots[dotstyle=*,dotsize=6pt](4,0)
\pnode(1,0){B} \pscircle*(B){3pt} \nput{135}{B}{B}%
\pnode(4,3){C} \pscircle*(C){3pt} \nput{45}{C}{C}%
\psline[linecolor=blue](B)(C)
\psline[linecolor=blue](1,3)(4,0)
\psIntersectionPoint(B)(C)(1,3)(4,0){A}
\psdots[dotstyle=*,linecolor=red,dotscale=2](A)   \nput{90}{A}{A}%
\end{psgraph}
\\ \hline  
\BSS{psIntersectionPoint}(B)(C)(1,3)(4,0)\AC{{\red A}}  \BSI{psIntersectionPoint}{pstricks-add} 
\\ \hline 
\end{tabular} 
  
%----------------------------------------

\SbSbSSCT{Points d'intersection avec polyIntersections} {Intersection points with polyIntersections}
%\subsubsection{polyIntersections}

\begin{tabular}{|c|c|} \hline  
\begin{psgraph}[axesstyle=none,xticksize=0 3cm,yticksize=0 4cm,subticks=0](0,0)(0,0)(4,3){4cm}{3cm}
\psline(0.5,.5)(3,.5)(2.5,2)(1,2.5)(.5,1)
\pnodes(1,1.5){A}(1.5,1){B} 
\polyIntersections{X}{Y}(A)(B)(0.5,.5)(3,.5)(2.5,2)(1,2.5)(.5,1)
\psdots(A)(B)\psline(X)(Y)
\psdots[dotstyle=*,linecolor=red](X)
\psdots[dotstyle=*,linecolor=red](Y)
\uput[0](X){X}\uput[-180](Y){Y}
\uput[0](A){A}\uput[-180](B){B}
\end{psgraph}
&  
\begin{psgraph}[axesstyle=none,xticksize=0 3cm,yticksize=0 4cm,subticks=0](0,0)(0,0)(4,3){4cm}{3cm}
\pnodes{P}(0.5,.5)(3,.5)(2.5,2)(1,2.5)(.5,1)
\pnode(1,1.5){A}\pnode(1.5,1){B}
\polyIntersections{X}{Y}(A)(B){P}{4}
\psnline(0,4){P}
\psdots(A)(B)\psline(X)(Y)
\psdots[dotstyle=*,linecolor=red](X)
\psdots[dotstyle=*,linecolor=red](Y)
\uput[-45](X){X}\uput[-180](Y){Y}
\uput[0](A){A}\uput[-180](B){B}
\end{psgraph}
\\ \hline  
\BSS{polyIntersections}\AC{X}\AC{Y}(A)(B) \BSI{polyIntersections}{pst-node} 
&  
\BS{pnodes}\AC{P}(0.5,.5)(3,.5)(2.5,2)(1,2.5)(.5,1)
\\
(0.5,0.5)(3,0.5)(2.5,2)(1,2.5)(0.5,1)
&
\BSS{polyIntersections}\AC{X}\AC{Y}(A)(B)\AC{P}\AC{4}
\\ \hline 
\end{tabular} 


\bigskip

\begin{tabular}{|l|} \hline  
\begin{psgraph}[axesstyle=none,xticksize=-1.5 1.5cm,yticksize=0 8cm,subticks=0](0,0)(0,-1.5)(13,1.5){12cm}{3cm}
\psset{algebraic}
\psplot[plotpoints=200]{0}{13}{sin(x)}
\fnpnodes[plotpoints=100]{0}{13}{sin(x)}{A}
\pnode(4,0.5){C} \pnode(6,.5){B}
\polyIntersections{X}{Y}(C)(B){A}{60}
\psline(X)(Y)
\psdots[dotstyle=*](B)
\psdots[dotstyle=*](C) 
\psdots[dotstyle=*,linecolor=red](X)
\psdots[dotstyle=*,linecolor=red](Y)
\psdots[dotstyle=*,linecolor=green](A60)
\uput[90](C){C}\uput[90](B){B}
\uput[0](X){X}\uput[180](Y){Y}\uput[90](A60){A60}
\end{psgraph}
\\ \hline  
\BS{fnpnodes}[plotpoints=100]\AC{0}\AC{13}\AC{sin(x)}\AC{A} \\
\BSS{polyIntersections}\AC{X}\AC{Y}(C)(B)\AC{A}\AC{60}
\\ \hline 
\end{tabular} 

\bigskip

\begin{tabular}{|l|} \hline  
\begin{psgraph}[axesstyle=none,xticksize=-1.2 1.2 ,yticksize=-1.5 1.5 , subticks=0](0,0)(-1.5,-1.2)(1.5,1.2){8cm}{3cm}
\psset{algebraic}
\parametricplot[plotpoints=200]{1}{5}{sin(t)|sin(2*t)}
\curvepnodes[plotpoints=100]{1}{5}{sin(t)|sin(2*t)}{A}
\pnode(0,0.5){C} \pnode(.5,.5){B}
\polyIntersections{X}{Y}(C)(B){A}{75}
\psline(X)(Y)
\psdots[dotstyle=*](B)
\psdots[dotstyle=*](C) 
\psdots[dotstyle=*,linecolor=red](X)
\psdots[dotstyle=*,linecolor=red](Y)
\psdots[dotstyle=*,linecolor=green](A75)
\uput[90](C){C}\uput[90](B){B}
\uput[0](X){X}\uput[180](Y){Y}\uput[90](A75){A75}
\end{psgraph}
\\ \hline 
\BS{curvepnodes}[plotpoints=100]\AC{1}\AC{5}\AC{sin(t)|sin(2*t)}\AC{A} \\
\BSS{polyIntersections}\AC{X}\AC{Y}(C)(B)\AC{A}\AC{75}
\\ \hline 
\end{tabular} 

 
 

\newpage
%-----------------------------------------------------------------------
\SbSSCT{Les 9 positions d'une figure par \BS{}psDefPSPNodes}{The 9 positions with \BS{}psDefPSPNodes}
\bigskip

\begin{tabular}{l}  
\begin{pspicture}(6,4)
\psaxes[xticksize=4,yticksize=6,axesstyle=none](6,4)
\psDefPSPNodes
\psdots[linecolor=red,dotscale=2](PSPbl)(PSPbc)(PSPbr)(PSPcl)(PSPcc)(PSPcr)(PSPtl)(PSPtc)(PSPtr)
 \uput[45](PSPbl){PSPbl} \uput[90](PSPbc){PSPbc}
\uput[90](PSPbr){PSPbr} \uput[45](PSPcl){PSPcl}
\uput[90](PSPcc){PSPcc} \uput[90](PSPcr){PSPcr}
\uput[45](PSPtl){PSPtl} \uput[90](PSPtc){PSPtc}
 \uput[90](PSPtr){PSPtr}
 \end{pspicture}
\\ 
\\ 
\\ \hline 
\BS{}begin{pspicture}(6,4)\\
 
\BSS{psDefPSPNodes}  \BSI{psDefPSPNodes}{pstricks-add} \\

\BS{}psdots({\red PSPbl})\\

\BS{}uput[45](PSPbl)\AC{{\red PSPbl}}
\\ 
\end{tabular}  

%--------------------------------

\SbSSCT{N\oe uds sur du texte avec  \BS{}psDefBoxNodes}{Nodes on text with \BS{}psDefBoxNodes}
\bigskip

\begin{tabular}{|l|} \hline  
 \psscalebox{8}{\psDefBoxNodes{nom}{\color{red!20} abcdefghij}}%
\shorthandoff{ :}
\uput[90](nom:tl){tl} \qdisk(nom:tl){3pt}
\uput[90](nom:tC){tC} \qdisk(nom:tC){3pt}
\uput[90](nom:tr){tr} \qdisk(nom:tr){3pt}
\uput[90](nom:Cl){Cl} \qdisk(nom:Cl){3pt}
\uput[90](nom:C){C} \qdisk(nom:C){3pt}
\uput[90](nom:Cr){Cr} \qdisk(nom:Cr){3pt}
\uput[90](nom:Bl){Bl} \qdisk(nom:Bl){3pt}
\uput[90](nom:BC){BC} \qdisk(nom:BC){3pt}
\uput[90](nom:Br){Br} \qdisk(nom:Br){3pt}
\uput[90](nom:bl){bl} \qdisk(nom:bl){3pt}
\uput[90](nom:bC){bC} \qdisk(nom:bC){3pt}
\uput[90](nom:br){br} \qdisk(nom:br){3pt}
\shorthandon{ :}
\\ \hline 
\\ 
 \BS{psscalebox}\AC{15}\AC{ \BSS{psDefBoxNodes}\AC{{\red nom}} \AC{ \BS{color}\AC{red!20} abcdefghij}}\\
 \BSS{shorthandoff}\AC{:} \footnotemark[1]  \\
\BS{uput}[90]({\red nom\xx tl})\AC{tl} \hspace{1cm} \BS{qdisk}({\red nom\xx tl})\AC{3pt} \\ \hspace{1cm} $\vdots $\\
 \BSS{shorthandon}\AC{:}
\\ \hline 
\end{tabular} 
 \footnotetext[1]{désactivation et ré-activation de \og : \fg  conflit entre ce module et Babel en français }


\subsection{ArrowNotch}

\begin{tabular}{|l|l|} \hline 
\multicolumn{2}{|c|}{ \BS{curvepnodes}[plotpoints=100]\AC{1}\AC{1.1}\AC{sin(t)|sin(2*t)}\AC{{\red A}} } 
\\ \hline
\begin{psgraph}[axesstyle=none,xticksize=0.5 0.6 ,yticksize=0.5 0.6 , subticks=0](0.8,0.8)(.9,.95){5cm}{5cm}
 \psset{algebraic}
\parametricplot[plotpoints=200]{1}{1.1}{sin(t)|sin(2*t)}
\curvepnodes[plotpoints=100]{1}{1.3}{sin(t)|sin(2*t)}{A} 
\ArrowNotch[arrowscale=10]{A}{0}{>}{X}
\psline[linecolor=red,arrowscale=5]{-D>}(X)(A0)
\psdots[dotstyle=*](X) \uput[-135](X){X}
\psdots[dotstyle=*](A0) \uput[-135](A0){A0}
\end{psgraph}
&  
\begin{psgraph}[axesstyle=none,xticksize=0.5 0.6 ,yticksize=0.5 0.6 , subticks=0](0.8,0.8)(.9,.95){5cm}{5cm}
\psset{algebraic}
\parametricplot[plotpoints=200]{1}{1.1}{sin(t)|sin(2*t)}
\curvepnodes[plotpoints=100]{1}{1.5}{sin(t)|sin(2*t)}{A} 
\ArrowNotch[arrowscale=10]{A}{0}{<}{V}
\psline[linecolor=red,arrowscale=5]{-D>}(V)(A0)
\psdots[dotstyle=*](V) \uput[-135](V){V}
\psdots[dotstyle=*](A0) \uput[-135](A0){A0}
\end{psgraph}
\\ \hline  
\BSS{ArrowNotch}[arrowscale=10]\AC{{\red A}}\AC{0}\AC{>}\AC{X} \BSI{ArrowNotch}{pst-node} 
&  
\BSS{ArrowNotch}[arrowscale=10]\AC{{\red A}}\AC{0}\AC{<}\AC{V}
\\
\BS{psline}[arrowscale=5]\AC{-D>}(X)(A0)
&
\BS{psline}[arrowscale=5]\AC{-D>}(V)(A0)

\\ \hline
\begin{psgraph}[axesstyle=none,xticksize=0.5 0.6 ,yticksize=0.5 0.6 , subticks=0](0.8,0.8)(.9,.95){5cm}{5cm}
 \psset{algebraic}
\parametricplot[plotpoints=200]{1}{1.1}{sin(t)|sin(2*t)}
\curvepnodes[plotpoints=100]{1}{1.3}{sin(t)|sin(2*t)}{A} 
\ArrowNotch[arrowscale=10]{A}{20}{>}{X}
\psline[linecolor=red,arrowscale=5]{-D>}(X)(A20)
\psdots[dotstyle=*](X) \uput[-135](X){X}
\psdots[dotstyle=*](A20) \uput[-135](A20){A20}
\end{psgraph}
&  
\begin{psgraph}[axesstyle=none,xticksize=0.5 0.6 ,yticksize=0.5 0.6 , subticks=0](0.8,0.8)(.9,.95){5cm}{5cm}
\psset{algebraic}
\parametricplot[plotpoints=200]{1}{1.1}{sin(t)|sin(2*t)}
\curvepnodes[plotpoints=100]{1}{1.3}{sin(t)|sin(2*t)}{A} 
\ArrowNotch[arrowscale=10]{A}{20}{<}{V}
\psline[linecolor=red,arrowscale=5]{-D>}(V)(A20)
\psdots[dotstyle=*](V) \uput[-135](V){V}
\psdots[dotstyle=*](A20) \uput[-135](A20){A20}
\end{psgraph}
\\ \hline  
\BSS{ArrowNotch}[arrowscale=10]\AC{{\red A}}\AC{0}\AC{>}\AC{X}
&  
\BSS{ArrowNotch}[arrowscale=10]\AC{{\red A}}\AC{0}\AC{<}\AC{V}
\\
\BS{psline}[arrowscale=5]\AC{-D>}(X)(A0)
&
\BS{psline}[arrowscale=5]\AC{-D>}(V)(A0)
\\ \hline 
\end{tabular} 



\subsection{Placement d'une étiquette à une distance donnée avec nlput }
\SbSSCT{Placement d'une étiquette à une distance donnée avec nlput }{nlput}

\begin{tabular}{|c|} \hline  
\begin{psgraph}[axesstyle=none,xticksize=0 2cm,yticksize=0 6cm,subticks=0](0,0)(6,2){6cm}{2cm}
\pnode(1,1){B}\pnode(5,1){C}
 \nlput(B)(C){1cm}{Texte}
\psdots[dotstyle=*,linecolor=blue](B) \nput{90}{B}{B} 
\psdots[dotstyle=*,linecolor=blue](C) \nput{90}{C}{C}%
 \end{psgraph}
\\ \hline  
\BSS{nlput}(B)(C)\AC{{\red 1cm}}\AC{Texte} \BSI{nlput}{pst-node} 
\\ \hline 
\end{tabular}  

\bigskip


\begin{tabular}{|c|c|c|} \hline 
\multicolumn{3}{|c|}{  \BSS{nlput}[\RDD{nrot}=\xx U](B)(C)\AC{1cm}\AC{\BS{red} Texte} \RDI{nrot}{pst-node}} \\ \hline  
\begin{pspicture}(5,2)
\pnode(0,0){A}
 \pnode(4,2){B}
 \ncline{A}{B}
 \nlput[nrot=:U](A)(B){1cm}{\red Texte}
 \end{pspicture}
&  
\begin{pspicture}(5,2)
\pnode(0,0){A}
 \pnode(4,2){B}
 \ncline{A}{B}
 \nlput[nrot=:D](A)(B){1cm}{\red Texte}
 \end{pspicture}
&  
\begin{pspicture}(5,2)
\pnode(0,0){A}
 \pnode(4,2){B}
 \ncline{A}{B}
 \nlput[nrot=90](A)(B){1cm}{\red Texte}
 \end{pspicture}
\\ \hline 
 nrot=\xx U  &  nrot=\xx U  & nrot=90
\\ \hline 
\begin{pspicture}(5,2)
\pnode(0,0){A}
 \pnode(4,2){B}
 \ncline{A}{B}
 \nlput[nrot=:L](A)(B){1cm}{\red Texte}
 \end{pspicture}
 &  
 \begin{pspicture}(5,2)
 \pnode(0,0){A}
 \pnode(4,2){B}
  \ncline{A}{B}
  \nlput[nrot=:R](A)(B){1cm}{\red Texte}
  \end{pspicture}
 &  
  \begin{pspicture}(5,2)
  \pnode(0,0){A}
  \pnode(4,2){B}
   \ncline{A}{B}
   \nlput(A)(B){1cm}{\red Texte}
\end{pspicture}
\\ \hline 
 nrot=\xx L  &  nrot=\xx R  & sans paramètre
 \\ \hline 
\end{tabular} 

 