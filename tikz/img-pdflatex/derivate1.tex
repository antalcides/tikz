\documentclass{article} 
\usepackage{tkz-euclide}
\usetkzobj{all}
\usepackage{tkz-fct}  
\usetikzlibrary{calc}
\usepackage[active,pdftex,tightpage]{preview}
\usetikzlibrary{positioning}
\PreviewEnvironment[]{tikzpicture}
\begin{document} 
\tkzfctset{tan style/.style={-,>=latex,blue}}  

\def\alpha{3} 

\newcommand\derivative[4]{%
    \tkzDefPointByFct[draw](#1) \tkzGetPoint{start}
  \tkzDefPointByFct[draw](#2) \tkzGetPoint{end}
  \draw[thin,|-|,yshift=-3pt] (start) -- node[black,fill=white,below] {#3}(start-|end);  
  \draw[thin,|-|,xshift=3pt] (start-|end) -- node[black,fill=white,right] {#4}(end); 
  \draw[thin] (start) --(end); 
} 

\begin{tikzpicture}[scale=2]     
  \tkzInit[xmin=-1,xmax=4.5,ymax=3]
  \tkzClip[space=1]
    \tkzAxeXY
  \tkzFct[domain=.1:5,samples=200,id=ln,line width=0.5pt,color=red]{log(x)+1} 
    \tkzDrawTangentLine[kl=1,kr=5](1)
    \derivative{1}{4}{$\Delta x$}{$\Delta y$}    
    \tkzText[draw=red,fill = red!20](4,2.75){$f(x)=\ln(x)+1$}
    \tkzText[blue,draw,fill = blue!20](2.8,3.5){$g(x)=x$}  
\end{tikzpicture} 

\end{document} 