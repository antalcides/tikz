% Author: Robert Felty
% Source: http://blog.robfelty.com/2007/02/14/pgf-gallery
% Model structure from Rumelhart \& McClelland (1986, p .222)%

\documentclass{article}
\usepackage{tikz}
\usetikzlibrary{arrows}
\usepackage{verbatim}

\begin{comment}
:Title: Distributed processing
:Tags: Foreach, Layers

A neural network example recreated from an illustration in [1]_. 

Note the use of a background layer to draw the lines connecting the nodes. This way
we don't have to worry about overlapping lines.  


:Author: Robert Felty
:Source: `Deep Thoughts by Robert Felty`__

.. __: http://blog.robfelty.com/2007/02/14/pgf-gallery

.. [1] Rumelhart, D. E. and `McClelland, J. L`__. (1986). *On Learning the Past Tenses of English Verbs*. In Volume 2 of Rumelhart, McClelland, and the PDP Research Group (1986), pp. 216-271. 

.. __: http://www.cnbc.cmu.edu/~jlm/
\end{comment}

\begin{document}

% Declare layers
\pgfdeclarelayer{background}
\pgfsetlayers{background,main}

% Styles
\tikzstyle{information text}=[text badly centered,font=\small,text width=3cm]
\begin{tikzpicture}[scale=.8,cap=round]
    % The graphic
    \begin{scope}[>=stealth', line width=1pt]
        \draw[->] (1,.9) node[below, information text]
            {Phonological representation of root form } -- (1,1.8);
        \draw[->] (5,-.2) node[below,information text]
            {Wickelfeature representation of root form } -- (5,.8);
        \draw[->] (11,-.2) node[below,information text]
            {Wickelfeature representation of past tense } -- (11,.8);
        \draw[->] (16,0.9) node[below,information text]
            {Phonological representation of past tense } -- (16,1.8);
    \end{scope}
    \draw (3,6) node[information text] { Fixed Encoding Network };
    \draw (8,6) node[information text, text width=4cm, ]
        { Pattern Associator Modifiable Connections };
    \draw (13.5,6) node[information text] { Decoding/Binding Network };
    % draw the nodes
    \foreach \x in {1,16}
        \foreach \y in {2,3,4} {
        \filldraw[fill=white] (\x,\y) circle (0.1);
        }
    \foreach \x in {5,11}
        \foreach \y in {1,2,3,4,5} {
            \filldraw[fill=white] (\x,\y) circle (0.1);
        }
    % The lines connecting the nodes are drawn in the background layer.
    % This way we can hide the lines behind the nodes and don't worry
    % about the width of each node.    
    \begin{pgfonlayer}{background}
        % we add the lines for the nodes starting in y 2,3, and 4
        \foreach \xa / \xb in {1 / 5, 5 / 11 , 11 / 5 , 16 / 11}
            \foreach \ya / \yb / \yc / \yd / \ye in {2 / 3 / 4 / 5 / 1, 
            3 / 4 / 5 / 1 / 2, 4 / 5 / 1 / 2 / 3} {
                \draw (\xa,\ya) -- (\xb,\ya);
                \draw (\xa,\ya) -- (\xb,\yb);
                \draw (\xa,\ya) -- (\xb,\yc);
                \draw (\xa,\ya) -- (\xb,\yd);
                \draw (\xa,\ya) -- (\xb,\ye);
            }
        % add remaining lines from y1 to y5
        \foreach \xa / \xb in {5 / 11 , 11 / 5}
            \foreach \ya / \yb in {1 / 5, 5 / 1} {
            \draw (\xa,\ya) -- (\xb,\ya);
            \draw (\xa,\ya) -- (\xb,\yb);
        }
    \end{pgfonlayer}
\end{tikzpicture}

\end{document}