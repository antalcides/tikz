% Pie chart
% Author: Robert Vollmert
\documentclass{article}

\usepackage{calc}
\usepackage{ifthen}
\usepackage{tikz}
\usepackage{verbatim}

\begin{document}

\begin{comment}
:Title: Pie chart
:Tags: Charts, Foreach

This example shows how to draw a basic pie chart. Note that labels are automatically 
aligned and placed in a smart way. This makes the code more complicated. However, 
charts can now bee drawn without worrying about overlapping labels. 

:Author: Robert Vollmert

\end{comment}


\newcounter{temp}

% calculate the anchor of the external label
\newcommand{\angledir}[1]{
  \setcounter{temp}{#1}

  \ifthenelse{\thetemp < 0}{\addtocounter{temp}{360}}{}
  \ifthenelse{\thetemp > 360}{\addtocounter{temp}{-360}}{}

  \ifthenelse{\thetemp < 11}{\def\piedir{right}}{
  \ifthenelse{\thetemp < 80}{\def\piedir{above right}}{
  \ifthenelse{\thetemp < 101}{\def\piedir{above}}{
  \ifthenelse{\thetemp < 170}{\def\piedir{above left}}{
  \ifthenelse{\thetemp < 191}{\def\piedir{left}}{
  \ifthenelse{\thetemp < 260}{\def\piedir{below left}}{
  \ifthenelse{\thetemp < 281}{\def\piedir{below}}{
  \ifthenelse{\thetemp < 350}{\def\piedir{below right}}{
    right}}}}}}}}%
}

% calculate the position of the internal label
\newcommand{\calcpiedist}[1]{%
  \ifthenelse{#1 > 120}{\def\piedist{0.50}}{
  \ifthenelse{#1 < 10}{\def\piedist{0.80}}{
    \setcounter{temp}{(80 * (120-#1) + 50 * (#1-10))/110}
    \def\piedist{0.\thetemp}}}
}

% draw a slice of a pie chart
\newcommand{\slice}[4]{%
  \setcounter{temp}{(#1+#2)/2}
  \begin{scope}[rotate=\thetemp]
    \draw[fill=black!10,join=round,thick]
                                (0,0)
                             -- (#1-\thetemp:1)
                            arc (#1-\thetemp:#2-\thetemp:1)
                             -- cycle;
    \angledir{\thetemp}
    \node [\piedir] at (1,0) {#4};
    \setcounter{temp}{#2-#1}
    \calcpiedist{\thetemp}
    \node at (\piedist,0) {#3};

  \end{scope}
}

\begin{tikzpicture}[scale=3]

\newcounter{a}
\newcounter{b}
\foreach \p/\t in {20/type A, 4/type B, 11/type C,
                   49/type D, 16/other}
  {
    \setcounter{a}{\value{b}}
    \addtocounter{b}{\p}
    \slice{360*\thea/100}
          {360*\theb/100}
          {\p\%}{\t}
  }

\end{tikzpicture}

\end{document}
