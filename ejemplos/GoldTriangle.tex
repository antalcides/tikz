\documentclass[]{scrartcl}
\usepackage[utf8]{inputenc} 
\usepackage[upright]{fourier}
\usepackage[usenames,dvipsnames,svgnames]{xcolor}
\usepackage{tikz}
\usepackage{amsmath,tkz-euclide,fullpage}
\usetkzobj{all}
\usepackage[frenchb]{babel}
\definecolor{fondpaille}{cmyk}{0,0,0.1,0}
\pagecolor{fondpaille}
\color{Maroon}
\parindent=0pt   

\begin{document}
\section{Pentagone régulier}   
Euclide construit un pentagone régulier (équilatéral et équiangle) inscrit dans un cercle. Son élément de base est le triangle d'or : un triangle isocèle dont les angles avec la base sont double de l'angle au sommet (et ainsi l'angle au sommet est le 5e de l'angle plat). 180/5=36

\subsection{La proportion d'or - l'équerre 1/2 }
[AB] un segment.

\begin{enumerate}
 
\item Placer I, le milieu, le milieu de [AB]  
\item Tracer la perpendiculaire en B à (AB)
\item Tracer l'arc de cercle de centre B de rayon BI = AB / 2, qui coupe [By) en C  
\item Tracer le segment [AC] 	  
\item Tracer l'arc de cercle de centre C de rayon CB, qui coupe [AC] en D   
\item Tracer l'arc de cercle de centre A de rayon AD, qui coupe [AB] en M 
\end{enumerate}

\[
   \frac{AM}{MB}=\Phi=\frac{1+\sqrt{5}}{2}
\]
\begin{tikzpicture}
  \tkzClip[space=1]
  \tkzDefPoint(0,0){A}
  \tkzDefPoint(8,0){B} 
  \tkzDefMidPoint(A,B)\tkzGetPoint{I}
  \tkzVecKOrth[-.75](B,A){C}
  \tkzInterLC(B,C)(B,I)\tkzGetSecondPoint{D}
  \tkzDuplicateLen(B,D)(D,A)\tkzGetPoint{E}
  \tkzDrawArc[delta=10](D,E)(B)
  \tkzInterLC(A,B)(A,E)\tkzGetSecondPoint{M}
  \tkzDrawArc[delta=10](A,M)(E)
  \tkzDrawLines(A,B B,C A,D)
  \tkzCompass(B,D)
  \tkzDrawPoints(I,D,C,M,E)
  \tkzLabelPoints(A,B,C,D,M,I,E)
\end{tikzpicture}



\newpage
\subsection{La proportion d'or - le rectangle d'or}

\begin{itemize}
   \item AB=1
   \item I milieu de [AB]
   \item CF=EB
\end{itemize}
\[
   \frac{AE}{AB}=\Phi=\frac{1+\sqrt{5}}{2}
\]

\begin{tikzpicture}
  \tkzInit[xmax=14,ymax=10]
  \tkzClip[space=1]
  \tkzDefPoint(0,0){A}
  \tkzDefPoint(8,0){B}
  \tkzDefMidPoint(A,B)\tkzGetPoint{I}
  \tkzDefSquare(A,B)\tkzGetPoints{C}{D} 
  \tkzDrawPolygon(A,B,C,D)
  \tkzInterLC(A,B)(I,C)\tkzGetSecondPoint{E}
  \tkzDrawArc[style=dashed,color=gray](I,E)(D)
  \tkzDefPointWith[colinear= at C](E,B)\tkzGetPoint{F}
  \tkzDrawPoints(A,B,C,D,E,F,I)
  \tkzLabelPoints(A,B,C,I,D,E,F)
  \tkzDrawSegments[style=dashed,color=gray](E,F C,F B,E)
\end{tikzpicture}


\newpage  
\subsection{Construction du triangle d'or}

Dans la figure jointe, I est le milieu de [AC], AC = AB, IB = ID, AD = AE = BF. Euclide démontre que le triangle ABF est un triangle d'or en utilisant des propriétés assez longues. (wikipedia)

De nos jours, la démonstration est plus simple car si on note AC = 1, on obtient 
\[
  IB=\frac{\sqrt{5}}{2}
\]   
\[
  AD=AE=BF=\frac{\sqrt{5}-1}{2}=\frac{1}{\Phi}
\]
 grâce au théorème de Pythagore
 où  est le nombre d'or
Les dimensions du triangle ABF sont donc 1 ; 1 et $\dfrac{1}{\Phi}$ . C'est bien un triangle d'or.


\bigskip

%\begin{tikzpicture}
%    \tkzPoint(0,0){A} 
%    \tkzPoint(5,0){C}
%    \tkzPoint[pos=above](0,5){B} 
%    \tkzDrawLine[](A,C)
%    \tkzDrawArc(A,C)(B)
%    \tkzDefMidPoint(A,C)\tkzGetPoint{I}
%    \tkzDrawSegment[color=gray,style=dashed](I,B)
%    \tkzDuplicateLen(I,B)(I,A)\tkzGetPoint{D} 
%    \tkzDrawArc[color=gray,style=dashed](I,B)(D)
%    \tkzDuplicateLen(A,D)(A,B)\tkzGetPoint{E}
%    \tkzDuplicateLen(A,D)(B,A)\tkzGetPoint{G} 
%    \tkzDrawArc[color=gray,style=dashed](A,D)(B)
%    \tkzInterCC(A,C)(B,G)\tkzGetPoints{K}{F}
%    \tkzDrawArc[color=gray,style=dashed](B,G)(F)
%    \tkzDrawPoint(F)
%    \tkzCompass(B,F)
%    \tkzDrawPolygon[color=red](A,B,F) 
%\end{tikzpicture}

\newpage  
\subsection{Construction du pentagone }

Euclide prouve qu'il peut construire un triangle d'or dans un cercle.
À partir du triangle d'or OA'C construire le triangle d'or CDA grâce à l'arc de cercle de centre A' et de rayon A'C
En prenant les bissectrices des angles C et D en les prolongeant jusqu'au cercle, il obtient les deux sommets B et E manquant.  


\begin{tikzpicture}
  \tkzDefPoint(0,0){A}
  \tkzDefPoint(0,5){B}
  \tkzDefPoint(5,0){C}
  \tkzDefPoint(54:5){F}
  \tkzDrawCircle[color=gray](A,C) 
  \tkzInterCC[with nodes](A,A,C)(C,B,F) \tkzGetPoints{a}{e}
  \tkzInterCC(A,C)(a,e)                 \tkzGetFirstPoint{b}
  \tkzInterCC(A,C)(b,a)                 \tkzGetFirstPoint{c}
  \tkzInterCC(A,C)(c,b)                 \tkzGetFirstPoint{d}
  \tkzDrawPoints(a,b,c,d,e)  
  \tkzDrawPolygon[color=red](a,b,c,d,e)
  \foreach \vertex/\num in {a/36,b/108,c/180,d/252,e/324}{%
  \tkzDrawPoint(\vertex)
  \tkzLabelPoint[label=\num:$\vertex$](\vertex){}% astuce   
  \tkzDrawSegment[color=gray,style=dashed](A,\vertex)
  }  
\end{tikzpicture}

\newpage  
\subsection{Calcul de $\cos\left(\dfrac{\pi}{5} \pi\right)$ }

\begin{tikzpicture}
   \tkzInit[xmin=-1,xmax=11,ymin=-1,ymax=6] 
   \tkzClip[space=1]   
   \tkzDefPoint(0,0){B} 
   \tkzDefPoint(10,0){C}
   \tkzDefPointBy[rotation=center B angle 36](C)
   \tkzGetPoint{A}
   \tkzDefLine[bisector](B,A,C)            \tkzGetPoint{a}
   \tkzDrawPoints(A,B,C) 
   \tkzLabelPoints(A,B,C)    
   \tkzDrawLine[add = 0 and 7,color=gray](A,a)  
   \tkzInterLL(B,C)(A,a)                    \tkzGetPoint{I}  
   \tkzDefPointBy[projection=onto B--A](I) \tkzGetPoint{H} 
   \tkzDrawPolygon[color=Maroon](A,B,C)
   \tkzDrawSegment[color=gray](I,H)
   \tkzMarkRightAngle(B,H,I)
   \tkzLabelPoints(A,B,C,I,H)
   \tkzMarkAngles[mark=||](H,A,I I,A,C)    
\end{tikzpicture}       


Soit un triangle d'or ABC ayant pour base [AC]
 \begin{itemize}
  \item[*]  BA=BC
  \item[*] $\widehat{ABC}=\frac{\pi}{5}$  
  \item[*]  $[AI)$ bissectrice de  $\widehat{BAC}$
  \item[*] $AC=1   $ 
\end{itemize}

On montre alors que :

\begin{itemize}    
  \item[*] $ AC=AI=BI=1   $
  \item[*] $ BA=2BH=2\cos(\frac{\pi}{5})$
  \item[*] $ IC=2\cos\left(\frac{\pi}{5}\right) -1$ 
  \item[*] $ \dfrac{AC}{AB}=\dfrac{IC}{AC}$ donc $IC = \dfrac{1}{2\cos(\frac{\pi}{5})}$ 
 \end{itemize}
 
Les deux dernières égalités donnent 
\[\left( 2\cos\left(\frac{\pi}{5}\right)\right)\left(2\cos\left(\frac{\pi}{5}\right) -1\right) =1 \]
 c'est à dire 
\[
  4 \cos^2\left(\frac{\pi}{5}\right) - 2 \cos\left(\frac{\pi}{5}\right) - 1 = 0
\]
Si on pose $\cos\left(\dfrac{\pi}{5} \right)=x $, la solution positive de l' équation $4x^2-2x+1=0$ est 

\[\cos\left(\dfrac{\pi}{5} \right)=\dfrac{1 +\sqrt{5}}{4}=\dfrac{\Phi}{2}   .\]       

\newpage   
\subsection{Pentagone inscrit dans un cercle}

On peut grandement simplifier la construction d'Euclide en conservant le même principe : construire des triangles d'or ou d'argent.

\begin{enumerate}
  
  \item Tracer un cercle $\Gamma$ de centre O et de rayon R (unité quelconque)
  \item Tracer 2 diamètres perpendiculaires 
 \begin{itemize}
 \item  les jonctions à $\Gamma$ formant les point A, B, C, D
 \item A étant diamétralement opposé à C
 \item B étant diamétralement opposé à D 
 \item Tracer un cercle $\Gamma'$ de diamètre [OA] (rayon R' = R/2) et de centre I. cercle $\Gamma'$  passe donc en O et A.
 \item   Tracer une droite (d) passant par B et I, 
   (d) intercepte cercle $\Gamma'$  en E et F (E est le plus proche de B)
 \item   Tracer 2 (arc de) cercles $\Gamma$1 et $\Gamma$2 de centre B et de rayons (respectivement) BE et BF
   $\Gamma$1 et $\Gamma$2 interceptent $\Gamma$ en 4 pts (D1, D2, D3, D4)
 \item   D, D1, D2, D3, D4 forment un pentagone régulier
   En effet, on vérifie que BOD2 est un triangle d'or, BOD1 un triangle d'argent (leurs bases valent respectivement  et  alors que leurs côtés valent R).
 \end{itemize}

\end{enumerate}

\begin{tikzpicture}
  \tkzInit[xmin=-6,xmax=6,ymin=-6,ymax=6] 
  \tkzClip
  \tkzDefPoint(0,0){O} 
  \tkzDefPoint(5,0){A}
  \tkzDefPoint(0,5){B}
  \tkzDefPoint(-5,0){C} 
  \tkzDefPoint(0,-5){D}
  
  \tkzDefMidPoint(A,O)     \tkzGetPoint{I}

  \tkzInterLC(I,B)(I,A)    \tkzGetPoints{F}{E}
  \tkzDrawArcAngles[color=gray,style=dashed](B,E)(180,360)
  \tkzDrawArcAngles[color=gray,style=dashed](B,F)(220,340)
  \tkzInterCC(O,C)(B,E)    \tkzGetPoints{D3}{D2}
  \tkzInterCC(O,C)(B,F)    \tkzGetPoints{D4}{D1}
  
  \tkzDrawLine[color=gray,add=.5 and .5](B,I)
  \tkzDrawCircle(O,A)
  \tkzDrawCircle[diameter](O,A)  
  \tkzDrawSegments[color=gray](B,D C,A) 
  \tkzDrawPolygon[color=red](D,D1,D2,D3,D4)
  \tkzDrawPoints(E,F,I,D1,D2,D4,D3)
  \tkzLabelPoints(A,B,C,D,O,I,E,F,D1,D2,D4,D3)  
\end{tikzpicture}  


\newpage  
\subsection{Pentagone inscrit dans un cercle inscrit dans un carré.}


\begin{tikzpicture}[scale=.8]
  \tkzInit[xmin=-6,xmax=12,ymin=-6,ymax=6] 
  \tkzClip
  \tkzDefPoint(-5,-5){A} 
  \tkzDefPoint(0,0){O}  
  \tkzDefPoint(+5,-5){B} 
  \tkzDefPoint(0,-5){F}    
  \tkzDefPoint(+5,0){F'}
  \tkzDefPoint(0,+5){E} 
  \tkzDefPoint(-5,0){K}
  \tkzDefSquare(A,B)         \tkzGetPoints{C}{D}

  \tkzInterLC(D,C)(E,B)      \tkzGetPoints{T'}{T}

  \tkzDefMidPoint(D,T)     \tkzGetPoint{I}
  \tkzInterCC[with nodes](O,D,I)(E,D,I)   \tkzGetPoints{H'}{H}
  \tkzInterLC(O,H)(O,E)     \tkzGetPoints{M'}{M}
  \tkzInterCC(O,E)(E,M)     \tkzGetPoints{Q}{M}
  \tkzInterCC[with nodes](O,O,E)(Q,E,M)    \tkzGetPoints{P}{E}
  \tkzInterCC[with nodes](O,O,E)(P,E,M)    \tkzGetPoints{N}{Q} 

  \tkzCompass(O,H)
  \tkzCompass(E,H)
  \tkzDrawArc[color=gray,style=dashed](E,B)(T)  
  \tkzDrawPolygon(A,B,C,D)
  \tkzDrawCircle(O,E)  
  \tkzDrawSegments[color=gray](O,H E,H E,F F',K)
  \tkzDrawPoints(T,M,Q,P,N)
  \tkzDrawPolygon[color=red](M,E,Q,P,N)   
\end{tikzpicture}    


\end{document}

