\documentclass[a4paper,12pt]{article}
\usepackage[latin1]{inputenc}
\usepackage{maximaplot}
\usepackage{graphicx}
\usepackage[T1]{fontenc}
\usepackage[spanish]{babel}
\usepackage{color}
\usepackage{pstricks}


\author{Jos� Manuel Mira}
\title{Gr�ficos  en \LaTeX\ con Maxima\\ II. Formatos abiertos}
\date{Julio 2006}

\begin{document}
\maketitle


\begin{abstract}
Este fichero tiene genera gr�ficos en formato abiertos, en el sentido de genera un fichero s�lo texto sobre el que si se desea es posible actuar a posteriori para modificarlo adapt�ndolo a las necesidades.

Hay varios formatos posibles: pictures, pspictures, eepic. Los �nicos universalmente compatibles son los pictures, que son tambi�n los m�s pobres en posibilidades. En el documento, separados en secciones, aparecen gr�ficos comentados con \verb|%| que pueden ser descomentados selectivamente seg�n el compilador a usar.
\end{abstract}



\section{Universalmente v�lidos:}

 El \texttt{set terminal latex} genera un entorno \texttt{picture} est�ndar de \LaTeX. El fichero resultante habr� de ser incorporado con un comando \verb|\input|. 

La ventaja que puede proporcionar este tipo de salida es que editando el fichero generado con un editor de texto pueden introducirse a posteriori modificaciones o comandos \LaTeX\ para modificar las caracter�sticas del texto o posicionar otros objetos \LaTeX\ utilizando el entorno \texttt{picture}. Las figuras~\ref{uno}, \ref{dos} y \ref{tres} son de este tipo.
 
 \begin{maximacmd}
  plot2d(sin(x),[x,-2*%pi,2*%pi],
  [plot_format, gnuplot],
  [gnuplot_preamble, 
  "set terminal latex; 
   set output 'mgrafico1.pic'"]);
 \end{maximacmd}


\begin{figure}
\begin{center}
\IfFileExists{mgrafico1.pic}{\input{mgrafico1.pic}}{}
\end{center}
\footnotesize
Usando este c�digo
\begin{verbatim}
\begin{maximacmd}
plot2d(sin(x),[x,-2*%pi,2*%pi],
  [plot_format, gnuplot],
  [gnuplot_preamble, 
  "set terminal latex; 
   set output 'mgrafico1.pic'"]);
\end{maximacmd}

\begin{center}
  \IfFileExists{mgrafico1.pic}{\input{mgrafico1.pic}}{}
\end{center}
\end{verbatim}
\caption{Gr�fico en formato picture de nombre mgrafico1.pic}\label{uno}
\end{figure}





\begin{maximacmd}
plot3d(atan(-x^2+y^3/4),[x,-4,4],[y,-4,4],
  [plot_format, gnuplot],
  [gnuplot_preamble, 
  "set terminal latex; 
   set output 'mgrafico2.pic'; set nokey;"]);
\end{maximacmd}


\begin{figure}
\begin{center}\color{red}
\IfFileExists{mgrafico2.pic}{\input{mgrafico2.pic}}{}
\end{center}
\footnotesize
Usando este c�digo
\begin{verbatim}
\begin{maximacmd}
plot3d(atan(-x^2+y^3/4),[x,-4,4],[y,-4,4],
  [plot_format, gnuplot],
  [gnuplot_preamble, 
  "set terminal latex; 
   set output 'mgrafico2.pic'; set nokey;"]);
\end{maximacmd}

\begin{center}
  \color{red}
  \IfFileExists{mgrafico2.pic}{\input{mgrafico2.pic}}{}
\end{center}
\end{verbatim}
\caption{Gr�fico en formato picture de nombre mgrafico2.pic}\label{dos}
\end{figure}

En el gr�fico de la figura~\ref{dos} se ha eliminado la leyenda (\verb|set nokey|). Si no se hubiera hecho (compru�belo) se habr�a producido un error al compilar debido a que cuando latex est� leyendo \texttt{mgrafico2.pic} se encuentra con \verb|atan(-x^2+y^3/4)| sin un entorno matem�tico. �Qu� puede hacerse? Hay dos posibilidades
\begin{enumerate}
\item Buscar a posteriori este texto en el fichero \texttt{mgrafico2.pic}  y sustituirlo por \verb|\arctan (-x^2+y^3/4)| y  continuaci�n poner \verb|%| al correspondiente entorno \texttt{maximacmd} para que no vuelva a generar el gr�fico \texttt{mgrafico2.pic} 
\par
Este ejemplo trivial ilustra lo que queremos decir con gr�ficos abiertos. Obs�rvese que debido a su car�cter ha podido cambiarse el color fuera del gr�fico, lo cual es imposible en los de formato cerrado.

\item Otra posibilidad es controlar a priori lo que aparecer� escrito en ese punto en el fichero \texttt{mgrafico2.pic}. Eso es lo que se hace en el la figura~\ref{tres}
\end{enumerate}



\begin{maximacmd}
plot2d([sin(x),x],[x,-2,2],
  [plot_format, gnuplot],
  [gnuplot_curve_titles, ["title '$\\sen x$'","title '$\ x$'"]], 
  [gnuplot_preamble, 
   "set terminal latex; 
    set output 'mgrafico3.pic';
    set key left top;"]);
\end{maximacmd}


\begin{figure}
\begin{center}
\IfFileExists{mgrafico3.pic}{\input{mgrafico3.pic}}{}
\end{center}
\footnotesize
Usando este c�digo
\begin{verbatim}
\begin{maximacmd}
plot3d(atan(-x^2+y^3/4),[x,-4,4],[y,-4,4],
  [plot_format, gnuplot],
  [gnuplot_preamble, 
  "set terminal latex; 
   set output 'mgrafico2.pic'; set nokey;"]);
\end{maximacmd}

\begin{center}
  \color{red}
  \IfFileExists{mgrafico3.pic}{\input{mgrafico3.pic}}{}
\end{center}
\end{verbatim}
\caption{Gr�fico en formato picture de nombre mgrafico3.pic}\label{tres}
\end{figure}


\section{V�lidos para latex+dvips}
El \texttt{set terminal pstricks} genera un entorno \texttt{pspicture} que requiere compilar con latex+dvips o bien con ps4pdf. La compatibilidad por tanto es menor que la generada por \texttt{set terminal latex}. El fichero gr�fico generado puede ser editado a posteriori como ocurre con los del entorno \texttt{picture} 

  
  
  
\begin{maximacmd}
plot2d([cos(x), 1-x^2/2, 1-x^2/2+x^4/24 ],[x,-2,2],
   [plot_format, gnuplot], 
   [gnuplot_curve_titles, ["title '$\\cos $'","title '$T_2 $'","title '$ T_4$'"]],
   [gnuplot_curve_styles, ["with lines 3", "with lines 4", "with lines 2"]], 
   [gnuplot_preamble, "set terminal pstricks; set output 'mgrafico4.pst';"]);
 \end{maximacmd}

\begin{figure}[p]
\begin{center}
\IfFileExists{mgrafico4.pst}{\input{mgrafico4.pst}}{}
\footnotesize
Usando este c�digo
\begin{verbatim}
\begin{maximacmd}
plot2d([cos(x), 1-x^2/2, 1-x^2/2+x^4/24 ],[x,-2,2],
 [plot_format, gnuplot], 
 [gnuplot_curve_titles, ["title '$\\cos $'","title '$T_2$'","title '$ T_4$'"]],
 [gnuplot_curve_styles, ["with lines 3", "with lines 4", "with lines 2"]], 
 [gnuplot_preamble, "set terminal pstricks; set output 'mgrafico4.pst';"]);
\end{maximacmd}

\begin{center}
  \IfFileExists{mgrafico4.pst}{\input{mgrafico4.pst}}{}
\end{center}
\end{verbatim}
\caption{Gr�fico en formato pstricks de nombre mgrafico4.pst}\label{cuatro}
\end{center}
\end{figure}


\begin{figure}[p]
\begin{center}
\IfFileExists{mgrafico4.pst}{\input{mgrafico4b.pst}}{}
\footnotesize
Usando este c�digo
\begin{verbatim}
\begin{center}
  \IfFileExists{mgrafico4b.pst}{\input{mgrafico4b.pst}}{}
\end{center}
\end{verbatim}
\caption{Gr�fico en formato pstricks de nombre mgrafico4b.pst modificado a partir del anterior}\label{cinco}
\end{center}
\end{figure}




% \footnotesize
% Usando este c�digo que es diferente del utilizado anteriormente
% \begin{verbatim}
% plot3d(-x^2+y^2,[x,-2,2],[y,-2,2],
%   [plot_format, gnuplot],
%   [run_viewer,true],
%   [gnuplot_preamble,
%     "set terminal postscript eps solid color; set output 'mgrafico2b.eps' "],
%   [gnuplot_pm3d, true]);
% \begin{center}
%   \IfFileExists{mgrafico2b.ps}{\includegraphics[scale=0.60]{mgrafico2b.eps}}{}
% \end{center}
% \end{verbatim}
% \caption{Gr�fico generado por Maxima de nombre mgrafico2b.ps}\label{tres}
% \end{figure}
% 


% Existe tambi�n un formato de salida \texttt{latex} que genera un entorno \texttt{picture} est�ndar susceptible de ser empleado tanto en salidas ps como pdf (figura~\ref{seis}).


% \begin{figure}[p]
% \begin{center}
% \IfFileExists{mgrafico5.psp}{\scalebox{0.5}{\input{mgrafico5.psp}}}{}
% \end{center}
% \footnotesize
% Usando este c�digo
% \begin{verbatim}
% plot2d(sin(x),[x,-2*%pi,2*%pi],
%    [plot_format, gnuplot],
%    [run_viewer,true],
%    [gnuplot_preamble, 
%     "set terminal pslatex solid color; 
%      set output 'mgrafico5.psp' "]);
% \begin{center}
% \IfFileExists{mgrafico5.psp}{\scalebox{0.5}{\input{mgrafico5.psp}}}{}
% \end{center}
% \end{verbatim}
% \caption{Gr�fico ps+picture de nombre mgrafico5.psp}\label{seis}
% \end{figure}

 



% # set terminal png transparent nocrop enhanced font arial 8 size 420,320 
% # set output 'surface2.9.png'
% set dummy u,v
% set key right below Right noreverse noinvert enhanced box linetype -2 linewidth 1.000 samplen 4 spacing 1 width 0 height 0 autotitles
% set parametric
% set view 50, 30, 1, 1
% set isosamples 50, 20
% set hidden3d offset 1 trianglepattern 3 undefined 1 altdiagonal bentover
% set ticslevel 0
% set title "Interlocking Tori" 0.000000,0.000000  font ""
% set urange [ -3.14159 : 3.14159 ] noreverse nowriteback
% set vrange [ -3.14159 : 3.14159 ] noreverse nowriteback
% set zrange [ * : * ] noreverse nowriteback 
% splot cos(u)+.5*cos(u)*cos(v),sin(u)+.5*sin(u)*cos(v),.5*sin(v) with lines,       1+cos(u)+.5*cos(u)*cos(v),.5*sin(v),sin(u)+.5*sin(u)*cos(v) with lines
% 
% 


 
\end{document}